\documentclass[blue]{GL2020}
\parindent=0pt
\begin{document}
\name{\bFPFCleric{}}

To be a Cleric of \cTechGod{} is to be a pillar of stability for the people of the \pTech{}. You are a spiritual beacon to your people, interpreting the will of \cTechGod{} for the masses. It is your solemn duty to both encourage the innovation that is the lifeblood of your society, and curb its excesses in order to permit only gradual, carefully controlled changes, lest the entire Workbench be upset  

\section*{A Cleric's Powers and Responsibilities}
The Temple strictly regulates use of magic and places limitations on technology. Ostensibly, they do this in the interest of maintaining the wellbeing, freedom, and equality of the people and the nation as a whole. In actuality, decisions about which technology and magic to permit and to whom are all too often arbitrary or corrupt, motivated as much by greed, lust for power, nepotism, or fear of change as they are by any high ideals. Whatever the motivations, the Priesthood jealously guards its sole authority to gatekeep technological advancements.

Clerics of \cTechGod{} are taught the ability to imbue a new piece of technology with magic, thus making it function. Some of the highest ranking also have the ability to authorize a design to be sent to production in order to make the technology available to the people of \pTech{}. (Technically, the High Priest, \cAntiChup{}, is not among those with this authority, but \cAntiChup{\they} should be able to contact the appropriate parties if the need arises.) Lower ranking Clerics have the power to diagnose and repair broken things. Much of their magical energy is directed toward imbuing the technology coming off production lines.

While the highest profile roles are related to the authorizing and imbuing new technology, there are other roles for a Cleric. In total, the Church has 3 departments aside from technology management. The vast majority of Clerics serve primarily as spiritual guides for communities of \pTechies{} (or in the middle management hierarchy that supports it). A rare few are charged with caring for the Celestial Beetles and recording their runic messages. Lastly, those who fit nowhere else end up in the Temple’s spy network. This network is a crucial tool for maintaining social stability, curtailing the blasphemous black market that deals in stolen tech, and maintaining a life preserving information edge against the other nations.

\subsection*{Temple History and Structure}
The Temple was founded at the beginning of time, at the birth of the Nation, by the Grace of \cTechGod{}. From the original Circle of Five, the Temple has grown to include many Clerics who serve all over the \pTech{}.

The current Circle of Five are the highest Officers of the Temple. They govern according to a written body of religious laws. A series of middle managers divide up the country into increasingly smaller areas, until individual Clerics are able to work directly with as few as a few dozen of \cTechGod{}'s followers under the supportive umbrella of the Temple at large.

\subsection*{The Celestial Beetles}
The Avatars of \cTechGod{}, the Celestial Beetles, are housed in the High Temple of \cTechGod{}, where they are kept safe by elite guards. As noted above, dedicated clerics are charged with caring for them and recording their messages. The Beetles write two kinds of messages: public and private. Messages meant for the public record are written in ordinary ink for all to see. Private messages sent to individual clerics — usually only the highest ranking ones — are written in special ink that vanishes as if it were never there after the message is read. The latter are rare, and difficult to verify beyond the recipient’s word, but \cTechGod{} works in mysterious ways.

\section*{Cults}
There is little tolerance for dissenting voices in the faith. Most cults to the minor gods have little power and can either be watched as they collapse under their own weight, or hurried along in their collapse with a well timed curse from the \pFarmers{} aimed at the leaders.

\section*{Becoming a Cleric}
To become a Cleric of \cTechGod{} is a long road of arduous study and proving one’s self over and over in the face of harsh judgements and often impossible standards. The halls of study for initiates of \pTech{} are ruthless and cutthroat. Few survive more than their first year without a sponsor. The rest drop out to find a less taxing calling — like responding to crises when technology goes awry. The more powerful the sponsor, the quicker the initiate rises through the ranks. It is rare, but not unheard of, for even members of the Circle of Five to support one or a handful of initiates.

After a minimum of four years of education on the roles of the Church, initiates may petition the department they wish to work in. If the department approves, they will arrange the ritual of clerichood for an auspicious day in the next quarter. How fancy the ritual and surrounding celebrations are depend largely on how much money the department has allocated to such things. Those who become spies tend to just kind of disappear. Members of the Circle of Five can grant access to any of the departments. 

Clerics are also theoretically allowed to pass on their personal mantle. A cleric who is compelled to leave the priesthood for some reason can select a successor who will step into the vacated role. This successor does not technically even have to be an initiate. The practice has fallen out of favor in the past few centuries as the Church has turned away from such blatant nepotism, preferring to conceal it behind miles of red tape and stacks of paperwork. But the rules remain on the books, so the practice remains valid and binding.

\section*{A Note About Rituals}
\emph{OOC: As an initiate or a cleric, you will have access to a number of rituals that you may be called upon to perform during the game. The Cleric Greensheet (to be released in mid to late summer) covers the minimum mechanical requirements for these rituals (these balance out the benefits provided by the rituals). There will also be an appendix with examples for what these rituals might look like in practice. Players are encouraged to create their own rituals that meet or exceed the requirements, or iterate on the examples provided, as long as doing so is fun for the player. Players may always use the examples provided instead. Some of the rituals listed here may require context you do not yet have (because it is coming in game docs not yet released) to fully understand the purpose of the ritual.}

Rituals with minimum mechanical requirements that will be on your Cleric Greensheet:
\begin{itemize}
  \item Ritual to Bless Something or Someone in the name of your patron.
  \item Ritual to Cleanse a Space.
  \item Examining a Relic to Determine Attunement.
  \item Reattuning a Relic.
  \item Inducting a New Devotee to your Patron.
  \item Promoting an Initiate to a Full Cleric.
  \item A \pTech{} specific unique ritual, to imbue technology with magic on a permanent basis.
\end{itemize}

\emph{Some characters may have additional rituals that have mechanical effects that are not known to the wider community of initiates and clerics. Initiates can only perform the first three rituals (your greensheet will have only these 3 described); the additional power available to Clerics is required for the more involved rituals.}

\emph{Players are also welcome to create additional rituals if you are so inspired, for example, weddings, funerals, etc. Such rituals will not have mechanical effects on the game without GM approval on a case by case basis per time the ritual is enacted.}

\end{document}
