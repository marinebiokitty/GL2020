\documentclass[blue]{GL2020}
\parindent=0pt
\begin{document}
\name{\bWorld{}}

\section*{World History}

\subsection*{The Creation Myth}

In the beginning, all of the Deities were equal in the nothingness. \cFarmGod{}, \cTechGod{}, and the twin \cEbb{\God}es \cEbb{} and \cFlow{} got together and decided, in their infinite wisdom, that what was needed was a world. And so they went to the other gods, and through persuasion and logic, trickery, bribery, and unexpected kindnesses, convinced the entire pantheon that they were correct. And so the Deities brought their power together and created \pEarth{}. Upon it they placed the single continent by the same name. This single land mass represents the cooperation of all the Gods, and the wonders that they could create when united. When the primary work was done, a council was called among all the gods to determine who would create the race of beings most like the Deities themselves. The vote was unanimous, and perhaps not unexpected. Those who had originally brought the idea should create it's crowning glory. And so \cFarmGod{}, \cTechGod{}, and the twin \cEbb{\God}es \cEbb{} and \cFlow{} set about creating humanity, each in their own way. The four Deities pooled their power together for a gift to humanity - the gift of Magic.  This gift came with a price, the price of the magical Storms, but the benefits of the Gods gift outweighs the cost.  And so upon \pEarth{} there walk 3 peoples, and their Patron Deities rule the Pantheon.

%In the beginning, all of the Deities were equal in the nothingness. A Deity whose name is now lost to time conceived of something new: a world - tangible and real, created by the Gods as a distraction from the endless existence. The other Deities were slow to agree at first, but as the project progressed, more and more saw the potential of this creation and sought to dominate it. Conflicts became more, and more violent. Eventually the struggle culminated with one God striking a mortal wound upon the God who had originally conceived the idea of a world. Never before had the Pantheon needed to face the loss of one of their number. In anger and fear, they rose up together and destroyed the God who had so wounded the creator. The dying Diety, for their part, was pained as much by the fighting among the Gods as by their wound. With their dying breath, they begged the Gods to set aside their quarrels and be content in each other.


%For a time, there was peace, but it was not to last. Eventually, \cFarmGod{}, \cTechGod{}, and the twin \cEbb{\God}es \cEbb{} and \cFlow{} coalessed into a powerful faction, made more powerful by the worship and conquest of their followers, that began overpowering other Deities, one by one. On \pEarth{}, the humans who served these Deities waged their own wars, and for a moment it looked like they might lose.  In that moment, the four Deities pooled their power together for a gift to humanity that would cement their worshiper's control of the world: Magic - a gift that would come a cost.  Afterwards, victory was achieved both in the place of the Gods, and on \pEarth{}.  The humans who served the Dieties spilled the blood of those who worshiped other gods, or subsumed them. Those who bowed to the new Gods were spared, and brought into the fold. Those who did not were killed.  And so \pEarth{} has been a world of three nations, and their four Patron Gods for as long as history remembers.

\section*{Geography}
\subsection*{Cengea}

%Mention the weather of each nation

\subsection*{The Children of the Sun}

%Mediterranean weather (think Italy / Greece / etc)

\subsection*{The Free People's Republic}

%Colorado weather (think high desert)

\subsection*{L'eau}
The \pShippies{} of \pShip{} live on their ships, in coastal towns on the mainland and many of the small islands surrounding the whole continent of Cengea. \pShippies{} contend with all kinds of weather, although individual ships tend to stick to their home waters as a crew prepard for the icy north would struggle in the tropical southern waters. 
\subsection*{The College of the Gods}
There is a lake in the middle of the continent of Cengea. It is not a particularly large lake, but it has many unusual properties due to the ley lines. Ley lines crisscross the ground of Cengea, providing sources of energy and natural conduits for those who know how to use them. Every major ley line in the world connects here. In the center of the lake, the lines rise from the ground, through the water. They twine about each other like vines, and together they reach toward the sky. Up and up until they reach the clouds, where they spread out again, blanketing the world.

The lake is made of something very like water. It tastes like water, quenches thirst like water, and cleans like water, but in other ways, it behaves like a much more viscous liquid. The surface of the lake is normally mirror smooth. No matter how strong the mundane wind blows, the water remains as smooth as glass. Human interference can cause ripples for a short time - i.e. throwing a stone will cause ripples, but they dissipate far more quickly than they should. It is only in the winds of the Storm that waves begin to form on the surface of the lake.

%<INSERT WHETHER CERTAIN CULTURES FORBID BATHING OR DRINKING THE WATER>

%Move this to your faction bluesheet:
%Agrarian - healing, wouldn't bathe in it (blasphemous) but used for healing
%Free People - priests are allowed to go to get water, but nobody else is permitted to touch it and it is only dispensed in a church

The lake is also unusual in that instead of having an island on the water, it has one above it. Floating over a hundred metras in the air is \pSchool{}, set on an island suspended in the twining ley lines of the world. The \pSchool{} is part teaching academy, part library, part archive, and part temple complex. The best and strongest students of magic from all three countries are sent to the \pSc{} to finish their studies and ultimately to participate in the ritual that controls the Storm.  The School can only be reached using strong magics, and when the winds of the Storms disrupt the surface of the lake, these magics become unstable and the school is not  reachable.

%metra is a standard unit of measure - somewhere between 1 and 5 feet in length.  3.14159 (pi) feet = 1 metra
%In one of the puzzles, need to be able to solve for a metra and see that it is equal to pi.

\subsection*{The Oceans and the Sea Serpents}
The continent of Cengea has a fairly large continental shelf, with shallow waters extending a good 101 kometras out from the coastline. This is lucky as the largest sea serpents won't venture onto the shelf, except during the spawning season, when the breeding serpents enter the brackish water of the deltas and birth live young. Alas the term ``sea serpent'' is a bit of a misnomer as the young live in fresh water, and even the largest have wings that can carry them short distances. They seem to be particularly attracted to any flying contraptions.  The smallest serpents are no more than 2 metras long, and live in fresh water. They swim upstream, settling in rivers and lakes across the continent, and even stretching their wings to reach terminal lakes. The only exception is the Lake under the \pSchool{}. No serpent has ever been recorded to take up residence there, so levitation to the school has always been safe.

%1 kometra = 1000 metras

\section*{The Gods}
Cengea has a pantheon of Gods. The most important are the Patron Gods: <INSERT GODS for Children and Free People>, and the twin \cEbb{\God}es \cEbbFull{}, and \cFlowFull{}. The Patrons set the rules of Cengea, granting power to those who follow their teachings, and punishing those who disrespect them. The Patrons are matched in power, and each shields their own faithful from the others. The Pantheon also contains many minor gods, who can grants small blessings, but lack the authority to issue punishment, or the power to protect worshipers from the Patron Gods.

%Creation Day - celebrate that magic exists, that Gods created that world, holiday celebrated by all nations.
%Each nations has celebrations of own God
%Each nation creates own afterlife mythos
%One shared morality rule - NO MURDER OF HUMAN LIFE!!!!!!!!! CONSEQUENCE OF THIS ERASURE OF MEMORIES!!!!!! (AKA AMNESIA)  can still do math, but don't know who your mom is / who you are.

\subsection*{Creation Day}
While each nation has their own Patron Diety, ultimately Cengea has a full Pantheon. Just because you serve primarily one Diety does not mean that it is wise to snub others when it is easy enough to metaphorically tip your hat. All of Cengea celebrates Creation Day on a day half way between the spring equinox and the summer solstice (opposite the day the storm starts to brew). Every major town throws a party of some kind on Creation Day. Some places put on reactment plays. Other places have parades. Some have street festivals. But the world over, it is a day of celebration of community and cooperation.

\subsection*{Morality and Murder}
Each Diety, and therefore each nation has their own views on morality, with one exception. Every religion condemns the taking of another human's life. Committing murder will earn you immdiate and complete amnesia; a punishment handed down directly by the Gods. This fact of life has led to some interesting quirks in the history of Cengea, including a strong bias toward weapons that neutralize enemies without killing them, and a very low number of widespread, violent conflicts. Very few people are committed enough to their goals to sacifice their memory and very sense of identity to see it through. Those groups who start out willing to do so usually dissolve quickly as a quorum of their membership forget what they were doing, and even who they are.

\subsection*{Magic}
Magic has been a part of Cengea since the beginning. The Patron Gods themselves granted humanity magic, breathing life into the ley lines. Each spoke to their most loyal followers and ordered them to build a school on top of the floating island in the middle of the continent. After a few decades spent mastering enough magic to figure out how to get up there, the people of Cengea did just that. It was not for several more decades that people came to understand why the \pSc{} mattered. About a century after the Gods granted humans magic, the Storms began. It was another 30 years before humanity learned to control the storms well enough to direct them in any particular direction and therefore mitigate the damage somewhat.

\pFarm{} magic is <INSERT DESCRIPTION>. \pTech{} magic is <Insert Description>. \pShip{} magic is subtle.  It works primarily by enhancing aspects that already existed in an object or person. Their magic is intuitive and personal, an extension of the craftsmenship and attention to detail that \pShippies{} bring to their work.
%Define in brief what kind of magic you expect from each country

\subsection*{The Storm}
%%Secret/unnoticed benefits of the storm ie rejuvenating the soil in \pFarm{}
Every \pCycle{} years, a Storm brews on the lake below the \pSc{}, blocking off access to the \pSc{} for several days starting around the half way point between the fall equinox and the winter solstice. After building in power for about a week, the storm spins off to wreak havoc on some part of Cengea. The students at \pSchool{} can work together, following an ancient ritual, and direct the storm toward one of three areas: The nation of \pFarm{}, \pTech{}, or to the coastlines and outlying islands where the \pShippies{} live and work. Every effort in the past to mitigate the damages done by the Storm has failed. No matter how people try to avoid the damage, the same failure occurs - the storm escapes control of the students and wreaks havoc over the entire continent before dissipating. In the end, more damage is done to all three countries than would have been done to a single country if the attempt had not been made. 

Thirty four years ago, the treaty was established that yielded the rotating pattern of who suffered the storm. Prior to that, the Time of Deciding was a chaotic power grab that left a brutal body count in it's wake, both at the \pSchool{} and in whichever country ended up losing out and getting hit that year. For 25 years, the treaty was honored, and each nation took the storm in turn. Forewarned, the nation targeted was able to devote resources to protecting their most valuable and vulnerable assets, and the others could concentrate on growth. When the treaty was broken 6 years ago, and \pShip{} was struck out of turn, the consequences were devastating. The \pShippies{} were completely unprepared, and lost nearly half of their fleet in the first day. When they were hit a second time, three years ago, they were slightly better prepared, but there wasn't much left to protect.

Some folks advocate for a return to the treaty - although opinions are split on who's turn it would be. Others point out how vulnerable relying on such things made the \pShip{}. Any attempt to reforge the treaty will be a fraught affair and successful adoption will require ratification by home governments, all of whom will be persnickety about the terms. \emph{OOC Note: There will be very limited communication in and out of the school during game, and any treaty that does not meet with the approval of the various home governments will not be successful.}

%Limited communication, permitted to attempt a new treaty but requires approval of folks out of game (ie: GMs can send nasty notes on behalf of nations when they send a useless treaty)  Must agree to send message before it can be sent, can send first message on first game, get response on second, send second reponse on second, send answer on third.  Point out in bluesheets most important points that countries need to get ratified so that players don't waste their time

%Amanda volunteers for treaty creation GM assistance and writing nasty letters from countries

Timeline of the Storms:
Storms - hit every 3 years
Present year - happening, don’t know where
T-3: \pShip{}//
T-6: \pShip{} (was \pTech{} turn)//
T-9: \pFarm{}//
T-12: \pShip{}//
T-15: \pTech{}//
Etc.

%Food on a player's plate =/= what the character is eating!

\end{document}
