\documentclass[blue]{GL2020}
\parindent=0pt
\begin{document}
\name{\bWorld{}}

\section*{\pEarth{}}
\pEarth{} is a single continent, divided among three nations. The \pFarm{} controls the southernmost two thirds of the continent, and the vast majority of the arable land. The \pTech{} dwell in and around the northern mountains of the \pSpine{}. The \pShip{} control a small wedge of land on the eastern coast, and dominate the seas, all the way around the continent and its myriad islands.

Each nation has its own culture, religion, and form of government. While the pantheon is shared, each nation particularly worships their Patron God. Over \pEarth{}’s long history, the nations have vacillated between cooperation and hostilities. Since on \pEarth{}, murder is forbidden by the Gods, much of the tension between nations is played out in ways other than war and bloodshed. A particular sticking point is control of the Storms. These magical Storms manifest once every three years, and will do devastating damage to some portion of the continent, so being able to send the Storm somewhere else is crucial to the advancement of each nation.

No other continents or cultures are known. \emph{(This is not a game about discovering a new continent or first contact with an alien race. No such plots exist, and players should focus their game on the situation on \pEarth{}.)}

\section*{World History}
The planet and the continent (both called \pEarth{}) were created by the Gods -- according to ancient religious texts, particularly the four Gods who are now the Patron Gods of each nation. They each created a people to their own taste, whose descendants still compose each nation (save the odd immigrant who emigrates from one nation to another). In this way, the people are more than just worshipers -- they are the Children of the Gods.

Even the best kept records in \pEarth{} that go back centuries and centuries do not include a time in which there were more than these three nations, eternally locked in uneasy codependence. Each nation needs things from the others to thrive. But cooperation leaves one vulnerable to betrayal. \pEarth{}’s history is littered with those, on both personal and national scales.

Still, it seemed like things might have permanently changed for the better after the most recent treaty was formed. 34 years ago, a treaty was established that regulated where the Storm would be sent, allowing each nation forewarning with which to prepare, and ensuring that the burden was shared around equally. But 6 years ago, the \pFarm{} and \pTech{} forged a new alliance and betrayed the \pShip{}, sending the Storm to that country out of turn.

\subsection*{The War}
Ever since, the \pShip{} have been at war with the other two nations. War in a world where murder is forbidden by the Gods is a strange thing. Since it is impossible to field armies (attempts have been made historically; they all ended in mass confusion quite quickly), the war is waged indirectly. The \pShip{} drive sea serpents onto the shore, where the creatures wreak massive damage before escaping back into the ocean. The \pShip{} have caused significant damage to settlements in proximity to bodies of water in both other countries this way, and have staked claim over many coastal and riverside areas around the continent as a result.

For the first three years of the war, the \pFarm{} and the \pTech{} had little answer to the \pShippies{}, but after much bloodshed, finally developed the technology to use massive trebuchets from \pTech{} to launch curses from \pFarm{} at \pShip{} ships. The trebuchets are massive undertakings to build (especially without the craftsmanship of the \pShip{} magic users), and must be built as permanent installations. There are a few grand and far off plans of mobile trebuchets, but for now, they are primarily useful as defensive installations.

In turn, the \pShip{} are innovating as well. They are starting to send landing parties with the sea serpents, cutting off their retreat and driving the serpents further inland than they would otherwise go. This tactic is too new to know whether it will have a significant impact on the war.

\section*{Geography of \pEarth{}}

\subsection*{The \pFarm{}}
The \pFarm{} control almost two thirds of the continent to the south, including most of the arable land. The weather is quite Mediterranean, with a long growing season, and vast farms, often stretching as far as the eye can see. The nobility work fertility magic that allows the land to grow 10x the amount of food it otherwise could, allowing the \pFarmers{} to grow most of the food that all three nations rely on. The common folk work with curses. And the Church controls music magic, which is used for some of the finest healing in all of \pEarth{}. \cFarmGod{} values <> and punishes harm to animals. Most \pFarmers{} are vegetarian on account of this, or will eat meat and fish only if it is prepared by another.

\subsection*{The \pTech{}}
The \pTech{} dwell in and around the northern mountains of the \pSpine{}. Much of their land is high desert -- poor for growing crops, and with wide fluctuations in temperature. Snow blankets large portions of the land, especially in the higher elevations for months of the year. The \pTech{} land is rich in minerals and this has led to them being the epicenter of new technology on \pEarth{}. Innovation is prized above almost everything else in \pTech{}, and inventors bring prototypes to the Church for approval. Once approved, the invention can be sent to production and imbued with the magic necessary to make it function. While many inventors are magical themselves, and could \emph{theoretically} imbue their own prototypes with magic, doing so incurs the wrath of \cTechGod{}.

\subsection*{\pShip{}}
The \pShippies{} of \pShip{} live on their ships, in coastal towns on the mainland, in a few settlements on the Eastern floodplain, and on the many small islands surrounding the whole continent. \pShippies{} contend with all kinds of weather, although individual ships tend to stick to their home waters. Before the war, the \pShippies{} provided protection from Sea serpents to the other two nations, as well as the majority of travel and shipping. They worship \cEbb{\full} and \cFlow{\full}, and their magic specializes in craftsmanship of all kinds. The \pShip{} magic is the subtlest of all, enabling everything from cooking to shipbuilding in unquantifiable ways. Only a magical weapon will pierce the hide of a sea serpent. Only a magical sieve can transform seawater to fresh. To leave a debt, of any kind, unpaid, is to disrespect the Goddesses.

\subsection*{The \pSchool{}}
The \pSchool{} sits at the center of \pEarth{}, and is the only truly neutral place on the continent. All three nations send their brightest, best connected, and most magically powerful to the \pSc{} -- for a hefty price. More information is available in the \pSchool{} bluesheet.

\subsection*{The Oceans and the Sea Serpents}
The continent of Cengea has a fairly large continental shelf, with shallow waters extending a mile out from the coastline. The waters are infested with sea serpents that come in every color of the rainbow. The serpents have always attacked ships and coastal settlements with distressing frequency. No ship that has sailed beyond the continental shelf has ever returned.

Alas the term ``Sea Serpent'' is a bit of a misnomer as the young live in fresh water, and even the largest have wings that can carry them vast distances. The babies are less than a foot long when they are born. They swim upstream from the brackish water in which they hatch, settling in rivers and lakes across the continent, and even stretching their tiny wings to reach terminal lakes. The oldest of the old, past breeding age, leave the water, and fly to the \pSpine{}. They take up residence in the \pSpine{}, and live out their days ambushing anyone foolish enough to venture into their caves.

\section*{Religion}
\pEarth{} has a pantheon of Gods. The most important are the Patron Gods: \cFarmGod{}, \cTechGod{}, and the twin \cEbb{\God}es \cEbbFull{}, and \cFlowFull{}. The religions are separate but intertwined, since they share a pantheon. Each nation primarily worships their Patron, but will offer small gestures to the others, even in their homeland, especially when asking a boon that falls under another God’s purview. When visiting another nation, it is considered a huge social faux pass to not engage with their religious services and traditions. Why risk snubbing the other Gods when it is easy enough to metaphorically tip your hat?

Religion is a deeply important part of the cultures on \pEarth{}, and in individual lives. Everyone knows at least one person who has spoken to the Gods in a dream, or through one of the \pEarth{}n Avatars. Many people have had prayers answered through obviously divine intervention. Those who do not take religion seriously are shunned, and so those few who feel less drawn to, or less supported by the system, tend to keep that to themselves and pretend in public.


\subsection*{The Creation Story}
In the beginning, all of the Deities were equal in the nothingness. \cFarmGod{}, \cTechGod{}, and the twin \cEbb{\God}es \cEbb{} and \cFlow{} got together and decided, in their infinite wisdom, that what was needed was a world. And so they went to the other gods, and through persuasion and logic, trickery, bribery, and unexpected kindnesses, convinced the entire pantheon that they were correct. And so the Deities brought their power together and created \pEarth{}. Upon it they placed the single continent by the same name. This single land mass represents the cooperation of all the Gods, and the wonders that they could create when united. When the primary work was done, a council was called among all the gods to determine who would create the race of beings most like the Deities themselves. The vote was unanimous. Those who had originally brought the idea should create its crowning glory. And so \cFarmGod{}, \cTechGod{}, and the twin \cEbb{\God}es \cEbb{} and \cFlow{} set about creating humanity, each in their own way. The four Deities then pooled their remaining power together for a gift for their new Children -- the gift of Magic.

\subsection*{The Gods}
The Patron Gods set the rules of \pEarth{}, granting power to those who follow their teachings, and punishing those who disrespect them. The Patrons are matched in power, so each individual need only follow the teachings of their patron God. The Pantheon also contains many minor gods, who can grant small blessings when invoked properly, but lack the authority to issue punishment, or the power to protect worshipers from the Patron Gods.

\subsection*{Clerics of the Gods}
Clerics are the keepers of religious lore and tradition. They lead ceremonies, provide counsel to individuals, and guidance to governments. As all magic comes from the Gods, they are also often the primary keepers of magic in a community. While people generally prefer to go to Clerics from their own nation, any Cleric should be able to provide sound, non-partisan advice to a person in need.

\subsection*{The Avatars}
Each Patron God has one or more avatars on \pEarth{}. These beings are conduits of the Gods’ power, and each takes the form of an animal. In \pFarm{}, the Avatars are hummingbirds the size of a small wagon. For the \pTech{}, a particular species of beetle. And in the \pShip{}, two sea serpents of immeasurable beauty -- one with iridescent silver scales, and the other gold. These Avatars will often serve as the mouthpieces for the Gods, bearing messages to chosen heroes and so forth. The Avatars are also a source of power and connection. Magic too difficult to do any other way can be done in the presence of an Avatar. If an Avatar were to be lost, it would severely cripple the God’s ability to impact life on \pEarth{}.

\subsection*{Morality and Murder}
Each Deity, and therefore each nation, has their own views on morality, with one exception: taking of another human's life. Committing murder will earn you immediate and complete amnesia; the punishment handed down directly by the Gods. 

Social opinions on those who have been struck with amnesia are split. Most people think of amnesiacs as unfortunate cases that now have the possibility to become better people. The circumstances that drive someone to kill are seen as a matter of nurture rather than nature -- so amnesiacs are generally accepted, reintegrated into society, and given the support they need to rebuild their lives. A smaller, but vocal minority treat these people as pariahs, and can never trust someone who has killed.

\subsection*{Rejecting the Gods}
To be an atheist is almost unthinkable. Even setting the vast societal pressures aside, claiming a patron Deity protects you from the rules of the others. Therefore, an atheist, far from being free from the Gods, is subject to 3 sets of rules and taboos rather than just one -- as well as the constant threat of displeasure from the divine with no recourse. And, of course, an atheist is denied the God’s gift to their children -- the right to use magic. It is a difficult and thankless lifestyle choice, chosen only by the most defiant and the truly foolish.

\subsection*{Magic}
Magic has been a part of \pEarth{} since the beginning. The Patron Gods themselves granted humanity magic, breathing life into the ley lines, then bade their people to build a school on top of the floating island in the middle of the continent. A century after the \pSchool{} was completed, the Storms began.

While there are many things in the world of \pEarth{} that an outsider might consider magical, the term ``magic’’ is reserved for the power that humans wield, and the things they create with it. Magical ability manifests at a young age, just beginning to show at around 6 or 7. While it’s hard to measure the raw potential of a completely untrained magic user, it is generally believed that the younger the user, the more raw power they wield. Of course, magic is nothing without control -- so the strongest magic users are young geniuses who learn to wield their power at a young age. For the purposes of the Ritual to Control the Storm, College age is considered the ideal balance between power levels and magical knowledge. 

This is not to say that elders in \pEarth{} who were magic in their youth are without magic now. They still have power, but it is more specialized and limited in scope. They must use skill and cleverness to address problems they might previously have been able to brute force.

\subsection*{The Storm}
Every \pCycle{} years, a Storm brews on the lake below the \pSc{}, blocking off access to the \pSc{} for several days at the halfway point between the fall equinox and the winter solstice. The Storm waxes and wanes in power as it builds, becoming most dangerous at night. After building in power for several days, the storm spins off to wreak havoc on some part of Cengea. 

The students at \pSchool{} can work together, following an ancient Ritual, and direct the storm toward one of three areas: The nation of the \pFarm{}, the \pTech{}, or to the coastlines and outlying islands where the \pShippies{} live and work. Every effort in the past to mitigate the damage done by the Storm has failed. No matter how people try to avoid the damage, whether by sending the Storm to a remote location or evacuating the area, the same failure occurs -- the Storm escapes control of the students and wreaks havoc over an even wider area before dissipating. In the end, more damage is done to all three countries than would have been done to a single nation if the attempt had not been made. 

\section*{The Time of Deciding}
This weekend is the ``Time of Deciding.’’ This weekend the Storm is brewing under and around the \pSchool{}, and must be sent somewhere. The people at the school now are the ones tasked with this decision. The whole process is quite complicated, involving a series of interconnected activities:

\subsection*{Voting Authority}
Four students from each nation are selected to stay at the \pSc{} during the Time of Deciding. They must take on the massive responsibility of directing the Storm. Students are selected based on their magical power and academic performance relative to their school year peers. As the selection process is concerned with selecting the exceptional rather than just the most tenured, students in higher years are not given preference over those in lower years. It is one of the highest honors a student can receive, and it can have a lot of influence on their future after they graduate.

The students will be ranked within their nation during the weekend -- by the teachers, the advisors, and by the Gods themselves, based on the actions the students take. The highest ranking Student will receive the most voting tokens on Sunday morning.

\subsection*{Influencing the Voting}
During the Time of Deciding, advisors from each nation come to the school, to guide and influence the students. Since the students are young adults, it is crucial that the advisors explain the consequences of the various options, given the current political landscape. The Teachers are ostensibly neutral parties in the discussion, but few have no opinion on where the Storm should go, and students often take council from teachers as well.

Since the advisors have no direct control over the voting, they take the task of influencing the students quite seriously. Different advisors will employ different tactics to try to influence the students, but outright threats of bodily harm are generally frowned upon. Since the advisors help rank the students, it is quite common for advisors to favor the student who agrees to vote the way the advisor wants. Since being ranked at the top opens almost any door for a student’s future, it is an attractive offer.

\subsection*{Storm Surges and The Bunkers}
During the Time of Deciding, the Storm is brewing. Three times over the course of the weekend, the Storm will ``Surge’’ in power, and it will become dangerous to be out and about at the \pSc{}. There are three Bunkers installed at the school, which provide protection from the physical, mental, and magical damage that the Surges can do to an unprotected person.

\subsection*{Voting to control the storm}
On Sunday morning, students will be given their voting tokens. Different students will receive different numbers of tokens, based on their final rankings, but each nation will receive the same number of tokens. The students will then need to cast their votes. Students are strongly advised to have a good idea of how they want to vote \emph{before} Sunday morning, as any votes not submitted by 11:30 am are forfeit.

\subsection*{The Ritual to Control the Storm}
Over the course of the weekend, the Ritual to Control the Storm must also be prepared. Several teachers are well versed in the preparations necessary, and will be leading the effort to organize the Relics and other preparations. On Sunday afternoon, the ritual will be enacted, requiring the participation of almost everyone at the \pSchool{}. Assuming everything was done properly, it will send the Storm wherever the Students voted to send it.

\subsubsection*{The Relics}
The Relics are objects of incredible magical power. Some are gifts from the Gods and others were forged by particular groups of humans, as epic and heroic undertakings. Regardless of their origin, the Relics are handed down as national heirlooms. The Relics are either stored at the \pSchool{} permanently, or brought there during the Time of Deciding to charge them with magical energy. The Relics can do incredible things that cannot be accomplished any other way, when they are charged up. One Relic from each nation must be used in the Ritual to Control the Storm, but others can be used to work miracles. Deciding which Relic to use is a crucial decision facing each nation.

\section*{A New Treaty?}
When the previous treaty was broken 6 years ago, and \pShip{} was struck out of turn, the tenuous peace was shattered. Still, after so many years of war, more and more people are willing to consider a peace treaty, or at least a ceasefire of some kind. Opinions are deeply divided on whose turn it would be to take the Storm’s fury. Others point out how vulnerable relying on such things made the \pShip{}. Any attempt to reforge the treaty will be a fraught affair and successful adoption will require ratification by home governments \emph{(OOC Note: managed by the GMs)}, all of whom will be persnickety about the terms.

\end{document}







REMOVED SECTIONS

\subsection*{The Creation Myth} For real
%In the beginning, all of the Deities were equal in the nothingness. A Deity whose name is now lost to time conceived of something new: a world - tangible and real, created by the Gods as a distraction from the endless existence. The other Deities were slow to agree at first, but as the project progressed, more and more saw the potential of this creation and sought to dominate it. Conflicts became more, and more violent. Eventually the struggle culminated with one God striking a mortal wound upon the God who had originally conceived the idea of a world. Never before had the Pantheon needed to face the loss of one of their number. In anger and fear, they rose up together and destroyed the God who had so wounded the creator. The dying Diety, for their part, was pained as much by the fighting among the Gods as by their wound. With their dying breath, they begged the Gods to set aside their quarrels and be content in each other.


%For a time, there was peace, but it was not to last. Eventually, \cFarmGod{}, \cTechGod{}, and the twin \cEbb{\God}es \cEbb{} and \cFlow{} coalessed into a powerful faction, made more powerful by the worship and conquest of their followers, that began overpowering other Deities, one by one. On \pEarth{}, the humans who served these Deities waged their own wars, and for a moment it looked like they might lose.  In that moment, the four Deities pooled their power together for a gift to humanity that would cement their worshiper's control of the world: Magic - a gift that would come a cost.  Afterwards, victory was achieved both in the place of the Gods, and on \pEarth{}.  The humans who served the Dieties spilled the blood of those who worshiped other gods, or subsumed them. Those who bowed to the new Gods were spared, and brought into the fold. Those who did not were killed.  And so \pEarth{} has been a world of three nations, and their four Patron Gods for as long as history remembers.


%Creation Day - celebrate that magic exists, that Gods created that world, holiday celebrated by all nations.
%Each nations has celebrations of own God
%Each nation creates own afterlife mythos
%One shared morality rule - NO MURDER OF HUMAN LIFE!!!!!!!!! CONSEQUENCE OF THIS ERASURE OF MEMORIES!!!!!! (AKA AMNESIA)  can still do math, but don't know who your mom is / who you are.

\subsection*{Creation Day}
While each nation has their own Patron Diety, ultimately Cengea has a full Pantheon. Just because you serve primarily one Diety does not mean that it is wise to snub others when it is easy enough to metaphorically tip your hat. All of Cengea celebrates Creation Day on a day half way between the spring equinox and the summer solstice (opposite the day the storm starts to brew). Every major town throws a party of some kind on Creation Day. Some places put on reenactment plays. Other places have parades. Some have street festivals. But the world over, it is a day of celebration of community and cooperation.


Timeline of the Storms:
Storms - hit every 3 years
Present year - happening, don’t know where
T-3: \pShip{}//
T-6: \pShip{} (was \pTech{} turn)//
T-9: \pFarm{}//
T-12: \pShip{}//
T-15: \pTech{}//
Etc.


%Limited communication, permitted to attempt a new treaty but requires approval of folks out of game (ie: GMs can send nasty notes on behalf of nations when they send a useless treaty)  Must agree to send message before it can be sent, can send first message on first game, get response on second, send second reponse on second, send answer on third.  Point out in bluesheets most important points that countries need to get ratified so that players don't waste their time

%Amanda volunteers for treaty creation GM assistance and writing nasty letters from countries

