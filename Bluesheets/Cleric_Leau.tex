\documentclass[blue]{GL2020}
\parindent=0pt
\begin{document}
\name{\bLeauCleric{}}

Being a Cleric of \cEbbFull{\full} and \cFlowFull{\full} is beautiful, wonderful, and all encompassing. You are their eyes, ears, and mouth to the \pShippies{}. You are expected to live your life according to the strictest standards of their edicts. If you cannot follow the tenets, how will you convince the faithful that it is not only possible, but meaningful and fulfilling to do so?

\section*{A Cleric's Powers and Responsibilities}
\cEbb{} and \cFlow{} grant their Clerics power in many areas. They grant the clarity of mind to guide decisions, the steadiness of hand to repair things in need of it, and the force of will to destroy that which has run its course. 

The greatest power a Cleric of \cEbb{} and \cFlow{} has is in listening without judgment to a supplicant's plight, and offering guidance and suggestions. It is important to keep in mind that the \pShip{} religion believes that no true peace can be achieved without Balance. If someone has been wronged, the most likely advice from a Cleric is: get even. The understanding is that nothing short of reciprocating in kind will allow someone to put down their hurts. If someone has wronged another, a Cleric will likely advise, ``make your peace with receiving back what you gave out. Remember how this feels before you act again.''

\section*{Cults}
There is little tolerance for dissenting voices in the faith. Most cults to the minor gods have little power and can either be watched as they collapse under their own weight, or hurried along in their collapse with a well timed curse from the \pFarmers{} aimed at the leaders.

\section*{Becoming a Cleric}
To become a Cleric is not easy in the \pShip{}. Most are not even allowed to try. \cEbb{} and \cFlow{} mark most of those with the aptitude for it before their 15th birthday. It is exceedingly rare for them to appear later (but not completely unheard of). The mark may appear anywhere on the person’s body, but the back of one or both wrists is the most common. Once the mark appears, the individual may be enrolled at a Monastery exclusively, or continue at their Academy, and take supplemental classes at the Monastery. Either way, a minimum of 4 years of study is typical before anyone will sponsor the Initiate (as they are now called) to the full priesthood. For details on the ritual to ordain a new Cleric, see ``Promoting someone to a Full Cleric'' in your ``Being a Cleric'' greensheet.

Before the ritual can begin, the Initiate must pick a Path. Once you are brought into the priesthood on a Path, you cannot change. Therefore, the decision is incredibly important, and Initiates often spend months, or even years, agonizing over it. Most Initiates ultimately choose to follow \cEbb{} or \cFlow{} primarily — but some few attempt to walk the difficult path of true Balance.
\begin{itemize}
  \item \cEbbFull{\full} is the keeper of ships, and the caretaker of things lost or taken away. Clerics of \cEbb{} know the value in loss, destruction, and removal. One cuts away at the wood to reveal the ship within. The end of one dream opens new and better opportunities. Sacrifice is honored highly. Loss and grief are comforted. Cleaning and clearing make way for the growth of \cFlow{}.
  \item \cFlowFull{\full} is the keeper of the rain, and caretaker of things created or given. Clerics of \cFlow{} know the value of starting, building, and nurturing. One starts a new project because they believe in the outcome. One gives things away to start ripples, which become waves of sharing and growth. Creation and generosity are revered. Quality, and the patience to create it are cultivated. Unbridled growth prepares for the shaping and cultivating of \cEbb{}.
  \item To walk the path of Balance between these two is the most challenging of all. Balance is about the moment of turning. When is it time to stop cutting things? When is it time to take a break from creating? How much iterating is too much? It's in the fullness at the top of the breath, and the emptiness at its bottom. The Path of Balance is all about turning points and change. It is about reining in the excess of either extreme, and bringing things into harmony. There are no more than two dozen Clerics of Balance among all of the \pShippies{}. Those rare few for whom marks appear late are often particularly good candidates for the Path of Balance. If someone insists upon the path of Balance, both Avatars must be summoned as part of the ritual.
\end{itemize}

\section*{A Note About Rituals}
\emph{OOC: As an initiate or a cleric, you will have access to a number of rituals that you may be called upon to perform during the game. The Cleric Greensheet (to be released in mid to late summer) covers the minimum mechanical requirements for these rituals (these balance out the benefits provided by the rituals). There will also be an appendix with examples for what these rituals might look like in practice. Players are encouraged to create their own rituals that meet or exceed the requirements, or iterate on the examples provided, as long as doing so is fun for the player. Players may always use the examples provided instead. Some of the rituals listed here may require context you do not yet have (because it is coming in game docs not yet released) to fully understand the purpose of the ritual.}

Rituals with minimum mechanical requirements that will be on your Cleric Greensheet:
\begin{itemize}
  \item Ritual to Bless Something or Someone in the name of your patron.
  \item Ritual to Cleanse a Space.
  \item Examining a Relic to Determine Attunement.
  \item Reattuning a Relic.
  \item Inducting a New Devotee to your Patron.
  \item Promoting an Initiate to a Full Cleric.
  \item A \pShippie{} specific unique ritual, TBD.
\end{itemize}

\emph{Some characters may have additional rituals that have mechanical effects that are not known to the wider community of initiates and clerics. Initiates can only perform the first three rituals (your greensheet will have only these 3 described); the additional power available to Clerics is required for the more involved rituals.}

\emph{Players are also welcome to create additional rituals if you are so inspired, for example, weddings, funerals, etc. Such rituals will not have mechanical effects on the game without GM approval on a case by case basis per time the ritual is enacted.}

\end{document}

