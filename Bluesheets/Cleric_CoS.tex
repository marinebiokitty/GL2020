\documentclass[blue]{GL2020}
\parindent=0pt
\begin{document}
\name{\bCoSCleric{}}

For those who become a Cleric of \cFarmGod{}, the path is full of beauty, wonder, and possibility. It is an opportunity for those amongst the working class to pursue a life of music, scholarship, or education; and it is an opportunity for the non or less magically gifted children among the Nobility to find purpose in life outside of childcare and estate management. For details about \cFarmGod{}'s precepts and domains, see the \pFarm{} bluesheet.

\section*{A Cleric's Powers and Responsibilities}
Clerics are expected to be sources of advice and support to their community. They lead rituals to honor \cFarmGod{}, mainly focused around blessing people and animals and especially healing of all forms, including the healing of a space or an emotional wound. They preside over important milestones in the life of a community and a family, especially the birth and watching over of children, marriage, and adoption. 

\cFarmGod{} grants \cFarmGod{\their} Clerics power in the finely honed musical healing arts. \pFarm{} Clerics with only rudimentary musical talents find that the songs of healing bloom in their hearts and spill easily from their lips after they dedicate themselves to \cFarmGod{} and accept \cFarmGod{\their} touch. Those Clerics strongest in the healing arts find that over time they develop something almost like a prescience when it comes to the health of the community. They will feel called to walk a certain path, only to come upon someone ill or injured on their travels.

In addition to their healing and life rituals, they guide wayward souls back to the path of goodness. They enforce the rules of society, through parables and advice. The clerics form the backbone of the \pFarm{} education system for the commoners. They work in tandem with the younger children of noble families who choose to go into education. 

For those for whom teaching does not suit, there is the opportunity to become a historian or a researcher. The Church of \cFarmGod{} keeps many of the records of the \pFarm{}. If one can earn a spot at one of the research institutions, one can study and research the most esoteric of religious doctrine to their hearts’ content. 

That said, there is little tolerance for dissenting voices in the faith. While the Church knows that most cults to the Minor Gods have little power and can either be watched or potentially dealt with with well timed curses aimed at leaders, 200 years ago a splinter faction known as the Aethers challenged well established Church canon that the Divine Realm had been severed from the mortal realm by the will of the Gods. They purportedly developed a ritual to travel between the two planes, showing great power where usually anyone outside of the Church’s power structure had none. While it is not known outside of those high up in the priesthood, the Church was terrified that if the Aether’s succeeded in their mission that the Church’s own power and purpose would be usurped. Such things could not be tolerated, and so it was decided to violently destroy all Aether members. Many memories were lost that day, but it was all in service of the greater good. 

The Church keeps a tight lid on information about the Aethers, particularly about their fate at the hands of the Church, as it would surely contribute to dissent and disapproval of the Church's actions, which could fuel the rise of other such cults.

\section*{Becoming a Cleric}

Clerics can come from any section of society, but typically come from nobles who do not have much magic or common folk who have been taught by the priests and feel called (and whose families can spare them from working the land or the trades). Clerics begin as initiates, and are taken under the wing of a senior cleric as an apprentice, traveling with them and learning the ways of \cFarmGod{}. Those of a more scholarly bent or who wish to rise high in the ranks of the priesthood also can attend a few schools dedicated to training Initiates. Noble Initiates tend to pursue this approach, though it is not unheard of for especially promising commoner youth.  

After a full apprenticeship of seven years and potentially additional study, the sponsoring cleric can decide that their Initiate is ready. Usually they send their Initiate on some sort of visit or mission by themselves to demonstrate their knowledge and healing ability. For those wishing to pursue scholarship, an exam is also held before one can become a cleric. The process can take many years, and becoming a full cleric is a big deal. 

Careful vetting is important because an Initiate’s final oath is accepted by an Avatar of \cFarmGod{} \cFarmGod{\them}self, a giant hummingbird. To be rejected by the Avatar is something that is deeply shameful to all parties involved, though it is a rare occurrence. When a person is deemed ready, they will take their oath before the community, their sponsoring cleric, and other clerics of the faith.

\section*{The Hummingbirds}
Fewer and fewer of the Avatars of \cFarmGod{}, the giant hummingbirds, have been seen in recent years.  While the decline started gradually somewhere in the last 150 to 200 years, it has become worse and worse since then. Some scholars point to other times in history when this has happened, and posit that it is a natural cycle. Others fret that it might be an ill omen of some kind. So far, the Priesthood has done its best to keep a lid on the fact that the hummingbirds are disappearing, but the rumors grow more difficult to contain with each passing year.

\section*{A Note on Rituals}

\emph{OOC: As an initiate or a cleric, you will have access to a number of rituals that you may be called upon to perform during the game. The Cleric Greensheet (to be released in mid to late summer) covers the minimum mechanical requirements for these rituals (these balance out the benefits provided by the rituals). There will also be an appendix with examples for what these rituals might look like in practice. Players are encouraged to create their own rituals that meet or exceed the requirements, or iterate on the examples provided, as long as doing so is fun for the player. Players may always use the examples provided instead. Some of the rituals listed here may require context you do not yet have (because it is coming in game docs not yet released) to fully understand the purpose of the ritual.}

Rituals with minimum mechanical requirements that will be on your Cleric Greensheet:
\begin{itemize}
  \item Ritual to Bless Something or Someone in the name of your patron.
  \item Ritual to Cleanse a Space.
  \item Examining a Relic to Determine Attunement.
  \item Reattuning a Relic.
  \item Inducting a New Devotee to your Patron.
  \item Promoting an Initiate to a Full Cleric.
  \item A \pFarm{} specific unique ritual, to heal.
\end{itemize}

\emph{Some characters may have additional rituals that have mechanical effects that are not known to the wider community of initiates and clerics. Initiates can only perform the first three rituals (your greensheet will have only these 3 described); the additional power available to Clerics is required for the more involved rituals.}

\emph{Players are also welcome to create additional rituals if you are so inspired, for example, weddings, funerals, etc. Such rituals will not have mechanical effects on the game without GM approval on a case by case basis per time the ritual is enacted.}

\end{document}

