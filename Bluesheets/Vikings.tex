\documentclass[blue]{GL2020}
\parindent=0pt
\begin{document}
\name{\bVikings{}}

The \pShippies{} are a proud people, with a long history of mercantilism and endless combat against the sea serpents of \pEarth{}. For countless years, they were the caretakers of \pEarth{}'s oceans and her people’s cargo. Not everyone in the \pFarm{} and the \pTech{} appreciated the \pShip{} ways of ``protecting'' the coast, but nevertheless an equilibrium existed that allowed all three countries to gain mutual benefit.

\section*{Geography}
The \pShip{} are mostly a waterfaring folk. They travel \pEarth{}'s oceans, as far out as the continental shelf extends. (No ship that has ventured out beyond the shelf has returned to tell the tale.) The \pShippies{} also float \pEarth{}'s rivers and lakes, using them as highways for travel, moving goods and people from the other two nations as well as their own.

Most \pShip{} towns cling to the coastline, no matter how rugged. Before the war, it was only on the eastern side of \pEarth{} that the \pShippies{} claimed any land in the interior, along a massive floodplain, where rivers abound to facilitate travel. Around the rest of the continent, the \pShip{} control small stretches of mainland coastline, and almost every island large enough to support a human settlement.

The only area of ocean that most \pShippies{} fear to sail is \pWod{}. \pWod{} is due west of \pEarth{}, on the border between 8th and 9th fleet. It is an area of unpredictable currents and jagged rock pillars looming out of the ever-present mist. Only the bravest — or most desperate — crews will sail \pWod{}. Scavenging from the shipwrecks may be profitable, but is also a constant reminder of the dangers of the area.

\section*{The Betrayal}
Six years ago, it was the \pTech{}'s turn to have the storm directed at them, but the storm was directed at the \pShip{} instead. The damage was tremendous. Unprepared for the storm to hit them, the \pShippies{} were about their regular business. As a result, more than half of her ships were destroyed in the initial 48 hours of the week-long onslaught.

Reports from \pSchool{} about what happened were frustratingly vague. Despite the consequences of murder in \pEarth{}, somehow the twelve students involved in controlling the storm that year ended up dead. If that wasn't fishy enough, subsequent investigation revealed that the vote was apparently unanimous. The leading theory is not a comforting one. Someone, or some\emph{thing}, sabotaged the voting and changed the votes after they had been cast, but before the actual Ritual to Control the Storm began.

In other nations, the people can hunker down and wait out the storm. In \pShip{}, that isn't really possible. Coastal towns can be torn from the cliffside by the massive waves. Ships can be smashed to pieces on the shore. The Storms stir up mud from the bottom that drive fish from their normal patterns and upset the sea serpents. With fishing disrupted, their fleet in tatters, and the sea serpents agitated, the \pShip{} struggled tremendously in the wake of the Betrayal.

\section*{Recent History}
Since the Betrayal, \pEarth{} has been at war. In a world where murder is punished by the Gods with complete and immediate amnesia, only something as dire as the plight of the \pShippies{} could drive a nation to war. Even \cEbbFull{} and \cFlowFull{} have sanctified the war on the \pFarm{} and the \pTech{}, occasionally bending the rules on murder in service of the war for their followers. Balance must be restored.

The \pShip{} are fighting for their lives. Everyone lost family or friends in the Storm 6 years ago. Nearly every ship clamored for retribution of some kind in the immediate aftermath. Because of this, \cLoud{\full} enjoyed a meteoric rise to prominence for unifying these cries for revenge. \cLoud{\They} championed the proposal for war before the Council of Storm-Watchers (see Form of Government below), and within months, the \pShip{} began raiding the coasts they had previously protected for their neighbors. The war officially began 5 years and 7 months ago.

At first, the \pShippies{} turned to the pirates for support — but when the pirates proved unwilling to cooperate with the Council, the \pShippies{} turned instead to their natural enemies, the sea serpents. The massive creatures spend most of their lives in the water (at least until they pass breeding age and abandon the water for the high caves of the \pSpine{}), but can be driven into the shallows, or even onto the shore, for short periods of time. Once there, the creatures ravage anyone and anything they can reach. Most of the \pShip{} offensive involves luring sea serpents into the shallows, harassing them until they are good and angry, and then driving them onto land belonging to the \pFarm{} and the \pTech{}. This way, no human hand is raised against another, and so no one's memories are lost.

What began as a campaign of causing destruction has changed somewhat in the intervening years. Especially once the second Storm hit three years ago, it became apparent that the \pShip{} needed more territory on the continent proper. \pShippies{} have started fielding landing parties that would go with the serpents and drive them further inland than could be done from the boats alone. After the carnage, the landing party could then go back and stake claim on the land left abandoned in its wake. Bit by bit, the \pShip{} have been encroaching on land of both of the other nations, although more successfully in the \pFarm{}.

For all their territorial gains, the war has taken a terrible toll on the \pShippies{}. The other two nations quickly pooled their resources and developed curse-launching trebuchets to combat raids from the sea; these terrible weapons have sunk many a ship. Cut off from trade with the other two nations, the \pShippies{} economy has contracted as well. The suffering is particularly great in the poorer 8th through 12th Fleets. While they still long for Balance against the wrongs committed by the other nations, an increasing number of \pShippies{} grow weary of war. But in order to actually achieve peace, a peace treaty would need to include both reparations (an unusual concept for the \pShippies{}, but one occasionally necessary when it is not possible to deliver consequences in kind back to the instigator) and a guarantee that no Storms will be sent to the \pShippies{} for at least the next few cycles.

\section*{Internal Politics and International Relations}
Before the Betrayal, most of \pShip{} were on reasonable terms with both the \pFarm{} and the \pTech{}. The \pShippies{} were available for hire to transport goods, or to chase off sea serpents. Different ships would charge different rates (some quite exorbitant), and if a section of the coast couldn't pay, then the \pShip{} couldn't be bothered to protect it. While eventually another ship would come along that was willing to protect the coast for a lower sum, the coast would be in danger until that occurred.

\subsection*{Pirates}
``Pirate Ships'’ are a particular subset of \pShip{} ships. Pirates never bothered with the song and dance of offering to protect a coast area. They would just show up and take things. Unlike most people in \pEarth{}, the pirate crews didn't seem particularly deterred by the risk of losing their memories, and every such crew soon found a contingent of people who would kill, take the consequences, and start again. Since no one was willing to tangle with such reckless people, the pirates were always able to take what they wanted.

For the first two years of the war, the \pShippies{} turned a blind eye, allowing pirate ships to congregate en masse in the poorest of the poor fleets, the 8th and 9th. But soon enough, pirate ships turned on other \pShippie{} vessels, and began to attack them as well, and have become a genuine problem for everyone.

When the war effort first got underway, it was expected that the pirates would fall in line with the national strategy and stop preying on \pShippie{} vessels, at least for a while. After all, with very few safe harbors in the 8th and 9th fleets, Storms are more devastating to pirates on average than the rest of the \pShip{}. But that has not been the case — if anything, the pirates have stepped up their attacks on their distracted fellow citizens, striking for their own gain at everyone else's expense.

\subsection*{Form of Government}
For most of their history, the \pShippies{} have had a consensus based representational democracy, and that is still ostensibly the case. Dolphins and gulls are used to spread messages calling groups to discuss policy, and disseminating decisions. The government consists of four “official” levels and, more recently, a less official fifth:
\begin{enumerate}    
    \item The \textbf{Ships:} The ships are the smallest unit within the \pShip{} government. Each ``Ship'' is comprised of the crew of each ship, or the members of each household in a town. Each ship gets to make its own rules, as long as they don't violate the (generally not very restrictive) edicts from the larger governing bodies. Everyone 14 years and older is allowed to participate in policy discussions. On some ships, these discussions continue until consensus is achieved, while other ships, especially those that regularly hunt sea serpents, observe a strict, militaristic hierarchy. With the onset of the war, the hierarchical structure has become significantly more common. When necessary, a representative (typically the first mate) will be sent on to the Raft.
    \item The \textbf{Rafts:} A ``Raft'' is composed of all of the Ships that call the same port home. They convene to discuss matters of local importance whenever they arise, or when called upon for a vote by the larger governing bodies.
    \item The \textbf{Fleets:} A ``Fleet'' is composed of all of the Rafts from a geographic area. Each geographic area is identified by a 1/12th arc of a clock, centered in the middle of the continent. The sections are numbered 1-12, starting N to NE (the 12-1 slice). Fleets are based on their port location, not the current location of a Ship.
    \item The \textbf{Council of Storm-Watchers:} The high council of the \pShip{}. This group of 12 elected officials, referred to as “Storm-Watchers,” is composed of one representative from each fleet. They are ostensibly the highest governing body — no nationwide policy is enacted without the approval of the Council.
    \item The \textbf{Warlord:} Ever since the Betrayal, \cLoud{\full}, known as ``The Warlord,’’ has risen in prominence in \pShippie{} politics, challenging the power of even the Council of Storm-Watchers. While technically answerable to that body, in practice the Warlord acts with considerable independence (some would say impunity) in the prosecution of the war, enjoying an increasingly fanatical following in every fleet. Thus an uneasy tension exists between the Council and the Warlord, and if the latter were to call for open defiance of the former, it is unclear what would happen. As of yet, this has not been put to the test.
\end{enumerate}

With the changes in the political climate brought on by the war, the \pShippies{} find themselves balanced on the knife's edge, wavering between consensus based democracy and fascism. Only time will tell which system prevails.

\section*{Religion, Morality, and Magic}
The \pShippies{} primarily serve the Twin Goddesses, \cEbb{} and \cFlow{}. \cEbbFull{\MYname} is the primary keeper of ships, and the caretaker of things lost or taken away. \cFlowFull{\MYname} is the primary keeper of the rain, and the caretaker of things created or given. They preside jointly over the sea and their people. Most Clerics focus their paths on one Goddess or the other, but some few choose to walk the difficult, middle path of Balance. Only those who bear the mark of \cEbb{} and \cFlow{} can enter training as an Initiate to become a Cleric. \cEbb{}’s religious symbols include waves, islands, ships, a golden sea serpent (Avatar, primary symbol), and limestone. \cFlow{}’s religious symbols include waves, falling raindrops, the moon, a silver sea serpent (Avatar, primary symbol), and limestone. The path of Balance is most often symbolized by golden and silver sea serpents twining together in an intricate, symmetrical pattern.

There are whispers in \pShip{} about the growing tide of a new cult, one that preaches ``true'' Balance and equality, but achieves it through questionable means. The wildest reports claim that entire ships have converted from the loving embrace of \cEbb{} and \cFlow{} to this unknown new Deity — but proof of these claims is in short supply.

\subsection*{Morals}
\cEbb{} and \cFlow{} demand a morality primarily of Balance. Give and take is fundamental to the \pShip{} worldview. Whatever you get, you are expected to give back in equal or greater measure. In theory, this creates a fair and equitable society — but in practice, it absolutely does not. Those who have, give to each other, containing their wealth and indeed multiplying it by passing it amongst each other. The rest struggle to make ends meet, as they can only pass scraps.

There is also an ebb and flow among the \pShippies{} of good and ill will. If someone does you a bad turn, you are expected to harm them in return. This creates localized escalations, and they can spread from these points of conflict, like ripples on a pond, to encompass vast swaths of \pShip{} territory.

\subsection*{The Creation of the \pShippies{}}
The most pervasive creation story of the \pShippies{} is as follows:\\
\emph{When the world was complete, \cEbb{} and \cFlow{} wanted beings in their own image on \pEarth{}. They came down to the earth on the Western shore. They walked out across the waves, each in a different direction — \cEbb{} to the north, and \cFlow{} to the south. In each place they stepped, up rose an island — land that was to be sacred to their people. They walked for three days and three nights, curving around the continent, and finally met on the eastern coast of \pEarth{}. There they sat to rest on an Island that grew up from the sea. Only one sliver of the ocean, \pWod{}, did not feel the trod of their feet.}

\emph{\cFlow{} began to build with the sand, but the dry sand would simply collapse — refusing to hold its shape, even for a \cFlow{\God}. \cEbb{} observed this, and drew the waves up the beach until the sand \cFlow{} was working with was soaked. The two of them knelt and built statues of a small group of humans together. \cFlow{} raised the Sun, and it dried the sand, creating firm bodies that would not be washed away by the water. \cEbb{} raised the Moon, and its soft light brought peace and wisdom into the statues. Both \cEbb{} and \cFlow{} called the wind, and it blew wanderlust into the heart of the \pShippies{}. Though they were tired, they looked upon their creation with joy.}

\emph{\cEbb{} and \cFlow{} took each other's hands and prepared to return to the place of the Gods, but the humans they had brought to life cried out, and begged them not to leave. The \pShippies{}'s plea moved the two, and their soft hearts ached. It seemed that nothing could be done, as they could not stay on \pEarth{}, and the humans could not live in the Realm of the Gods. Very little stops a \cEbb{\God} for long, though, and \cEbb{} and \cFlow{} soon had a solution. They each took up a fistful of sand, and threw it to the night sky. There the grains lodged as the first stars. If \pShippies{} live their lives right by the two, and by each other, then they will have a place among the stars, alongside the Moon and Sun, and \cEbb{} and \cFlow{} — once their mortal lives are spent, they will spend eternity either as a star in the Divine Realm, or a grain of sand, forever confined to the mortal world.}

\subsection*{Magic}
The magic of the \pShip{} is a subtle one. It lives in the hands of the shipwright, the fisher, the harpoon-maker, and the cook. Their magic is tied to crafting, the sea, and the purpose of the object made. So subtle is their magic that there are some in the \pFarm{} and the \pTech{} that look down on the \pShippies{}, and claim their magic is of no consequence. But by the will of \cEbb{\full}, and the skill of a master shipwright (among the most revered magic users of the \pShippies{}), a ship built by the \pShippies{} will always be the highest quality. And so is anything else a \pShippie{} craftsperson puts their skill to. Tools they make for land based activities such as farming such as plows, or in support of scientific advancement such as microscopes are desperately sought after and passed down as family heirlooms. For those with eyes open, the magic of the \pShippies{} makes daily life possible in \pEarth{}. Without it, the \pShippies{} would be no more.

\section*{Economics and Industry}
In pre-war times, the domestic economy of the \pShip{} was based on trade and artisanal crafts. Economic interactions with the other two countries were a mix of fees (paid freely or extorted) to guard against sea serpents, delivery fees to transport goods and materials, and fish and guano for fertilizer traded to the land dwellers. The \pShippies{} most expensive export was always their magically crafted goods. They did not export a huge quantity of them, since there were never many to go around — good craft took \emph{time} to complete, but neither were the \pShippies{} stingy with their magical gifts. Every family that could afford it had something made by \pShippie{} craftsfolk, be it a clock that would never wind down, or a whetstone that put an edge on a farming tool in half the time, or a work surface that healed itself from the damage of use.

Ever since the start of the war, with shipments of food from the \pFarm{} and raw materials and tech from the \pTech{} cut off, the \pShippie{} economy has contracted and prices have soared. This is only partially offset by goods seized during raids or obtained through the black market.

\section*{Immigration and Emigration}
Before the betrayal, attempts at immigration to the \pShip{} were relatively common compared to other nations. The appeal of a seemingly carefree life was strong. But joining the \pShippies{} is dangerous. For those not born on a ship, there is a 3 year probationary period, which is harsh and particularly dangerous while they become accustomed to life at sea. Emigration was fairly common as well. The austere life of the \pShip{} as well as the whims of the sea drive many to seek safety on dry land.

\section*{Culture}
The \pShip{} culture is one of wanderlust and restlessness. Adventurous spirits pervade, and detailed planning is eschewed in favor of following where the wind takes you. Camaraderie among a crew is paramount, with fellowship within a chosen family holding much stronger sway than any blood tie.

Many \pShippies{} from the richer fleets live in at least 5 different communities before settling down more permanently in their 30s. Life for those in the poorer fleets, particularly 8th - 12th Fleets, is a lot more restricted. Moving is costly, both to the individual and to the community receiving the new mouths. Without enough resources to go around, ships will often reject requests to join, and a ship that strikes unusually good fortune will quickly be overwhelmed by requests to join. There is also rampant discrimination, both subtle and overt, that prevents poorer people from moving to ships in richer fleets and thereby escaping their poverty.

\subsection*{Art and Entertainment}
\pShippies{} don't go in much for pure aesthetics. If it isn't functional, it's dead weight, and dead weight could mean the difference between running afoul of a reef and sailing free and clear. The most skilled crafters can make something that is both functional and beautiful; this form of minimalism is highly prized. \pShippies{} also tend towards song, dance, and storytelling in their artistry.

\subsection*{Marriage and Family Structure}
While marriage is an option among the \pShippies{}, most people don't bother with it, instead negotiating among consenting adults as to how they want their relationship to be structured. Both monogamy and polyamory are common, with little expectation of relationships being permanent. Couples and groups often separate on amicable terms. This is very much at odds with both the \pTech{} and \pFarm{}, who see formalizing permanent ties as an important part of family and social structure.

Children are raised communally, and children as young as 7 are regularly allowed to change ships if they so desire, as long as the other ship will accept them. There is no particular pressure to have children or maintain ties with parents, siblings, partners, or children.

Names are composed of a single personal name, which people change not infrequently, the name of their ship, and the fleet to which it belongs, as a suffix. For example: \cLoud{\full}.

\subsection*{Educational System}
Children are ``homeschooled’’ on their ship or in their neighborhood from 4 through 13 years old. Students are then expected to study at least 4 years at one of the 12 Academies (one per fleet). Most of the Academies have entrance exams, but the Academies in 6th through 9th fleet accept anyone. The Academies offer advanced study courses beyond a 10th year of education, but attendance dwindles significantly with each additional year.

The \pShip{} \pSchool{} students are elected from a pool of candidates by the whole nation, and the students campaign with the support of their academy. Since the academy pays the tuition to the \pSc{}, it is theoretically possible for anyone to attend. In reality, the poorer fleets can't mount the necessary campaign to get a student elected, and couldn't afford the tuition anyway. The rare student from such fleets who manages to get elected to the school does so under the sponsorship of one or a few wealthy patrons that have taken a personal interest for one reason or another.

\subsection*{Taboos}
A \pShippie{} should not go a year and a day without bathing in sea-water. They should never waste fresh drinking water. They should never fail to give honor to \cEbb{} and \cFlow{} for anything they create, and it should always bear the seal of the maker's Patron \cEbb{\God}. And finally, they should never let an action done to them go unmatched, for good or ill.

\subsection*{Clothing}
\pShip{} clothing tends toward greens, blues, purples, blacks, and silver. They often dress simply, but in layers. Fabric can be plain, or have patterns that evoke light on the water and other natural phenomenons. The most common kinds of jewelry pay homage to the ocean, or to \cEbb{} and \cFlow{}. Above all, the \pShippies{} eschew flashiness and extravagance. \pShippies{} who are based primarily on ships tend to wear tighter fitting clothes that offer good movement and minimal chance of entanglement \emph{(OOC: with a stereotypical pirate theme/influence.)} Wave Riders who are based primarily in coastal towns tend to wear loose, flowy outfits that look cool billowing in the wind. The \pShippies{} are the most likely to wear leather, furs when it is cold, and silk when it is warm.

\section*{Opinions About People from the Other Nations}
It is hard to have a good opinion of anyone who condones the kind of betrayal that the \pShippies{} went through. The common opinion on the other nations are as follows:

The \pTech{} thinks technology is the solution to everything. They never think to solve something with common sense, or hard work — only a flashy, new technological solution will do. They hoard \emph{stuff} like nobody's business — who can breathe with all that clutter they call ``art'” everywhere? With their heads in the clouds, always dreaming about their next ``innovation,'’ it's a miracle they don't trip over their own feet. What's worse, others pay the price for their hubris. Their pipedream of ending the Storms forever is what unleashed the Betrayal upon us, a crime which they have yet to begin attoning for. Only when Balance is restored can we begin to contemplate forgiving them.

The \pFarm{} are every bit as bad in their own way. The life of a \pFarm{} citizen is one of stability, and assurance. It is safe and reliable. What do they have to complain about compared to the daily dangers a \pShippie{} faces? Sure it makes them a little boring, but one ought to be grateful for a boring but safe life. They are also so uptight about tradition. Sure it takes the \pShip{} a long time to reach consensus and change something, but that's nothing to how long it takes to convince the Nobility of the \pFarm{} that something needs changing. Speaking of the Nobility, they're also such a regimented society. Whatever you are born to, you are stuck with. There's no chance of hard work being rewarded, or incompetence being punished.

\end{document}

