\documentclass[blue]{GL2020}
\parindent=0pt
\begin{document}
\name{\bTeacherBlue{}}

Only the best of the best are invited to teach at the \pSchool{}. Being a Teacher at the \pSc{} is a tremendous honor and a tremendous responsibility. It is also a careful balancing act. You are ostensibly neutral, having shed any national ties when you accepted the teaching position here. A few Teachers make a serious effort to actually achieve impartiality — but most retain close contacts via regular correspondence, and keep careful tabs on domestic and international politics.

A large number of Teachers resigned after the tragic incident six years ago in which the twelve Students who remained for the Time of Deciding all died mysteriously after the Ritual to Control the Storm. Therefore, a lot of Teachers present are relatively new hires, arriving at the school between 4 and 6 years ago. Even for Teachers who have been teaching longer, it is not common for the same Teachers to be asked to stay for the Time of Deciding many times in a row. The burden is heavy, and most prefer to go home to their families during this uncertain time.

Those of the staff who do stay at the \pSc{} during the Time of Deciding have a multifaceted challenge ahead. You must prepare the Ritual to Control the Storm, lest the magic get out of hand and do even more damage than a controlled Storm would. This includes the particular responsibility of deciding which Relics to use for the ritual, which will directly influence where the Storm goes. You must guide the Students, and provide a buffer between them and the Advisors, who often forget that the Students are human beings, not just pawns in a game of high stakes chess. You must preserve the veneer of impartiality lest you be ostracized by your colleagues, or if your actions are sufficiently egregious, be fired by the Principal. Finally, you must weigh its value against the risk that people dear to you could die if the Storm is sent to your home nation.

Though these tasks are daunting, you are not in this alone. Every Teacher at the Time of Deciding is trying to find the same balance. You can and should be a resource for each other as you navigate this time. All of you have been colleagues for years, some for decades. Many of you have become close friends. Who better to hold your moments of doubt than another Teacher, or the immortal Principal? But it's probably best if the Students don't hear your uncertainties. They look up to you for reassurance and stability.

You will almost certainly need to field concerns around the ethics and morality of where to send the Storm. You may find yourself questioning, or be called on to defend, the traditions of the \pSc{}, the nations, and the religion. How will you support your favorite Students? How will you protect your homeland? And how will you make \pEarth{} a better place?

\begin{itemz}[Notes]
	\item We will establish at least 1 tradition of the school during the in-person pre-game workshops.
\end{itemz}

\begin{itemz}[Goals]
    \item Decide which Student you will give your Voting Stone to. Each of you has been entrusted with ONE. You may use whatever criteria you like to decide, but most Teachers will want to grant their stone to a Student that holds similar views to themselves. (See the World Guide bluesheet for details on Voting Stones.)
    \item Prepare the Ritual to Control the Storm, and then execute your part in the ritual itself. The Librarian, \cLibrarian{\full}, is nominally in charge of ensuring this vital task is completed, but the work requires many hands, and so any and all who are interested are encouraged to assist. Deciding how to attune the Relics to be included is crucial to this preparation, as the attunement status of each Relic used will influence where the Storm ultimately goes.
    \item Attend the Ceremony of Excellence in support of the Students and this sacred tradition of the school.
    \item Three of your colleagues, \cMusic{\full}, \cBeetle{\full}, and \cChupSecond{\full} are competing to be selected as the next Principal of the \pSc{}, a decision which will affect you greatly, as well as future Teachers for generations to come. Decide which of them you support, and make your preferences known to the current Principal, \cPrincipal{\full}.
    \item Preserve the school, and its traditions.
\end{itemz}

\end{document}


