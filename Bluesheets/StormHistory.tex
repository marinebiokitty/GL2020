\documentclass[blue]{GL2020}

\usepackage{array}
\usepackage{xcolor}
\usepackage{hyperref}
\usepackage{multicol}
\usepackage{ltablex}
\usepackage{tabularx}
\renewcommand{\tabularxcolumn}[1]{m{#1}}
{\renewcommand{\arraystretch}{1.5}
\setlength{\columnsep}{1cm}

\parindent=0pt
\begin{document}
\name{\bStormHistory{}}

In preparation for this weekend, you have brushed up on what you know about the history of the storm, and various attempts to control it.

\subsection*{Where has the Storm been sent in the past?}
For the last 6 years, the ``End of Suffering'' treaty has been in place. Previously, the ``Time of Peace'' treaty dictated a steady rotation in where the storm should be sent. This is where the storm has been sent for the past 18 years. The pattern seen in 18, 15, and 12 years ago continues back in time without disruption to 34 years ago. Prior to that, each ``Time of Deciding'' was a complicated and intense political conflict.

\begin{tabular}{ | c | c | }
\hline
 \textbf{ Num years ago} & \textbf{Storm Sent to} \\ 
\hline
 This year & ?? \\
\hline  
 3 & \pShip{} \\
\hline
 6 & \pShip{} \\
\hline
 9 & \pFarm{} \\
\hline
 12 & \pShip{} \\
\hline
 15 & \pTech{} \\
\hline
 18 & \pFarm{} \\
\hline
\end{tabular}

\subsection*{How much damage does the storm do?}
Over generations, the damage that the storm does has been quantified into approximate ``units'' that represent 1 year's worth of work to repair the physical damage to infrastructure. Obviously one cannot repair the associated loss of life, but it is fair to assume that 2 units of damage also involves 2x as many deaths as 1 unit.

The ``Time of Peace'' treaty provided certainty in when the storm would strike a nation. This was an important breakthrough because it allowed nations to prepare only when necessary. A nation \textbf{unprepared} for the storm sustains 6 units of damage. A nation that has had time \textbf(1 year or more) to prepare sustains only 3 units of damage. By rotating which nation would receive the storm, the people of \pEarth{} were able to go from having to spend every 3rd year preparing just in case, to only needing to prepare 1 year out of 9, and still have 3 years with which to repair the sustained damage, and 5 years of growth.

\subsection*{What happens if we send the storm to something other than a Nation?}
The following consequences, measured in ``units'' assume that the nation has dedicated the prior year to preparations. If the nation is \textbf{unprepared}, double the units of damage.

\begin{tabularx}{\textwidth}{ |>{\centering\arraybackslash}X |>{\centering\arraybackslash}X | }
\hline
 \textbf{Target} & \textbf{Consequence} \\ 
\hline
Sent to 1 nation & 3 units of damage to that nation if prepared, 6 units if unprepared. \\
\hline
Sent ``everywhere'' & 2 units of damage to each nation. \\
\hline  
 Sent ``nowhere'' & 3 units of damage to each nation. \\
\hline
 Uncontrolled storm (a.k.a. ritual not performed) & 4 units of damage to every country. \\
\hline
 Sent to \textbf{part} of a nation (e.g low population density or evacuated) & 4 units of damage across nation, not just in the part targeted \\
\hline
\end{tabularx}

\subsection*{What is the situation of the Wave Riders at this point?}
Obviously, with the storm hitting the \pShip{} out of turn 6 years ago caused major damage. The introduction of war is also a massive drain on each nation's resources, but the \pShip{} are investing the most here. It is reasonable to question how much longer the \pShip{} could sustain the current situation.

Under the ``Time of Peace'' treaty, it is estimated that each nation had ~ 16 units worth of damages that it could sustain before it would be meaningfully wiped out. The \pShip{} sustained 6 units of damage 6 years ago, and another 3 units 3 years ago. Waging war costs ~1/3 unit of resources per year per enemy nation, meaning that the \pShip{} can wage war for 3 years and prepare for the storm, but not repair any damage previously sustained.

If the war continues, and the storm continues to be sent to the \pShip{}, the \pShip{} are estimated to survive this storm cycle, plus 2 more, so a total of 6 years past today. If the war were to end, but the storm continued to be sent to the \pShip{}, they could repair 2 units of damage per cycle, greatly extending their runway, but still always losing ground and eventually disappearing.

For the \pFarm{} and the \pTech{}, who only need to devote ~1/3 of a unit of resources per year to fighting the \pShip{} (since they aren't fighting each other too), these nations can still have 2 years worth of growth across each 3 year storm cycle, which is better than the ratio during the ``Time of Peace'' treaty.

\subsection*{Development of the Mechanisms to Control the Storm}
The voting stones for the students is the oldest mechanism to Control the Storm. Their magical development is lost to the annals of time, along with the origin of the Ritual itself. Perhaps they were created at the same time? Speculation has it that part of what happened six years ago involved the voting stones themselves being tampered with somehow.

The restriction protocol for the Pedestals that hold the Relics during the Ritual is somewhat newer. Some 500 years ago or so is the first record of the teachers attempting to control which relics could be used in the ritual. What followed was a magical arms race of sorts between the restriction protocol and those seeking to circumvent it. Over time, the number of successful attempts decreased. In the last 100 years, the restriction protocol has been overcome only 3 times, although a number of attempts were made. Since the ``Time of Peace'' treaty was signed 40 years ago, no known attempt has been made to overcome the restricting protocol.

The effect of the treaty is by far the most recent mechanism to be discovered. It was not until the ``Time of Peace'' treaty was ratified, being the first treaty of its kind in recorded history, that this effect was recorded. Since it is relatively new in comparison, very little is known about if or how this mechanism could be subverted.

\end{document}

