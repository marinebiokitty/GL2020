\documentclass[green]{GL2020}
\usepackage{enumitem}
\setlist{nosep}
\parindent=0pt
\begin{document}
\name{\gSeaSerpent{}}

The care and feeding of baby sea serpents is not something to be found in a book. People don’t raise baby sea serpents. They are wild animals for goodness sake! And yet, here we are, with a freshly hatched baby that clearly needs looking after. Oh and it has the biggest, cutest, baby eyes. \emph{(OOC Note: Unless a task actively requires it, you should probably keep the baby here in its hiding place for as long as possible; there’s a very good chance someone will try to kill it if they find it. The GMs can hang on to the phys rep stuffed animal for you whenever the baby is in its hiding place.)}

The following are a list of tasks that need to be completed to ensure the continued good health of the baby serpent. These tasks can be completed in any order (unless a specific time is specified) When a task is completed, cross it off of this list and increase the CR on the sea serpent item card by +1.

\begin{enumerate}
  \item \textbf{Give the baby a bath:} Take the sea serpent to a source of water and spend \textbf{2 minutes} roleplaying giving an unruly baby a bath. For \textbf{20 minutes} afterwards, you and anyone who helps you are soaking wet. Tell players as much when you encounter them.
  \item \textbf{Feed the baby each day:} Find a \iFish{\MYname} somewhere and bring it to the baby. Spend 2 minutes roleplaying feeding the baby. (feeding each day is a separate +1 to CR for the baby)
  \begin{enumerate}
    \item Feed on Friday
    \item Feed on Saturday
    \item Feed on Sunday
  \end{enumerate}
  \item \textbf{Build a toy to entertain the baby and play with it:} Bring the following materials to the Makers space and spend \textbf{2 minutes} roleplaying building a toy boat. Then retrieve ``\iToyBoat{}’’ from ``\sSignH{}’’.
  \begin{enumerate}
    \item 1 ``\iWoodenBlock{}'' (to carve the hull of the boat)
    \item 1 ``\iEagleFeather{}'' (to make a sail)
    \item 1 ``\iThread{}'' (to make tiny rigging)
  \end{enumerate}
  Once built, you can store the toy in the ``\sStudentBookCaseTwo{}’’ with the baby. Play with the baby and the toy for at least \textbf{2 minutes.}
  \item \textbf{Take the baby on a walk:} Fresh air is good for babies right? Carry the baby around for at least \textbf{10 minutes.} Visit at least \textbf{3 different locations} (e.g.: The graveyard, the training grounds, and the library.)
  \item \textbf{Acquire some medicine:} The baby seems to have a little cough. Acquire a ``\iStrength{}’’ and expend it on the baby.
  \item \textbf{Give the baby a name:} The baby isn’t a thing. It is a living creature, and it deserves a name. Someone should name it! Write the name here: \underline{\hspace{2cm}}
\end{enumerate}

Once \textbf{all} of the tasks have been completed (baby’s CR =11), the baby has grown too big to fit in its little hiding place. Who knew they grew so fast? Write on the item card that it is now \textbf{1 hand bulky, and unstashable.} Someone now has to carry it around (or leave it somewhere out in the open) and hope that it is big and scary enough now that no one will try to hurt it.

\end{document}
