\documentclass[green]{GL2020}
\parindent=0pt
\begin{document}
\name{\gPreparingTheRitual{}}

The ritual to control the storm is \textbf{very complicated.} To help reduce the risk of things going wrong, there are a number of preparations that need to be made. The responsibility for these preparations generally falls to the teachers at the \pSchool{} with \cLibrarian{\full} expected to take the lead. Students who express interest should not be turned away.

Some tasks must be completed by the end of Friday or the end of Saturday. Others just need to be completed before lunch on Sunday, but tasks may always be completed early. If one or more tasks are not completed by their deadline, or only partially completed, just let a GM know and move on.

\textbf{Tasks to be accomplished before close of game on Friday:}
The ``Ritual Space’’ must be physically cleaned. This activity is simulated via a cooperative, abstracted mechanic (no one will be actually physically cleaning.) You will find a box of ball pit balls (physical items) in the ``Ritual Space’’ with further instructions. GMs will evaluate how many balls remain in the starting box before the game starts up on Saturday morning to assess how successful this task was.

\textbf{Tasks to be accomplished before close of game on Saturday:}
\begin{enumerate}
  \item The pedestals for the relics must be assembled. Instructions for how to assemble the pedestals (\iPedestal{}) are available in the ``Maker Space’’, on ``\sSignF’’. Once constructed, place the pedestals in the ``Ritual Space.’’
  \item An exceptionally powerful Tass vessel, ``The Storm Seed’’ must be charged to full potential. You will find The Storm Seed as a sign in the ``Ritual Space’’ with further instructions on how to charge it up. The GMs will count the magical energy in the container before the game starts up on Sunday morning to assess how successful this task was.
  \item The Teachers Oath Recital:
  \begin{enumerate}
    \item In the ``Ritual Space’’ there is a plaque that contains The Teacher’s Oath.
    \item A group of at least 6 teachers must be collected at some point to recite the oath. Further instructions on how to do so are available on the sign.
    \item If a GM was not present for the recitation, make sure someone informs a GM when this task is completed.
  \end{enumerate}
\end{enumerate}

\textbf{Tasks to be accomplished before lunch time on Sunday:}
\begin{enumerate}
  \item At least 2 clerics must cleanse the space (they have a specific mechanic for this). If a GM was not present for the cleansing, make sure someone informs a GM that this task was completed.
  \item The ritual circle itself must be renewed with the ``Sacred Chalk’’ (phys rep: painters tape):
  \begin{enumerate}
    \item Six different colors of painters tape will be provided in the ``Ritual Space’’ area to use in laying out a ritual circle of the player’s own design on the ground. The design must include all 6 colors.
    \item These will be in-game items that could theoretically go missing, so you may want to keep an eye on them.
    \item Once the tape has been put down, it can be removed only for the purposes of cosmetic adjustment. It is not possible to ``undo’’ the ritual circle.
    \item When you are done, snap a picture of the completed ritual circle and show a GM.
  \end{enumerate}
  \item Ensure all of the students vote for where to send the storm.
  \item Ensure that 3 Relics are selected to help direct the ritual, and put in place.
  \begin{enumerate}
    \item In the past, organizers have striven for 1 relic attuned to each nation, to provide equal potential for protection from the storm, however there is no hard and fast rule for this, or even a strict requirement that the relics must be attuned at all. Still, the relics only serve a protective function if they are attuned. Not having relics at all runs the risk of the ritual going awry.
    \item \textbf{At noon sharp the pedestals will lock down.} The relics currently on them will be locked in place and can no longer be removed or switched. If a pedestal is empty, it will no longer be possible to place a relic upon it. - this is a kludge so that the GMs can assess the state of the relics before the ritual commences.
    \item 
    \item 
  \end{enumerate}
\end{enumerate}
\end{document}
