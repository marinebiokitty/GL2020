\documentclass[green]{GL2020}
\usepackage{enumitem}
\setlist{nosep}
\parindent=0pt
\begin{document}
\name{\gPreparingTheRitual{}}

The ritual to control the storm is \textbf{very complicated.} To help reduce the risk of things going wrong, there are a number of preparations that need to be made. The responsibility for these preparations falls to the teachers at the \pSchool{} with \cLibrarian{\full} expected to take the lead. Students who express interest should not be turned away.

Some tasks must be completed by the end of Friday or the end of Saturday. Others just need to be completed before lunch on Sunday, but tasks may always be completed early. If one or more tasks are not completed by their deadline, or only partially completed, just let \cPrincipal{\full} or a GM (if you can’t find \cPrincipal{}) know and move on.

There may be characters who have modifications or additions to these preparations that they wish to have included as well. Any teacher can approve these additions, but it is a always risk that an addition or modification amounts to sabotage.

\textbf{Tasks to be accomplished before close of game on Friday:}
The ``Ritual Space’’ must be physically cleaned. This activity is simulated via a cooperative, abstracted mechanic (no one will be actually physically cleaning.) You will find an envelope full of pieces of colored construction paper attached to the ``\sCleaningInstructions{}’’ sign in the ``Ritual Space’’ with further instructions. \textbf{Warning:} This task must be completed all in one go, you probably want to gather help to ensure that you can complete the task efficiently. \cPrincipal{} will evaluate how many pieces of construction paper are left outside of their final locations once you declare the task complete or give it up.

\textbf{Tasks to be accomplished before close of game on Saturday:}
\begin{enumerate}
  \item The pedestals for the relics must be assembled. Instructions for how to assemble the pedestals (\iPedestalForRelic{}) are available in the ``Maker's Space’’, on ``\sSignF{}’’. Once constructed, place the pedestals in the ``Ritual Space.’’
  \item An exceptionally powerful Tass vessel, ``\sStormSeed{}’’ must be charged to full potential. You will find The Storm Seed as a sign in the ``Ritual Space’’ with further instructions on how to charge it up. \cPrincipal{} will count the magical energy in the container before the game starts up on Sunday morning to assess how successful this task was.
  \item The Teachers Oath must be recited:
  \begin{enumerate}
    \item In the ``Ritual Space’’ there is a plaque that contains The Teacher’s Oath.
    \item A group of at least 6 teachers must be collected at some point to recite the oath. Further instructions on how to do so are available on the sign.
    \item If \cPrincipal{} was not present for the recitation, make sure someone informs \cPrincipal{\them} when this task is completed.
  \end{enumerate}
\end{enumerate}

\textbf{Tasks to be accomplished before lunch time on Sunday:}
\begin{enumerate}
  \item At least 2 clerics must cleanse the space (they have a specific mechanic for this). If \cPrincipal{} was not present for the cleansing, make sure someone informs \cPrincipal{\them} that this task was completed.
  \item The ritual circle itself must be renewed with the ``\iChalk{}’' which the \cPrincipal{} should have on hand (phys rep: multiple colors of painters tape):
  \begin{enumerate}
    \item Six different colors of painters tape will be provided in the ``Ritual Space’’ area to use in laying out a ritual circle of the teacher characters’ own design on the ground. 
    \item The design must include all 6 colors.
    \item The design must include specific places for the Relic Pedestals.
    \item The chalk pieces are in-game items that could theoretically go missing, so you may want to keep an eye on them.
    \item Once the tape has been put down, it can be removed only for the purposes of cosmetic adjustment. It is not possible to ``undo’’ the ritual circle. There may be mechanics that provide an exception.
    \item When you are done, show \cPrincipal{}.
  \end{enumerate}
  \item Ensure all of the students vote for where to send the storm. Votes can be cast any time between 9:30 am and Noon on Sunday.
    \item The Pedestals may optionally be ``restricted’’ to accept only relics attuned to a designated location. Each pedestal can be ``restricted’’ to 1 attunement location, or can be left open to accept any attunement. It is not possible to ``restrict’’ a pedestal to multiple locations simultaneously. This restriction protocol is how the Teachers are expected to express their opinion on where the storm should go.
\begin{enumerate}
	\item Pedestals are restricted individually by a simple majority (7/12) of teacher opinions. At any point that 7 or more teachers are present at the ``Ritual Space’’, and an unrestricted pedestal is available, a restriction can be declared. Write it on the item card for the pedestal. (e.g.: \pFarm{})
	\item Once a pedestal is restricted, a 2/3rds majority (9/12) of teachers must agree to change the restriction. (Cross out the old restriction and write in the new one.)
	\item  A unanimous agreement (12/12) teachers is required to remove a restriction entirely from a pedestal. (Cross out the listed restriction  but do not write any new one down.)
	\item  Each pedestal can be restricted to a different single location, all to the same single location, or some mix.
	\item Only relics attuned to match the restricted location will stay on the pedestal. Find \cPrincipal{} or a GM who will start a 15 minute timer. At the end of that time, they will check if the attunement of the relic on the pedestal matches the restrictions. If it does not, they will remove the relic from the pedestal, placing it down on the floor or a nearby table.
	\item Each attuned relic will influence where the storm ultimately goes. See your ``\gVotingStones{}’’ Greensheet for more details.
	\item It is common to re-attune relics to match pedestal restrictions, even if that location does not match the nation the relic is associated with. (e.g. The \iPitcher{} could be attuned to the \pShip{}.)
\end{enumerate}
\end{enumerate}

In ages past, people have found ways to overcome the pedestal restrictions. The restriction protocols were improved, closing those loopholes, but new ones were still found at some rate. In the last 100 years, the restriction protocol has been overcome only 3 times, although a number of attempts were made. Since the ``Time of Peace’’ treaty was signed 40 years ago, no known attempt has been made to overcome the restricting protocol.

Relics that are used will be de-attuned by the ritual, and there may not be time to re-attune them if their abilities are needed in the last 30 minutes or so before the end of game, so it is still important to consider \textbf{which} relics are ultimately used in the Ritual, not just where they are temporarily attuned to.

If there are no \textbf{attuned} relics on any pedestals (either no relic or all de-attuned), there is a risk that the Ritual to Control the Storm goes awry. That would be very bad.

\textbf{At noon sharp on Sunday, the pedestals will lock down.} Any relics currently on them will be locked in place and can no longer be removed or switched (which means they can’t be (re)attuned or de-attuned either). If a pedestal is empty, it will no longer be possible to place a relic upon it. The only exception is if a relic does not match the restriction of the pedestal that it is on, characters will have 5 minutes to make 1 good faith attempt to find and place a relic that does match the restriction instead, or change the restriction and attempt to place the same relic again. \emph{(OOC: This cut off for interacting with the pedestals and relics exists so that the GMs can do necessary accounting before the Ritual actually begins.)}

\end{document}

