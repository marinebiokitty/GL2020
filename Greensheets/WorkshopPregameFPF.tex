\documentclass[green]{GL2020}

\usepackage{enumitem}
\setlist{nosep}
\parindent=0pt
\begin{document}
\name{\gPreGameFPF{}}

\section*{(50-60 min)}

Hello and welcome. Once again my name is X and my pronouns are Y. If you are here, it is because your character is from the \pTech{}. If you are not from this nation, you're in the wrong place!

Over the next hour we are going to:
\begin{enumerate}
	\item Introduce ourselves and our characters.
	\item Create some social and cultural norms for the \pTech{}, which can help us feel like a more coherent group.
	\item Reflect on the social inequalities present in our society and where we fall within them.
	\item Reflect on the impact the war has had on the nation.
\end{enumerate}

\subsection*{Player / Character Introductions (5 min)}
We’re going to go around the circle and introduce \textbf{players}. Please share your name, pronouns, and what state you live in normally. Use the 1st person for this. $<$GM models$>$

We’re going to go around again, this time introducing your \textbf{character}. Please share your character’s name, pronouns, and whether they are a student, teacher or an advisor. Use the 1st person for this too. $<$GM models$>$

\subsection*{Social and Cultural norms (30 min)}
Split yourselves up into 3 groups of 4-5. I am going to have each person count to a number of 1-3, starting here.  That will be your group.  I’m going to give each group a different question. You’ll decide the answer in your group, then we’ll come back together and share it with the rest of the group. The answer can be words, or actions, but shouldn't take more than 1 sentence to explain. We’ll split off again with a second and then third question. We’ll split off again with a second and then third question. We will give each of these questions a few minutes to come up with an answer and then spend the next few minutes sharing our answers.  Each round should only take five minutes total. In order to save time, we will be using the same groups each time.  

Round 1:
\begin{enumerate}
	\item How do we greet each other?
	\item How do we express approval and disapproval?
	\item How do we swear?
\end{enumerate}


Round 2:
\begin{enumerate}
	\item What is a sign of moderate wealth? What about great wealth?
	\item What symbolic object is present in almost every home (or on every ship)?
	\item What is one thing we judge people negatively for doing?
\end{enumerate}

Round 3:
\begin{enumerate}
	\item What action or activity do most of us do regularly that is steeped in tradition?
	\item What holiday is the biggest celebration for us? When is it?
	\item How often are group religious services held? Do lay people have roles in these services, or just clerics and initiates?
\end{enumerate}


We are going to get back into a group to have some time to do free world building.  We have a few topics for folks to discuss if you feel stuck, but we are going to open it up first to any topics that this group wants to discuss.  Workshop leaders will make sure that we get to multiple areas of discussion and that conversation is not dominated by a few group members.  As such, if you want to speak, we ask that you raise your hand so we can call on you after another group member is done speaking.

Here are some of the prompts:
\begin{itemize}
	\item What do folks do for fun?
	\item What are some new trends popping up in the culture?
	\item What regional differences exist in the country?
	\item What foods do we eat?
\end{itemize}  

\subsection*{Social Inequalities (20min)}
Let’s all come back together. We are going to reflect on the social structure of our nation, where the characters fall within it, and what changes might or might not be possible for us to enact.  Players should answer these questions based on what they think is generally known about their character.

Where does your character fall in the social hierarchy? Pay attention to this; as it will likely influence how you treat each other. \textbf{Have players group themselves by what they think is generally known about their character:}

Do you publically express negative opinions or disagreements with the church and their control of technology? Let's see a show of hands. 

Then have players form lines for each of the following questions:
\begin{enumerate}
	\item How good are you personally at inventing and adapting technology? Form a line from no interest/no talent to genius inventor
	\item How rich is your family? In terms of money and/or social capital? Form a line from richest to poorest
\end{enumerate}

\textbf{Discussion Questions:}
Get into the following groups:
\begin{itemize}
	\item \cHeir{}, \cAmbition{}, \cLibrarian{},  \cAssistantScientist{}
	\item \cChupInventor{}, \cHeadScientist{}, \cTechStar{}, \cEthics{}
	\item \cBeetle{}, \cAntiChup{}, \cScholarship{}, \cDiplomat{}
\end{itemize}

\begin{enumerate}
	\item How much do you believe the adage that talent is all you need, and station is no impediment? Do all people really have equal access to opportunities?
	\item What are your opinions on the Faledon Family as a legacy family with a guaranteed seat on the council and often the swing vote on things? Are you looking forward to the day that \cHeir{} takes over?
\end{enumerate}


\subsection*{Impact of The War (5 min)}
Now we are going to reflect on the impact of the war. We are going to stay in our same groups from the previous questions. I’m going to ask a couple of questions, for personal reflection. We won't discuss these now, but you may want to talk about them in character later:
\begin{enumerate}
	\item Where in the country does your character live when not at the College? How close is your home to the fighting? Has your home been damaged or destroyed by the war?
	\item How many people did you know personally who have been hurt or killed?
\end{enumerate}
\textbf{Discussion Question:} How has rationing affected the country?

\subsection*{Send Off (5 min tops)}
Now that the workshops are complete, think back to any goals you may have set for your player experience of this game, and any expectations you had going in this afternoon. 
\begin{itemize}
	\item Did any of those change organically during the workshops? 
	\item Do you want to specifically and intentionally change any of them now? 
	\item What concrete steps can you take to help you get what you want out of this weekend?
\end{itemize}

Think about these things in the back of your mind over the next few hours. You have between now and 6:30 to unload the cars, get unpacked and get into costume. Dinner is at 6:00. Once you finish eating, you’ll have the remaining time before the game starts to get into costume. Game starts at 7:00 pm at X location. Please don’t be late; we won’t wait.

\end{document}