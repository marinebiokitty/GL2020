\documentclass[green]{GL2020}
\usepackage{enumitem}
\setlist{nosep}
\parindent=0pt
\begin{document}
\name{\gTreaty{}}

\pEarth{} has been at war for almost 6 years. This weekend, you have the opportunity to negotiate with other advisors to try to hammer out a solution. However, you do not have the power to ratify such a treaty, the best you can do is send impassioned letters home alongside the drafted treaty to try to convince your government to agree to whatever treaty you propose. 

In the Great hall, there are treaty templates and blank paper available for this mechanic. \textbf{These items are in-game as soon as they are written upon.}

\textbf{Proposing Treaties:}
You may write whatever you want in your proposal - new alliances, new arrangements for the storm, reparations, formal apologies, extradition clauses, etc. You may write as much or as little as you want. However, since back and forth communication to your government is limited, the approach of many small proposals will likely run out of time before you accomplish much. - comprehensive proposals are *highly recommended* to achieve meaningful change.
\begin{itemize}
  \item Treaties may be proposed between any 2 or more entities (aka nation). 
  \begin{itemize}
    \item Only the 3 nations are recognized as entities, so treaties can only be between nations as of game start.
    \item A treaty may be proposed between 1 or more currently recognized entities, and a proposed new entity, strictly for the purposes of formalizing that the current entities recognizes the new one as an entity worthy of having treaties with. Such a case must also identify qualified ``Advisor Signatories’’ for the new entity.
  \end{itemize}
  \item Any proposed treaty that you want a government to review or possibly accept \textbf{MUST} be signed by at least \textbf{2 advisors} from each entity involved.
  \item Proposals \textbf{may} also include the signature of a Mediator, who may \textbf{not} also be one of the signatories, but may be from a nation named in the proposal. There is no requirement that the mediator be an advisor. Proposals that include a mediator will be received more positively by \textbf{all} entities.
  \item \textbf{Once a proposal is signed by characters at the Time of Deciding, bring it to GM HQ} so we can make a copy of it for our records. You may then refer to the proposal by number in letters home, rather than needing to recount the details in every letter.
\end{itemize}

\textbf{Writing Letters Home to Convince Governments to Accept the Proposals:}
Your home governments have NO IDEA of what is being discussed here unless you tell them (although they may learn of events that transpire). And nothing you discuss here is OFFICIAL unless the relevant governments sign off on it - the people present at this event do not have sufficient political capital to enforce their will on the entire continent. \textbf{If you have this greensheet, you have the ability to send letters home to your government.}
\begin{itemize}
  \item Unless you know otherwise, you may only send letters to your own government (which will also reach church leadership). For the purpose of this mechanic, there are no factions within the government you can reach out to specifically.
  \begin{itemize}
    \item \pFarm{}: \cQueen{\full}.
    \item \pTech{}: The High Council.
    \item \pShip{}: The Council of StormWatchers + the Warlord
  \end{itemize}
  \item You \textbf{MUST} include both who is sending the message (truthfully), and which government it is to be sent to.
  \item You may send letters on behalf of others, but if the letter is not about a treaty, or if it is on behalf of someone not a citizen of that nation, your government may not respond.
  \item Drop off any correspondence you wish to send to your home governments in the envelope for “\sSignT{}”, in GM HQ.
  \begin{itemize}
    \item Unless you know otherwise, you may not send game items, only written letters.
  \end{itemize}
  \item Return correspondences will be distributed at meal times. Responses will be addressed to whoever sent the letter. GMs will return your original correspondence with the response so you can remember what you wrote.
  \item If you send a letter within 1 hour of the next meal, or after 9pm (for breakfast the next day), you should not expect a response until the subsequent meal. Submitting multiple letters will likely delay response. There will not be time to respond to letters sent after 11 am on Sunday, so that will be your last chance for treaty ratification.
  \item The more well crafted your letter, and the arguments in it, the more likely you will be able to persuade your government to shift its position and agree to a proposal it might otherwise have rejected. Responses on the other hand will be short and to the point, because the GMs are very busy. Short messages alone are not sufficient evidence that a government is displeased. If they don’t like something, they will say so directly.
\end{itemize}

\textbf{Ratified Treaties:}
If at any point, you have approval in hand from each nation’s government that is involved in a proposal, the proposal is considered ratified. Bring the approvals to the GMs, and the GMs will post the new ceasefire or treaty publicly. Governments will be \textbf{very reluctant} to break a ratified treaty (after all, the last time they did that, it started a war), but may be convinced to amend an existing treaty with new clauses that don’t contradict previous ones.

\textbf{Any ratified treaty which stipulates where the storm should be sent will carry metaphysical weight and have some influence on the storm’s actual destination for this Time of Deciding. See your ``\gVotingStones{}’’ Greensheet for more information.}
\end{document}

