\documentclass[green]{GL2020}
\parindent=0pt
\begin{document}
\name{\\gBadLuckCurse{}}

You have the bad luck (pun intended) to have had a bad-luck curse placed upon you. This curse causes you to have bad luck. Not all of the time at first, but enough of the time. The longer the curse is on you, the worse it gets.

The only known way to end such a curse is to fulfill whatever requirement was stipulated in its making. This is most commonly to go to a certain location, or accomplish a certain task. Unfortunately, not everyone who places a bad-luck curse has the courtesy to tell their victim what the conditions for ending it are. It is possible there are other ways to dispel this, but you don’t know for sure. The best you can do is ask around and hope!

\textbf{You’ll need your D20  for this mechanic.}

The curse has the following effects on you:
\begin{itemize}
  \item You cannot notice waylays directed at you. Unless \textbf{someone else} notices them, they will automatically succeed.
  \item At breakfast each morning, shuffle the items in your possession. Pick one at random, and ``lose it'' by dropping it somewhere in game space before noon. Your character does not remember where they lost it, or when; you should not become ``aware'' of having lost something until after lunch, unless someone prompts you for it specifically (e.g.: Can I have my pen back that I lent you?)
  \item In combat, roll your D20 before each round. On a Nat 1 or 2, your CR is reduced to zero for that round, and any ``upstage'' attack directed at you will act like a ``knock out.''
  \item At meal times, roll your D20. If the result is 4 or lower (20% chance), you do not draw CR stones up to your maximum.
  \item Any time you attempt an action that requires a die roll, \textbf{you must roll twice and take the lower.}
\end{itemize}

\end{document}
