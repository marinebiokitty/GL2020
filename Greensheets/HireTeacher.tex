\documentclass[green]{GL2020}
\usepackage{enumitem}
\setlist{nosep}
\parindent=0pt
\begin{document}
\name{\gHireTeacher{}}

Normally a board of directors is in charge of managing the employment of teachers for the \pSchool{}. As the Principal you \emph{can} act without the board, although you rarely do. The board does a good job, and you have many other things to worry about.

You have the authority to \textbf{hire} a new teacher from among the advisors, or the 3rd year students - it would be very unorthodox to hire someone with no teaching experience, but it is not unheard of, and it would be very difficult for someone to get your decision overturned, even if they objected, even after you retire.

You also have the authority to \textbf{fire} a current teacher - technically you can do so for any reason, but it’s really bad form to do so, unless you have a really good reason. Some good reasons might include: dereliction of their teaching duties, mistreating a student, or violation of the neutrality teachers are supposed to maintain towards international politics (treaties, wars, etc.) and in advising students.

Hiring or firing someone will only have post-game consequences. Changing the roster in this manner will not impact any mechanics - an advisor hired as a teacher still needs to participate as an advisor in mechanics requiring advisors, etc.

If you choose to hire or fire someone, you may create as much or little ceremony around it as you like. We recommend including ``The Teacher’s Oath’’ in the ``Ritual Space.’’

\emph{(OOC: When you choose your successor and formally transfer the title, give them this greensheet.)}

\end{document}

