\documentclass[green]{GL2020}
\parindent=0pt
\begin{document}
\name{\gLifeAfterDeath{}}

\textbf{If you haven't played your body for 10 minutes; go back and do that first! When the time has elapsed, leave all of you items in a pile, along with your character badge.}

Unfortunately your character was killed at The Time of Deciding. That your murderer is now an amnesiac may not be much consolation to you. Your spirit will eventually move on to the afterlife (whatever that entails), but for the next little while, your spirit is in a transitional period on the Divine Plane, still able to interact with and influence the mortal plane in some ways.

\textbf{Set your player badge to the ``Not Here’’ side.}

While all of your items stayed with your physical body, you retain all knowledge and abilities you had in life. If you can find a way to gain access to any required items, you can continue to use all of your abilities and greensheets as long as your target is on the Divine Plane (they do not need to be corporeal here unless they are trying to hand you objects).

You are now limited to the following:
\begin{enumerate}
  \item You may freely explore the Divine Plane, and may converse freely with any one else who is also dead, or any living being who manages to reach the Realm of the Gods.
  \item \textbf{Once per day} you may speak directly to your Patron Deity (find a GM).
  \item You may generally read any objects or signs on the mortal plane as if you met the interaction criteria. \textbf{Exception:} Signs that only contain OOC information remain off limits. (e.g.: ``you may not interact with this sign unless you know otherwise.’’)
  \item You may go among the living invisibly and listen in to conversations that are not magically protected (use the OOC symbol; people in a magically protected conversation will tell you so.)
  \item You may not normally interact with people or things on the Mortal Plane. However, \textbf{three times per day} you may partially manifest on the material plane. See the ability ``\aGhostlyMessage{}’’ for details. 
\end{enumerate}

\end{document}
