\documentclass[green]{GL2020}
\usepackage{enumitem}
\setlist{nosep}
\parindent=0pt
\begin{document}
\name{\gCeremonyOfExcellence{}}

The Ceremony of Excellence is an age old tradition at the \pSchool{}, and is an important opportunity to celebrate the students. The speeches made during the Ceremony often shape the political discussions occurring at the ``Time of Deciding.’’ The Ceremony is scheduled for 4:00 pm on Saturday.

This crucial task normally falls to the principal of the school, but since \cPrincipal{\full} is retiring this year, \cPrincipal{\they} \cPrincipal{\have} passed this task to \cMusic{\full} as a way to assess \cMusic{\their} readiness for such an important role. \cPrincipal{} will be assessing \cMusic{}’s ability to execute this as smoothly as possible.

Instructions for The Ceremony of Excellence:
\begin{enumerate}
  \item \cMusic{} is in charge of picking people for each part, making sure they are properly prepared, and managing the ceremony. Having the appropriate audience is also quite important. - Since only \cMusic{} and \cPrincipal{} actually know exactly what is supposed to happen at each step, the coaching is crucial to making sure people fill their roles appropriately.
  \item Traditionally the teacher and student speakers rotate between countries, and this year it would traditionally be the \pShippies{} turn, but \cMusic{} has ultimate decision making power, and is allowed to go ``off script'' as it were, as long as some teacher and some student speaks.
  \item Several people have expressed interest in certain roles in the ceremony. \cMusic{} can pick from among those options, or choose someone else, if they can be convinced to take up the task.
  \item The Ceremony itself consists of:
  \begin{enumerate}
    \item The Ceremony leader giving a quick speech (up to 5 min) to set the stage: \cMusic{}.
    \item A Cleric should bless the space.
    \item A single teacher should speak (up to 5 min) on the unique excellence of the students here this weekend. \cFlowPriest{\full} and \cPirate{\full} have expressed an interest in this.
    \item A single student should speak (up to 5 min) on their vision for the future. \cWarlordDaughter{\full} and \cPirateChild{\full} have both expressed an interest in this.
    \item A single advisor should speak (up to 5 min) on the responsibility that comes with power and authority. \cHedonist{\full} and \cCurse{\full} have expressed an interest in this.
    \item A pair of students should engage in a magical combat exhibition to show off their skills. The students should use the “upstage”intention in their combat, to avoid actually injuring each other, if they misjudge their opponents strength. \cTechStar{\full}, \cHeir{\full}, \cLibAssist{\full}, and \cAmbition{\full} have all expressed interest in being one of the duelists.
  \end{enumerate}
\end{enumerate}

The audience would ideally consist of everyone present at the school during the ``Time of Deciding,’’ but that is pretty much impossible, too much is going on this weekend. It is particularly important that as many students as possible attend, and moderately important for the teachers to attend. \cPrincipal{} will have to make a subjective call on how hard \cMusic{} tries to get everyone to attend, and how effective it ends up being. Advisors are notoriously difficult to persuade, and \cPrincipal{} should not penalize \cMusic{} for failing to round up the advisors.

\end{document}

