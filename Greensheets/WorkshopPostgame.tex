\documentclass[green]{GL2020}

\usepackage{enumitem}
\setlist{nosep}
\parindent=0pt
\begin{document}
\name{\gPostGame{}}

\section*{Deroll (15 min)}
We have 4 options for this 15 minute deroll time:
\begin{enumerate}
	\item X will be running a sitting and thinking silent, self-reflection deroll - no talking, just thinking. You’ll be able to do as much or little of this as you want.
	\item Y will be running a walking and talking deroll. You’ll be able to do as much or little of this as you want.
	\item Z will be coordinating a super preliminary clean up and teardown - in case you’d rather do something with your hands that is mostly unrelated to the emotional experience of game.
	\item You can do whatever you want. Take a walk, have a cuddle puddle, go journal, go meditate, go pack your own stuff, whatever feels right for you right now. Just be back in 15 min if you want to do ``game wrap'' with us.
\end{enumerate}

Your decision now for what to start with does not lock you in. You can change your mind, tap out, switch activities, whatever you need, whenever you need it. This time is for you, we’re just offering a couple of activities that you might find supportive.

\subsection*{Sitting and Thinking Deroll:}
\begin{enumerate}
	\item Stepping away:
	\begin{enumerate}
		\item Get comfortable as best you can. Close your eyes and focus on your breathing. (4 square).
		\item Think of your character in your mind. Think of being that character, like you’ve been all weekend. How they walked, how they talked, etc. Imagine that you are inside this character.
		\item Now visualize looking up, look out a little distance and see yourself as you, the player. See how you normally style your hair, the clothes you normally wear, etc.
		\item Now I’m going to slowly count down from 10. As I count down, I want you to imagine moving, step by step, across that distance. Leave behind the character, and allow yourself to inhabit the player instead.
		\begin{enumerate}
			\item 10… take a step. 9… take another step. 8… 7… … 1.
		\end{enumerate}
		\item Settle down wherever you are now, and open your eyes.
	\end{enumerate}
	\item Take your namebadge off from around your neck, and look at it. See the character’s name there. Take that nametag out to reveal your player name badge behind it. You can put the badge back on for now, or not, as you prefer.
	\begin{enumerate}
		\item You are not your character. Your character is still with you; the experiences you had this weekend were real, but you are not your character.
	\end{enumerate}
	\item Now we’re going to reflect on a few thoughts privately:
	\begin{enumerate}
		\item How are you feeling right now, physically? Are you tired? Are you hungry? Are you energized?
		\item How are you feeling mentally and emotionally right now? Are you excited? Are you sad? Are you confused about something that happened?
		\item Is there anything from the game you’d like closure on? Is there a player you’d like to talk to about something? We can’t guarantee that your fellow player will be up for that conversation, but knowing you want to have it is prerequisite to asking for it.
		\item Are there any questions you want to ask or clarifications you need from the GMs? We’ll cover as much as we can during Game Wrap, and unstructured time after, and of course we’ll be available over the discord later.
		\item What is 1 thing you can plan to do for yourself either tonight or tomorrow after you get home, or to wherever you are staying tonight, to take care of yourself?
	\end{enumerate}
\end{enumerate}

\subsection*{Talking and Walking Deroll:}
\begin{enumerate}
	\item Walking away:
	\begin{enumerate}
		\item Step over here with me. 
		\item Right here, standing here, is your character. You are occupying the same space with them right now. Take your character namebadge out of the badge holder and look at it. This is the character you played this weekend.
		\item Now look up, look out a little distance away and imagine yourself as you, the player, over there somewhere. See the distance between the character and the player.
		\item Now I’m going to slowly count down from 10. As I count down, I want you to step across that distance. Leave behind the character behind (feel free to literally leave the name badge), and allow yourself to inhabit yourself more and more with each step.
		\begin{enumerate}
			\item 10… take a step. 9… take another step. 8… 7… … 1.
		\end{enumerate}
		\item Settle down wherever you are now.
	\end{enumerate}
	\item Look at your namebadge now. It has your player name displayed now. You are not your character. Your character is still with you; the experiences you had this weekend were real, but you are not your character.
	\item Now we’re going to reflect on a few questions in small groups of \textbf{no more than 3}. You can walk around this area while you talk, or sit together. Don’t go too far though, so I can give you the next question. You can always pass on answering a question out loud to the others.
	\begin{enumerate}
		\item How are you feeling right now, physically? Are you tired? Are you hungry? Are you energized?
		\item How are you feeling mentally and emotionally right now? Are you excited? Are you sad? Are you confused about something that happened?
		\item Is there anything from the game you’d like closure on? Is there a player you’d like to talk to about something? We can’t guarantee that your fellow player will be up for that conversation, but knowing you want to have it is prerequisite to asking for it.
		\item Are there any questions you want to ask or clarifications you need from the GMs? We’ll cover as much as we can during Game Wrap, and unstructured time after, and of course we’ll be available over the discord later.
		\item What is 1 thing you can plan to do for yourself either tonight or tomorrow after you get home, or to wherever you are staying tonight, to take care of yourself?
	\end{enumerate}
\end{enumerate}

\section*{Game Wrap (1 hour, 15 min)}

\subsection*{GM Epilogue (5 min)}
The GMs will summarize the most broadly reaching epilogue points
\begin{enumerate}
	\item Where was the storm sent?
	\item Was a treaty signed - if so, what did it contain? Has it already been broken?
	\item What is the fate of the school?
	\item Does \cPrince{} ascend the throne?
	\item Who takes over as head of the Faledon family?
	\item Is \cLoud{} still the warlord?
	\item Who is the new principal (if relevant)?
	\item Was a permanent portal to the Divine Plane established?
	\item Are the \pGoaties{} out in the open now? Or still working in the shadows?
\end{enumerate}

\subsection*{Character Narratives (55 min):}
Think carefully about what you’re going to say for this. There are 37 of you, and we do want to stick to the time we have allotted while still giving everyone a chance to be heard if they want. We’re going to go around the circle, and each player will have the chance to say:
\begin{enumerate}
	\item Player name/pronouns
	\item I played $<$Character name$>$
	\item 1 sentence summarizing who your character was (especially any secret parts of their identity) at the beginning of the game, and what they wanted most.
	\item 1 sentence summarizing whether what they wanted most changed over the course of the game, and if they got it in the end or not. 
	\item 1 sentence sharing a pivotal or impactful moment for your character.
\end{enumerate}

\subsection*{Questions (15 min):}
Time for players to ask the GMs and each other questions about what happened with X, what Y was for, how did Z turn out, etc. this time is squishy and will bleed into final tear down and goodbyes.

Optional: Finding a check in buddy! - we’ll want to check in with our buddy approx 1 day, 3 days, and 1 week from now.


\section*{Teardown and Load Out}
We gotta pack up and get out of here. This last hour is unstructured time to hang out and chat amongst yourselves as needed. We encourage you to seek out players with whom  you had intense scenes or significant roleplay, and check in with them and yourself.

\section*{Post Game on the Discord}

\subsection*{Epic Tales:}
A place for players to synthesize their character’s experience into a short story-esque retelling if you like.

\subsection*{Intro Thread:}
We’ll be posting an OOC intro thread on Tuesday for folks who would like to introduce themselves OOC to each other, and share a little about who they are and what they like to do.

\end{document}
