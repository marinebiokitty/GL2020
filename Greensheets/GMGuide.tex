\documentclass[green]{GL2020}

\usepackage{enumitem}
\setlist{nosep}
\parindent=0pt
\begin{document}
\name{\gGMGuide{}}

A good reference doc/guide for things we can anticipate players might want, and notes to remind us of stuff.

{\large ITEM NUMBERS}
\begin{enumerate}
	\item ARE 4 digits long!
	\item Notes to/from \cChupInventor{} start with item number ``21\ldots ''
	\item Only 1 item: the \iFolderOfNotes{} from \cBunker{} that \cDisney{} has start with ``31\ldots ’’
	\item Technology that can be imbued by the FPF Clerics must start with ``45\ldots ''
	\begin{enumerate}
		\item \iMagicMusicBox{}
		\item \iVidCom{}
	\end{enumerate}
	\item The sea serpent baby starts with ``51…’’
	\item Not \textbf{items}, but the divine realm locations all have 8 digit locations ##x##x##x##.
	\item The most commonly needed things:
	\begin{enumerate}
		\item \iRitualCandle{}
		\item \iGlassVial{}
	\end{enumerate}
	\item Other items have completely random/arbitrary numbers starting with ``9\ldots ’’
\end{enumerate}

{\large Pregame stuff:}
\begin{enumerate}
	\item Make sure everyone gets a D20.
	\item Check with \cPrincipal{\MYplayer} for whatever the Principal's badge of office is.
	\item Give each of the \pGoaties{} a 4 leaf clover pin, or check that they have their own 4 leaf clover symbol.
	\item Record what each cursemaker chooses as their maker’s mark.
	\begin{enumerate}
		\item Fill in other marks for the curses \cHedonist{\MYplayer} has.
		\item Copy \cPrince{}’s mark on \cInitiate{}’s bad luck curse.
	\end{enumerate}
\end{enumerate}

{\large In Game stuff:}
\begin{enumerate}
	\item Curses - GM will drop in the appropriate folder in GM HQ for the cursemaker to pick up after 15-30 minutes.
	\begin{enumerate}
		\item Cursemakers will let you know when they start brewing something. Hopefully.
		\item GMs will have only 15 minutes to get a copy of the curse/cure into the little envelope attached to ``\sCurses{}'' sign in GM HQ that matches that player’s badge #. We will have a few printed out initially, but if you notice demand is heavy for one in particular, make sure we print /assemble more before we are completely out.
		\item \cCurse{\full}, badge # \cCurse{\MYnumber{}}
		\item \cPrince{\full}, badge # \cPrince{\MYnumber{}}
		\item \cAdopted{\full}, badge # \cAdopted{\MYnumber{}}
		\item \cLibAssist{\full}, badge # \cLibAssist{\MYnumber{}}
		\item \cHeir{\full}, badge # \cHeir{\MYnumber{}}

	\end{enumerate}
	\item If a character dies:
	\begin{enumerate}
		\item Make sure you send them back to play their dead body for 10 minutes because the Amphora can resurrect people up to 5 min after death.
		\item Make sure they get both the greensheet and a yellow headband from GM HQ (\sMurdered{}).
		\item Print the character a new badge that swaps the description for “A ghost”.
	\end{enumerate}
\item Inventions - GM must approve.
	\begin{enumerate}
		\item The technology must already exist; only minor modifications are allowed. The scitzotech of the world makes this hard, but in general:
		\item \textbf{NO:} batteries, engines, weapons, flight, memory restoration, anything that duplicates magic that is exclusive to a subset of characters (no making a machine to heal, or a curse making machine), anything as powerful or more powerful than the relics, or anything that negates the big public plots (e.g.: building new bunkers from scratch).
		\item \textbf{MAYBE:} Can only make \iVidCom{} if you have the \iVCDBlueprint{} in hand. If it is going to short circuit a plot. If it has the same functionality as an existing curse (cures are probably a no).
			\item Black market can provide ``\iMagitechParts{}'' item (chup inventor has a mechanic for regulating risk).
		\item 1 \iMagitechParts{} parts for up to 1 hands bulky. 2 or 3 hands would require consuming 2 \iMagitechParts{} parts. Anything that would reasonably be bigger than 4-hands bulky is almost certainly out of the scope of game to build.
		\item Once you okay an invention, go straight to GM HQ and prepare the item card for the inventor, as it only takes them 10 minutes, and we can't have these prepared ahead of time.
	\end{enumerate}
	\item Advisors have the ability to send letters home. Will need to respond; responses only need to be returned at the next meal 1+ hr after the letter home is received. Use prepared wax seals.
	\begin{enumerate}
		\item Letters sent after 11 am on Sun can be ignored unless a GM has time to kill somehow.
	\end{enumerate}
	\item \cWarlordDaughter{}’s RN has them write a letter to \cQuiet{} asking if anything suspicious happened in the weeks leading up to \cLoud{}'s sudden personality change.
	\begin{enumerate}
		\item GMs should write a response that acknowledges the specific message the player wrote while also revealing that: \cLoud{} and \cQuiet{} blamed the Council of Stormwatchers for being too soft as the crux of the problem/ultimate reason for the betrayal. They contracted \cCurse{\full} to create a curse that would empower one of them to rise in national politics and be the backbone they thought the country needed. But the curse was way more potent than they had bargained for. They never wanted \textbf{this}, but they don’t know how to stop it.
		\item At the meal AFTER we receive a \iPanacea{} from \cWarlordDaughter{}, send the WR advisors the official resignation letter: \cChupLeader{}, \cJuniorStatesman{}, \cBunker{}, and \cEbbPriest{}. Send \cWarlordDaughter{} the personal resignation letter. Both are stored in GM HQ.
		\item After the \iPanacea{} happens, it is much harder for the \gAssassinateWarlord{} plot to succeed due to the letters being harder to persuade. But also, the warlord stepping down achieves an approximately equivalent outcome as far as \cEvil{} and \cDiplomat{} are concerned, so they may well drop that plot at this point. -- Letter from \cWarlordDaughter{} still works for sure, letter from black market has lower chance of success. Letter from advisor(s) fails completely.
	\end{enumerate}
	\item Changing Patrons or Abandoning Gods - Warn the player that this will be very difficult, then give them a copy of \gAbandonGods{} from GM HQ.
	\item Avatars - GMs generally need to play Avatars and the Deities.
	\begin{enumerate}
		\item Resurrecting the Ebb Avatar requires a GM. You must consult the current list of FoG members to see whether the ritual feels weird. If any FoG members participate, tell the ritual leader something felt weird.
		\begin{enumerate}
			\item At the end of the ritual, switch out the \iBabySeaSerpent{} item card for the \iAvatarSeaSerpent{} Item card. Add +3 to the CR of the baby sea serpent to set the CR of Ebb’s new avatar.
		\end{enumerate}
		\item Summoning the Avatar of Genesis requires a GM in case they want to talk to it. Swap out the \iRabbitStatue{} card for the \iAvatarRabbit{} card.
		\item Talking to the \iAvatarBeetle{} requires a GM, as does the Ritual to seal the Faledon Heir. IF \cAmbition{} is the heir-apparent in the ritual, you MUST consult with \cHeir{} about whether \cHeir{\they} succeeded at the ritual to make \cAmbition{} an equivalent entity for the blood curse. 
\item If they manage to figure out where the beetle is in game, and incorporate it into the ritual, you will also be asked to make a statement from Kero about the future of the Faledon family under this head of house. - consult with the heir apparent to learn what is the 1 biggest thing they will push for with the family; that thing will succeed, as long as it is within reasonable scope for what the family can do. aka they can’t end the war or the storms on their own.
	\end{enumerate}
	\item Inducting someone into the FoG requires a GM: 
	\begin{enumerate}
		\item \textbf{FoG headcount - need to keep a list of who has been inducted, and when.}
		\item Add the new person’s name to the GM list of FoG members.
		\item Have the person set their \textbf{V-Score = 1}
\item Give the person a Luck of Genesis ability (gain advantage on a die roll x1/day).
	\end{enumerate}
	\item Promoting a Cleric or Consecrating an Avenger requires a GM:
	\begin{enumerate}
		\item GM plays the avatar; accept the person, do a formal ``swearing in’’ (make it up!)
		\item For Consecrating an Avenger, make sure they change their \textbf{V-score=2}.
	\end{enumerate}
	\item Many of the Preparations for the Storm need to be checked by a GM or the NPC or have a the NPC or GM be told the thing.
	\begin{enumerate}
		\item We have a copy of the preparations in GM HQ. Check things off and make any notes about incomplete/poorly completed tasks.
		\item Players (or another GM or NPC who is running off to do something else) may ask you to check if a relic matches the ``restriction’’ on a pedestal. To do this, set a 15 minute timer. Between 10 min later and when the timer goes off, go and check the attunement of the relic by opening the attached envelope and reading the piece of paper inside it. Put the paper back and don’t tell anyone what it says. If the pedestal has a ``restriction’’ the location will be written on the item card for the pedestal. If the relic attunement \textbf{does not} match, put the relic either on the floor next to the pedestal, or on a nearby table/etc. If the location \textbf{does} match the ``restriction.’’ Just put the relic back on the pedestal.
\begin{enumerate}
\item The time delay is \textbf{important} here. It prevents this mechanic from being abused to side step other mechanics in the game. Don’t cut it short.
\item \textbf{Relics attuned to the school} always match whatever restriction is listed
		\end{enumerate}
		\item Ritual leaders for the ritual to control the storm must submit their plan by Noon on Sunday so GMs can prepare over lunch.
	\end{enumerate}
	\item Voting Mechanics
	\begin{enumerate}
		\item There is a GM Voting tally greensheet for actually counting up the votes and figuring out where the storm is going to be sent.
	\end{enumerate}
	\item Repairing the bunkers:
	\begin{enumerate}
	\item Mu Trivia requires asking a GM if you have the answers right. Here are the Q/As, and what to say if not all of them are correct.
		\begin{enumerate}
			\item If all answers are correct: \textbf{The brick comes loose in your hand; you can now retrieve whatever is inside the attached envelope.}
		\item If \textbf{even one} answer is \textbf{incorrect:} The brick doesn't budge; at least one of your answers is not correct.
			\item What entity governs the \pFarm{}?\textbf{ \cQueen{full}}
			\item What entity governs the \pTech{}? \textbf{The Council}
		\item What entity governs the \pShip{}? \textbf{The Council of Stormwatchers}
		\item  Who was the \pSchool{}’s first Principal? \textbf{Katharina Friedrich}
			\item How many Patron Deities are there? \textbf{4 Deities}
			\item What animal serves as \cFarmGod{}'s Avatar? \textbf{Hummingbirds}
			\item Which family holds a hereditary position in the \pTech{} government? \textbf{The Faledon Family}
			\item In the geography of \pEarth{}, where is The Graveyard located, and why is it feared? \textbf{Between 8th & 9th Fleets, because of jagged rocks, mist, and unpredictable currents}
		\item How was the Old Wing of the College Destroyed? \textbf{The then principal left the island.}
			\item How many tiers are in the library? \textbf{4 Tiers (this is the Q they are most likely to get wrong.}
	\end{enumerate}
	\end{enumerate}
	\item The Library: Most of the Library choose-your-own-adventure does not require a GM, but there are a couple parts that do:
	\begin{enumerate}
		\item Characters can conduct research at the Tier 1 Catalog (page 12) and the Catalog of Esoterica (page 46). Look in the Library choose-your-own-adventure booklet for the precise details. In summary: At the Tier 1 Catalog, they can just ask the GMs a question, and if it is \textbf{public} knowledge, they get an answer. At the Catalog of Esoterica, they can ask more esoteric questions that are not necessarily public knowledge, and how quickly they get an answer depends on the size of their research group and whether any of them are a “Subject Matter Expert.” They write the question down, place it in Envelope 13, we write the answer on the same piece of paper, and they return and pick it up once the requisite amount of time has passed.  Answers \textbf{should not short-circuit an existing plot} (i.e. information that is gated by a research notebook or greensheet). In such cases, give them a hint or suggest where to look in game to find more information. For example, if a group is researching what will happen to the Ley Lines if the Storm strikes the school, they should find a book with torn out pages (torn out by \cChupstudent{\full}), and either a clue hinting that \cChupstudent{} is the culprit or pointing them to one of the characters in game who has some of the information they’re looking for (\cHistory{\full} and \cBeetle{\full}). Maybe one or the other of them was the last person to check out the book so their name is in it!
		\item If people want to do something in the Library that is not among the listed options, they can ask a GM. Feel free to tell them they can’t if it doesn’t seem reasonable or would break things. Otherwise, you are free to make up the results.
	\end{enumerate}
\end{enumerate}

{\large End of Game stuff:}
See EOG workshop doc.

\end{document}

