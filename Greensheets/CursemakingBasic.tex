\documentclass[green]{GL2020}

\usepackage{array}
\usepackage{xcolor}
\usepackage{hyperref}
\usepackage{multicol}
\usepackage{ltablex}
\usepackage{tabularx}
\renewcommand{\tabularxcolumn}[1]{m{#1}}
{\renewcommand{\arraystretch}{1.5}
\setlength{\columnsep}{1cm}

\usepackage{enumitem}
\setlist{nosep}
\parindent=0pt
\begin{document}
\name{\gCursemakingBasic{}}

The umbrella term is ``curse,'' but spells with negative effects are more exactly known as curses, and spells with positive effects are more exactly known as cures (even if they aren’t directly ``curing’’ something). With basic curse-making abilities, you are limited to recipes that require no more than 2 ingredients, but you know a good handful of recipes that might be useful this weekend. If you are having trouble finding an ingredient you need, other characters may be able to help, especially nobles from Children of the Sun, some of whom have the ability to make flowers grow on command.

You have a small kettle on your character that you normally brew your curses in. Like your other magical foci, you can technically use any container, like a pot or a mug. Since these are freely available, for simplicity of the mechanic, the kettle is not actually an item card, and cannot be taken from you. This means that once you finish with the active steps of brewing a curse, no one can interrupt you or steal the work in progress. \textbf{Your only limit is that you may only brew ONE curse/cure at a time.}

\textbf{Crafting Curses:}
\begin{enumerate}
	\item Find a source of fire. (any sign in game that describes a stove, fire, or an open flame will do.)
	\item Place each ingredient, one at a time, into the kettle while reciting its name out loud. 
	\item Place the kettle over the fire and stir in a clockwise direction for thirty seconds, then stir in a counter-clockwise direction for thirty seconds. -\textbf{ after this step is complete, this process is no longer interruptible.}
	\item The ingredients are consumed at this point. Discard the ingredient item cards to the nearest stock. If you are interrupted before this, the ingredients are not consumed.
	\item Tell a GM ASAP so we can prepare the item-envelope for you during the charging time.
	\item Remove the kettle from heat and allow the curse/cure to charge for the required time. The default charging time is \textbf{15 minutes}, but some curses may require longer charging time.
	\item After the charging time, the curse is complete. Go to the GM HQ and take the item-envelope from the sign with your badge number on it. - if the curse isn’t there, find a GM ASAP!
	\item \textbf{Draw/write your maker's mark on the item-envelope.} You must place the mark as faithfully as you can. You get to pick what your makers mark looks like, but you should pick something easy to write/draw that you can do consistently, and that \textbf{does not} immediately implicate your character as the maker (e.g.: don’t use your character or player signature.) The GMs will need to know what you pick as part of check-in for game. Note that \textbf{only} Advanced Cursemakers can actually see maker’s marks; if you are not an Advanced Cursemaker, you lack the skill to read other people's marks.
\end{enumerate}

\textbf{Activating Curses:}
The rules document describes how to activate a curse on someone. Unless a mechanic says otherwise, curses with conditional triggers (e.g.: only trigger if this person is from X bloodline), or embed curses (e.g. an object that will curse the first person who touches it) take weeks or months to craft, and are therefore outside of the scope of this game.

\textbf{Recipes:}
Below is a list of recipes you know. Other recipes may exist. If someone brings you a recipe that requires 2 or fewer ingredients, you can make it. Feel free to write the recipe down on this sheet for future reference.

\begin{tabularx}{\textwidth}{| >{\centering\arraybackslash} m{4cm} | >{\centering\arraybackslash} m{4cm} | X |}
\hline
	\textbf{Curse Name (item number)} & \textbf{Ingredients (item number)}  & \textbf{Effect} \\
\hline
\hline
	\iCourage{}	&	\iMoonflower{}, \iLimestone{}   & +3 CR  for \textbf{10 minutes} after the curse is activated; cannot stack with additional instances of this curse.	\\
\hline
	\iWeakness{}	& \iNightshade{}, \iBlackCrocus{}   &	-3 CR  for \textbf{10 minutes} after the curse is activated; cannot stack with additional instances of this curse. \\
\hline	
	\iInsight{}	& \iMorningGlory{}, \iEagleFeather{}	& A moment of clairvoyance - in the Library, at \textbf{one juncture/page}, you may look at where \textbf{all} of the options lead before picking one. After doing this, the effect fades. \\
\hline	
	\iBabble{}	& \iClay{}, \iSpiderWeb{}	&	The target character can speak only gibberish for \textbf{5 minutes}. \\
\hline	
	\iRestraint{}	& \iSpiderWeb{}, \iLily{}	&	Applies the \textbf{restrained} status to the target character for \textbf{5 minutes.} They are magically restrained and cannot break this status before 5 minutes have passed by natural means.\\
\hline	
	\iStrength{}	&	\iFlameOrchid{}, \iObsidian{} &	Allows the user to carry up to 4 hands bulky worth of objects normally (instead of 2), for \textbf{1 hour} after the curse is activated. \\
\hline	
	\iBlindness{}	&	\iMorningGlory{}, \iClay{} & Causes the target to become blind for \textbf{5 minutes}.\\
\hline	

\end{tabularx}



\end{document}

