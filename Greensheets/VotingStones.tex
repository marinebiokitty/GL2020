\documentclass[green]{GL2020}
\usepackage{enumitem}
\setlist{nosep}
\parindent=0pt
\begin{document}
\name{\gVotingStones{}}

The primary reason for the ``Time of Deciding'' is in order to prepare and execute a ritual to control where the magical ``Storm'' will go. There are many preparations toward this end. In order to measure where the students want the Storm to go, a system of voting stones is used. The attunement of the relics used in the final Ritual, as well as any declarations of where the Storm should go that are made in any ratified Treaties will also have some influence on where the storm goes.. \textbf{Ties resolve in favor of the majority student opinion.}

\section*{The Treaty}
Any ratified treaty which stipulates where the storm should be sent will carry metaphysical weight and have some influence on the storm's actual destination for this Time of Deciding. At the beginning of game, the ``End of Suffering'' treaty is in effect, which specifies the storm to be sent to the \pShip{}. 

The specified location will receive \textbf{18 votes}.

\section*{The Attunement of Relics used in the Ritual}
\textcolor{red}{This is a change from information previously provided: Relics attuned to a given location that are included in the ritual will count as votes \textbf{towards} that location, rather than being protective.}

The teachers have the ability to restrict the pedestals that hold the Relics for the Ritual. The restriction limits it so that only relics matching the attunement listed will stay on the pedestal. If the attunement does not match, the relic will be rejected. See \cPrincipal{} or GM if you place a Relic on a restricted pedestal. They will adjudicate the mechanic.

A declaration of neutrality from the teachers would be represented by 1 relic attuned to each location. \textcolor{red}{\textbf{This is not a common stance for the teachers to take.}} Since the teachers are generally expected to leave treaty negotiations to the Advisors, and provide the students with neutral guidance at this crucial juncture of their young lives, deciding on the relic attunement restrictions is how they express their opinion on where the Storm should be sent.

\emph{(OOC Note: There are no actual mechanical restrictions that prevent teachers from getting involved in treaty discussions or to obligate them to dispense neutral advice to students. Players may choose to have their character do so if and when it makes sense for their character. \textbf{However}, players should be prepared to face social and possibly significant post-game consequences for doing so (e.g. contempt from colleagues, losing one's job, etc.) )}

The attuned-to location of each relic will receive \textbf{6 votes}, for a total of \textbf{18 votes,} split among as many as 3 locations. 

\section*{The Student Vote}
The student vote is managed through a set of 36 voting stones. The students will need to cast their votes on Sunday.

\subsection*{The mechanics of transferring voting stones are as follows:}

\begin{enumerate}
\item Every character (except for Principal \cPrincipal{}) starts with \textbf{1} voting stone. This includes students.
\item Teachers and Advisors must pick a student to give their stone to. Only students can cast votes. \emph{(OOC: Don't be a jerk and refuse to give your stone to a student under any circumstance. Let someone persuade you.)}
\item No single student may receive more than 4 additional stones. If a fifth teacher or advisor tries to give them another voting stone, the student must decline. This means that a single student cannot cast more than 5 votes total (their own starting stone, plus 4 from Teachers and Advisors).
\item Teachers and advisors are \textbf{strongly} advised to speak to multiple students and take their time deciding who to give their stone to, even if you start game thinking you know who you want to give your stone to; things might change! Saturday evening is a reasonable time to pass over your stone, though waiting until early on Sunday is not uncommon. 
\item Teachers and Advisors are also strongly encouraged to consider giving their voting stone to students who have not yet received any stones. This will likely require asking around for how many stones different students have received so far.
\item Students are encouraged to speak to multiple advisors to get perspective on the best place to send the Storm.
\end{enumerate}

\textbf{Unless a mechanic explicitly states otherwise, stones may not be given/taken, stolen, or traded after they are in the hands of a student.} Some teachers may have the ability to transfer votes from one willing student to another. 

\subsection*{The mechanics of casting voting stones are as follows:}

Students with multiple voting stones may choose to split their votes however they like between multiple locations. Votes may be cast any time between \textbf{Sunday morning game on and Noon.} To cast your votes, follow these steps:

\begin{enumerate}
  \item Write your name on the outside of the envelope attached to the voting stone just before submitting. Make sure to write legibly. You cannot put anyone else's name on it.
  \item Open the envelope, remove the enclosed piece of paper, write your vote for the destination of the Storm on the paper, and replace it inside the envelope.
  \item Place the envelope (with the stone attached) in the voting box.
\end{enumerate}

You must complete steps 1-3 with 1 vote before starting another (it is theoretically possible for someone to do something to one of your other voting stones while you are submitting the first one). You may \textbf{not} fill in the voting slip inside the voting stone and then carry the voting stone around. To the best of your ability, it \textbf{must} go straight into the ballot box after you have filled it out. Only an explicit mechanic can interrupt this.

\textbf{Each voting stone counts as 1 vote}. The student vote as a whole represents \textbf{36 votes}.

\section*{The Final Result}
GMs will tally the votes over lunch on Sunday, and reveal the result at the appropriate point in the Ritual to Control the Storm. The final vote breakdown will not be provided in game — only the revelation of where the Storm is headed. \emph{(OOC: Don't worry, we'll give you the breakdown after the game ends.)}

\end{document}


