\documentclass[green]{GL2020}

\usepackage{array}
\usepackage{xcolor}
\usepackage{hyperref}
\usepackage{multicol}
\usepackage{ltablex}
\usepackage{tabularx}
\renewcommand{\tabularxcolumn}[1]{m{#1}}
{\renewcommand{\arraystretch}{1.5}
\setlength{\columnsep}{1cm}

\usepackage{enumitem}
\setlist{nosep}
\parindent=0pt
\begin{document}
\name{\gCursemakingAdvanced{}}

\textbf{This greensheet builds directly on the ``\gCursemakingBasic{}'' greensheet that you should also have.}

You have learned the secret of enhancing curses, allowing you to create more complex spells. Beyond the basic curse preparation, you now know how to create more advanced and dangerous curses. You may create curses with up to\textbf{ 4 ingredients}.


\textbf{Crafting Curses:}
The basic method for crafting curses with the Advanced skill is the same, with 1 exception regarding the Maker’s Mark.

With your advanced skill you can \textbf{once per day} cause the Maker’s Mark on a curse you make to appear different from your normal mark (draw/write a completely different symbol). You may not deliberately cause the mark to look specifically like someone else’s mark unless you have one of their curses on hand. As long as the curse is un-activated or not yet worn off, you may copy the mark to the best of your ability. Having someone stand next to you who has a curse active on them counts for this.

\textbf{Recipes:}
Below is a list of recipes you know. Other recipes may exist. If someone brings you a recipe that requires 2 or fewer ingredients, you can make it. Feel free to write the recipe down on this sheet for future reference.

\begin{tabularx}{\textwidth}{| >{\centering\arraybackslash} m{3cm} | >{\centering\arraybackslash} m{5cm} | X |}
\hline
	\textbf{Curse Name (item number)} & \textbf{Ingredients (item number)}  & \textbf{Effect} \\
\hline
\hline
	\iSlowActingPoison{}	&	\iWeakness{}, \iRestraint{}, \iFish{}, \iFlameOrchid{} & The target will die in several days \emph{(OOC this will kill the character post-game)}\\
\hline	
	\iSlowActingPoisonCure{}	&	\iCourage{}, \iLily{}, \iStoneFlower{} & Cures the effects of the \iSlowActingPoison{} before any harm is done\\
\hline	
	\iHonesty{}	&	\iInsight{}, \iHollyhock{}, Blood from the target & Target must answer the next 1 question truthfully. \textbf{Additional Instructions: } You must write the name of the person who's blood was used on the curse once you receive it. It will not work on anybody else.\\
\hline	
	\iBlindness{}	&	\iRestraint{}, \iMorningGlory{}, \iClay{} & Causes the target to become blind for \textbf{5 minutes}.\\
\hline	
	\iSight{}	&	\iCourage{}, \iEagleFeather{}, \iSunflower{} & Immediately ends blindness caused by the \iBlindness{}.\\
\hline	
	\iSkill{}	&	\iStrength{}, \iBlackCrocus{}, \iLimestone{} & Through cleverness and strength the target is able to reduce the time required for any \textbf{one} mechanic by half. E.g. If something requires meditating for 5 minutes, they may accomplish that thing in 2.5 minutes. If a cooldown is 10 minutes long, they may wait only 5, etc. This includes the charging time for curses.\\
\hline	
	\iBadLuckCurse{}	&	\iWeakness{}, \iBabble{}, \iBlackCrocus{}, \iCharcoal{} & The target is \textbf{permanently} cursed with terrible luck, affecting every aspect of their life. \textbf{Additional Instructions:} You must have a GM approve a specific ``release'' location or action. (You may pick any location that is reasonable to expect someone could reach in epilogue. Likewise, you may pick a task that could be completed in epilogue; however, you may not specify a task which is forbidden by the target’s Patron Deity.) In addition to the release condition, the original Cursemaker is able to release the target from the curse at any time.\\
\hline	
\end{tabularx}



\end{document}
