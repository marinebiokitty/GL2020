\documentclass[green]{GL2020}
\usepackage{enumitem}
\setlist{nosep}
\parindent=0pt
\begin{document}
\name{\gWRCleric{}}

This greensheet describes the rituals associated with being a Cleric. Initiates know a subset of these rituals. Characters will call on you to perform these throughout the weekend. 

\subsection*{General Guidelines to Playing a Cleric Character:}
Be a supportive and listening ear for anyone who needs one. Guide those who follow your patron Deity to stick to their tenants and embody them yourself. Emphasize the cultural touchpoints for your nation that will be developed during workshops for yourself and others.

Your default answer when someone asks for your help should be ``yes.'' Only refuse to help if you have a good \textbf{in character} reason to do so. When it comes to creating and playing out rituals in game, many of your participants in the rituals will \textbf{not know how to participate (IC or OOC).} You should be prepared to coach people ahead of time, and guide them during the ritual. You may consult this sheet at any time, and you can and should incorporate this preparation into your roleplay.

\subsection*{Rituals You Can Perform:}
\textbf{All rituals are interruptible.} Items that are consumed are only consumed at the \textbf{end} of a \textbf{completed} ritual; if it is interrupted, the items are not consumed.

You are empowered in and out of character to create these rituals before or during game if that is enjoyable for you. At the end of this sheet is a series of appendices that have \textbf{examples of one possible way} to run each of these rituals. You may use it verbatim, riff on it, or create something entirely new. \textbf{However,} any increases in the time, or material or personnel requirements, to a ritual beyond what is listed here \textbf{must have} buy-in from all players involved before you start.

There are a few, rare circumstances in which someone may have a different or modified version of a ritual they ask you to perform. While it is appropriate to view such requests with some suspicion, it is not completely unheard of, especially in times of crisis when the clerichood of one Deity may need to call upon help from another to make quorum.

Below is the list of rituals you are trained in, along with their basic requirements and limitations:

\textbf{Ritual to Bless Something or Someone in the name of your patron:}
  \begin{enumerate}
    \item Should take no less than 15 seconds, and no more than 1 minute.
    \item Most invoke the name of your Patron Deity.
    \item Requires 1 cleric or initiate, the person being blessed, or the object(s) being blessed and the person who wants you to bless it. 
    \begin{itemize}
      \item You may not bless an object for yourself. Another cleric must do it for you.
      \item You may not bless an unwilling person unless a mechanic specifically allows for it. If you are not sure, ask a GM.
    \end{itemize}
  \end{enumerate}

\textbf{Ritual to Cleanse a Space:}
  \begin{enumerate}
    \item Should take no less than 30 seconds, and no more than 2 minutes.
    \item Requires 1 cleric or initiate and 1 ``\iRitualCandle{}''. The candle must be consumed by the ritual.
    \item Most invoke the name of your Patron Deity at least once.
  \end{enumerate}
   
\textbf{Examining a Relic to Determine Attunement:}\\
This ritual allows you to examine a Relic to determine its Attunement status. Relics can exist in one of 4 known states - they can be attuned to one of the three nations, or they can be unattuned (neutral). To do this:
  \begin{enumerate}
    \item Should take no less than 2 minutes, and no more than 5 minutes.
    \item Requires 1 cleric or initiate, one other person to help, the relic to be examined, and 1 ``\iCrystalLens{}'' (rumor has it that these can be found in the library somewhere.) The lens must be consumed by the ritual. 
    \item At the end of the ritual, you may open the relic and pull out the paper inside to read the current attunement. Replace the paper immediately, without letting anyone else read it \textbf{(not even your helper).}
  \end{enumerate}
   
\textbf{Reattuning a Relic:}
  \begin{enumerate}
    \item Should take no less than 5 minutes, and no more than 10 minutes.
    \item This ritual must happen at the ``\sLeyLinesNexus{}’’ \textbf{after} the ley lines have been renewed.
    \item Requires 2 clerics (or 1 cleric and 1 initiate), 3 people from the location you wish to attune the relic to, 1 ``\iRitualCandle{}'', and 1 ``\iTuningFork{}''.  The candle must be consumed by the ritual.
    \item You must take the relic to a GM at the completion of the ritual who will adjust the attunement of the relic (no one gets to learn what the relic was attuned to before this ritual was conducted.)
  \end{enumerate}
   
\textbf{Inducting a New Devotee to your Patron:}\\
You have the ability to induct a new devotee to your patron. You can only do this with someone who freely consents to this, and whom you are convinced is genuinely committed to the change. In 99\% of cases, this is usually a step along someone’s path to immigrating to your country. As far as you know, everyone in the game ascribes to the Patron Deity of their nation. If you find out otherwise, it is appropriate to be shocked and even scandalized. 
  \begin{enumerate}
    \item Should take no less than 2 minutes, and no more than 5 minutes.
    \item This ritual must happen in the temple.
    \item Requires: 1 cleric, the person changing to a new patron deity, 1 person who remains a follower of their previous patron, 2 people who already follow the new patron deity.
		\item Send the person to a GM at the end.
  \end{enumerate}
   
\textbf{Promoting Someone to a Full Cleric:}\\
This ritual is almost exclusively used to promote someone who has studied as an initiate for years to the status of a full Cleric. There are some weird edge cases where someone who is not an initiate may be promoted to a full Cleric. Ask a GM if you are not sure.

Becoming a full Cleric is a huge responsibility. Don’t hesitate to roleplay extensively, asking the candidate serious questions about their commitment to this path. You must be convinced through roleplay that this person is ready for the increase in responsibility. Once you are satisfied, organize the ritual:
  \begin{enumerate}
    \item Should take no less than 5 minutes, and no more than 10 minutes.
    \item This ritual must happen in the temple.
    \begin{itemize}
      \item 2 clerics (at least 1 of whom must be a cleric of the same patron as the initiate)
      \item The initiate becoming a full Cleric.
      \item At least 4 witnesses (a minimum of ½ rounded up must be followers of the patron to which the initiate is pledging).
      \item The ``\iOakStaff{}'' (which should probably be returned to its place in the temple when you are done).
      \item One of the 2 relics traditionally associated with the nation/Patron Deity. Unlike most uses of the relics, this does not require the relic to be attuned to a \textbf{specific} location, they just have to be attuned to somewhere. Further, this ritual will not de-attune the relic.
      \item A GM.
    \end{itemize}
    \item The ritual must summon the avatar of the Deity (whom the GM will play). If the avatar is not able to access the mortal plane (e.g. if their mortal animal vessel has been killed), this ritual will fail. 
  \end{enumerate}
   
\textbf{Seeking Answers on the Wind}\\
There are very few ways to get direct answers from the Deities. While the Celestial Beetles of \cTechGod{} are the closest anyone comes to being handed solutions, \cEbb{} and \cFlow{} are quite happy to offer guidance and direction. You may ask a question for yourself, or on behalf of someone else. You and anyone else participating in the ritual will all receive the same answer, whispered on an errant sea breeze that blows in all the way from the coast, bearing the scent of saltwater.

To ask the Goddesses for guidance:
\begin{enumerate}
    \item This ritual has a 20 min cool down after you use it (successfully) before it can be used again.
    \item Takes no less than 30 seconds, and no more than 2 minutes.
    \item Requires an ``\iEagleFeather{}'', a ``\iRitualCandle{}'', and the question written down on a piece of paper. The feather and the candle must be consumed by the ritual. The question must be submitted to a GM.
		\item The GM will provide a smoewhat vague hint of a productive direction of investigation to find the answer to the question.
  \end{enumerate}

\subsection*{Appendix: Ritual Examples}

\textbf{Ritual to Bless Something or Someone in the name of your patron:}
  \begin{enumerate}
    \item Take the object in your hands, or mime taking the other person’s hands in yours. (you may ask for OOC permission and then actually take their hands if you are both comfortable with the physical contact, but the default is no actual contact.)
    \item Say ``I call upon $<$\cEbb{} or \cFlow{}$>$ to wash over and purify  $<$person’s name$>$ or this $<$ name of object$>$’’
    \item If you are blessing a person, you can have them respond with ``I accept this blessing''. If you are blessing an object, have whoever asked you to bless the object say this.
  \end{enumerate}

\textbf{Ritual to Cleanse a Space:}
  \begin{enumerate}
    \item Acquire a \iRitualCandle{}.
    \item Stand in the middle of the space you wish to cleanse. Call upon \cEbb{} and \cFlow{} to bring their attention to the space.
    \item Walk slowly around the edge of the space, holding the candle out in front of you. As you walk, you may do so silently, or hum, or chant as you choose.
    \item Return to the center of the circle. Call upon \cEbb{} and \cFlow{} to extend their power to protect the space and those who enter it from undo harm.
  \end{enumerate}
The ritual is now complete. Discard the candle item in the nearest stock; it has been consumed
   
\textbf{Examining a Relic to Determine Attunement:}
  \begin{enumerate}
    \item Find a ``\iCrystalLens{};'' rumor has it there are several in the Library. 
    \item Spend 5 uninterrupted minutes holding the Relic with your helper without taking any other action (conversation is fine). At the end of the 5 minutes, the Crystal Lens shatters (discard the item card in the nearest stock). 
    \item Open the envelope associated with the Relic and read the paper inside to see where it is currently attuned. \textbf{No one else may look inside the envelope (not even your helper), even if they also possess the ability to examine relics. Only one pair may examine a given Relic at a time.}
  \end{enumerate}
   
\textbf{Reattuning a Relic:}
  \begin{enumerate}
    \item Gather the following in the Old Wing of the College, at the ``\sLeyLinesNexus{}''.
    \begin{itemize}
      \item A \iTuningFork{} and 1 \iRitualCandle{} (for the cleansing ritual)
      \item 1 additional Cleric or Initiate (besides yourself)
      \item 3 more people all of whom must be  from the target attunement location. (e.g.: 3 people from \pTech{} to attune a relic to \pTech{})
    \end{itemize}
    \item A cleric must lead the ritual.
    \begin{enumerate}
      \item The cleric or initiate not leading the ritual should cleanse the space (see ritual above) with all of the participants already inside the space.
      \item Arrange the ritual participants in a circle, and place the relic in the middle.
      \item The ritual leader steps \textbf{out} of the circle, and walks slowly around the outside of the circle. They may do this in silence, or hum, sing, chant, etc, as they feel so moved. They will return to their place at the end of the walk.
      \item Each person does this, in turn. Start with the person to the cleric’s \textbf{left.} Each person should circle in the \textbf{opposite} direction that the previous person did.
      \item The ritual leader calls upon the magic of the ley lines to imbue those assembled with the power to change the attunement of the relic at hand.
      \item The ritual leader steps \textbf{into} the circle, touches the tuning fork to the relic, and steps back to their place the circle. They then pass the tuning fork to the person on their \textbf{right.}
      \item Each person around the circle in turn touches the tuning fork to the relic. The tuning fork should end back in the hands of the ritual leader.
    \end{enumerate}
    \item Take the item to a GM. The GM and ritual leader will discuss in secret where the ritual leader intends the relic to be attuned. The GM will then check the current alignment of the relic and adjust the alignment if necessary.
    \begin{itemize}
      \item No one in the ritual gets to learn what the previous attunement was. This ritual cannot be used as a replacement for the ``Examine the Relic'' ability.
    \end{itemize}
  \end{enumerate}
The \iTuningFork{} is not consumed, but is traditionally passed to the other cleric who participated.
   
\textbf{Inducting a New Devotee to your Patron:}\\
Changing your patron deity, and your home nation, is a big deal. Prepare for this in the following way:
  \begin{enumerate}
    \item Speak at length with the person. Impress upon them the tenants they are about to take on.
    \item Ensure that they have also spoken with a Cleric of their current patron about their decision.
  \end{enumerate}
  
Once you are convinced of someone’s sincere wish to take up a \cEbb{} and \cFlow{} as their new patron deity:
  \begin{enumerate}
    \item Assemble the following in the Temple:
    \begin{enumerate}
      \item The inductee
      \item At least 1 witness from the patron the inductee is leaving behind.
      \item At least 3 witnesses from the patron the inductee is joining.
    \end{enumerate}
    \item Arrange the inductee and their 1 witness on one side of the space. Arrange the 3 witnesses from your patron on the other side of the space, with everyone facing each other. The two groups should be at least 6 paces apart.
    \item Begin the ritual by opening your arms to indicate both groups of people. Make a short explanation of what is happening (e.g.: This person wishes to take on \cEbb{} and \cFlow{} as their patron. We are here to witness this.)
    \item Bring everyone into a proper frame of mind for the ritual by having everyone follow you in humming / chanting for at least 15 seconds or until you feel everyone’s attention is present and focused on the task at hand.
    \item Have the inductee and the witness turn to face each other and bow to acknowledge one another. Then have the witness say ``As a representative of $<$Their patron deity$>$. My heart is heavy with this goodbye.’’
    \item The Inductee replies ``May knowing that I go home lighten your burden.’’ The inductee should then turn to face the other 3 witnesses.
    \item The 3 witnesses should say ``We are the representatives of \cEbb{} and \cFlow{}.’’
    \item They must call out together. ``Come home to us.’’ over and over. Each time they say the phrase, the inductee should take 2 steps toward the group.
    \item When the inductee reaches the group, you must say ``\cEbb{} and \cFlow{} welcome you. May you find comfort, and may you be a credit to us.’’
    \item Send the person to a GM at their earliest convenience in case we need to swap any mechanics for them.
  \end{enumerate}
The ritual is now complete, the person now has your patron deity as theirs. Celebration should ensue.
   
\textbf{Promoting Someone to a Full Cleric:}
  \begin{enumerate}
    \item Assemble the following in the temple:
    \begin{enumerate}
      \item You, the primary cleric sponsor to lead the ritual. You can only promote a cleric of your own patron (in this case, you can lead a ritual for a new cleric of \cEbb{}, \cFlow{} or the Path of Balance; it is still traditional that you only lead a ritual for your patron, and that you lead jointly with an appropriate counterpart if someone insists on balance.)
      \item A second cleric from your nation, or  2 clerics from other nations, or 5 non-clerics from your nation.
      \item The initiate to be promoted.
      \item 1 person to be the initiate’s sponsor. This should be someone who can speak about the character’s character.
      \item One relic traditionally associated with your nation (its current attunement does not need to be known).
      \item The ``\iOakStaff{}'' (One is available in every temple across \pEarth{})
      \item At least 5 additional witnesses. This is a \textbf{big} deal that whole communities generally celebrate together.
    \end{enumerate}
    \item Have a cleric cleanse the space. If possible, have someone other than yourself do this.
    \item Organize your audience arrayed around the initiate, either in a semicircle, or a circle.
    \item Ask the initiate: ``Are you ready to embark on the path of \cEbb{} /\cFlow{}/Balance?''
    \item Ask the witnesses: ``are you ready to bear witness to this oath?'' Guide them to respond if necessary.
    \item Summon the relevant avatar by holding up the relic and calling the avatar forth. If someone wishes to take up the Path of Balance, you must summon both avatars.
    \begin{enumerate}
      \item If the mortal vessel of the avatar is not present on the mortal plane, the ritual will fail at this point (the GM will tell you so.)
    \end{enumerate}
    \item Once the Avatar(s) manifests, prompt the candidate to make a short statement as to their intention to become a cleric. (e.g.: ``I desire to pledge to the path of \cEbb{}/\cFlow{}/Balance'')
    \begin{enumerate}
      \item Audience should respond with ``we witness.''
    \end{enumerate}
    \item Prompt the sponsor to make a short statement or ``speech'' (no more than a few sentences), vouching for the good character of the candidate.
    \begin{enumerate}
      \item Audience should respond with ``we witness.''
    \end{enumerate}
    \item Have a cleric (can be the ritual leader) make a short statement or ``speech'' (no more than a few sentences), vouching that the candidate is properly prepared and devoted.
    \begin{enumerate}
      \item Audience should respond with ``we witness.''
    \end{enumerate}
    \item Have the candidate make a longer statement or ``speech'' (no more than two paragraphs) explaining why they wish to become a Cleric, and what good they believe they can do on behalf of their patron deity.
    \item The Avatar(s) will accept the candidate, and initiate the formal oath taking.
    \begin{enumerate}
      \item Technically it is possible for the avatar to reject a candidate, which is just embarrassing all around, but this happens only rarely, and only when the clerics, the sponsor, or the initiate themselves is dis-ingenious.
      \item Applause is appropriate at this time.
    \end{enumerate}
    \item The Cleric leading the ritual should thank the Avatar(s) and release them.
  \end{enumerate}
Major celebration should ensue! This is a really big and exciting deal.

\textbf{Seeking Answers on the Wind}\\
This ritual has a 20 minute cool down before you can use it again.
  \begin{enumerate}
    \item Write your question down on a piece of paper, or have the person asking write it down. If someone else is writing it down, you must still read the question before starting the ritual.
    \item Acquire an ``\iEagleFeather{}'' and a ``\iRitualCandle{}''.
		\item Find a GM; having them on hand is important to make sure everyone gets the answer in a timely manner.
		\item This ritual must be performed in an \textbf{in game outside space}. Most of game will be OOC outside, but IC some spaces are indoors and some are outdoors.
		\item Hold the \iEagleFeather{} above your head and call out to the Goddesses: ``\cEbb{\full}, we seek your council. \cFlow{\full} we seek your council.’’
		\item Roleplay lighting the feather on fire with the \iRitualCandle{} (no actual flames will ever be used in game due to fire hazard), and watch it burn away. 
		\item Blow on the item card as if you were scattering the ashes of the feather.
		\item Roleplay lighting the piece of paper on which the question is written on fire with the ritual candle.
		\item Blow on the item card as if you were scattering the ashes of the piece of paper. If this is someone else’s question, you can invite them to take the paper and scatter the ashes themselves.
		\item Cup a hand to one ear as if listening for something faint.
  \end{enumerate}

The ritual is now complete. Return the bird feather and ritual candle item cards to the nearest stock or GM; they have been consumed by the ritual. Give the piece of paper with the question on it to a GM. The GM will provide a suggestion on a productive direction of investigation. (e.g.: ``The thing you seek is in the Obsidian Greenhouse in the library’’, or ``A cursemaker should be able to help you.’’)

\emph{(OOC NOTE: Players may always come to a GM for guidance if the player feels stuck / isn’t having fun. This ritual allows players to ask for a smaller version of that help without breaking character and provides an in-character opportunity for other characters to become engaged with that character and invested in the things they care about.)}


\end{document}
