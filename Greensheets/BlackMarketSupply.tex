\documentclass[green]{GL2020}
\usepackage{enumitem}
\setlist{nosep}
\parindent=0pt
\begin{document}
\name{\gBMSupply{}}

The \pGoaties{} have established a Black Market that crosses national borders on Cengea. While the market pre-dates the war, the demand for it has redoubled since the war started. None of the established nations \textbf{like} the Black Market, but the \pTech{} is particularly vicious in its desire to stamp the market out.

You are one of the ``Market Managers’’ for this Black Market. This means you get to broker trades and purchases along with the other Market Manager: \cChupSecond{\full}. You are technically the one in charge of supplying the market, while \cChupSecond{} handles the actual deal-making most of the time. It’s convenient for \cChupSecond{} to be the ``face'' that people associate with the market. It gives you plausible deniability if shit were to hit the fan.

Supplying the market usually involves one of four things:
\begin{enumerate}
  \item Imbuing Technology - Inventors who have been rejected by the Church (just like you were) increasingly turn to the market to get their technology imbued with magic. This of course creates a convenient dependance where the inventor must keep coming back to the market with each new instance of their technology. Most eventually just sell the un-imbued items to the market, you activate them, and then sell it on for a tidy profit. To imbue a piece of technology for perpetual use:
  \begin{enumerate}
    \item You \textbf{must} have a V-score $>$ 0 to use this. If you lose your V-score (for example by leaving the \pGoaties{}, you lose \cGenesis{}’s gift that allows you to imbue technology!)
    \item A Cleric-like ritual that takes no less than 30 seconds, and no more than 2 minutes.
    \item Requires the item of technology to be imbued. Only items with an item number starting ``45…’’ will benefit from being imbued.
    \item Requires the expenditure of 1 unit of magical energy contained in a piece of Tass (e.g. a ``\iChargingStone{}). You have 6 \iChargingStone{\MYname}s with you this weekend. Four are drained, 2 still have 1 unit of magical energy left in them.  You are considered to have the ability to draw magical energy from Tass for the purpose of this task. (Discard the CR stone from the item to the nearest stock when you use it.)
    \item Write ``Imbued’’ on the item card after the ritual is complete.
  \end{enumerate}
  \item You can write to your contacts in the \pTech{} to see if they can supply you with the requested item. \textbf{You may only attempt to source one thing at a time.} This is most common for if someone is asking about forged documents — things small enough to send in an envelope and not attract attention. To do this:
  \begin{enumerate}
    \item Write a note to your contacts in the \pTech{}. The note is an \textbf{in game item}. Include \textbf{both} what it is you want, and the price you and \cChupSecond{} intend to charge for it.
    \item Hand the note to a GM, or drop the note in ``\sSignR{}’’ in GM HQ if you can’t find a GM easily.
    \item The GMs will write back to you as soon as we can. Expect an average turn around of 2 hours. Return to the sign to check for replies; if convenient, GMs may hand you the reply directly, but please do not assume that we will always be able to find you in a timely fashion.
    \item GMs will either supply the requested item, or let you know that it is not possible to fulfill that request, or not for that price (too risky for the reward.) The return notes will also be \textbf{in game items.}
    \item You may freely destroy both your own notes and the ones from your contacts.
  \end{enumerate}
  \item At greater risk, you may request your contacts to send you ``\iMagitechParts{}.’’ These are useful for building things yourself (see option 4 below), or for other inventors here this weekend. Being harder to conceal than just papers, these run a risk of your supply line being discovered. You may only attempt to source one thing at a time. You may not ask for another document or ``\iMagitechParts{}’’ until you hear back about your previous request.
  \begin{enumerate}
    \item The first time you request \textbf{``Misc. Magitech Parts’’,} there is NO chance that your supply line is discovered. However, subsequent requests will incur stacking 5\% chances that the supply line is nearly discovered.  \textbf{You will still receive the requested item, even if your supply line is nearly discovered. This will only impact your ability to make a subsequent request.} Starting with the second request for parts, roll a D20.
\begin{enumerate}
    \item On your second request, if you roll a 1, your supply line has almost been discovered.
    \item Each subsequent time you request parts (3rd time, 4th time, etc), the chance of near discovery increases by 5\%. (a 1 or a 2 for the third request with a 10\% risk, 1-3 for the fourth request with a 15\% risk etc.)
    \begin{enumerate}
       \item Requesting documents (see above) does not increase the risk of discovery.
    \end{enumerate}
\end{enumerate}
    \item \textbf{Once your supply line is nearly discovered, you will not be able to request anything for 6 hours} while your contacts relocate to a different safe-house. This could easily make it impossible for you to source important things during a crucial period of time. 
    \begin{enumerate}
      \item \textbf{Start a 6 hour timer or set an alarm for 6 hours from now.} (Or if 6 hours from now would be the middle of the night, you just can’t use this mechanic until tomorrow.) You may not submit requests for additional supplies until this time has passed.
    \end{enumerate}
    \item \textbf{The risk counter will reset to 5\% after a near discovery. The risk will never return to zero.}
  \end{enumerate}
  \item You can build things yourself. For example, if someone wants an automatic scribing machine that records dictation, you can probably make one! To do this:
  \begin{enumerate}
    \item \textbf{Talk to a GM.} GMs must approve your construction as something that you have both the time and resources to construct. This is both for game balance, and so that we know what you are making and can go prepare the item card for you!
    \item Acquire 1 unit of ``\iMagitechParts{}’’ You have 3 of these stored in your secret maker’s space (The ``\sAlchemyLabOne{}’’ in the Old Wing of the College).
  \begin{enumerate}
    \item You may freely navigate to the ``\sAlchemyLabOne{}’’, bypassing the normal CR requirement for access. You may bring people with you, if you want, else you can just ``disappear into the ruins somewhere'’ and leave people behind.
    \item You may freely bypass the rolling mechanic at  the ``\sAlchemyLabOne{}’’.
    \item You may freely review the items available at this location, and take as many or as few as you like at any time. (You are also immune to the cooldown requirement).
    \item You are also \textbf{the only one} who can \textbf{store} things here. The space can hold up to 5 items. At game start, it is holding 3 ``\iMagitechParts{}.’’ You \textbf{cannot} store \textbf{any} bulky items here, except for the item with item number \iAvatarRabbit{\MYnumber}.
  \end{enumerate}
  \item Take the ``\iMagitechParts{}’’ to either your secret engineering laboratory hidden in the old wing of the college (recommended) or the public maker space. You are the only one who can use the ``\sAlchemyLabOne{}’’ as if it were a Maker’s Space.
  \item Spend \textbf{5 minute} roleplaying building the item. For especially complicated tasks, the GMs may add additional conditions like needing a helper, or a longer build time.
  \item Return the ``\iMagitechParts{}’’ to the nearest stock. See a GM for the item card for the item you just constructed. You cannot take the raw materials and use them to build something else. This is both for game balance, and to allow the GMs to prepare your item card ahead of time for you.
   \item \textbf{Choose wisely what you are going to build — as explained above, resupplying magitech components is both slow and risky.}
  \end{enumerate}
\end{enumerate}

Since imbuing technology requires Tass (e.g. your \iChargingStone{}s), you will probably want to have some people pay for things in magical energy to recharge the \iChargingStone{}s. You’ll need to find a way to drop off and pick up the Tass somewhere, or do it through \cChupSecond{} or \cChupSecond{\their} agent. If you do it yourself, you’ll expose yourself as an agent of the Black Market, but sometimes that might be worth it.

There are a number of well known inventors in game who might be interested in buying magitech parts from the Black Market or getting their devices imbued. These include \cTechStar{}, \cAssistantScientist{}, \cBunker{}, and \cHeadScientist{}; be careful approaching them, though, as you have never sold to any of them and don’t know their feelings on the Black Market. \cHeadScientist{}, at least, is a good friend of yours and may be sympathetic.


\end{document}

