\documentclass[green]{GL2020}
\usepackage{enumitem}
\setlist{nosep}
\parindent=0pt
\begin{document}
\name{\gCoSCleric{}}

This greensheet describes the rituals associated with being a Cleric. Initiates know a subset of these rituals. Characters will call on you to perform these throughout the weekend. 

\subsection*{General Guidelines to Playing a Cleric Character:}
Everyone who claims a Patron Deity is a ``follower’’ of that Deity. A subset of ``followers’’ undertake religious studies, becoming ``initiates.’’ ``Initiates’’ who complete the required years of study can apply to be promoted to full ``clerics.’’ ``Clerics’’ and ``initiates’’ are still ``followers.’’

Be a supportive and listening ear for anyone who needs one. Guide those who follow your patron Deity to stick to their tenets and embody them yourself. Emphasize the cultural touchpoints for your nation that will be developed during workshops for yourself and others.

Your default answer when someone asks for your help should be ``yes.'' Only refuse to help if you have a good \textbf{in character} reason to do so. When it comes to creating and playing out rituals in game, many of your participants in the rituals will \textbf{not know how to participate (IC or OOC).} You should be prepared to coach people ahead of time, and guide them during the ritual. You may consult this sheet at any time, and you can and should incorporate this preparation into your roleplay.

Music is sacred to \cFarmGod{}, and with very few exceptions, followers are only allowed to make music/sing songs/etc., under the guidance of a Cleric, for a specific, religious purpose. Feel free to embrace this. You are always allowed to pull out your phone to play music, look up lyrics, or whatever you need to support your roleplay. You are not however obligated to use music at all; whatever makes you feel comfortable.

\subsection*{Rituals You Can Perform:}
\textbf{All rituals are interruptible.} Items that are consumed are only consumed at the \textbf{end} of a \textbf{completed} ritual; if it is interrupted, the items are not consumed.

You are empowered in and out of character to create these rituals before or during game if that is enjoyable for you. At the end of this sheet is a series of appendices that have \textbf{examples of one possible way} to run each of these rituals. You may use it verbatim, riff on it, or create something entirely new. \textbf{However,} any increases in the time, or material or personnel requirements, to a ritual beyond what is listed here \textbf{must have} buy-in from all players involved before you start.

There are a few, rare circumstances in which someone may have a different or modified version of a ritual they ask you to perform. While it is appropriate to view such requests with some suspicion, it is not completely unheard of, especially in times of crisis when the Clergy of one Patron Deity may need to call upon help from another to make quorum. As you are all under one pantheon (technically 1 larger religion) your rituals are compatible with each other.

Below is the list of rituals you are trained in, along with their basic requirements and limitations. The Ritual to Bless Something or Someone, and The Ritual to Cleanse a Space have no mechanical effects on their own. You can use them for roleplay purposes, and many other game mechanics will call for one, the other, or both.

\textbf{Ritual to Bless Something or Someone in the name of your patron:}
  \begin{enumerate}
    \item Should take no less than 15 seconds, and no more than 1 minute.
    \item Most invoke the name of your Patron Deity.
    \item Requires 1 cleric or initiate, the person being blessed, or the object(s) being blessed and the person who wants you to bless it. 
    \begin{itemize}
      \item You may not bless an object for yourself. Another cleric must do it for you.
      \item You may not bless an unwilling person unless a mechanic specifically allows for it. If you are not sure, ask a GM.
    \end{itemize}
  \end{enumerate}

\textbf{Ritual to Cleanse a Space:}
  \begin{enumerate}
    \item Should take no less than 30 seconds, and no more than 2 minutes.
    \item Requires 1 cleric or initiate and 1 ``\iRitualCandle{}''. The candle must be consumed by the ritual.
    \item Most invoke the name of your Patron Deity at least once.
  \end{enumerate}
   
\textbf{Examining a Relic to Determine Attunement:}\\
This ritual allows you to examine a Relic to determine its Attunement status. Relics can exist in one of 4 known states — they can be attuned to one of the three nations, or they can be unattuned (neutral). To do this:
  \begin{enumerate}
    \item Should take no less than 2 minutes, and no more than 5 minutes.
    \item Requires 1 cleric or initiate, one other person to help, the relic to be examined, and 1 ``\iCrystalLens{}'' (findable in Tier 2 of the Library). The lens must be consumed by the ritual. 
    \item At the end of the ritual, you may open the relic and pull out the paper inside to read the current attunement. Replace the paper immediately, without letting anyone else read it \textbf{(not even your helper).}
  \end{enumerate}
   
\textbf{Reattuning a Relic:}
\begin{enumerate}
    \item Should take no less than 5 minutes, and no more than 10 minutes.
    \item \textbf{The Ley Line Nexus in the Old Wing of the college must be renewed in order for this ritual to be possible.}
    \item Requires 2 clerics (or 1 cleric and 1 initiate), 2 additional people, 1 ``\iRitualCandle{},’’ and 1 ``\iTuningFork{}.'' The candle must be consumed by the ritual.
    \item If the Relic is to be reattuned to the \pTech{}, this ritual must be performed in “The Clockwork Vault” (Library Tier 3). If the Relic is to be reattuned to the \pShip{}, this ritual must be performed in “The Upside Down Grotto” (Library Tier 3). If the Relic is to be reattuned to the \pFarm{}, this ritual must be performed in “The Chamber of the Sun” (Library Tier 3). If the Relic is to be reattuned to another location, you would need to find a mechanic in game to do so.
    \item You must take the relic to a GM at the completion of the ritual who will adjust the attunement of the relic (no one gets to learn what the relic was attuned to before this ritual was conducted.)
  \end{enumerate}
   
\textbf{Inducting a New Devotee to your Patron:}\\
You have the ability to induct a new devotee to your patron. You can only do this with someone who freely consents to this, and whom you are convinced is genuinely committed to the change. In 99\% of cases, this is usually a step along someone’s path to immigrating to your country. As far as you know, everyone in the game ascribes to the Patron Deity of their nation. If you find out otherwise, it is appropriate to be shocked and even scandalized. 

Changing your patron deity, and your home nation, is a big deal. Prepare for this in the following way:
  \begin{enumerate}
    \item Speak at length with the person. Impress upon them the tenants they are about to take on.
    \item Ensure that they have also spoken with a Cleric of their current patron about their decision.
  \end{enumerate}

Once you are convinced of someone’s sincere wish to take up a \cFarmGod{} as their new patron deity:
  \begin{enumerate}
    \item Should take no less than 2 minutes, and no more than 5 minutes.
    \item This ritual must happen in the temple.
    \item Requires: 1 cleric, the person changing to a new patron deity, 1 person who remains a follower of their previous patron, 2 people who already follow the new patron deity.
		\item Send the person to a GM when you are done.
  \end{enumerate}
   
\textbf{Promoting Someone to a Full Cleric:}\\
This ritual is almost exclusively used to promote someone who has studied as an initiate for years to the status of a full Cleric. There are some weird edge cases where someone who is not an initiate may be promoted to a full Cleric. Ask a GM if you are not sure.

Becoming a full Cleric is a huge responsibility. Don’t hesitate to roleplay extensively, asking the candidate serious questions about their commitment to this path. You must be convinced through roleplay that this person is ready for the increase in responsibility. Once you are satisfied, organize the ritual:
  \begin{enumerate}
    \item Should take no less than 5 minutes, and no more than 10 minutes.
    \item This ritual must happen in the temple.
    \begin{itemize}
      \item 2 clerics (at least 1 of whom must be a cleric of the same patron as the initiate)
      \item The initiate becoming a full Cleric.
      \item At least 4 witnesses (a minimum of ½ rounded up must be followers of the patron to which the initiate is pledging).
      \item The ``\iOakStaff{}'' (which should probably be returned to its place in the temple when you are done).
      \item One of the 2 relics traditionally associated with the nation/Patron Deity. Unlike most uses of the relics, this does not require the relic to be attuned. This ritual will not de-attune an attuned relic.
      \item A GM.
    \end{itemize}
    \item The ritual must summon the avatar of the Deity (whom the GM will play). If the avatar is not able to access the mortal plane (e.g. if their mortal animal vessel has been killed), this ritual will fail. 
  \end{enumerate}
   
\textbf{Healing Someone:}\\
You have the ability to heal, using the abilities granted to you by \cFarmGod{}. This is a rare and precious ability.
  \begin{enumerate}
    \item This ritual has a 20 min cool down after you use it (successfully) before it can be used again.
    \item Takes no less than 30 seconds, and no more than 2 minutes.
    \item Requires 1 cleric and the person to be healed.
    \item Requires some musical element (singing/humming/chanting/clapping etc, or you can play an instrument, or a sound recording from your phone.)
    \item This ability can be used to: 
    \begin{itemize}
      \item Restore 1 magical unit to an individual who has somehow given away some of the magical essence, rather than waiting for it to regenerate on its own. You cannot increase someone’s CR past their maximum this way.
      \item Immediately revive an unconscious target.
    \end{itemize}
  \end{enumerate}

\subsection*{Appendix: Ritual Examples}

\textbf{Ritual to Bless Something or Someone in the name of your patron:}
  \begin{enumerate}
    \item Take the object in your hands, or mime taking the other person’s hands in yours. (You may ask for OOC permission and then actually take their hands if you are both comfortable with the physical contact, but the default is no actual contact.)
    \item Say ``I call upon \cFarmGod{}’s light to warm $<$person’s name$>$ or this $<$name of object$>$’’
    \item If you are blessing a person, you can have them respond with ``I accept this blessing''. If you are blessing an object, have whoever asked you to bless the object say this.
  \end{enumerate}

\textbf{Ritual to Cleanse a Space:}
  \begin{enumerate}
    \item Acquire a \iRitualCandle{}.
    \item Stand in the middle of the space you wish to cleanse. Call upon \cFarmGod{} to bring their attention to the space.
    \item Walk slowly around the edge of the space, holding the candle out in front of you. As you walk, you may do so silently, or hum, or chant as you choose.
    \item Return to the center of the circle. Call upon \cFarmGod{} to extend their power to protect the space and those who enter it from undo harm.
  \end{enumerate}
The ritual is now complete. Discard the candle item in the nearest stock; it has been consumed.
   
\textbf{Examining a Relic to Determine Attunement:}
  \begin{enumerate}
    \item Find a ``\iCrystalLens{};'' rumor has it there are several in the Library. 
    \item Spend 5 uninterrupted minutes holding the Relic with your helper without taking any other action (conversation is fine). At the end of the 5 minutes, the Crystal Lens shatters (discard the item card in the nearest stock). 
    \item Open the envelope associated with the Relic and read the paper inside to see where it is currently attuned. \textbf{No one else may look inside the envelope (not even your helper), even if they also possess the ability to examine relics. Only one pair may examine a given Relic at a time.}
  \end{enumerate}
   
\textbf{Reattuning a Relic:}
  \begin{enumerate}
    \item \textbf{The Ley Line Nexus in the Old Wing of the college must be renewed in order for this ritual to be possible.}
    \item If the Relic is to be reattuned to the \pTech{}, this ritual must be performed in “The Clockwork Vault” (Library Tier 3). If the Relic is to be reattuned to the \pShip{}, this ritual must be performed in “The Upside Down Grotto” (Library Tier 3). If the Relic is to be reattuned to the \pFarm{}, this ritual must be performed in “The Chamber of the Sun” (Library Tier 3). If the Relic is to be reattuned to another location, the ritual must be performed \textbf{at that location}.
    \item Gather the following at the required reattunement location:
    \begin{itemize}
      \item A \iTuningFork{} and 1 \iRitualCandle{} (for the cleansing ritual)
      \item 1 additional Cleric or Initiate (besides yourself)
      \item 2 more people
    \end{itemize}
    \item A cleric must lead the ritual.
    \begin{enumerate}
      \item The cleric or initiate not leading the ritual should cleanse the space (see Ritual to Cleanse a Space above) with all of the participants already inside the space.
      \item Arrange the ritual participants in a circle, and place the relic in the middle.
      \item The ritual leader steps \textbf{out} of the circle, and walks slowly around the outside of the circle. They may do this in silence, or hum, sing, chant, etc, as they feel so moved. They will return to their place at the end of the walk.
      \item Each person does this, in turn. Start with the person to the cleric’s \textbf{left.} Each person should circle in the \textbf{opposite} direction that the previous person did.
      \item The ritual leader calls upon the power of the appropriate Patron God to imbue those assembled with the power to change the attunement of the relic at hand.
      \item The ritual leader steps \textbf{into} the circle, touches the tuning fork to the relic, and steps back to their place the circle. They then pass the tuning fork to the person on their \textbf{right.}
      \item Each person around the circle in turn touches the tuning fork to the relic. The tuning fork should end back in the hands of the ritual leader.
    \end{enumerate}
    \item Take the item to a GM. The GM and ritual leader will discuss in secret where the ritual leader intends the relic to be attuned. The GM will then check the current attunement of the relic and adjust the attunement if necessary.
    \begin{itemize}
      \item No one in the ritual gets to learn what the previous attunement was. This ritual cannot be used as a replacement for the ``Examine the Relic'' ability.
    \end{itemize}
  \end{enumerate}
The \iTuningFork{} is not consumed, but is traditionally passed to the other cleric who participated. The \iRitualCandle{} is consumed; discard it to the nearest stock.

   
\textbf{Inducting a New Devotee to your Patron:}\\
Changing your patron deity, and your home nation, is a big deal. Prepare for this in the following way:
  \begin{enumerate}
    \item Speak at length with the person. Impress upon them the tenants they are about to take on.
    \item Ensure that they have also spoken with a Cleric of their current patron about their decision.
  \end{enumerate}
  
Once you are convinced of someone’s sincere wish to take up a \cFarmGod{} as their new patron deity:
  \begin{enumerate}
    \item Assemble the following in the Temple:
    \begin{enumerate}
      \item The inductee
      \item At least 1 witness from the patron the inductee is leaving behind.
      \item At least 3 witnesses from the patron the inductee is joining.
    \end{enumerate}
    \item Arrange the inductee and their 1 witness on one side of the space. Arrange the 3 witnesses from your patron on the other side of the space, with everyone facing each other. The two groups should be at least 6 paces apart.
    \item Begin the ritual by opening your arms to indicate both groups of people. Make a short explanation of what is happening (e.g.: This person wishes to take on \cFarmGod{} as their patron. We are here to witness this.)
    \item Bring everyone into a proper frame of mind for the ritual by having everyone follow you in humming / chanting for at least 15 seconds or until you feel everyone’s attention is present and focused on the task at hand.
    \item Have the inductee and the witness turn to face each other and bow to acknowledge one another. Then have the witness say ``As a representative of $<$Their patron deity$>$. My heart is heavy with this goodbye.’’
    \item The Inductee replies ``May knowing that I go home lighten your burden.’’ The inductee should then turn to face the other 3 witnesses.
    \item The 3 witnesses should say ``We are the representatives of \cFarmGod{}.’’
    \item They must call out together. ``Come home to us.’’ over and over. Each time they say the phrase, the inductee should take 2 steps toward the group.
    \item When the inductee reaches the group, you must say ``\cFarmGod{} welcomes you. May you find comfort, and may you be a credit to us.’’
    \item Send the person to a GM at their earliest convenience in case we need to swap any mechanics for them.
  \end{enumerate}
The ritual is now complete, the person now has your patron deity as theirs. Celebration should ensue.
   
\textbf{Promoting Someone to a Full Cleric:}
  \begin{enumerate}
    \item Assemble the following in the temple:
    \begin{enumerate}
      \item You, the primary cleric sponsor to lead the ritual. You can only promote a cleric of your own patron.
      \item A second cleric from your nation, or  2 clerics from other nations, or 5 non-clerics from your nation.
      \item The initiate to be promoted.
      \item 1 person to be the initiate’s sponsor. This should be someone who can speak about the character’s character.
      \item One relic traditionally associated with your nation (its current attunement does not need to be known).
      \item The ``\iOakStaff{}'' (One is available in every temple across \pEarth{})
      \item At least 5 additional witnesses. This is a \textbf{big} deal that whole communities generally celebrate together.
    \end{enumerate}
    \item Have a cleric cleanse the space. If possible, have someone other than yourself do this.
    \item Organize your audience arrayed around the initiate, either in a semicircle, or a circle.
    \item Ask the initiate: ``Are you ready to embark on the path of being a Cleric of \cFarmGod{}?''
    \item Ask the witnesses: ``are you ready to bear witness to this oath?'' Guide them to respond if necessary.
    \item Summon the relevant avatar by holding up the relic and calling the avatar forth.
    \begin{enumerate}
      \item If the mortal vessel of the avatar is not present on the mortal plane, the ritual will fail at this point (the GM will tell you so.)
    \end{enumerate}
    \item Once the Avatar manifests, prompt the candidate to make a short statement as to their intention to become a cleric. (e.g.: ``I desire to pledge to the path of \cFarmGod{}.'')
    \begin{enumerate}
      \item Audience should respond with ``we witness.''
    \end{enumerate}
    \item Prompt the sponsor to make a short statement or ``speech'' (no more than a few sentences), vouching for the good character of the candidate.
    \begin{enumerate}
      \item Audience should respond with ``we witness.''
    \end{enumerate}
    \item Have a cleric (can be the ritual leader) make a short statement or ``speech'' (no more than a few sentences), vouching that the candidate is properly prepared and devoted.
    \begin{enumerate}
      \item Audience should respond with ``we witness.''
    \end{enumerate}
    \item Have the candidate make a longer statement or ``speech'' (no more than two paragraphs) explaining why they wish to become a Cleric, and what good they believe they can do on behalf of their patron deity.
    \item The Avatar will accept the candidate, and initiate the formal oath taking.
    \begin{enumerate}
      \item Technically it is possible for the avatar to reject a candidate, which is just embarrassing all around, but this happens only rarely, and only when the clerics, the sponsor, or the initiate themselves is dis-ingenious.
      \item Applause is appropriate at this time.
    \end{enumerate}
    \item The Cleric leading the ritual should thank the Avatar and release them.
  \end{enumerate}
Major celebration should ensue! This is a really big and exciting deal.

\textbf{Healing Someone:}\\
This ritual has a 20 minute cool down before you can use it again (if successful). This ability has two possible uses:
  \begin{enumerate}
    \item You can restore 1 magical unit to an individual who has somehow given away some of the magical essence, rather than waiting for it to regenerate on its own. You cannot increase someone’s CR past their maximum this way.
    \item An unconscious target will wake up as soon as the ritual is complete.
  \end{enumerate}

To cast this ritual:
  \begin{enumerate}
    \item Reach your hands out toward the person you wish to heal.
    \item Invoke some musical or rhythmic element for at least 30 seconds. (e.g.: Singing, humming, chanting, clapping, etc. You may use a musical instrument or play music from your phone if you prefer.)
    \item Call out the following: ``\cFarmGod{}, heed my call. See $<$Name of the person you are targeting$>$ here before me. Lend us your healing warmth.’’ Then, state the effect you wish to have happen out loud (e.g.: Restore their magical energy.) You cannot conceal from anyone listening what your intentions are.
    \item Clap your hands 3 times.
  \end{enumerate}
  
The ritual is now complete. Explain to the person what effect they experience. 
  \begin{enumerate}
    \item If you are trying to restore magical energy to a person who’s CR is already at max, the ritual fails.
    \item If you are trying to return someone to consciousness who is already conscious, the ritual fails.
  \end{enumerate}

If your ritual fails, you may use it again immediately, for the same or another purpose, with no cool down time required.

\end{document}

