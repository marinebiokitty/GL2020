\documentclass[green]{GL2020}

\usepackage{enumitem}
\setlist{nosep}
\parindent=0pt
\begin{document}
\name{\gPreGameSafety{}}

\section*{Summary}
\begin{enumerate}
	\item 1 - 1:15 Orientation
	\item 1:15 - 1:45 Logistics Briefing
	\item 1:50 - 3:20 Safety Briefing and Combat Workshop
	\item 3:25 - 4:25 Workshop by Country
	\item 4:30 - 5:00 Workshop by Cohort
	\item 5:00 - 5:30 Load In
	\item 5:30 - 6:30 Dinner
	\item 7:00 GAME START
\end{enumerate}

\section*{Workshop Orientation}
Welcome to Grand LARP 2022, ``The Children of the Gods.’’ This game has been a long time coming, and we are so excited to finally share it with you. Over the next few hours, we will be conducting a series of workshops and briefings that are important for your safety and enjoyment, and that of your fellow players, during this event. 

I know some of you may think you’ve heard it all before, or that this stuff isn’t important, or feels weird or uncomfortable, but please try to pay attention anyway. You will be held to the standards we discuss in workshops, as well as the ones covered in the rules document.

Before we jump in, I’d like to introduce a few people, starting with myself.
\begin{enumerate}
	\item The present members of the writing team. (Eric wasn't able to join us.)
	
	\begin{enumerate}
		\item Acata (she/her)
		\item Aaron (he/him)
	\end{enumerate}
	\item Our Run Time GM - Jeremy Cole (he/him)
	\item Louis (He/They) and David (He/They) who will be assisting with workshops.
	\item The head of logistics: Kate Hill (She/They) - you’ll hear more from her later.
	\item Retreat Venue Staff
\end{enumerate}

Now that you have faces to go with some of the names that have been emailing you for the past few months, let’s review the plan for the afternoon. After this orientation, we’ll turn the floor over to Kate who will introduce the logistics team and review some important reminders about the space. \emph{$<$Optional$>$ The Staff from the Greater Commision will have a few things to share at that time as well.} After that, we have 3 workshop slots, each a different length. Everyone will be doing the same workshops at the same time, but some will be broken up into smaller groups that reshuffle each time. If you look on the manila envelope that is your character packet, you will see who will be leading each session you are in. 

\textbf{PAUSE FOR QUESTIONS.}

The time is now \emph{X}. We'll be starting the logistics briefing at 1:15 pm. You have about \emph{X} minutes before this workshop starts. If you need to grab something from your car like water, sunscreen, or a folding chair, now is the time. Don't go too far, we do need to start on time.

\section*{Logistics Briefing}
Kate and the logistics team will take care of this part. 

The next workshop is ``Safety, Calibration, and Combat,’’ which we will be doing all together. We will start this workshop in 5 minutes, at \emph{X} time (ideally 1:50 or earlier)

\section*{Safety, Combat, and Calibration Workshop (80-90 min)}
Hello and welcome. Once again my name is X and my pronouns are Y.

Over the next hour we are going to:
\begin{enumerate}
	\item Introduce ourselves and our characters.
	\item Discuss pronouns
	\item Introduce Game Space, and time for discussing mobility considerations.
	\item Review and practice our safety mechanics.
	\item Have space for questions on how calibration and negotiation are used in this game.
	\item review and practice the IC combat mechanic.
\end{enumerate}

\subsection*{Player / Character Introductions (5 min)}
We’re going to go around the circle and introduce \textbf{players}. Please share your name, pronouns, and what state you live in normally. Use the 1st person for this. $<$GM models$>$

We’re going to go around again, this time introducing your \textbf{character}. Please share your character’s name, pronouns, what nation they are from, and whether they are a student, teacher or an advisor. Use the 3rd person for this. $<$GM models$>$

Now is a good time to check that your player name badge is showing in your badge holder, and that behind it is your character name badge.


\subsection*{Pronouns (5 min)}
Pronouns are important, both for players and for characters. We understand that you may not be familiar with or well practiced with all pronouns, but using the correct pronouns for someone is a matter of respect, and falls under the LRS code of conduct. If you notice someone using the wrong pronouns for you or someone else, gently correct them. If it happens repeatedly, get a GM.

\textbf{PAUSE FOR QUESTIONS.}

\subsection*{Introduce Game Space, and time for discussing mobility considerations. (10 min)}
I know it's early in the day, but we are going to introduce the game space now so that we can talk about accessibility, because accessibility is a community concern.

\textbf{``IC Outdoor Spaces"} Garden, Training Field, Graveyard. 

\textbf{``IC Indoor Spaces"} Temple, Maker's Space, Great Hall/Ritual Space, Student Lounge/Bunker 1, Teacher Lounge/Bunker 2, Advisor Lounge/Bunker 3, Old Wing of the College, and the Library.

\textbf{``OOC Spaces''} Player OOC space, GM HQ.

Given the cold and possible wet, some of our attendees may have less mobility than we had all otherwise hoped for. Can I get hands up for people who expect to need to stay in one location for most of game? You can pick where that is going to be, but we recommend someplace that makes sense for your character, so others have an intuition for where to find you, and/or a central game location. E.g.: A cleric may choose to settle in the temple. Let us know where you decide and we'll announce it to everyone. If in doubt, the Great Hall is the hub of everything

For those who are more easily able to traverse from location to location, now is the time to think about these characters (indicate/reintroduce the characters) and come up with 1) reasons to seek them out specifically, 2) reasons to be okay or even better, actively interested, in doing favors for them, like finding and fetching things they might need, and 3) things your character is trying to do this weekend for which the location is flexible.  Pick the locations these players are based in whenever possible. If you want to bring these characters to a particular place in game as part of including these players in your play, let a GM know and we can temporarily transplant locations, aka brnig the game to them.

You the players make this game what it is. We the GMs need your help making sure everyone gets to play.

\textbf{PAUSE FOR QUESTIONS.}

\subsection*{Review and Practice Safety Mechanics: (30 min)}
Safety mechanics are introduced in section 4.3 of the rules document, but we will be going over them, and practicing them. 


\subsubsection*{OOC Hand Gesture}
The most important. \textbf{If you’re not sure, fall back on this one!} Since hopefully none of us are in a state of distress or dysregulation right now, it is important to acknowledge that practice with safety mechanics now can only go so far. So please do your best to internalize the idea that if all else fails, ``put your hand on your head, and find a GM.''

\begin{enumerate}
	\item Demonstrate Usage
	\item \textbf{What is it for:} If you need to talk player to player with someone, because you need to step out to use the restroom or another OOC reason, or because a mechanic says to do so. Do not go OOC just to avoid someone IC.
	\item \textbf{Practice:} break up into groups of 3 or 4. In these groups, we’re going to start a conversation, then somebody is going to use the OOC symbol for one of the 3 reasons: to talk player to player, to step away for an OOC reason, or because a mechanic requires it. We’ll give you 2 minutes.
\end{enumerate}
	
If an in-game mechanic has you go OOC while you are around other characters, you should briefly explain what they see/what is going on. e.g. ``I vanish before your eyes.'' If you are going OOC for an out of game reason, just go. No explanations required.

\subsubsection*{Okay Check in}
\begin{enumerate}
	\item Demonstrate Usage
	\item \textbf{What is it for:} Checking in with someone to make sure the \textbf{player} under the character is okay. If they are not, we want to get them help as soon as possible. Use this mechanic to check in with someone who appears to be in distress (e.g.: crying), who’s demeanor has suddenly changed (e.g.: stopped talking and has a 50-yard stare), or you otherwise want to check in on whether the player under the character is doing okay. If you get a ``thumbs down'' or ``thumbs sideways'' response, pause game play and take a moment out of character to check in with the player. Encourage them to check in with themselves if they need to change anything about current play or the current environment. If you don’t feel able to check in with them at that time, find a GM or send the player to GM HQ.
	\item \textbf{Practice:} Break into 3 circles, one with me, one with Louis, and one with David Neubauer. Think quietly to yourself about what distress feels like to you, physically and emotionally. Consider how that might present in game. For example: crying, freezing up/spacing out, exploding/getting angry. I’m not asking you to embody that right now, but I do need you to think about it. We are giong to use this mechanic to ask the person next to us in the circle. Once the respond, they will ask the next person, and so forth until everyone has had a chance to both ask and respond. You may pick your response from among the 3 options. The last person will ask us.
\end{enumerate}

\subsubsection*{Brake / Largo}
	\begin{enumerate}
		\item Demonstrate Usage
		\item \textbf{What to use it for:} Use this to de-escalate the intensity of an interaction for a player’s safety or comfort. (e.g.:Ask a player to speak more quietly, even though their character is still yelling, or take a few steps back, even if their character is still physically blocking yours in.) This is not a negotiation - the person being asked to ``brake'' should immediately comply for the comfort and safety of their fellow players.
		\item \textbf{However}, and please pay attention here; unlike in some other games, ``Brake’’ cannot be used to change the \textbf{IC situation}. If a mob is out for your character’s blood, you cannot use ``brake’’ to get rid of them. You can use it to ask the \textbf{players} to stop actually shouting at you even if their \textbf{characters} are still shouting.
		\item \textbf{Practice in pairs:} One person is going to get loud, heated, or physically close to the other player. That player should use ``brake.’’ When asked to brake, that player should take it down a notch OOC, and continue. Take a moment to reset, then switch roles.
\end{enumerate}

\subsubsection*{``Game Halt’’}
	\begin{enumerate}
		\item Demonstrate Usage
		\item \textbf{What to use it for:} Used to halt game play in a whole area, either for safety or a mechanic. This call should be used to halt game play to sort out a safety issue such as a player being unable to continue at the moment, or in need of a longer player to player discussion for which ``brake'' is insufficient, \textbf{and} everyone in the scene needs to be involved. If not everyone needs to be involved, just use the OOG symbol and pull whoever you need to aside so the scene can continue without you. This call should also be used for physical danger, for example: repositioning players so no one is at risk of falling down a cliff. This call is also used by some mechanics to pause the game to fetch a GM for something.
		\item \textbf{Practice:} Mingle in a group and chat about whatever. I’m going to tap someone on the shoulder, and once I move off, they will call for a ``Game Halt.’’ everyone should pause their conversations, stop moving, and look toward the person who called the Game Halt. Either that person, or someone they assign, should then come get me as the GM.
	\end{enumerate}
	
\subsubsection*{``Badge Off’’}
	\begin{enumerate}
		\item Demonstrate Usage
		\item \textbf{What to use it for:} Use this to indicate that you are exiting game for the night, or need to take an extended break from game (longer that would be comfortable to maintain the Out of Game Hand Signal). If you are going ``Badge Off'' for a game related reason, feel free to come find a GM if we can be of any help. Players should assume that another player without their badge is out of character for an extended period. You may take no game actions toward someone who isn’t wearing their character badge. If you are looking for their character, assume they cannot be found.
		\item \textbf{Practice:} Again, break up into 3 groups, each group should begin to chat. One by one people should step away from the conversation and take their badge off until everyone has done it.
	\end{enumerate}
	
\subsubsection*{``Open Door’’}	
	\begin{enumerate}
		\item Describe Usage
		\item \textbf{What to use for:} If you need to leave the game space, do so. \textbf{PLEASE} let a GM know that you are leaving so we don’t start a search and rescue operation assuming you are lost in the mountains somewhere.
	\end{enumerate}

\textbf{PAUSE FOR QUESTIONS.}

\subsection*{Calibration and Negotiation (5 min)}
Some of you may come from games where calibration and negotiation are handled very differently than in this game. Details of this are available in section 4.2 of the rules document in case you want to take this opportunity to refresh your memory. Now is our time to ensure that everyone is on the same page about how we are going to use these tools in this game.

The default for this game is \textbf{not} to break character for calibration or negotiation. To the extent possible this game is designed to have already addressed these things in casting, or for the negotiation to be IC, reducing the need to do these things OOC. That said, player safety is paramount, and it may become necessary anyway.

If you wish to do any pre-play calibration at dinner tonight, first ask ``are you open to pre-game calibration with me?'' without further detail.  Remember that other players may not be aware that they have a plot with your character, may not believe calibration is necessary (IC or OOC), or may prefer to go in without expectations. In such a case, respect your fellow players' choices and contact a GM if you want support for yourself.

\textbf{PAUSE FOR QUESTIONS.}

\subsection*{Combat Mechanic (30 min)}
Everyone has a set CR; therefore combat is deterministic. CRs generally range from 3-5 in this game. All combat is magical. We recommend you decide on a flavor of your magic (e.g.: fire magic, plant magic, etc).

Everyone should have a small bag with stones in it. Now is a good time to find that in your character packet. The number of stones should match your CR. Some mechanics in game will require you to surrender one or more of your CR stones. This reduces your current CR. \textbf{Unless you know otherwise,} you may draw stones from the stock back up to your maximum at the next meal.

By default your CR is the same for attacking and defending. Some items may change your CR for one or both.

\textbf{Structure of a Round of Combat:}
\begin{enumerate}
	\item Someone decides to attack someone else.
	\item Attacker announces attack intent (knock out, upstage, or restrain) + CR value up to their max.
	\begin{enumerate}
		\item Knock out = wake up in 5 min. People can freely go through the character’s pockets.
		\item Upstage = bested in combat; no actual harm. - this is commonly used for sparring, exercise, entertainment, etc. If you are bored or uncertain of what to do at any point in the game, challenge someone to an ``upstage'' duel; it’s fun to throw magic at each other.
		\begin{enumerate}
			\item The nobility of the CoS also use it to settle matters of honor. There is generally no need to knock your opponent out. Showing off your superiority is sufficient.
		\end{enumerate}
		\item Restrain = attempt to restrict the movement of your target. After successful attack, requires 2 hands to maintain the restrain.
	\end{enumerate}
	\item Defender may respond with:
	\begin{enumerate}
		\item ``Resist'' if their CR is $>=$ to the CR of the attack.
		\item ``Yield'' if their CR is $<$ CR of Attack, or they wish to allow the attack to succeed anyway. If the defender does not otherwise respond, the default is that they ``yield.''
		\item \textbf{OR} A counter attack. If the defender wishes to make a counterattack instead of resisting, they may (for example hoping for aid from an ally, or that their attacker’s defensive CR is lower than their attacking CR if the attacker is holding a weapon). \textbf{This will count as a ``yield'' for the original attack.} If you want to resist and then counter attack, you need to say ``Resist'' first, and resolve this round. Then launch your own attack in a new round.
	\end{enumerate}
	\item Attacker then restates the attack in a cinematic way. E.g. ``I throw a fireball at you.''
	\item The Defender restates the resist, yield, or counter attack in a cinematic way e.g.: ``I create a bit of ice on the ground to help me slide out of the way in time.'' Or ``I try to throw up a wall of ice, but the fireball melts right through it.'' Keep these to 1 sentence, we aren’t narrating entire anime combat sequences.
		\item \textbf{Note that ``yield'' is a CHANGE from how darkwater combat is handled in some other games. It is necessary to clarify for the cinematic combat.}
\end{enumerate}

\textbf{Do not pause for questions here; many questions people will have are about to be answered. Ask people to hold their questions till we've run through some examples.}

\subsubsection*{1v1 Combat}
\begin{enumerate}
	\item Model CR 3 attacks CR 2.
	\item Model CR 3 ``pulls the punch'' on CR 2.
	\item Model CR 2 attacks CR 3.
	\item Practice at least one of these combinations in pairs.
	\item Observe that if the CRs are the same, you could beat on each other all day with neither gaining the upper hand.
\end{enumerate}

\subsubsection*{2v1 Combat}
Being outnumbered in this system is really bad, and we're about to demonstrate why.
\begin{enumerate}
	\item Model an attack with assistance (CR2 + CR 2 vs CR 3)
	\begin{enumerate}
		\item CR 3 can either yield, or counterattack \textbf{one person}
	\end{enumerate}
	\item You cannot assist defends. We know this might feel funny, but it is a balance thing, and to help avoid combat dragging on. The system is designed to be as efficient as possible, so we can go back to roleplaying as soon as possible.
	\item Practice in 3s.
	\item Observe that having allies is valuable.
\end{enumerate}

Some items or signs in game may describe objects or entities with a CR. These things may be attacked (typically using a ``knock-out’’ attack) as long as you meet all listed requirements (e.g.: can only be attacked at night). They will not attack back. - if you aren’t sure if something can be attacked in this way, ask a GM.

\textbf{PAUSE FOR QUESTIONS.}

\subsubsection*{Waylay}
Anyone may use waylay as a combat attack. It is a 5-count; other mechanics may also use the same technique but may have a different count.
\begin{enumerate}
	\item Model a waylay: Peace sign, count ``waylay-1\ldots waylay-5'' out loud (emphasize that normally the player should do this silently, and you are talking out loud for demonstration purposes). Then say ``Waylay <intent>.''
	\item If you notice a waylay, make it obvious (e.g.: say ``I notice you''). You may choose not to notice a waylay on yourself or someone else. (e.g. deliberately distracting someone so your friend can sneak up on them.)
	\item If the waylay count completes, the attack just works. No CR involved. You don’t get to retroactively notice a waylay.
	\item Practice in groups of 3. Pick your option:
	\begin{enumerate}
		\item Have someone succeed.
		\item Have the target notice.
		\item Have the other person notice.
	\end{enumerate}
\end{enumerate}

\textbf{PAUSE FOR QUESTIONS.}

\subsubsection*{Killing Blows}
Killing blows are disallowed until 6 pm on Saturday. If you do succeed in killing someone, go the GM HQ and look for a sign about ``\sMurderConsequences{}''. If your character is killed, your GMs have plans, so don’t despair. \textbf{Play your body for 10 minutes!} Then go to the GM HQ and look for a sign about ``\sMurdered{}''.

\textbf{Requires:}
\begin{enumerate}
	\item At least 1 free hand.
	\item A 10 count. Each phrase should take at least 1 sec to say, and be loud enough for anyone in the room to hear.
	\item Anyone who is conscious can stop you (including your target if they are only restrained.)
	\item If you are trying to KB something represented by an object or a sign, you need to\textbf{ knock it out first}.
	\item Model a KB. - interrupted vs. complete.
\end{enumerate}

\textbf{PAUSE FOR QUESTIONS.}

Alright, up next is a workshop by Nation. Check your manila envelope for who is going to run your workshop.

\end{document}
