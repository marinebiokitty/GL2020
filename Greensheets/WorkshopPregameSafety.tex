\documentclass[green]{GL2020}

\usepackage{enumitem}
\setlist{nosep}
\parindent=0pt
\begin{document}
\name{\gPreGameSafety{}}

\section*{Summary}
\begin{enumerate}
	\item 1 - 1:15 Orientation
	\item 1:15 - 1:30 Logistics Briefing
	\item 1:30-2:15-Safety Briefing and Combat Workshop
	\item 2:20-2:55 Calibration Workshop
	\item 3:00-3:40 Workshop by Cohort
	\item 3:50-4:50 Workshop by Country
	\item 5:00 - 6:00 Load In and Costuming
	\item 6:00 - 6:50 Dinner
	\item 7:00 GAME START
\end{enumerate}

\section*{Workshop Orientation(15 minutes)}
Welcome to, ``The Children of the Gods.’’ This game ran in 2022 in California, and we are excited to bring this experience to a new audience.Over the next four hours, we will be conducting a series of workshops and briefings that are important for your safety and enjoyment, and that of your fellow players, during this event. 

You all should have read the rules doc, and there may be some repeats, especially around safety, but we ask you to engage and listen. You will be held to the standards we discuss in workshops, as well as the ones covered in the rules document.

Before we jump in, I’d like to introduce a few people, starting with myself.
\begin{enumerate}
	\item The present members of the writing team, who are also your run time GMs. 	
	\begin{enumerate}
		\item Acata (she/her)
		\item Aaron (he/him)
		\item Olivia (she/they)
		\item Kit (they/she)
	\end{enumerate}
	\item Our NPC who will also be helping with workshops, Samara Metzler (she/her).  Samara will be playing Principal Treva.  
	\item Retreat Venue Staff
\end{enumerate}

Now that you have faces to go with some of the names that have been emailing you for the past few months, let’s review the plan for the afternoon. After this orientation, we'll review some important reminders about the space. \emph{$<$Optional$>$ The Staff from the center will have a few things to share at that time as well.} After that, we have 4 workshop slots, each a different length. Everyone will be doing the same workshops at the same time, but some will be broken up into smaller groups that reshuffle each time. If you look on the manila envelope that is your character packet, you will see who will be leading each session you are in. 

\textbf{PAUSE FOR QUESTIONS.}

The time is now \emph{X}. We'll be starting the logistics briefing at 1:15 pm. You have about \emph{X} minutes before this workshop starts. If you need to grab something from your car like water, sunscreen, or a folding chair, now is the time. Don't go too far, we do need to start on time.

\section*{Logistics Briefing{{15 minutes}}}

Welcome to Silver Lake Retreat Center!  We have a few important bits of information to share with you about logistics. In your manilla folder you will find a map of the site on your weekend schedule.  Please pull it out while we go over some of the spaces. 

\subsection*{Reminders of game spaces!}
We are using 3 buildings: The Cedars (where we are now), 1st floor and basement of The Glen, and the Retreat Center.
\begin{itemize}
	\item The Cedars: The Great Hall, The Ritual Space, The Training Field, The Maker's Space, and The Garden.
	\item Glen Main Floor: Advisor Lounge/Bunker 3, The Temple, and a room the GMs are using to store game materials. Please Do NOT go in there unless directed to do so.
	\item Glen Basement: The Student Lounge/Bunker 1, The Library, and The Old Wing of the College.
	\item Retreat Center: The Teacher Lounge/Bunker 2, The Graveyard, and GM HQ! We will try to always have a GM, NPC, or a walkie talkie you can use to reach us at HQ.
	\item The other group that is here this weekend is using ``The Lodge.'' Please be respectful of that group and leave them alone. If one of them has a question about what we are doing, you can send them to a GM.
\end{itemize}

For some mechanics, it also matters what in game spaces are Indoors vs Outdoors. The relevant mechanics will remind you whether the space is indoors, but for your reference now: \\
\textbf{``IC Outdoor Spaces''} Garden, Training Field, Graveyard. \\
\textbf{``IC Indoor Spaces''} Temple, Maker's Space, Great Hall/Ritual Space, Student Lounge/Bunker 1, Teacher Lounge/Bunker 2, Advisor Lounge/Bunker 3, Old Wing of the College, and the Library. ``B'' in the maps stands for ``Bunker.''\\
\textbf{``OOC Spaces''} GM HQ, everyone's sleeping spaces.

We also know that there may be some players who have mobility concerns.  These players may choose to establish themselves primarily in one location.  Is there anyone who would like to do this? If so, let us all know where you plan to station yourself. For everyone else, please remember to bring these players into the game.

All play spaces are accessible from the first floor or via mobility lift (in the case of Glen).  Some rooms in Cedar on the second floor are not accessible, but these are all sleeping spaces that are out of character.

\subsection*{Sleeping:}

In terms of sleeping, you should be able to find your linen packet here in this space if you requested one.  If you did not get your linen packet during registration please grab it after the workshops.  Please do not take one if you did not order one.  We have a sheet of players who did that we will be checking.  

Also remember that sleeping spaces are Out of Character!  We know there is a lot of space in some of these, but we do want to make sure everyone has a space to decompress, sleep, and take a break.

Quiet time at the site is at 10:00 PM.  Not coincidentally, this is also a game off for your GMs. This means that at this point, we will not be available to answer questions or guide scenes.  We have gotten permission from the site to have quite RP inside and lights on after 10 PM.  Please be respectful of those sleeping, fellow guests and nearby residents.  Do not do RP outside after this time, and do not have loud scenes after this time. If you absolutely must scream at someone, wait till the morning.

\subsection*{Food and Drink:}

Meals are on the schedule and all will take place in the Cedar’s dining hall.  The other group present this weekend is handling their own meals, so we probably won't cross paths with them in the dining hall.  Also note that dinner on Saturday will also be out of character.  You don’t have to take any meals in character if you do not wish.  We just ask you to indicate that you are OOC (which we will be showing you how to do later).  Please let a GM know if you have any problems during meals.

Smoking, drinking of alcohol and use of other substances is not allowed at this site.  Please refrain from using them. This site is cool. We'd like to not get kicked out in the middle of our event, and even maybe have the opportunity to come back in the future.

We also have some serious allergies among our players, specifically to eggplant and chocolate.  Please do not consume these in public spaces.   Hot water, tea, water, and coffee will be available throughout the day in the dining hall.

Please clean up after yourself if you do eat, especially in your room!

\subsection*{Other logistics:}

No weapons, not even fake ones, are allowed on site.  We did get special permission for one prop and for wands, but please do not use anything that could be looked on as a weapon.

We are in a rural situation.  There is wifi available, but cell reception is not strong and the wifi is limited.  Please be considerate of others and limit your use of wifi so your GMs especially can use it if needed!  Wifi is available in the cedars and the retreat center in terms of the spaces we are using.  Glen does not have wifi.

You are allowed to use the buildings of The Retreat Center, Cedars and Glen.  We will also have access to the dining hall at meal times, which is located in Cedars.  We also can take advantage of outdoor spaces, though we ask you to generally stay around our area to be able to be found if another player or GM is looking for you!  This is a very large site.

The next workshop is ``Safety, Calibration, and Combat,’’ and you will see what group you are in by looking in your manilla folder.  We will start this workshop in 5 minutes, at \emph{X} time (ideally 1:50 or earlier)

\subsection*{Safety and combat splits}
Splits for Safety And Combat Workshop:
\begin{enumerate}
	\item Leaders: Aaron, Kate | \cChupStudent{}, \cChupInventor{}, \cChupSecond{}, \cChupAvenger{}, \cChupLeader{}, \cHeadScientist{}, \cAssistantScientist{}, \cTechStar{},\cPresident{}, \cBunker{}, \cLibrarian{}, \cInterpol{}
	\item Leaders: Olivia, Samara | \cDisney{}, \cHedonist{}, \cScholarship{}, \cBeetle{}, \cAntiChup{}, \cInitiate{}, \cWarlordDaughter{}, \cPirate{}, \cFlowPriest{}, \cEbbPriest{}, \cPrince{}, \cWildCard{}
	\item Leaders: Acata, Andrea | \cAmbition{}. \cHeir{}, \cDiplomat{}, \cEvil{}, \cPirateChild{}, \cMusic{}, \cAdopted{}, \cLibAssist{}, \cCurse{}, \cHistory{}, \cJuniorStatesman{}, \cEthics{}
\end{enumerate}

\section*{Safety and combat(45 minutes)}
Hello and welcome. Once again my name is X and my pronouns are Y.

Over the next hour we are going to:
\begin{enumerate}
	\item Introduce ourselves and our characters.
	\item Discuss pronouns
	\item Review our safety mechanics.
	\item practice the IC combat mechanic.
\end{enumerate}

\subsection*{Player / Character Introductions (5 min)}
We’re going to go around the circle and introduce \textbf{players}. Please share your name, pronouns, and what state you live in normally. Use the 1st person for this. $<$GM models$>$

We’re going to go around again, this time introducing your \textbf{character}. Please share your character’s name, pronouns, what nation they are from, and whether they are a student, teacher or an advisor. Use the 3rd person for this. $<$GM models$>$

Now is a good time to check that your player name badge is showing in your badge holder, and that behind it is your character name badge.


\subsection*{Pronouns (5 min)}
Pronouns are important, both for players and for characters. We understand that you may not be familiar with or well practiced with all pronouns, but using the correct pronouns for someone is a matter of respect, and falls under the LRS code of conduct. If you notice someone using the wrong pronouns for you or someone else, gently correct them. If it happens repeatedly, get a GM.

\textbf{PAUSE FOR QUESTIONS.}

\subsection*{Review Safety Mechanics: (10 min)}
Safety mechanics are introduced in the rules document, but we will be reviewing them briefly here, and have time for questions.

\subsubsection*{OOC Hand Gesture}
The most important. \textbf{If you’re not sure, fall back on this one!} Since hopefully none of us are in a state of distress or dysregulation right now, it is important to acknowledge that practice with safety mechanics now can only go so far. So please do your best to internalize the idea that if all else fails, ``put your hand on your head, and find a GM.''

\begin{enumerate}
	\item Demonstrate Usage
	\item \textbf{What is it for:} If you need to talk player to player with someone, because you need to step out to use the restroom or another OOC reason, or because a mechanic says to do so. Do not go OOC just to avoid someone IC.
\end{enumerate}	
	
If an in-game mechanic has you go OOC while you are around other characters, you should briefly explain what they see/what is going on. e.g. ``I vanish before your eyes.'' If you are going OOC for an out of game reason, just go. No explanations required.

\subsubsection*{Okay Check in}
\begin{enumerate}
	\item Demonstrate Usage
	\item \textbf{What is it for:} Checking in with someone to make sure the \textbf{player} under the character is okay. If they are not, we want to get them help as soon as possible. Use this mechanic to check in with someone who appears to be in distress (e.g.: crying), who’s demeanor has suddenly changed (e.g.: stopped talking and has a 50-yard stare), or you otherwise want to check in on whether the player under the character is doing okay. If you get a ``thumbs down'' or ``thumbs sideways'' response, pause game play and take a moment out of character to check in with the player. Encourage them to check in with themselves if they need to change anything about current play or the current environment. If you don’t feel able to check in with them at that time, find a GM, NPC,  or send the player to GM HQ.
\end{enumerate}

\subsubsection*{``Game Halt’’}
	\begin{enumerate}
		\item Demonstrate Usage
		\item \textbf{What to use it for:} Used to halt game play in a whole area, either for safety or a mechanic. This call should be used to halt game play to sort out a safety issue such as a player being unable to continue at the moment, or in need of a longer player to player discussion for which \textbf{and} everyone in the scene needs to be involved. This call should also be used for physical danger, for example: repositioning players so no one is at risk of falling down a cliff. This call is also used by some mechanics to pause the game to fetch a GM for something.
\end{enumerate}
	
\subsubsection*{``Badge Off’’}
	\begin{enumerate}
		\item Demonstrate Usage
		\item \textbf{What to use it for:} Use this to indicate that you are exiting the game for the night, or need to take an extended break from game). If you are going ``Badge Off'' for a game or safety related reason, feel free to come find a GM if we can be of any help. Players should assume that another player without their badge is out of character. You may take no game actions toward someone who isn’t wearing their character badge. If you are looking for their character, assume they cannot be found.
\end{enumerate}		
	
\subsubsection*{``Open Door’’}	
	\begin{enumerate}
		\item Describe Usage
		\item \textbf{What to use for:} If you need to leave the game space, do so. \textbf{PLEASE} let a GM know that you are leaving so we don’t start a search and rescue operation assuming you are lost in the woods somewhere.
	\end{enumerate}


\textbf{PAUSE FOR QUESTIONS.}

\subsection*{Combat Mechanic (20 min)}

You should have already read over combat rules.  What we are going to do right now is do a quick demonstration, and then have folks have time to ask questions and if we have any remaining time, practice.

Example to demonstrate:


\subsubsection*{1v1 Combat}
\begin{enumerate}
	\item Model CR 3 attacks CR 2.
	\item Model CR 3 ``pulls the punch'' on CR 2.
	\item Model CR 2 attacks CR 3.
	\item Model counter attack.
	
	\item Observe that if the CRs are the same, you could beat on each other all day with neither gaining the upper hand.
\end{enumerate}

\subsubsection*{2v1 Combat}
Being outnumbered in this system is really bad, and we're about to demonstrate why.
\begin{enumerate}
	\item Model an attack with assistance (CR2 + CR 2 vs CR 3)
	\begin{enumerate}
		\item CR 3 has the \textbf{option} to counterattack \textbf{one person}, but it is not mandatory.
	\end{enumerate}
	\item You cannot assist defends. We know this might feel funny, but it is a balance thing, and to help avoid combat dragging on. The system is designed to be as efficient as possible, so we can go back to roleplaying as soon as possible.
	\item Observe that having allies is valuable.
\end{enumerate}


Now we are going to talk a bit about cinematic combat.  After we resolve combat, we now can describe how it looks.  Now we are going to describe what the above combat may look like described in a cinematic style.
\begin{enumerate}
	\item One GM describes their initial attack “I pull plants up from the ground and wrap them around your legs”
	\item the second GM describes how they assist “I see what GM one is doing, manipulate the ties on GM 2’s clothing and use that to bind them further, intertwined with the plants”
	\item The defending GM sees this coming and sends a counterattack “I lash out with a gust of wind aimed right at GM 1’s one’s head, attempting to push them back and knock them out”
	\item GM 1 falls unconscious and GM three is wrapped up, leaving GM 2 to deal with a messy situation.
\end{enumerate}

Some items or signs in game may describe objects or entities with a CR. These things may be attacked (typically using a ``knock-out’’ attack) as long as you meet all listed requirements (e.g.: can only be attacked at night). They will not attack back. - if you aren’t sure if something can be attacked in this way, ask a GM.


\textbf{PAUSE FOR QUESTIONS.}

\subsubsection*{Waylay}
Anyone may use waylay as a combat attack. It is a 5-count; other mechanics may also use the same technique but may have a different count.
\begin{enumerate}
	\item Model a waylay:llama sign, count ``waylay-1\ldots{} waylay-5'' out loud (emphasize that normally the player should do this silently, and you are talking out loud for demonstration purposes). Then say ``Waylay $<$intent$>$.''
	\item have another gm state “I notice you” and stop the attack
	\item Model a waylay that is not noticed and goes through.  Finish the count to five and then perform a knockout (without CR).

\end{enumerate}

\textbf{PAUSE FOR QUESTIONS.}

\subsubsection*{Killing Blows}
Killing blows are disallowed until 9 pm on Saturday. If you do succeed in killing someone, go to GM HQ and look for a sign about ``\sMurderConsequences{}''. If your character is killed, your GMs have plans, so don’t despair. \textbf{Play your body for 10 minutes!} Then go to the GM HQ and look for a sign about ``\sMurdered{}''.

\begin{enumerate}
	\item Model a KB. - interrupted vs. complete.
\end{enumerate}

\textbf{PAUSE FOR QUESTIONS.}

Alright, up next is the calibration workshop. Please check your manilla folders for your GM.
\end{document}

