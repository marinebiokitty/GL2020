\documentclass[green]{GL2020}
\usepackage{enumitem}
\setlist{nosep}
\parindent=0pt
\begin{document}
\name{\gBMRun{}}

The \pGoaties{} have established a Black Market that crosses national borders on \pEarth{}. While the market pre-dates the war, the demand for it has redoubled since the war started. None of the established nations \textbf{like} the Black Market, but the \pTech{} is particularly vicious in its desire to stamp the market out.

You are one of the ``Market Managers’’ for this Black Market. This means you get to broker trades and purchases along with the other Market Manager: \cChupInventor{\full}. You are technically the one in charge of the market.

People find you, or your agent \cLibAssist{\full} mostly by asking around clandestinely. Once someone expresses an interest in acquiring something on the Black Market:

\begin{enumerate}
  \item If you don’t have it on hand, contact \cChupInventor{} to see about sourcing it.
  \item You set the price, sometimes in consultation with \cChupInventor{}. A 1-to-1 trade for another item (especially curses), or charging up a piece of Tass are common prices, but you may also issue “Black Market credits,” or accept any other form of payment at your discretion. Bigger, or more dangerous items should fetch a higher price. Your contacts outside of game may decline to source something for you if they feel the price is too cheap for the risk involved.
  \item Once the buyer agrees to the price, and you have the item on hand, you can make the exchange yourself, through your agent, or via arranging some sort of a drop off.
\end{enumerate}

Hand offs should be done as discreetly as possible. Remember, your goal is to avoid anyone outside of the \pGoaties{} knowing \textbf{you} are the Black Market lynchpin. But also, if people don’t use the black market, the \pGoaties{} lose a major source of funding and power.  
\end{document}

