\documentclass[green]{GL2020}
\parindent=0pt
\begin{document}
\name{\gBMRun{}}

The Followers of Genesis have established a black market that crosses national borders on Cengea. While the market pre-dates the war, the demand for it has redoubled since the war started. None of the established nations \textbf{like} the black market, but the \pTech{} is particularly vicious in its desire to stamp the market out.

You are one of the``Market Managers’’ for this black market. This means you get to broker trades and purchases along with the other Market Manager: \cChupInventor{\full}. You are technically the one in charge of the market, while \cChupInventor{} just supplies your needs, but you probably don’t want to alienate \cChupInventor{\them} by not at least pretending to consult \cChupInventor{\them} sometimes.

People find you, or your agent \cLibAssist{} mostly by asking around clandestinely. Once someone expresses an interest in acquiring something on the black market:

\begin{enumerate}
  \item If you don’t have it on hand, contact \cChupInventor{} to see about sourcing it.
  \item You set the price, sometimes in consultation with \cChupInventor{}. A 1 to 1 trade for another item (especially curses), or charging up a piece of Tass are common prices, but you may also issue “black market credits,” or accept any other form of payment at your discretion. Bigger, or more dangerous items should fetch a higher price.
  \item Once the buyer agrees to the price, and you have the item on hand, you can make the exchange yourself, through your agent, or via arranging some sort of a drop off.
\end{enumerate}

Hand offs should be done as discreetly as possible. - remember your goal is to avoid anyone outside of the \cFoG{} knowing *you* are the black market lynchpin. But also, if people don’t use the black market, it is not obvious to the FoG how indispensible you are.

\end{document}
