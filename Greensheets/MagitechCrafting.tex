\documentclass[green]{GL2020}
\usepackage{enumitem}
\setlist{nosep}
\parindent=0pt
\begin{document}
\name{\gMagitechCrafting{}}

As an experienced engineer, you have the ability to craft magitech devices, \textbf{if you have access} to the necessary parts. Crafting prototypes of brand new technology is beyond the scope of game, as it takes more than a single weekend, but you can craft small devices which already exist in-world or slight modifications thereof. 

\begin{enumerate}
  \item \textbf{Talk to a GM.} GMs must approve your construction as something that you have both the time and resources to construct.
  \item You will need to acquire at least 1 unit of ``\iMagitechParts{}’’ to build your device. (The GM will tell you if you need more than 1) 
  \item You can only use this ability in the Maker’s Space. 
  \item Spend \textbf{10 minutes} roleplaying building the item. 
  \item Discard the ``\iMagitechParts{}’’ to the nearest stock. 
  \item \textbf{See a GM} for the item card for the item you just constructed. We will probably be in GM HQ making it for you.
\end{enumerate}

Once you have constructed an item of technology, it does not automatically function, since everything is powered by magic. Unless it is ``imbued’’ with magical energy, \textbf{it will not function.}

You have the ability to \textbf{temporarily} imbue any device \textbf{you} have built: \textbf{1x/day}, you may sacrifice 1 of your CR tokens to imbue your device for \textbf{30 minutes.} For thta duration, the device will work for you as agreed upon between you and the GMs.

If you want to permanently imbue one of your devices so it will work in perpetuity, and can be used by anyone, you will need to convince a priest of Kero to do so for you.


\end{document}
