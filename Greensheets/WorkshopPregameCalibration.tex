\documentclass[green]{GL2020}

\usepackage{enumitem}
\setlist{nosep}
\parindent=0pt
\begin{document}
\name{\gCalibrationWS{}}

\section*{(30 minutes)}


Hello everyone.  Once again my name is X and my pronouns are Y.  

We are going to be breaking you all into groups.  Please follow your assigned GM:

These are the groups

Group One: 
\cPrince{}, \cPirate{}, \cFlowPriest{}, \cCurse{}, \cChupLeader{}, \cAdopted{}, \cMusic{}

Group Two: 
\cLibAssist{}, \cEvil{}, \cHistory{}, \cChupSecond{}, \cAntiChup{}, \cScholarship{}, \cChupInventor{}

Group three: 
\cBeetle{}, \cEthics{}, \cJuniorStatesman{}, \cInterpol{}, \cLibrarian{}, \cChupAvenger{}, \cHeadScientist{}, \cAssistantScientist{}

Group four: 
\cDiplomat{}, \cHeir{}, \cAmbition{}, \cChupStudent{}, \cPresident{}, \cBunker{}, \cInitiate{}

Group Five: 
\cTechStar{}, \cPirateChild{}, \cDisney{}, \cWildCard{}, \cWarlordDaughter{}, \cHedonist{}, \cEbbPriest{}

Over the next half hour, we are going to:
\begin{enumerate}
	\item go over how calibration, especially OOC calibration happens in this game
	\item break into five groups to do some sharing, at a high level, of OOC expectations 
and any jitters or concerns. 
\end{enumerate}


\subsection*{Calibration {5 minutes}}

Some of you may come from games where calibration and negotiation are handled very differently than in this game. Details of this are available in section 4.2 of the rules document in case you want to take this opportunity to refresh your memory. Now is our time to ensure that everyone is on the same page about how we are going to use these tools in this game.

The default for this game is \textbf{not} to break character for calibration or negotiation. To the extent possible this game is designed to have already addressed these things in casting, or for the negotiation to be IC, reducing the need to do these things OOC. That said, player safety is paramount, and it may become necessary anyway.

While calibration for how you want play to proceed in terms of steering and plot should follow the rules above, we encourage stepping out of the game if you find yourself unable to continue with a scene or a situation for your own personal safety and wellbeing. If you feel comfortable, speaking with a player out of game about what you need for the scene or content to continue is an important form of OOC collaboration that is often challenging to do while still in character. In these situations, please step out of character and take the time you need. If what you need is some time out of game, please let a GM know so we can help other characters who may be looking for you. We especially recommend doing this if you need to de-escalate a scene that has become too intense for any reason. If you need help re-engaging or have other concerns, please do come see a GM or your NPC, \c{Principal}{}.

If you wish to do any pre-play calibration today at dinner tonight, first ask ``are you open to pre-game calibration with me?'' without further detail.  Remember that other players may not be aware that they have a plot with your character, may not believe calibration is necessary (IC or OOC), or may prefer to go in without expectations. In such a case, respect your fellow players' choices and contact a GM if you want support for yourself.

\subsection*{Sharing our Hopes and fears as players {25 minutes}}

We are now going to break up into predetermined groups. When you hear your name called, please follow the GM who has named you and they will take you to a designated space.  

A bit about what we will be doing in these groups: We are going to be sharing OOC expectations for the game (this can also be called “hopes”) and also worries or things we would like to avoid in game. Again this is us as players, not our characters.  

This game is a game that does have limited transparency and some of these fears or concerns may be related to character secrets.  We ask that you do not share these secrets with others during this section. Instead we are aiming for high level sharing.  

For example, let’s say my character has a romantic relationship I really want to pursue that is not yet established. I may state this as ``I really hope for some deep interpersonal roleplay, be it friendship or even romance''  instead of ``I am really excited to have my character open up about their romantic feelings for one of their best friends.'' 

For the second part, your character may be a spy  and you are nervous about keeping this secret long enough for you to have fun with it.  You may say ``I am nervous that i won’t be able to keep secrets enough in this game and will ruin things for others.''  Not ``I am secretly a spy and I worry that I won’t be able to keep this a secret for long enough to do fun spy things.''

I hope this makes sense!  Do the best you can.  If you don’t want to share anything for either question, that is also fine.  Please just say ``Pass'' when it becomes your turn.  We do ask you to still stay in the workshop to listen, even if you do not want to share.

We are doing this for a few reasons.  One is that it can be helpful for us to know what others are looking forward to and what they are nervous about or do not want to see in their play so we can help steer play in character in ways that will be fun for the other players we are collaborating with.  It also can help us when we are checking in with others out of character.  Finally, sharing some of our nerves can be useful so we all know that we are not alone in pre-game jitters.  We are all here as friends and therefore we want to build a supportive space where we know we are in this together as we work to build a story.

With that, in a circle, take turns asking:
\begin{enumerate}
\item What are your expectations and/or hopes for this game?  How can we as fellow players support you?
\item What are your fears and/or worries about this game?  How can we as fellow players help you with them? 
\end{enumerate}

Please do not take longer than a minute or so per question.  We want to make sure everyone has a chance to express themselves.

After this you will heading to a workshop by cohort, meaning student, teacher or advisor.  Please look in your manilla folder for where you should go next.

\end{document}