\documentclass[green]{GL2020}
\usepackage{enumitem}
\setlist{nosep}
\parindent=0pt
\begin{document}
\name{\gUnsuspectingBirdie{}}

Being able to see and hear through the eyes of another, without them ever knowing, would be a tremendous boon for anyone. More-so for anyone involved in the spy business. \cAntiChup{}, \cScholarship{}, and \cPrince{} are conspiring to make this into a reality through the use of a ``\iMagicTag{}'', and a ``\iScryingAmulet{}''. The plan involves 3 parts. Parts 1 and 2 may be accomplished in any order, but both parts must be completed before part 3 can be done.

There is only 1 amulet, and only 1 tag, so this process can only be completed on 1 person, even though you have discussed 2 possible targets: \cPirate{\full} or \cInitiate{\full}. You will need to agree on who to actually target with this mechanic. For the other possible target, you can investigate if any blackmail can be uncovered, or fabricated, to hold over them. They will obviously know they are being blackmailed to pass information, but even a reluctant spy can yield key information.

These preparations can be made in either order:
\begin{enumerate}
  \item Place the ``\iMagicTag{}'' on the target.
  \begin{enumerate}
    \item \cScholarship{} starts with 1 magical tag in their possession. \textbf{If you are in possession of this greensheet, the trigger does not apply to you.} Do not open the magical tag. That trigger is for whoever you choose to target with this mechanic and thus give the tag to.
    \item It is expected that these tags will be delivered secretly (e.g. via a pickpocket mechanic) so as to avoid arousing suspicion around where they came from.
  \end{enumerate}
  \item Activate the ``\iScryingAmulet{}''.
  \begin{enumerate}
     \item Get a ``\iBlindness{}'' made. Conveniently \cPrince{} is a cursemaker.
     \item Once the curse has been brewed, have a cleric (conveniently  \cAntiChup{} is a cleric) pour the curse over the amulet. Then discard the curse item card to the nearest stock; it has been consumed.
  \end{enumerate}
\end{enumerate}

You can only attune the Amulet to the target after the tag is placed on the target and the Amulet has been activated:
\begin{enumerate}
  \item Place the activated ``\iScryingAmulet{}'' in the target’s possession for a minimum of 2 minutes. (you can knock them out and place it on them, pick pocket it onto them and then retrieve it, or something else of your own devising.)
  \item Open the \iScryingAmulet{} item envelope and write the name of the character that the amulet is attuned to on the paper inside. Replace the paper inside the amulet.
\end{enumerate}

The magic in the tag and the amulet will take 72 hours to settle (meaning this is a strictly \textbf{post game} effect, no matter how quickly you accomplish it). Once it does, whoever holds the amulet will be able to see and hear through the attuned target’s eyes at any time, without the target knowing. Make sure you agree on who is leaving the School with the Amulet in their possession.

The tagged person will almost certainly notice the tag at some point, and will have the opportunity to try to remove it before the end of the weekend. If they succeed in removing the tag before they leave the \pSc{} (whether or not you have attuned the amulet to them), your plan fails. So there may be some strategy in when you choose to place the tag.


\end{document}
