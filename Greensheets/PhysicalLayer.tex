\documentclass[green]{GL2020}
\usepackage{enumitem}
\setlist{nosep}
\parindent=0pt
\begin{document}
\name{\gPhysicalLayer{}}

Repairing the physical protections on the bunker will involve actual building.

Ask a GM for the lego set. Your goal is to construct a protective structure. You may collaborate with as many people as you like.

\textbf{Important rules:}
\begin{enumerate}
  \item The lego set cannot be stolen - Be extra careful about dropping them on the ground!
  \item The lego set must always stay within 3 ZoC of a damaged bunker (you will not be able to transport it from one game location to another, so make sure you pick a good place IC and OOC to set up to build. We recommend a place with a table.)
\end{enumerate}

\textbf{Requirements for the Structure:}
\begin{enumerate}
  \item The structure must include a fully encircling wall, and be able to hold 13 of the smallest pieces inside it, without stacking them.
  \item 1 structure per broken bunker; each structure must be unique. This means if more than one bunker is broken, you will need more than 1 structure design.
  \item The structure must use exactly \textbf{4 colors} - 1 for each patron deity.
  \item \cBunker{\full} and the GMs must sign off on the finished product.
\end{enumerate}

\textbf{Once \cBunker{} and the GMs sign off on the lego construction,} we will update the capacity of the appropriate bunker(s) by \textbf{+3.}  If this is the \textbf{last} repair, (the bunker can now protect 13 people) you or the GM may remove the ``Damaged Bunker’’ sign to reveal the one below it.

\end{document}
