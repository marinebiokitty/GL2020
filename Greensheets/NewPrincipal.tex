\documentclass[green]{GL2020}
\parindent=0pt
\begin{document}
\name{\gAssassinateWarlord{}}

Being the immortal principal of the \pSchool{} is not all fun and games. You are technically in charge, which mostly makes everything that goes wrong somehow your responsibility, along with the impossible task of trying to keep the peace this weekend.

\cPrincipal{\\full} is ready to retire after nearly 200 years as the \pSc{}’s principal. \cPrincipal{\They} \cPrincipal{have} done a truly admirable job over \cPrincipal{\their} tenure. Whoever steps into the role will have some big shoes to fill. The two candidates currently under consideration for this are: \cMusic{\full} and \cBeetle{\full}. Both candidates should give serious consideration to the implications of becoming immortal and taking on these responsibilities.

To help judge how prepared each of the candidates are, \cPrincipal{} has assigned them each one task that would normally fall to the principal to complete:
\begin{itemize}
  \item \cMusic{} is in charge of the Ceremony of Excellence.
  \item \cBeetle{} is in charge of the ritual to renew the Ley Lines.
\end{itemize}

\cPrincipal{} knows what goes into each of these tasks \emph{(OOC: they have the instructional greensheets),} and will be assessing both whether the tasks are completed, and the grace, skill, and tact with which they were accomplished. Other things may also sway \cPrincipal{}’s decision, such as recommendations from your fellow teachers or the students. And while it might be possible to sabotage your competition in a variety of ways, if you get caught at it, it will very likely reflect poorly on your bid for the position.

\cPrincipal{} is expected to make a decision about who to pass the title of Principal off to on \textbf{Sunday.} A small ceremony should be convened to commemorate the occasion at some point before lunch. We recommend the following:
\begin{itemize}
  \item An audience of at least 6 teachers.
  \item A speech from \cPrincipal{} explaining \cPrincipal{\their} decision. (approx 2-3 min)
  \item An acceptance speech from the winning candidate. (approx 2-3 min)
  \item A cleric’s blessing of the winning candidate.
  \item The passing of a symbol of the office.
\end{itemize}

\emph{(OOC: You are welcome to make this process more elaborate / include more ritual elements if it will enhance the play experience of all involved.)}

Once this ritual is complete, the role of principal has been transferred. A GM will facilitate the transfer of any abilities/greensheets/etc that the new Principal now has access to, and the former principal no longer does. The former principal should probably pass on any secret knowledge about the school that they have.



\end{document}
