\documentclass[green]{GL2020}
\usepackage{enumitem}
\setlist{nosep}
\parindent=0pt
\begin{document}
\name{\gNewPrincipal{}}

Being the immortal principal of the \pSchool{} is not all fun and games. You are technically in charge, which mostly makes everything that goes wrong somehow your responsibility, along with the impossible task of trying to keep the peace this weekend.

\cPrincipal{\full} is ready to retire after nearly 200 years as the \pSc{}’s principal. \cPrincipal{\They} \cPrincipal{\have} done a truly admirable job over \cPrincipal{\their} tenure. Whoever steps into the role will have some big shoes to fill. The three candidates currently under consideration for this are: \cMusic{\full}, \cBeetle{\full}, and \cChupSecond{\full}. 

All three candidates should give serious consideration to the implications of becoming immortal and taking on these responsibilities. Candidates are highly encouraged to talk through this with friends, colleagues, family, and partners this weekend. This is your last chance to back out.

To help judge how prepared each of the candidates are, \cPrincipal{} has assigned them each one task that would normally fall to the principal to complete:
\begin{itemize}
  \item \cMusic{} is in charge of the Ceremony of Excellence.
  \item \cBeetle{} is in charge of the Ritual to Renew the Ley Lines.
  \item \cChupSecond{} is in charge of the execution of the Ritual to Control the Storm.
\end{itemize}

\cPrincipal{} knows what goes into each of these tasks \emph{(OOC: they have the instructional greensheets),} and will be assessing both whether the tasks are completed, and the grace, skill, and tact with which they were accomplished. Other things may also help sway \cPrincipal{}’s decision, such as recommendations from your fellow teachers or the students. And while it might be possible to sabotage your competition in a variety of ways, if you get caught at it, it will very likely reflect poorly on your bid for the position.

\cPrincipal{} is expected to make a decision about who to pass the title of Principal off to on \textbf{Sunday.} A small ceremony should be convened to commemorate the occasion. It is recommended but not required that this happen at some point before lunch. \textbf{Note:} This could potentially put \cChupSecond{} at a disadvantage since the actual execution of the Ritual to Control the Storm happens \textbf{after} lunch. \cPrincipal{} should consequently weigh the effort \cChupSecond{} puts into the lead up (getting the ritual written, having people plan, discuss, and practice, etc) heavier than prep for the other tasks. 

We recommend the following for the Ceremony to name the new Principal:
\begin{itemize}
  \item An audience of at least 6 teachers.
  \item A speech from \cPrincipal{} explaining \cPrincipal{\their} decision. (approx 2-3 min)
  \item An acceptance speech from the winning candidate. (approx 2-3 min)
  \item A cleric’s blessing of the winning candidate.
  \item The passing of a symbol of the office.
\end{itemize}

\emph{(OOC: You are welcome to make this process more elaborate / include more ritual elements if it will enhance the play experience of all involved.)}

Once this ritual is complete, the role of principal has been transferred. A GM will facilitate the transfer of any abilities/greensheets/etc that the new Principal now has access to, and the former principal no longer does. \cPrincipal{} should probably pass on any secret knowledge about the school that they have. 

For any mechanics that call for it, \cPrincipal{} can still participate as a teacher. If something were to happen to the new principal, the responsibilities default back to \cPrincipal{} who may be forced to choose a new successor. If something were to happen to \cPrincipal{}, well, that would be a trick in and of itself given that \cPrincipal{} is immortal until \cPrincipal{\they} transfer\cPrincipal{\verbs} that to \cPrincipal{\their} successor. (Being immortal means that any mechanic that would otherwise kill your character automatically fails. You are not immune to any other mechanics or their effects.)

\end{document}

