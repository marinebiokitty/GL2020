\documentclass[green]{GL2020}
\usepackage{enumitem}
\setlist{nosep}
\parindent=0pt
\begin{document}
\name{\gFoGCleric{}}

This greensheet describes the rituals associated with being a Cleric of Genesis. You may need some of these rituals in support of the various goals of the FoG. Technically you could perform these rituals for other people too, but that's probably not a good idea.

Everyone who claims a Patron Deity (or, in this case, \cGenesis{}) is a ``follower’’ of that Deity. A subset of ``followers’’ undertake religious studies, becoming ``initiates.’’ ``Initiates’’ who complete the required years of study can apply to be promoted to full ``clerics.’’ ``Clerics’’ and ``initiates’’ are still ``followers.’’

\subsection*{Rituals You Can Perform:}
\textbf{All rituals are interruptible.} Items that are consumed are \textbf{only consumed }at the \textbf{end} of a \textbf{completed ritual}; if it is interrupted, the items are not consumed. Some of these rituals are identical to the rituals available to other clerics. A few (denoted by an asterisk ($*$) are modified versions, or something unique to Clerics of \cGenesis{}. Be careful who you let see you doing them. Honestly, be careful who you let see you doing any of this; you aren't supposed to know or be able to perform even the rituals that are the same as other clerics.

You are empowered in and out of character to create these rituals before or during game if that is enjoyable for you. At the end of this sheet is a series of appendices that have \textbf{examples of one possible way} to run each of these rituals. You may use it verbatim, riff on it, or create something entirely new. \textbf{However,} any increases in the time, or material or personnel requirements, to a ritual beyond what is listed here \textbf{must have} buy-in from all players involved before you start.

Below is the list of rituals you are trained in, along with their basic requirements and limitations. The Ritual to Bless Something or Someone, and The Ritual to Cleanse a Space have no mechanical effects on their own. You can use them for roleplay purposes, and many other game mechanics will call for one, the other, or both. You may \textbf{not} want to reveal that you can do these to just anyone though. It will raise a \textbf{lot} of suspicion as to how you know how to do this.

\textbf{Ritual to Bless Something or Someone in the name of your patron:}
  \begin{enumerate}
    \item Should take no less than 15 seconds, and no more than 1 minute.
    \item Most invoke the name of your Patron Deity.
    \item Requires 1 cleric or initiate, the person being blessed, or the object(s) being blessed \textbf{and} the person who wants you to bless it. 
    \begin{itemize}
      \item You may not bless an object for yourself. Another cleric must do it for you.
      \item You may not bless an unwilling person unless a mechanic specifically allows for it. If you are not sure, ask a GM.
    \end{itemize}
  \end{enumerate}

\textbf{Ritual to Cleanse a Space:}
  \begin{enumerate}
    \item Should take no less than 30 seconds, and no more than 2 minutes.
    \item Requires 1 cleric or initiate and 1 ``\iRitualCandle{}''. The candle must be consumed by the ritual.
    \item Most invoke the name of your Patron Deity at least once.
  \end{enumerate}
   
\textbf{Examining a Relic to Determine Attunement:}$*$\\
This ritual allows you to examine a Relic to determine its Attunement status. Relics can exist in one of 4 commonly known states - they can be attuned to one of the three nations, or they can be unattuned (neutral), or, unbeknownst to anyone outside of the Followers, they may be attuned to the School. To do this:
  \begin{enumerate}
    \item Should take no less than 2 minutes, and no more than 5 minutes.
    \item Requires 1 cleric or initiate, one other person to help, the relic to be examined, and 1     \item Requires 1 cleric or initiate, one other person to help, the relic to be examined, and 1 ``\iCrystalLens{}'' (findable in Tier 2 of the Library). The lens must be consumed by the ritual. 
    \item This ritual may not occur in the temple.
    \item At the end of the ritual, you may open the relic and pull out the paper inside to read the current attunement. Replace the paper immediately, without letting anyone else read it \textbf{(not even your helper(s)).}
  \end{enumerate}
   
\textbf{Reattuning a Relic to the School:}$*$\\
NOTE: Relics attuned to the School will count as 8 votes, not 6. Further, such attunement will circumvent the normal pedestal restrictions. A Relic attuned to the school will not be rejected, no matter the restriction.
  \begin{enumerate}
    \item Should take no less than 2 minutes, and no more than 5 minutes.
    \item This ritual may happen anywhere \textbf{except} the temple.
    \item Requires 1 cleric, 2 other \pGoaties{}, 1 ``\iRitualCandle{}'', and 1 ``\iGlassVial{}'' filled with water.  Both items must be consumed by the ritual.
    \item You must take the relic to a GM at the completion of the ritual who will adjust the attunement of the relic (no one gets to learn what the relic was attuned to before this ritual was conducted.)
  \end{enumerate}
   
\textbf{Inducting a New Devotee to your Patron:}$*$\\
You have the ability to induct a new devotee to \cGenesis{}. You must be convinced through roleplay that this person is committed to the change, and understands the consequences, and how crucial secrecy is to your mission. \textbf{The last thing you need this weekend is to induct someone, reveal your secrets, and have them immediately bail on you, outing you to the authorities.} But you need new recruits for your plans to work, so it's a risk you’ll have to take - just make sure it’s calculated.
  \begin{enumerate}
    \item Should take no less than 2 minutes, and no more than 5 minutes.
    \item This ritual may happen anywhere \textbf{except} the temple.
    \item Requires: 1 cleric, the person accepting \cGenesis{} as their new patron deity, and 1 other person who already follow \cGenesis{}.
	\item \textbf{When the ritual is complete; It is VERY IMPORTANT you send the person to a GM for a few mechanics.}
  \end{enumerate}
  
\textbf{Consecrating a new Avenger:}$*$\\
This ritual is used to consecrate someone as an Avenger of Genesis, able to kill other people without being punished with amnesia; instead, each time they kill, someone in their life will forget they exist. The clerics performing the ritual should determine through role play that the person to be consecrated is fully devoted to the cause and understands the sacrifice they are making: to be slowly forgotten by the world.

Avengers are a \textbf{crucial} part of the \pGoaties{} strategy over all. Even if you do not have immediate plans to have anyone killed this weekend, having access to an Avenger \textbf{just in case} is important. Avengers are your trump cards. That said, you can only consecrate \textbf{one} new avenger this weekend. Further, someone \textbf{cannot be both} a Cleric of \cGenesis{} and an Avenger of \cGenesis{}

  \begin{enumerate}
    \item Should take no less than 5 minutes, and no more than 10 minutes.
    \item This ritual may happen anywhere \textbf{except} the temple.
    \item Requires:
    \begin{enumerate}
    	\item 2 Clerics of Genesis.
			\item The person to be consecrated as an Avenger.
			\item \textbf{Either} the Avatar of Genesis (if it has been summoned) \textbf{or} the \textbf{attuned} \iHorseshoe{}. If the \iHorseshoe{} of Genesis is used, it becomes de-attuned at the end of the ritual.
    \end{enumerate}
	\item Once the ritual is complete, have the new Avenger change their ``V-Score'' to 2.
 \end{enumerate}
   
\textbf{Nullifying a Bad Luck Curse:}$*$\\
Using this ritual will raise serious questions as to how you came by this ability.

\begin{enumerate}
    \item This ritual has a 20 min cool down after you use it (successfully) before it can be used again.
    \item Takes no less than 30 seconds, and no more than 2 minutes.
    \item Requires 2 FoG members (you count as 1), and the person afflicted with the bad luck curse.
    \item The ritual \textbf{must} have both FoG invoke Genesis’ name directly as their Patron. There is no way to execute this ritual without the person who’s curse you are removing learning who your patron deity is.
		\item This ritual \textbf{must} have the effect of knocking the person out. When they wake up, their bad luck curse is gone \emph{(OOC: tell them to discard their ``Bad Luck Curse’’ greensheet to the nearest stock.)} If you incorporate the ``\iHorseshoe{}’’ into the ritual, the person will \textbf{not} be knocked out.
  \end{enumerate}

\subsection*{Appendix: Ritual Examples}

\textbf{Ritual to Bless Something or Someone in the name of your patron:}
  \begin{enumerate}
    \item Take the object in your hands, or mime taking the other person’s hands in yours. (you may ask for OOC permission and then actually take their hands if you are both comfortable with the physical contact, but the default is no actual contact.)
    \item Say ``I call upon \cGenesis{}’s luck to favor $<$person’s name$>$ or this $<$ name of object$>$’’
    \item If you are blessing a person, you can have them respond with ``I accept this blessing''. If you are blessing an object, have whoever asked you to bless the object say this.
  \end{enumerate}

\textbf{Ritual to Cleanse a Space:}
  \begin{enumerate}
    \item Acquire a \iRitualCandle{}.
    \item Stand in the middle of the space you wish to cleanse. Call upon \cGenesis{} to bring their attention to the space.
    \item Walk slowly around the edge of the space, holding the candle out in front of you. As you walk, you may do so silently, or hum, or chant as you choose.
    \item Return to the center of the circle. Call upon \cGenesis{} to extend their power to protect the space and those who enter it from undo harm.
  \end{enumerate}
The ritual is now complete. Discard the candle item in the nearest stock; it has been consumed
   
\textbf{Examining a Relic to Determine Attunement:}$*$
  \begin{enumerate}
    \item Find a ``\iGlassVial{}''and fill it with water.
		\item Find someone to help you.
    \item \textbf{This step MAY NOT be done inside the temple.} Spend *2 uninterrupted minutes* roleplaying examining the Relic with your helper, without taking any other action (conversation is fine). At the end of the 2 minutes, empty the vial of water over the relic (discard the item card in the nearest stock). 
    \item Open the envelope associated with the Relic and read the paper inside to see where it is currently attuned. \textbf{No one else may look inside the envelope (not even your helper), even if they also possess the ability to examine relics. Only one pair may examine a given Relic at a time.}
  \end{enumerate}
   
\textbf{Reattuning a Relic:}$*$
  \begin{enumerate}
    \item Gather the following anywhere \textbf{except} the temple. - somewhere secret is probably best; you really don’t want to be caught doing this.
    \begin{itemize}
      \item A \iGlassVial{} filled with water and 1 \iRitualCandle{}
      \item at least 2 additional \pGoaties{}
      \item The Relic you want to attune to the School.
    \end{itemize}
    \item Place the Relic in the middle, and arrange the participants in a circle around it. 
		\item Use stomping, clapping, or pounding to create a rhythm. You can set everyone in the circle to do the same thing, or assign different people to different sounds/actions. Do this for at least \textbf{30 seconds} to bring everyone participating in the ritual into sync.
		\item Pass the \iRitualCandle{} around the circle to the \textbf{left}. While the candle is being passed, invoke \cGenesis{}: ``\cGenesis{}, we guide the storm to enact your reform.’’
		\item Once the candle returns to you, place it on top of the relic.
		\item Hold the vial of water out in front of you. finish the invocation by saying``Through luck and skill, we impose our will.’’
		\item Pass the vial of water to the \textbf{right}. Each person must finish the invocation by repeating ``Through luck and skill, we impose our will.’
		\item Once the vial returns to you, empty it over top of the relic / candle. The candle and the vial are consumed (return them to the nearest stock at your earliest convenience).
		\item Take the item to a GM. The GM will check the current alignment of the relic and adjust the alignment if necessary to be attuned to the \pSc{}.
    \begin{itemize}
      \item No one in the ritual gets to learn what the previous attunement was. This ritual cannot be used as a replacement for the ``Examine the Relic'' ability.			
    \end{itemize}
  \end{enumerate}

   
\textbf{Inducting a New Devotee to your Patron:}$*$\\
To induct a new devotee to Genesis:
\begin{enumerate}
  \item Assemble the following anywhere \textbf{except} the Temple - somewhere secret is probably best; you really don’t want to be caught doing this.
  \begin{enumerate}
    \item The inductee
    \item At least 1 witness who already follows Genesis.
  \end{enumerate}
  \item Arrange the inductee and the witness(es) on opposite sides of the space. The inductee should face out, away from the witness(es). The witnesses should face in, toward the inductee.
  \item Begin the ritual by opening your arms to indicate both (groups of) people. Make a short explanation of what is happening (e.g.: This person wishes to take on Genesis as their patron. We are here to witness this.)
  \item Bring everyone into a proper frame of mind for the ritual by having everyone follow you in humming / chanting for at least 15 seconds or until you feel everyone’s attention is present and focused on the task at hand.
  \item The inductee should say ``I forsake $<$Insert their current Patron Deity$>$’’ and then turn around to face the witness(es).
  \item The witness(es) should say ``I(We) represent Genesis.’’ and then call out together. ``Come home to us.’’ over and over. Each time they say the phrase, the inductee should take 2 steps toward the group.
  \item When the inductee reaches the group, you must say ``Genesis welcomes you. May you find comfort, and may you be a credit to us.’’
  \item When the ritual is complete; send the person to a GM at the first convenient opportunity for a few mechanics.
\end{enumerate}

The ritual is now complete, the person now has your patron deity as theirs. Quiet celebration should ensue.

\textbf{Consecrating a New Avenger:}$*$\\
\begin{enumerate}
  \item Assemble the following anywhere \textbf{except} the Temple - somewhere secret is probably best; you really don’t want to be caught doing this.
  \begin{enumerate}
    \item 2 Clerics of \cGenesis{}; choose 1 to be the ritual leader.
    \item The person to be consecrated as an Avenger.
    \item the Avatar of Genesis (if it has been summoned) \textbf{or} the \textbf{attuned} \iHorseshoe{}.
    \item A GM
  \end{enumerate}
  \item Have a cleric cleanse the space. 
  \item Have one cleric stand in front of the person to be consecrated, and the other behind them.
  \item Ask the person to be consecrated the following questions, giving them time to answer each in turn: 
  \begin{enumerate}
    \item ``Are you ready to embark on the path of vengeance in the name of Genesis?''
    \item ``Do you understand that the path you are about to choose is a lonesome one, that though your deeds will be great, you will pass from this world unsung and forgotten by all?''
    \item If their answers are acceptable, you may proceed.
  \end{enumerate}
  \item Summon the presence of Genesis by holding up the \iHorseshoe{} or the Avatar and calling forth Genesis to bear witness.
  \item Prompt the one to be consecrated to make a short statement as to their intention to become an Avenger. (e.g.: ``I desire to pledge to the path of righteous bloodshed in the name of Genesis.'')
  \begin{enumerate}
    \item The clerics should reply with something like, ``we witness.''
  \end{enumerate}
  \item Have a cleric (can be the ritual leader) make a short statement or “speech” (no more than a few sentences), vouching that the candidate is properly prepared and devoted.
  \begin{enumerate}
    \item The other cleric should reply with something like, ``I witness.''
  \end{enumerate}
  \item Have the one to be consecrated make a longer statement or “speech” (no more than two paragraphs) explaining *why* they wish to become an Avenger, and what good they believe they can do on behalf of Genesis.
  \item If Genesis accepts the new Avenger, the Avatar (played by the GM) will say so, or the \iHorseshoe{} will flash with brilliant light \textbf{and become de-attuned.}
  \begin{enumerate}
    \item Technically it is possible for Genesis to reject a candidate, which is just embarrassing all around, but this happens only rarely, and only when someone involved in the ritual is disingenuous.
    \item To de-atune the relic, open the item envelope, remove the paper inside it, and discard to the nearest stock \textbf{without reading it.}
  \end{enumerate}
  \item The Cleric leading the ritual should thank Genesis, welcome the new Avenger, and conclude the ritual.  
\end{enumerate}
   
\textbf{Nullifying a Bad Luck Curse:}$*$\\
This ritual has a 20 minute cool down before you can use it again (if successful). Luck, both good and bad, falls under the domain of Genesis. As Genesis has grown in power thanks to the expansion of their followers, the abilities they can grant has grown as well. You have the ability to nullify a ``Bad Luck Curse’’ as created by a cursemaker from the Children of the Sun. This is something commonly believed to be impossible - the only way to end such a curse is to fulfill the release conditions (usually accomplishing a particular task, or going to a particular location). \textbf{Using this ritual will raise serious questions as to how you came by this ability.}

To nullify a Bad Luck Curse:
\begin{enumerate}
  \item Gather the following anywhere \textbf{except} the temple: yourself, a second Follower of Genesis, and the person with the Bad Luck Curse.
  \item Invoke Genesis' name directly. Call on them to claim dominion over their purview of luck and undo this curse. Have the other FoG member do the same. -- There is no way to do this without the person you are helping learning that both you and your helper follow Genesis and not one of the Patron Gods.
  \item Mime touching the person’s forehead, then the backs of both of their hands, and then both of their palms, with your four-leaf clover. (if you get OOC permission first, you may do this for real)
  \item Have the other Follower of Genesis knock the person out. (you should probably warn the person that this is part of the ritual, and ask them not to resist.)
	\item \emph{OOC: Tell the person that when they wake up, the curse has been nullified, and that they should destroy or disregard the ``\gBadLuckCurse{}’’ greensheet once they wake up.}
\end{enumerate}
  
\end{document}

