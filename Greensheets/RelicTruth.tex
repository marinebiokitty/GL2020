\documentclass[green]{GL2020}
\usepackage{enumitem}
\setlist{nosep}
\parindent=0pt
\begin{document}
\name{\gTruthRelic{}}

You may use this Relic to compel truthful answers from a target.

\textbf{Number of Uses: U; Relic must be attuned somewhere.}

To use this relic to compel truth:
\begin{enumerate}
  \item The ``\iLariat{}'' \textbf{MUST} be attuned to someplace in order to use this ability.
  \begin{enumerate}
    \item Check the item-envelope to see if there is a piece of paper inside it (do not read the paper; do not even pull it out of the envelope if you can avoid it).
    \item If there is \textbf{no} paper, this ability \textbf{cannot} be used until the Relic is re-attuned to someplace.
    \item Using this ability will cause the Relic to become de-attuned.
  \end{enumerate}
  \item You must make a successful ``Restrain’’ attack against the target, and declare that you are using the \iLariat{}. This attack follows normal combat rules.
   \begin{enumerate}
    \item The Lariat does not change your CR, but you may accept assistance as per usual combat.
    \item The person holding the \iLariat{} is ``the wielder.’’
    \item The person the attack is directed at is ``the target.’’ They will \textbf{not} actually become restrained; you will have to explain this to them.
  \end{enumerate}
  \item \textbf{If the attack is successful,} start a \textbf{5 minute} timer. The wielder may ask the target up to \textbf{3 questions} during this time.
  \begin{enumerate}
    \item Either when the target has answered the 3 questions, or at the end of the 5 minutes (whichever is shorter), the effect ends. In other words, you can’t spend 30 minutes debating what questions to ask after successfully activating this effect.
    \item The questions must be direct, simple and straightforward. You cannot ask compound questions like: ``who are you working for and what is your ultimate goal?’’ You may ask these as two separate questions if you like. A good rule of thumb is that your question may only have 1 ``information gathering’’ word (who, what, when, where, why, how), either explicitly or implicitly.
  \end{enumerate}
  \item The target must answer these questions as truthfully and thoroughly as they are able.
    \begin{enumerate}
    \item The magic of the lariat is such that no amount of shouting will cause the target to be unable to hear the question, or the wielder unable to hear the answer. It might affect other people’s ability to hear what is said, but the target and the wielder should feel free to step away from loud noises, or lean close together in order to hear each other.
    \item The effect of the relic is locked to the target and the wielder at the time of the attack. If the wielder loses or gives up control of the \iLariat{}, the effect ends immediately, even if the wielder recovers the lariat before the 5 minute timer is up.
    \item The target is not compelled to answer questions from anyone other than the wielder, truthfully or otherwise, but neither do such questions count against the 3.
  \end{enumerate}
  \item At the end of the 5 minute timer, once the 3 questions are answered, or if the effect is ended by the wielder losing control of the Lariat:
  \begin{enumerate}
    \item Open the item-envelope for the ``\iPitcher{},'' take out the piece of paper inside without reading it, and discard it to the nearest stock or give it to the nearest GM.
    \item \textbf{DO NOT} read what the piece of paper said.
  \end{enumerate}
\end{enumerate}

\end{document}
