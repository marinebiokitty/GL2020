\documentclass[green]{GL2020}
\usepackage{enumitem}
\setlist{nosep}
\parindent=0pt
\begin{document}
\name{\gDeathRelic{}}

\textbf{Number of Uses: U; Relic must be attuned somewhere.}

In order to use the \iScythe{} for something, it must be attuned to someplace. Using the effects listed will cause it to deattune:
\begin{enumerate}
  \item Check if the ``\iScythe{}'' is attuned to someplace before beginning.
  \begin{enumerate}
    \item Check the item-envelope to see if there is a piece of paper inside it (do not read the paper; do not even pull it out of the envelope if you can avoid it).
    \item If there is \textbf{no} paper, this ability \textbf{cannot} be used until the Relic is re-attuned to someplace.
  \end{enumerate}
  \item Deattune the relic after you you use it successfully:
  \begin{enumerate}
    \item Open the item-envelope for the \iScythe{}, take out the piece of paper inside without reading it, and discard it to the nearest stock or give it to the nearest GM. DO NOT read what the piece of paper said.
    \item Only the Killing Blow ability can ``fail'' by being interrupted. In all other uses, the relic is always deattuned afterwards.
    \end{enumerate}
\end{enumerate}

You may use this relic to increase your CR for 1 attack by +4. To do this, hold the \iScythe{}'' in your hand, and make your attack with the increased CR (e.g. Your CR is normally 2, but you can make a ``Knock Out 6’’ attack.) You may take assists on the attack, per usual. Afterwards, deattune the relic as described above.

You may use this relic to initiate an attack against an Avatar of one of the Deities should they appear. To do this, hold the scythe in your hand, and make your attack on the Avatar. You may take assists on the attack, per usual. Afterwards, deattune the relic as described above. The relic is deattuned, even if the attack is successfully resisted.

You may use this Relic to murder someone without suffering the amnesia that is usually a hallmark of such a taboo. \textbf{To do this}:
\begin{enumerate}
  \item \textbf{It must be AFTER 9 pm on Saturday.}
  \item The character must be unconscious.
  \item Hold the \iScythe{} in both hands and hold it out at arms length as you perform the 10-count (``Killing blow 1,’’ ``Killing blow 2’’... ``Killing blow 10’’). This mechanic \textbf{can be interrupted,} just like any other n-count.
  \item If you succeed in killing your target, do not go to the GM HQ and interact with the ``You killed someone. Now what?’’ sign. The relic shields your identity such that the deities cannot perceive who enacted the murder. It offers no protection against other humans seeing you do this, or figuring out it was you.
  \item If you succeed in killing your target, deattune the relic as described above.
\end{enumerate}

You may use two or even all three of these effects simultaneously in a single attack, but the relic will become deattuned after that attack. 

\end{document}

