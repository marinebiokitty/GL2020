\documentclass[green]{GL2020}
\usepackage{enumitem}
\setlist{nosep}
\parindent=0pt
\begin{document}
\name{\gWRInitiate{}}

This greensheet describes the subset of rituals associated with being a Cleric that initiates know. Characters will call on you to perform these throughout the weekend. 

\subsection*{General Guidelines to Playing a Cleric Character:}
Everyone who claims a Patron Deity is a ``follower’’ of that Deity. A subset of ``followers’’ undertake religious studies, becoming ``initiates.’’ ``Initiates’’ who complete the required years of study can apply to be promoted to full ``clerics.’’ ``Clerics’’ and ``initiates’’ are still ``followers.’’

Be a supportive and listening ear for anyone who needs one. Guide those who follow your patron Deity to stick to their tenants and embody them yourself. Emphasize the cultural touchpoints for your nation that will be developed during workshops for yourself and others.

Your default answer when someone asks for your help should be ``yes.'' Only refuse to help if you have a good \textbf{in character} reason to do so. When it comes to creating and playing out rituals in game, many of your participants in the rituals will \textbf{not know how to participate (IC or OOC).} You should be prepared to coach people ahead of time, and guide them during the ritual. You may consult this sheet at any time, and you can and should incorporate this preparation into your roleplay.

Many players quickly associate the \pShip{} with sea shanties. Feel free to embrace this. You are always allowed to pull out your phone to play music, look up lyrics, or whatever you need to support your roleplay. If sea shanties aren’t your thing, you aren’t obligated to use them! Make up your own take, or don’t use music at all; whatever makes you feel comfortable.

\subsection*{Rituals You Can Perform:}
\textbf{All rituals are interruptible.} Items that are consumed are only consumed at the \textbf{end} of a \textbf{completed} ritual; if it is interrupted, the items are not consumed.

You are empowered in and out of character to create these rituals before or during game if that is enjoyable for you. At the end of this sheet is a series of appendices that have \textbf{examples of one possible way} to run each of these rituals. You may use it verbatim, riff on it, or create something entirely new. \textbf{However,} any increases in the time, or material or personnel requirements, to a ritual beyond what is listed here \textbf{must have} buy-in from all players involved before you start.

There are a few, rare circumstances in which someone may have a different or modified version of a ritual they ask you to perform. While it is appropriate to view such requests with some suspicion, it is not completely unheard of, especially in times of crisis when the clerichood of one Deity may need to call upon help from another to make quorum.

Below is the list of rituals you are trained in, along with their basic requirements and limitations. The Ritual to Bless Something or Someone, and The Ritual to Cleanse a Space have no mechanical effects on their own. You can use them for roleplay purposes, and many other game mechanics will call for one, the other, or both.

\textbf{Ritual to Bless Something or Someone in the name of your patron:}
  \begin{enumerate}
    \item Should take no less than 15 seconds, and no more than 1 minute.
    \item Most invoke the name of your Patron Deity.
    \item Requires 1 cleric or initiate, the person being blessed, or the object(s) being blessed and the person who wants you to bless it. 
    \begin{itemize}
      \item You may not bless an object for yourself. Another cleric must do it for you.
      \item You may not bless an unwilling person unless a mechanic specifically allows for it. If you are not sure, ask a GM.
    \end{itemize}
  \end{enumerate}

\textbf{Ritual to Cleanse a Space:}
  \begin{enumerate}
    \item Should take no less than 30 seconds, and no more than 2 minutes.
    \item Requires 1 cleric or initiate and 1 ``\iRitualCandle{}''. The candle must be consumed by the ritual.
    \item Most invoke the name of your Patron Deity at least once.
  \end{enumerate}
   
\textbf{Examining a Relic to Determine Attunement:}\\
This ritual allows you to examine a Relic to determine its Attunement status. Relics can exist in one of 4 known states - they can be attuned to one of the three nations, or they can be unattuned (neutral). To do this:
  \begin{enumerate}
    \item Should take no less than 2 minutes, and no more than 5 minutes.
    \item Requires 1 cleric or initiate, one other person to help, the relic to be examined, and 1 ``\iCrystalLens{}'' (findable in Tier 2 of the Library). The lens must be consumed by the ritual. 
    \item At the end of the ritual, you may open the relic and pull out the paper inside to read the current attunement. Replace the paper immediately, without letting anyone else read it \textbf{(not even your helper).}
  \end{enumerate}
   
\subsection*{Appendix: Ritual Examples}

\textbf{Ritual to Bless Something or Someone in the name of your patron:}
  \begin{enumerate}
    \item Take the object in your hands, or mime taking the other person’s hands in yours. (you may ask for OOC permission and then actually take their hands if you are both comfortable with the physical contact, but the default is no actual contact.)
    \item Say ``I call upon $<$\cEbb{} or \cFlow{}$>$ to wash over and purify  $<$person’s name$>$ or this $<$name of object$>$’’
    \item If you are blessing a person, you can have them respond with ``I accept this blessing''. If you are blessing an object, have whoever asked you to bless the object say this.
  \end{enumerate}

\textbf{Ritual to Cleanse a Space:}
  \begin{enumerate}
    \item Acquire a \iRitualCandle{}.
    \item Stand in the middle of the space you wish to cleanse. Call upon \cEbb{} and \cFlow{} to bring their attention to the space.
    \item Walk slowly around the edge of the space, holding the candle out in front of you. As you walk, you may do so silently, or hum, or chant as you choose.
    \item Return to the center of the circle. Call upon \cEbb{} and \cFlow{} to extend their power to protect the space and those who enter it from undo harm.
  \end{enumerate}
The ritual is now complete. Discard the candle item in the nearest stock; it has been consumed
   
\textbf{Examining a Relic to Determine Attunement:}
  \begin{enumerate}
    \item Find a ``\iCrystalLens{};'' rumor has it there are several in the Library. 
    \item Spend 5 uninterrupted minutes holding the Relic with your helper without taking any other action (conversation is fine). At the end of the 5 minutes, the Crystal Lens shatters (discard the item card in the nearest stock). 
    \item Open the envelope associated with the Relic and read the paper inside to see where it is currently attuned. \textbf{No one else may look inside the envelope (not even your helper), even if they also possess the ability to examine relics. Only one pair may examine a given Relic at a time.}
  \end{enumerate}
   
\end{document}

