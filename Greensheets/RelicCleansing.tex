\documentclass[green]{GL2020}
\parindent=0pt
\begin{document}
\name{\gStatusCleanse{}}

You may use this Relic to cleanse illness or status effects from a target. 

\textbf{Number of Uses: U; Relic must be attuned somewhere.}

To use this relic to cleanse a status effect:
\begin{enumerate}
  \item The ``\iChalice{}'' \textbf{MUST} be attuned to someplace in order to use this ability.
  \begin{enumerate}
    \item Check the item-envelope to see if there is a piece of paper inside it (do not read the paper; do not even pull it out of the envelope if you can avoid it).
    \item If there is \textbf{no} paper, this ability \textbf{cannot} be used until the Relic is re-attuned to someplace.
    \item Using this ability successfully will cause the Relic to become de-attuned.
  \end{enumerate}
  \item You must fill the chalice with clean water from someplace in-game.
  \item Hand the chalice to the person and have them drink the water from it.
  \begin{enumerate}
    \item You may compel a restrained person to drink. - The default is to mime this; do not touch another player without obtaining prior consent.
    \item You may not compel an unconscious person to drink (they could choke).
  \end{enumerate}
  \item If they have any illness or status effects (positive or negative) the most recently obtained one is removed.
  \begin{enumerate}
    \item This includes curses and cures. This \textbf{does not} include the ability to restore memories lost due to having murdered someone. See a GM if you aren’t sure if something can be cleansed by this relic.
    \item If the person has no effects that can be cleansed, nothing happens.
    \item If the person has more than one effect active, only the most recently obtained effect is cleansed.
  \end{enumerate}
  \item \textbf{If an effect was cleansed}, open the item-envelope for the ``\iChalice{},'' take out the piece of paper inside without reading it, and discard it to the nearest stock or give it to the nearest GM.
  \begin{enumerate}
    \item \textbf{DO NOT} read what the piece of paper said.
  \end{enumerate}
\end{enumerate}
\end{document}
