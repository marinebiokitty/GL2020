\documentclass[green]{GL2020}
\parindent=0pt
\begin{document}
\name{\gRitualControlStorm{}}
\textbf{This greensheet is publicly viewable by everyone.}

At each Time of Deciding, the students design their own Ritual to Control the Storm. Below follows a list of common components of such rituals, and various considerations you may wish to take into account. \emph{OOC Note: If you have a cool idea for something to do for this ritual that isn’t listed, feel free to do that instead/as well!)}

While anyone can offer advice on designing the ritual, the students have the final say. The ritual \textbf{must} have at least one designated ritual leader who will guide the actual execution of the ritual starting at 1:30 pm on Sunday Afternoon. Having more than 3 ritual leaders is not recommended. \textbf{The ritual leader(s) must write down the plan and give it to the GMs by 1:00 pm on Sunday so that the GMs can make their own preparations for when and how to interrupt with descriptions of what is happening in the environment as the ritual proceeds.}

\textbf{Requirements for your design:}
\begin{itemize}
  \item The relics, their pedestals, and The Storm Seed must be included somehow.
  \item The ritual should take between 10 and 30 minutes to complete.
  \item The ritual should involve as many characters as possible in some active way. Characters can be grouped for this (e.g.: all teachers clap a rhythm.) - participants should get to feel cool doing whatever they are being asked to do.
  \item Be aware of mobility issues and ensure your plans allow for participants to be seated/stationary for part or all of the ritual if they need it, and still be able to participate.
\end{itemize}

\textbf{Components often Employed in this Ritual:}
\begin{itemize}
  \item Positioning the relics and the \sStormSeed{} within the ritual circle.
  \item Having a Grand Procession into the Ritual Space 
  \item Arranging people around/within the ritual circle in specific groups.
  \item One or more clerics blessing the space, the relics, or the participants.
  \item One or more characters making speeches.
  \item The use of ritual candles.
  \item Passing objects around the circle.
  \item Call-and-repeat sequences.
  \item Repetitive vocal (e.g. singing, humming, etc.) and/or physical activities (e.g. clapping, stomping, hand gestures).
  \item Moments of silence / stillness.
\end{itemize}

Depending on exactly how long this ritual takes, characters will have approximately 30 minutes after the conclusion of this ritual before the game ends.
\end{document}
