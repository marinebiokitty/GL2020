\documentclass[green]{GL2020}
\usepackage{enumitem}
\setlist{nosep}
\parindent=0pt
\begin{document}
\name{\gPunishKidnappers{}}

You have been planning to teach your adopted family a lesson for a long time. With \cLibAssist{\full}’s help, you have been working for almost a year to bring this to fruition. This weekend will provide you access to the last few things you need. And then you can hold your adoptive family hostage, just for a few months - until they tell you where your birth family is, and how to contact them, and agree to stop kidnapping children like they did you. You have never been the most religious person, but you just cannot believe in your heart of hearts that \cFarmGod{} \textbf{wants} things to be this way in the \pFarm{}.

Cursing an entire bloodline was not your first choice; you’ve tried for as long as you can remember to convince them through other means. This is your last resort because they are so blinded by their own righteousness that they refuse to admit their folly. It feels like a particularly poetic solution though, since curses are generally considered beneath the nobility. So you, a common-born magic user, will use a curse to bring them low. The curse and cure involved in this process are both of your own invention. The curse is a modification of the ``Bad Luck Curse,’’ which you don’t know exactly how to make, but you’ve read enough about to manipulate the theory behind it.

\textbf{Genealogy Ability:} You may, at any time, go OOC and ask someone if they have a ``\mCPacket{\MYname}’’ If they do, it means that they are part of the \cAdopted{\formal} bloodline and will be affected by the curse.

Your plan involves 2 parts, which you can (and probably should) work on simultaneously)

\begin{enumerate}
  \item Finish the curse to apply to the \cAdopted{\formal} family.
  \begin{enumerate}
    \item Collect:
    \item Brew a curse from 1 ``\iFlameOrchid{}'' and 1 ``\iObsidian{}''.This is a curse of your own invention. Normally you would ask the GMs for the curse, but since you're going to use it immediately, there is no intermediate item card.
    \item Expend the curse on your ``\iWIPCurse{}.''
    \item Wait \textbf{12 hours} for the final curse to mature. If you are in a hurry, you can choose to stay outside of the bunkers during one of the storm surges, the curse will mature much faster with exposure to a storm surge, and it will be ready as soon as the surge is over. You are encouraged to work in part 2 (see below) during this wait time.
    \item At the end of the time, open the \iWIPCurse{} envelope and take out the \iWithering{} insde. Discard the \iWIPCurse{} item. You can now activate the curse whenever you want. \textbf{HOWEVER} you probably want to make sure you’ve finished the other part first!
    \item To activate the ``\iWithering{}’’ simply discard the item to the nearest stock, and tell everyone you see to ``open your C-Packet if you have one’’ and ask them to pass the word along to everyone they see, and so forth. Please also tell a GM so we can make an announcement at the next meal.
  \end{enumerate}
  \item Finish the protection to allow one person to be spared from this curse.
  \begin{enumerate}
    \item You do actually like your \cMusic{\auncle{}} \cMusic{} well enough. As a teacher at the \pSchool{}, \cMusic{\they} are far removed from the politics of the \pFarm{} nobility. Without extra precautions, \cMusic{} would also be affected by your curse. To prevent this you can finish a \iProtection{} using your \iWIPProtection{} that you’ve spent months preparing.
    \item Get a cleric to bless the ``\iWIPProtection{}.’’
    \item Get a ``\iSlowActingPoisonCure{}’’ from an advanced cursemaker.
    \item Have \cMusic{} (or whoever you want to make immune to the \iWithering{}) hold first the ``\iSlowActingPoisonCure{}’’ in their hands, and breathe upon it, and then repeat the process with your ``\iWIPProtection{}’’.
    \item Sit in meditation with the ``\iSlowActingPoisonCure{}’’ and your ``\iWIPProtection{}’’ for at least 2 minutes. Meditate somewhere that reminds you of \cFarmGod{}.
    \item Open the ``\iWIPProtection{}’’ envelope and retrieve the ``\iProtection{}'' inside. Discard the ``\iSlowActingPoisonCure{}’’ and ``\iWIPProtection{}’’ items to the nearest stock. They have been consumed.
    \item  You may now activate the protection whenever you want. To do so, simply discard \iProtection{} to the nearest stock, and tell the person from step 2.d to destroy their ``C-Packet.’’ If you have not activated \iWithering{} yet, they don’t get to know what was inside their ``C-Packet.’’ If you have already activated \iWithering{}, then they will still discard their ``C-Packet’’ and no longer suffer the effects of \iWithering{}, \textbf{but} they will know the full extent of the effect.
  \end{enumerate}
\end{enumerate}

Once you have prepared both ``\iWithering{}'' \textbf{and} ``\iProtection{}'' (complete steps 1.f and 2.f), regardless of whether you activate them, you may open your ``\mPacketThree{}.’’ The content inside represents your advancing skill in cursemaking.

Getting rid of the curse is easy (for you), and impossible for anyone else (a feature of the ``Bad Luck Curse’’ on which this is based). All you have to do is decide to end it. However, this will end it for \textbf{all} affected. And this took you over a year to prepare. It would suck to have to give up this bargaining chip without any concessions from your adopted family. Still, you can do it if necessary.

\textbf{There is one other emergency contingency,} which is that you could take the curse from *one other* person, while leaving it active on everyone else, by taking it onto yourself. This is a very last resort option however, because as the creator and caster of this curse, if you take it onto yourself, you will \textbf{never} be able to remove it. You could still dispel the curse, and it would lift for everyone else, but not for you. - If you take this option, you will slowly die from it over the next 6 months or so, and no one will be able to save you. If you choose to take the curse onto yourself, open your ``\mCPrimePacket{\MYname}.''


\end{document}
