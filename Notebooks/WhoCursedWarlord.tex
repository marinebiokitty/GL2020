\documentclass[notebook]{GL2020} %% [notebook] or [greennotebook]
\begin{document}

\startnotebook{\nWhoCursedWarlord{}}

\begin{page}{first}
Before the war, your ship was a joyous, loving place, with no grand, bloodthirsty designs beyond her rails. But not long after the first storm hit the \pShip{} out of turn, one of your main parental figures, \cLoud{\full}, underwent a significant personality shift, almost overnight and started calling for war. First your ship, and then your fleet, and then your nation rallied behind \cLoud{\them}. At first, you went along with everything, but after a few years, you started to have your doubts, especially after you were shipped off to the \pSchool{}. Over the past two years you have become increasingly convinced that something was \textbf{done} to \cLoud{} - something magical - something that you wish to reverse. As more and more voices call for peace, you worry for \cLoud{} if \cLoud{\theyare} magically compelled to continue warmongering. Once the game starts, open \nbref{second}.
\end{page}

\begin{page}{second}
So who had the motivation to do something like this? Maybe some warmonger in the \pShip{} government? Speak to the advisors of the \pShip{} until you find one that is strongly pro-war. Could they have caused this somehow? Once you have found at least one pro-war \pShippie{} advisor, open \nbref{third}.
\end{page}

\begin{page}{third}
Maybe you’ve found a meaningful lead? But you don’t remember anything particularly odd happening in the weeks before that might explain \textbf{how} it happened. Maybe your \cQuiet{\parent} \cQuiet{\full} does? Write to them using your ``\aLettersHome{}.’’ Once you have received and read \cQuiet{\their} response, open \nbref{fourth}.
\end{page}

\begin{page}{fourth}
This is why people shouldn't mess with magic they don’t understand (which is totally, definitely not what you’re about to do). It has unforeseen consequences that start the first international scale war in all of history. So how do you fix it? \textbf{Bring a cursemaker with you} to the ``Curse Carrel’’ in the 2nd Tier of the library and search through the books there for 3 minutes. This should allow you to learn a little about the theory behind curse and cure creation, and the types of ingredients that might accomplish what you want. Once you have completed the research, open \nbref{fifth}.
\end{page}

\begin{page}{fifth}
\emph{(OOC: Read this part aloud; to anyone who helped you research.)} The work of cursemaking hinges both on the inherent properties of the ingredients used, and the intentions imbued by the maker. While a true panacea that would reverse \textbf{any} curse cannot be constructed, it is possible to make something very close. \emph{The rest of the page is badly smudged; making it impossible to read. OOC: stop reading aloud here.}

It seems that Ebb and Flow aren’t the only Deities looking out for you. In the sudden warmth of midsummer’s noon, you feel the brush of \cFarmGod{}’s presence, and an image of a room lit by the radiance of the sun enters your mind. Travel to the “Chamber of the Sun” in the 3rd Tier of the library, gaze into the light, and earn a “Memory of Light.” Then open \nbref{sixth}.
\end{page}

\begin{page}{sixth}
With \cFarmGod{}’s blessing comes the knowledge of how to craft the Cure you need. To create a Cure that will return \cLoud{}’s personality to their former disposition:
\begin{enumerate}
	\item Collect the following ingredients: \iCharcoal{} and \iSight{}
	\item Obtain a sample of your own blood (you will need to find a mechanic to do this) to lock the intention of the cure, so no one can use it for something else.
\end{enumerate}

Once you have the three things you need, open \nbref{seventh}.
\end{page}

\begin{page}{seventh}
Get a cursemaker to make you a ``\iPanacea{}'' by using the two items plus the sample of your blood. They must be skilled enough to create Curses that use 3 components, so \cPrince{} or \cCurse{} are your best bets. (Creating the Cure will destroy the individual components). The Cure will take \textbf{1 hour} to brew (They have a mechanic that says that someone else can bring them a recipe to make a cure they don’t otherwise know how to make for just such a situation as this). Once you have the Cure in hand, open \nbref{eighth}
\end{page}

\begin{page}{eighth}
You may now send the Cure along with a letter home to \cLoud{}, to try to convince them to drink this cure. While your ability does not normally let you send items, this mechanic overrides that.

Hopefully \cLoud{} will consume the Cure immediately, and be returned to \cLoud{\their} former self. Either way, you hope for a letter in return after the next meal.

\end{page}

\endnotebook

\end{document}

