\documentclass[notebook]{GL2020} %% [notebook] or [greennotebook]
\begin{document}

\startnotebook{\nGetItOff{}}

\begin{page}{first}
This ``\iMagicTag{}’’ item is definitely not something you had Friday morning. Where did it come from? And more importantly, why is it stuck to your skin and won’t come off!? Get a cleric to examine the tag for 30 seconds, then open \nbref{second}.

As a reminder, you can show this item to anyone, but you cannot drop it, give it away, destroy it, or have it taken from you - it is somehow magically bonded to you.
\end{page}

\begin{page}{second}
\emph{(OOC: Read this part out loud to the cleric who examined the ``\iMagicTag{}’’)} This is definitely a cursed object of some kind. Definitely for a nefarious purpose, but what exactly is obscured. What’s worse is that there is a forgetting spell on it. At the end of the weekend, whoever has this tag on them will forget it exists. Meaning, if they want to get rid of it, they must do so before the end of the weekend. \emph{(Stop reading out loud here.)}

You will need to research a cure. Go to the ``Curse Carrel’’ in the Mid-Tier area of the library and search through the books there for 3 minutes. This should allow you to learn a little about the theory behind curse and cure creation, and the types of ingredients that might accomplish what you want. \textbf{You must bring at least one cursemaker along with you} to help explain what you find. Once you have completed the research, open \nbref{third}
\end{page}

\begin{page}{third}
\emph{(OOC: Read this part aloud; to anyone who helped you research.)} The work of cursemaking hinges both on the inherent properties of the ingredients used, and the intentions imbued by the maker. While a true panacea that would reverse \textbf{any} curse cannot be constructed, it is possible to make something very close. \emph{The rest of the page is badly smudged; making it impossible to read. OOC: stop reading aloud here.}

It seems that \cFarmGod{} has noticed your plight. In the sudden warmth of midsummer’s noon, the knowledge of \cFarmGod{} enters your mind. To create a Cure that will help you remove this magical tag:
\begin{enumerate}
	\item Collect the following ingredients: \iFish{}, \iCharcoal{}, and \iSight{}
	\item  Obtain a sample of your own blood (you will need to find a mechanic to do this) to lock the intention of the cure, so no one can use it for something else.
\end{enumerate}

Once you have the four things you need, open \nbref{fourth}
\end{page}

\begin{page}{fourth}
Get a cursemaker to make you a ``\iPanacea{}'' by using the three items plus the sample of your blood. They must be skilled enough to create curses that use 4 components, so \cPrince{} or \cCurse{} are your best bets. (creating the cure will destroy the individual components). The cure will take \textbf{2 hours} to brew (They have a mechanic that says that someone else can bring them a recipe to make a cure they don’t otherwise know how to make for just such a situation as this). Once you have the Cure in hand, open \nbref{fifth}
\end{page}

\begin{page}{fifth}
Find someone you trust for this next part. You will need to activate the ``\iPanacea{}’’ on yourself, and then have the other person knock you out (the normal effects of this cure don't apply for this mechanic).  Lend them this notebook to complete the next step \textbf{(OOC: they must give it back to you once you regain consciousness.)} When the person is knocked out, open \nbref{sixth}
\end{page}

\begin{page}{sixth}
This notebook has been lent to you to complete (or not) the next step in removing the ``\iMagicTag{}’’. \textbf{(OOC: You must return this notebook regardless of what you choose to do.)} If you want to remove the tag, ask them OOC for the tag item card and destroy it. If you don’t want to destroy it, take no action other than to return this notebook. When the original owner of this notebook wakes up from being knocked out, open \nbref{seventh}
\end{page}

\begin{page}{seventh}
\emph{(OOC Note: While you the player knows already whether the ``magical tag’’ was destroyed while you were unconscious, you should still check in-character so you can react in character.)}

\textbf{If the tag was destroyed:} Thank the Deities that’s gone. Now maybe you’d better figure out who stuck this on you in the first place, and why.

\textbf{If the tag was not destroyed:} The tag is still on you. But that should have worked! Why didn’t the person you asked to remove it, remove it? Were they not worthy of your trust? If you want to try again to remove the ``magical tag’’ you will need to acquire a new ``\iPanacea{}’’ by bringing all of the ingredients to a cursemaker again.

\end{page}

\endnotebook

\end{document}
