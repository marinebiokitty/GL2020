\documentclass[greennotebook]{GL2020} %% [notebook] or [greennotebook]
\begin{document}

\startnotebook{\nCreationOne{}}

\begin{page}{first}
You are very interested in researching the creation of \pEarth{}. You have two allies in this investigation, \cHeadScientist{\full} and \cScholarship{\full}. Each of you has a different but related line of investigation that when combined all together should yield an exciting and interesting discovery. When the game starts, open \nbref{second}.
\end{page}

\begin{page}{second}
Go to the ``\sOldPhilosophyOne{}’’ in the Old Wing of the \pSc{}. If possible, bring \cPrincipal{} along as their knowledge may provide interesting perspective or important context for whatever you find. Once you find the ``\sOldPhilosophyOne{}’’ open \nbref{third}.
\end{page}

\begin{page}{third}
In perusing the old writings here you have gleaned something important \emph{(OOC: Read this part to anyone who helped you, e.g. \cPrincipal{}).} Many of the texts from this time period observe the ways in which the Deities are not so different from humans - they too are subject to strong emotions like jealousy and regret. \emph{(OOC: Stop reading here.}

Once \textbf{\cScholarship{}} has also opened page 3, discuss your findings with them for at least 2 minutes. Then open \nbref{fourth}.
\end{page}

\begin{page}{fourth}
Putting together the information you’ve found, and the information that \cScholarship{} has found leads you to your next line of inquiry. How can we learn more about what the Deities thought and felt at the time of creation? You think on the problem and settle on a risky proposition. There’s an old, old recipe for a curse, stored safely in the archives of the 7th fleet, Southern Atoll Monastery. It is supposed to allow one to send their consciousness back in time an arbitrary distance - when the effect fades, the person recalls mostly emotions, but sometimes gets specific details also. Could you use it to go all the way back to the beginning? Open \nbref{fifth}.
\end{page}

\begin{page}{fifth}
You will need to find a cursemaker willing to brew this curse for you (they do not know how to make it inherently, but part of their mechanic allows for someone to bring them a recipe, precisely for a situation like this). It only takes 2 ingredients, which are easy enough to come by, but the recipe has been kept secret for a reason. Do be careful who you share it with.

To make the curse, combine a \iHollyhock{} (for remembrance and nostalgia) with a \iObsidian{} (for cutting through the fog of time). The curse will take 1 hour to brew. Once you have the curse in hand from the cursemaker, open \nbref{sixth}.
\end{page}

\begin{page}{sixth}
You will need to activate the curse on yourself. It is highly recommended you do this somewhere safe, and surrounded by people you trust. For \textbf{10 minutes} after you activate the curse on yourself, you will experience a tumultuous stream of intense emotions, and are unaware of actions going on around you. \emph{(OOC: you do not lose physical awareness; please don’t stumble off a cliff, or run into other players.)} When the 10 minutes are up, open \nbref{seventh}.
\end{page}

\begin{page}{seventh}
As your senses slowly return to your body, the first thing you are aware of is a pounding headache. But as you concentrate on what you just experienced, your strongest impressions are \textbf{the emotions of anger, fear, and shame.} A few details become more clear as you contemplate further: \textbf{Conflict over the new world of \pEarth{} led to two Deities being killed somehow. The first that the pantheon knew of death!} Once both \cHeadScientist{} and \cScholarship{} have opened page 7 of their notebooks too, you can open \nbref{eighth}.
\end{page}

\begin{page}{eighth}
Tell \cPrincipal{} to open \cPrincipal{\their} ``\mWOne{\MYname}'' - something about this investigation has jogged \cPrincipal{\their} memory with an important detail. Then, invite \cPrincipal{} to come discuss your findings with you and your fellow researchers ( \cScholarship{} and \cHeadScientist{}) and anyone else they wish to include.  Once everyone is assembled, open \nbref{ninth}.
\end{page}

\begin{page}{ninth}
Speculate with the assembled group about what all of your findings mean. Once you agree on a possible alternative creation story that incorporates the things you have discovered, go to the great hall, and lift the sign with the ``\sCreationMythOfficial{}'' on it. Read the sign below it, and reposition the ``\sCreationMythOfficial{}'' next to it, so both can be read freely by anyone.
\end{page}

\endnotebook

\end{document}
