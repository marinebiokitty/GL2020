\documentclass[notebook]{GL2020} %% [notebook] or [greennotebook]
\begin{document}

\startnotebook{\nFindAssassin{}}

\begin{page}{first}
This is a shared research notebook between \cJuniorStatesman{} and \cPresident{}. You should work closely together in this research. If at any time one of you has progressed further (opened more pages) than the other, you may discuss briefly and the other may open pages to match.

Figuring out what happened to \cHeadDiplomat{} will be difficult enough without access to \cHeadDiplomat{\their} home. You can’t afford to dilly dally and risk the trail going cold. When the game starts, open \nbref{second}.
\end{page}

\begin{page}{second}
The first step, like with almost anything else, is to meditate on the problem and see if \cEbb{} and \cFlow{} have any guidance for you. Bring at least 2 clerics or initiates (preferably of \cEbb{} or \cFlow{}, but any cleric will do in a pinch) to the temple with you and meditate together for 3 minutes on the question of ``What happened to \cHeadDiplomat{}?’’ When the time has elapsed, open \nbref{third}
\end{page}

\begin{page}{third}
\emph{(OOC: Read this part aloud; to anyone who meditated with you.)} The Goddesses whisper in your ears about ripples from a rock thrown into still water. They urge you to find the ``Chamber of the Fates’’ in the 2nd Tier of the library. \emph{OOC: stop reading aloud here.}

Once you reach the ``Chamber of the Fates,’’ open \nbref{fourth}
\end{page}

\begin{page}{fourth}
At first the Chamber of the Fates seems unremarkable. A dead end? But as you look up at the tangled threads above you, your attention is drawn this way and that way, following an aimless walk through the pattern. You stare up at the threads, trying to capture an elusive thought that drifts through the room like the dust motes.

Who benefits from \cHeadDiplomat{} being out of the picture? Maybe one of the advisors stood to gain enough to orchestrate an attempted assassination. Once you have interrogated at least 3 advisors on who benefits from \cHeadDiplomat{} being out of the picture, open \nbref{fifth}
\end{page}

\begin{page}{fifth}
Lots of advisors have some possibility of benefiting, but nothing that seems worth organizing murder. Unless someone’s hiding something really deeply? It’s time to turn to \cHeadDiplomat{}’s notes, which \cJuniorStatesman{} brought along just in case (\iMadroneNotes{}). There’s nothing obvious in the writing, but maybe, just maybe, there’s something hidden?

Find a \iCrystalLens{} and use it to look at the notes (this will not consume the item). When you do so, open \nbref{sixth}
\end{page}

\begin{page}{sixth}
There are hidden notes here! And quite a lot of it too, scribbled in margins with ink only visible through an enchanted lens like this one. \cHeadDiplomat{} has a whole theory about some group called the ``Followers of Genesis.’’ \cHeadDiplomat{} was concerned that this group would infiltrate the highest levels of government,start sabotaging things, and ultimately try to upend all of \pShippie{} society. The notes suggest they may even be plotting to interfere with the Ritual to Control the Storm in some way, and that they have an operative here among the Wave Rider advisors (\cEbbPriest{}, \cChupLeader{}, \cBunker{}, and \cJuniorStatesman{})! It looks like \cHeadDiplomat{} was preparing to go to the Council of Stormwatchers about it. If the Followers knew this, it would make sense that they needed to stop \cHeadDiplomat{}. And a fringe group like that surely wouldn’t hesitate to use a tool like murder. Open \nbref{seventh}
\end{page}

\begin{page}{seventh}
You shouldfind these ``Followers of Genesis’’ and take the most senior member of the group that you can find back to \pShip{} for further investigation. And so the Council can determine proper justice if necessary. They should either go with you willingly, or be unable to resist you at the end of game (restrained or unconscious). Vigilante justice will bring no peace; killing them here doesn’t make this right.
\end{page}

\endnotebook

\end{document}

