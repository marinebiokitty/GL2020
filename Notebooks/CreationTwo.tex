\documentclass[notebook]{GL2020} %% [notebook] or [greennotebook]
\begin{document}

\startnotebook{\nCreationTwo{}}

\begin{page}{first}
You are very interested in researching the creation of \pEarth{}. You have two allies in this investigation, \cEbbPriest{} and \cScholarship{}. Each of you has a different but related line of investigation that when combined all together should yield an exciting and interesting discovery. When the game starts, open \nbref{second}.
\end{page}

\begin{page}{second}
Search the library for the ``New Philosophy Room’’. If possible, bring \cHistory{} along as their study of History may help provide important context. Once you find the ``New Philosophy Room’’ open \nbref{third}.
\end{page}

\begin{page}{third}
In pursuing these (fairly) modern treatises something stands out to you. \emph{(OOC: Read this part to anyone who helped you, e.g. \cHistory{}).} None of the writing attributes the ability to create something from nothing to any of the Patron Deities. But all are adamant that \pEarth{} is the first and only planet created by the Deities. So where did the idea to create\pEarth{} come from? \emph{(OOC: Stop reading here.}

Once \textbf{\cEbbPriest{}} has also opened page 3, discuss your findings with them for at least 2 minutes. Then open \nbref{fourth}.
\end{page}

\begin{page}{fourth}
Putting together the information you’ve found, and the information that \cEbbPriest{} has found leads you to the next key realization: Maybe there’s more to the idea that  the Gods are not altruistic beings - anything they do for humanity must have a benefit to them, especially big things, like granting humanity magic. Which begs the question, What did the Patron Gods get in return? Open \nbref{fifth}.
\end{page}

\begin{page}{fifth}
You wonder if it has something to do with conflict. Study the underpinnings of magic by observing combats between people of the same religion 2x, and people of different religions 2x. At least 1 combat must end in someone being knocked out, but the rest can use the upstage mechanic, so no one gets hurt.

Take your finding and meditate (uninterrupted) for 5 min somewhere. Set a timer, close your eyes and don’t engage with anyone. If you are interrupted, you’ll have to start over. Once the time has elapsed, open \nbref{sixth}.
\end{page}

\begin{page}{sixth}
Increase your permanent CR by 1, and take an extra CR stone from the nearest stock to remind you of this. Then engage in combat to test your learnings. You may spar with people using the ``upstage'' mechanic. Once you have won at least 1 combat, open \nbref{seventh}.
\end{page}

\begin{page}{seventh}
You conclude that the conduit of power goes both ways - the stronger the followers on \pEarth{}, the stronger the God. A Deity or deities wanting to get the upper hand in a conflict against other Deities could do so by strengthening their followers on \pEarth{}. For example, by granting their followers magic. Once both \cScholarship{} and \cEbbPriest{} have opened page 7 of their notebooks too you can open \nbref{eighth}.
\end{page}

\begin{page}{eighth}
Tell \cHistory{} to open \cHistory{\their} “\mWTwo{}” - something about this investigation has jogged \cHistory{\their} memory with an important detail. Then, invite \cHistory{} to come discuss your findings with you and your fellow researchers ( \cScholarship{} and \cEbbPriest{}) and anyone else they wish to include.  Once everyone is assembled, open \nbref{ninth}.
\end{page}

\begin{page}{ninth}
Speculate with the assembled group about what all of your findings mean. Once you agree on a possible alternative creation story that incorporates the things you have discovered, go to the great hall, and lift the sign with the ``\sCreationMythOfficial{}'' on it. Read the sign below it, and reposition the ``\sCreationMythOfficial{}'' next to it, so both can be read freely by anyone.
\end{page}

\endnotebook

\end{document}
