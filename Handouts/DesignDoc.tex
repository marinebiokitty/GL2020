\documentclass[sheet]{GL2020}

\usepackage{graphicx}
\graphicspath{ {./images/} }
\usepackage{xcolor}
\usepackage{hyperref}
\usepackage{multicol}
\usepackage{ltablex}
\usepackage{tabularx}
\usepackage{indentfirst}
\renewcommand{\tabularxcolumn}[1]{m{#1}}
\setlength{\columnsep}{1cm}

%% document-wide tweaks
\interlinepenalty10000
\setstretch{1}
\def\mytype{Rules and Scenario}
\lfoot{}\rfoot{}

\begin{document}

%% layout for cover page
\thispagestyle{empty}
\parskip0pt

%% title box
\begin{center}\LARGE\bf\begin{tabular}{|c|}
  \hline \gamename\\ \gamedate\\ Design Document\\ \hline
\end{tabular}\end{center}

\vfill\vfill

This game and all materials thereof are copyright 2021 by Acata Felton, Aaron Sunshine, Eric Fritz, Jeremy Cole, Kelsey Miranda, Koh Henderson, Olivia Montoya, and Kate Hill.\\

\vfill\vfill

\begin{center}\bf
  BROUGHT TO YOU BY THE LUMINARY ROLEPLAY SOCIETY
\end{center}

\vfill

\clearpage

%% layout for Table of Contents page
\thispagestyle{empty}
\tableofcontents

\clearpage

%% layout for main body of rules
\setcounter{page}{1}
\parskip5pt
\vfill
\section{Introduction}

The following is the Design Document for {\em\gamename}, a real-time, real-space, roleplaying game sponsored by the Luminary Roleplay Society. \textcolor{red}{This version of the Design Document is released as of <DDMMMYYYY>. Any subsequent changes, additions, or clarifications necessary will be presented in red text, just like this.}

\subsection{Basic Expectations for Players}
You are responsible for knowing and following the rules of this game. It is through the constraints created by these rules and other mechanics that the intended play experience is made possible. Many of these rules are nigh-impossible for GMs to enforce, and rely upon the honor system. Do not cheat. Do not abuse loopholes. Play fair. To do otherwise will deprive both yourself and your fellow players of the experience you signed up for.

The number of pages of reading for this game is between 40 and 50 pages for most characters. This page estimate is across all of the documents, including this design document, and the rules document. The Design and Rules Documents are available through our website now (\url{https://childrenofthegodslarp.wordpress.com/}). Players should expect to receive your character materials (20-30 pages total) about 2 months before game. Please make sure you set aside enough time to prepare for game by familiarizing yourself with all of the documents made available to you. You do not need to memorize anything (except maybe your CR stat; see the rules document for what a CR is), but you should read everything in detail at least once, and know where to look up information. Print copies of everything will be provided at game.

\subsection{GM Commitment}
The \textbf{gamemasters} (\textbf{GMs}) run the game. If you have any problems or questions concerning the game, contact a GM. Rulings the GMs make are final.  We may violate the letter of the rules to preserve the spirit.  The GMs promise to be as fair and reasonable as possible. Neither they nor these rules are perfect, so we ask for your understanding and flexibility.

\subsection{Disclaimers and Acknowledgments}
This game is a work of fiction. The attitude of the characters and the social milieu of the game world do not necessarily represent the opinions of the GMs, or of the player playing that character. While any work of fiction must draw some inspiration from real world situations, it is not intended to be a replication or direct allegory of any real world experience. We understand that the player experience is necessarily grounded in their own lived experiences in life, and have done our best to be mindful of the ways that game content can interact with that. 

\subsection{Game Style}
This game is Secrets and Powers (S\&P) style or ``Lit Form''. Extensive game content is created by the GMs, and hooks are given to players to start various plots. Characters are pre-written, with deep and interconnected backstories for players to really sink their teeth into. Character goals may be mutually exclusive, leading to character vs. character conflict. Players will get the most out of the game by embracing their character, playing to discover the world created by the game writers, and reacting to new revelations from other PCs in character. Players will likely not come out of this game having accomplished everything their character wanted.

While characters are often ``playing to win'', players should look to the best stories. Embrace the most narratively impactful moment to have your character back down, admit a failure, spill a secret, or change their mind about something. Also keep in mind that you do not have complete narrative control over your character's story. Your secrets may come out for reasons that are out of your control. Other characters may succeed in stymieing your goals, or sneak their opposing goals in under your nose. We encourage you to embrace these additions to your character's story. You can use a ``yes, and'' approach to respond to the actions and revelations of other characters that impact your character, as you would in a more free-form game. This approach to storytelling allows players to lean into characters who want to keep a secret hidden while still enjoying the full breadth of their character arc.

This game also contains significant interactions between PCs and the environment. These interactions are regulated through predefined mechanics that allow players to interact with elements of the environment without waiting for a GM to narrate something or make a ruling. This generally leads to less sitting around waiting out of character, and more time playing and interacting in character. The trade off is that there is some reading required during the game. The reading is rarely more than 1 paragraph at a time, and any new mechanics you discover will be similar to ones you've seen in other contexts.

This is not a ``what you see is what you get'' (WYSIWYG) game. You will need to explain to each other out of character what characters perceive in situations where it is not obvious out of game; e.g.\ ``My character's hands are covered in blood.'' {\bf Metagaming} is inferring in-game knowledge that is inappropriate for your character from out-of-game information. While some players may be more familiar with games with higher levels of transparency, not every player enjoys having to firewall a lot of information, or having their character's secrets revealed out of game before they are revealed in game. Please do not volunteer, solicit, or discuss information contained in any documents other than this design doc or the rules guide. There are intentional places in this game where characters have different amounts of information, and for game balance, we would like to keep it that way. 

However, if necessary for safety, players may discuss secret character information ahead of time. Please include a GM in any such conversations so we can be aware of any safety concerns and prepare to support you if needed. We generally find that these pre-game discusions are necessary less frequently than you might think in this style of game. Trust your fellow players to enjoy unexpected twists in the story as new information is revealed to players and characters simultaneously.

\subsubsection{Game-Wide Secrets}
Secrets and Powers style games often have game-wide secrets that are revealed part way through the event. These can change the way characters view and interact with the world, and therefore character priorities. This game has the potential for several \textbf{world shifting} secrets to come to light, and for characters to take \textbf{world changing} actions. The GMs will do our best to make sure that any such secrets are interesting and satisfying, rather than leaving players feeling like the reveal invalidates their character and character goals. 

\subsubsection{Player Experience}
Players should expect to spend time before the weekend of game familiarizing themselves with game material, and preparing any costuming they choose. There is no need to connect with other players before the game -- there will be time for that during Friday workshops (onsite). Please do NOT pre-plan anything - doing so limits both your ability to experience the game organically, and other players', who's character wanted to interact with yours.  During the game, players should plan to spend most of their game time in character, interacting with each other and the world. Players should not expect to need to spend extensive periods of time out of character negotiating things. Scenes can typically go wherever the characters take it, unless someone invokes a safety mechanic (see rules document for details).

\subsubsection{Costuming and Weather} 
\textbf{Costumes are always admired but never required.} Costume shaming of any kind will not be tolerated. While it can be fun to have a costume, making something from scratch, thrifting and altering, or finding the prefect premade item can be time consuming and expensive. Not everyone has time and money for that. If you choose to costume, keep in mind that the game space requires traveling between a few buildings, and that therefore good footwear will be important. We encourage layers, as weather permitting, some parts of game will take place on outdoor decks adjacent to indoor spaces, and it can be quite cool at night in October. Please also prepare for the possibility of rain, by including an umbrella, or rain jacket in your packing. (If it is raining, we will bring all play spaces inside)

\subsubsection{Covid-19 Safety}
We will be using wastewater Covid levels to judge the prevelence of Covid-19 in the month leading up to game. At minimum, players will be expected to be up to date on vaccinations (inculding any recommended boosters available 2 months before the event), and present a negative test within 24 hours of the start of the event. Please contact us if you are contraindicated for the vaccine, and we will see what accomodations we can make. Other safety measures we reserve the right to employ include masking (KN95 or better), Hepa filters, <etc>.

\textbf{Please ensure that you are able to wear a mask (for covid protection) with your costume.} We won't be making the call on whether masks are necessary until the last minute (approximately 1 week before game), so you should plan for it just in case.

\section{Logistics}
\subsection{Basic Information}
\begin{itemize}
  \item \textbf{Dates and Times:} Players will be expected to be on site at the event location from 1 pm  on Fri, Oct 4th - 5 pm on Sun, Oct 6th. Arriving late or leaving early requires special dispensation from the GMs. Please request any such arrangements well in advance.
  \item {\textbf{Location:} Silver Lake Conference Center; 223 Low Road Sharon, CT 06069 . (approx. ~3 hr 15 min drive from BOS Airport, ~2h 15 min from JFK Airport, or ~1 hr 30 min from Albany or Bradley Airports.)
  \item {\textbf{Ticket Prices:}} The average ticket price for this event will be \$425, but please see the section on Accessibility below if this presents a barrier to your attendance.
\end{itemize}

\subsubsection{Sleeping Arrangements}
Sleeping spaces at the camp are split between two buildings. In the Retreat Center, all beds are single beds, in rooms with 3 or 4 beds total, except for 2 ADA compliant rooms (which we are reserving for those in need of such spaces). The remaining bedrooms are up a short flight of stairs. In the Lodge, some beds are singles, and some are bunk beds, but all are of high quality, better than one might expect from a summer camp space. The lodge is all on a single floor, with no stairs.

\subsubsection{Accessibility}
Accessibility is a crucial priority for our team. We are balancing as many needs as we can, and are always happy to chat about your unique situation, and what we can do to support you attending this event. We are already preparing the following physical, mental, and financial accomodations:

\begin{itemize}
  \item All rooms have electricity for charging electronics, running CPAP machines, etc. The rooms are also winterized to help manage any low temperatures we might encounter.
  \item Sleeping Spaces have Wi-Fi. However, there is \textbf{no cell service} at the venue. In addition to regularly circulating, the GMs will have walkie talkies stationed in each building so we are reachable at any time.
  \item Paved paths are available between all buildings, and the 3 buildings we will be playing in and around are no more than a 5 minute walk between. There are some small hills, but players who cannot traverse them for any reason are welcome to drive between buildings. When traveling between buildings players are restricted to ``soft RP'' only. No game actions can be taken during this travel to ensure that folks who are seperated between walking and driving are not missing out.
  \item All play spaces are wheelchair accessible. One space has a motorized chair lift to enable accessibility. We have confirmed with the venue that this equipment will be carefully serviced before our arrival to ensure it is functional. Not all sleeping spaces are wheelchair accessible, so unfortunately we will need to know if you use a wheelchair when we do room assignments.
  \item All sleeping spaces are strictly out of game spaces. Players are always welcome to retire to their room if they need to take a break. If we don't see you at a meal for which we were expecting you, we may come check in on you. If you prefer not to be disturbed, please let a GM know.
  \item If staying on site is a barrier for you, please contact us and we will see what arrangements can be made. Unfortunately, we cannot reduce a ticket price due to staying off site, as the play spaces are what we are primarily paying for. \textbf{Please note} that this is an accessibility accomodation that will impact your experience of the event.
  \item As part of our preparation for game, we will be soliciting dietary restrictions from all players, communicating these to the venue, and determining if they cannot accomodate any of them. We will have access to full kitchens in all 3 of our buildings, and such players may bring and prepare their own food. We will work with you to adjust your ticket price to reflect that you don't need to pay for meals you can't eat. \textbf{Please note} that this is an accessibility accomodation that will impact your experience of game; we strongly discourage players from trying to opt out of the meals just to try to save a few dollars.
  \item The LRS employs a ``pay what you can'' model for our events. Players will request a ticket price as part of applying for the game. We will balance sponsorship donations against scholarship requests to allow as many people to attend as possible, regardless of financial means to do so.
  \item The venue is run by a Christian organization, and there are some Christian paraphanaila on a few walls (e.g. a large cross). We understand that this can be uncomfortable for some folks. We have discussedd extensively with the venue their policies around harassment. We are satisfied that their staff will respect our community, and that any issues from staff or other campers who might be present will be treated as seriously by the venue as we will treat it. You can read more about Silver Lake's stand on things like protecting trans campers and Black Lives Matter here: \url{}.
  \item Most restrooms at the venue are already marked as gender neutral. We have confirmed permission to mark any remaining ones as such.
\end{itemize}

\section{Game Content}
As this is a S\&P game, the GM team has prepared extensive information about the premise and direction of the game, as well as the types of personal plots and stories that are likely to show up. If you are unable to play a game that includes the topics and content warnings described below, this may not be the game for you. You can always reach out to us with additional questions, but these are not topics we can remove from the game.

\subsection{The Premise}
\emph{Welcome to the world of \pEarth{}. The Gods have gifted the people with magic, but at a cost -- the ravages of magical devastation roaming across the land once every 3 years. Only a chosen few students can manipulate the path of these magical Storms. Until recently, a treaty ensured a tenuous peace. The three nations agreed that the Storm should hit each nation in turn, sharing the devastating burden. Although the Storms always took their toll, no one nation suffered too greatly. That all changed when the treaty was broken, and a nation betrayed. Although their architects and diplomats struggle to repair what was lost, the work won't be complete by the next Time of Deciding.}

\emph{Our story begins at the center of the known world. Here at the \pSchool{}, students are trained in the rituals necessary to change the path of the Storm. Instructors maintain the veneer of impartiality -- but in reality, each fights for their own nation's interests to varying degrees, all vying for control of the magical disaster. But despite their efforts, no nation has the complete picture -- and new interests conspire to change the fate of the world. Where the magic will ravage next -- and what consequences it will have -- are for you to decide.}

\subsection{Themes and Topics}
This game is set in a fantasy world with strong magical and religious elements. Not all is well however, and the people in this world are facing many difficult choices. We ask that players prepare to engage with these themes, as they constitute much of the fabric of the world.

While having fun is a primary goal of most players it is important to recognize that some of the themes and topics addressed in the game are quite serious. All of the characters are facing very significant moral and ethical choices that have real world analogies in the abstract, if not the specific. It is possible to both have fun at this game, and not dismiss the gravity of the decisions being made, and the consequences they will have for this fantastic world.

\subsubsection{Magic and its Price}
Magic in this world is the source of almost everything. Food can be grown without magic, but not nearly enough to feed the population. Ships cannot be built well enough to withstand a sea serpent attack. Technology runs on magical energy, and cannot be made to function without it. There is no corner of life on \pEarth{} that is not dependent on magic, and the loss of magic would be as unimaginable as you the player living in a world where electricity stopped working.

The Storms are also magical in origin and nature. Every three years, a magical Storm brews over the lake where the \pSc{} resides, and then whirls off to cut a swath of destruction across the continent. To date, no one has been able to figure out how to stop this from happening. The best anyone could do was use the Ritual to direct the Storm toward a particular country. This is not a trivial problem to be solved overnight -- it has shaped the land, the lives, and the politics of \pEarth{} since the storms began.

\subsubsection{Religion}
Religion is a big part of this world, and the lives of all of the characters. To help bring the world of \pEarth{} to life, we ask players to engage with the trappings of the religion, at least casually. As part of workshops, the players from each nation will create several common practices they share. The clerics of each religion will also pick and announce a time for a short religious ceremony of some kind (recommend no more than 15 minutes) during the weekend. If you miss it, it is reasonable to expect social consequences, even if you have a very good explanation.

This serves a few purposes. Firstly, this \textbf{is} a world with Gods. The Gods exist. They gave people magic. They appear to people, or speak through their Avatars frequently. It would take almost total isolation and ignorance to make someone a true atheist, or even agnostic in this world. The most lively areas of study all involve innovative magic use, and there is no way to separate the intellectual from the magical from the religious in this world. This is an intentional part of the fiction of the world. 

Secondly, participating in shared rituals helps create a deep sense of connection and community between the characters, which is an experience we hope to foster for the players. 

Lastly, for those few characters that do reject part or all of the religion, the experience of trying to navigate a religious world can only happen if the rest of the players help create that world for them to navigate. Unless your character sheet specifically describes your character as questioning or rejecting the religion, you should assume that your character is moderately to highly religious, and play them accordingly. 

\subsubsection{Morality and Ethics}
On a national scale, morality is dictated by religion, and governments endeavor to set rules and laws in place that reinforce this. On a personal level, many characters have opinions, not all of which align with the national and religious agenda. This game is absolutely meant to be a space where characters can engage in discussions about morality and ethics. However, we ask players to set aside some time to think seriously about the fact that your characters have grown up in this world, with very few dissenting voices. You should be prepared to argue for and support the status quo unless you have a specific reason outlined in your character sheet to question it.

We know that some of the morals that the fictional societies uphold are likely to be orthogonal, or even contradictory, to what some of our players believe in real life. We ask you to resist the urge to modify your character's opinions to match your own too early in the weekend. This phenomenon is common in games (sometimes called the ``21st century morality problem'’) and can very quickly cause tension and conflict to collapse. Many of these characters have hills they are willing to die on (sometimes literally), and the game will be more interesting if you are willing to stick to that until another character successfully challenges that worldview. 

\subsection{Content Warnings}
This game deals with a number of serious and potentially triggering or upsetting topics. We have done our best to provide a list below of the most widely reaching content. Almost every character will bump up against these. Some characters will have large swaths of their play built around them. Players are expected to treat these topics with the gravity they deserve.

\subsubsection{War}
There is a war going on in the world of this game. It has been going on for almost 6 years. Many characters have been directly or indirectly involved, and everyone has been impacted in some way. War is not a conflict we introduce lightly. A lot of people end up dead, or mentally or physically disabled as a result of war, even in a fictionalized world. But much like in real life, many people are convinced that it is necessary, or that it is just, or that there is no way out that saves face (and that saving face is important.) While there is an opportunity in game for the characters present to try to forge a cease-fire or a peace treaty, the ultimate decision to ratify a treaty does not rest with the players. Instead, characters can submit correspondence to be sent to their respective governments (managed by the GMs), who will ultimately approve or reject proposals. This structure exists, not to take agency away from players, but to avoid trivializing the process of finding a solution. 

\subsubsection{Nationalism and Jingoism}
Aside from the immediate effects of war, nations have done everything they can to amp up national pride, and a sense of righteousness for their own cause. Leaders in every country are happy to villainize those from elsewhere, to create a boogeyman that keeps the fear and anger that started the war burning hot. The students are somewhat insulated from these effects inside the \pSchool{}, but the advisors are bringing an extra-heavy dose of this into game. Some of these interactions may feel very similar to racism, and while players are encouraged to lean into the nationalism, no real world racism will be tolerated.

\subsubsection{Religion and Spirituality}
We understand that not all of our players are religious or spiritual in their own lives, and that some folks may have trauma around organized religion (one of your GMs does). Being a part of this highly religious world may feel strange or uncomfortable, especially at first. We ask players to take particular care to avoid sharing a cavalier attitude toward religion and spirituality during and around this game, or to assume that ``everyone'' believes something or other out of character. \textbf{Players} should avoid proselytizing while out of character, whether about religion, spirituality, atheism, or other. There isn't much proselytizing in character either; \textcolor{red}{each nation is happy to have people change their patron deity to that nation's patron deity, but people are not pushy about it.} Let people have their own lives, do not assume that their character attitudes reflect their player attitude, and do not offer unsolicited advice or opinions about religion or spirituality.

\subsubsection{Murder}
In the world of \pEarth{}, the Gods declared murder the ultimate crime, and enforce punishment themselves. The person who took another life loses their memories. Pretty much all memories, usually immediately, regardless of the circumstances. Some characters are interacting closely with plots that more directly address this, others are more removed, but it is a fact of this world. There are mechanical consequences for killing another character in game. See the section on ``Combat'' in the rules document for more information.

While it is possible for characters to die during the course of the event, we have made extensive preparations to ensure that those players are able to continue participating, enjoying themselves, and influencing the story. You may find that the best thing you did for someone else' story was kill their character.

\subsubsection{Memory Loss}
Even though character memory loss is a possibility, ableism is \textbf{not} part of this game. Characters can be suspicious of another character claiming not to remember things, but you should be careful about jumping to the conclusion that it is divine amnesia; plenty of people lie all the time about not remembering one thing or another for a variety of reasons. It is also important to distinguish that some \emph{players} may have difficulty remembering things. Be kind and patient with each other if someone asks for you to clarify or repeat something you said previously, or needs to refer to a game document.

\subsubsection{Potential Apocalypse}
It may be possible, through certain character actions or inaction, to cause an apocalypse level event that will drastically change the course of \pEarth{}n history. No change so drastic can come without significant fallout among many inhabitants, and the worst parts of it are likely to fall on the most marginalized communities even if they are specifically prepared (for example: if they are instigating the change.) Any decision you make that drives toward something like this should be made bearing this weight in mind.

An Apocalypse scenario would include imminent risk of death for the player characters. While not every character would die in such a situation, many characters could, and an individual character's survival may be outside of the full control of the player of that character. We encourage you to consider this content warning in the context of the game style, which includes mutually exclusive goals and character vs. character conflict that does not have pre-negotiated resolutions.

\subsubsection{Classism}
The world of \pEarth{} is riddled with direct and indirect classism. Some characters are more cognizant of this than others, and some have significant levels of privilege. Classism figures heavily in the backstory and motivations of several characters, and some of the plots and stories in game center around trying to address this in either an individual or a systemic way. Addressing classism is not a trivial thing, and players should be cautious about trying to ``white-knight'’ a solution by trying to fix a perceived problem without any input from the marginalized people they are attempting to help.

\subsubsection{Content Warnings impacting PART of game}
\begin{itemize}
  	\item Family issues. Some characters have difficult relationships with biological or non-biological family members, including estrangement.
	\item Established romance -- some characters are in established romantic relationships; at the discretion of the players involved, these can be re-imagined as platonic life partnerships.
	\item Potential relationships -- some characters are interested in exploring new romantic or platonic relationships with other characters. None of the feelings would necessarily be unwelcome, but some characters may not realize yet that someone else is interested in a relationship with them.
	\item Polyamory -- Some of the existing or potential relationships involve more than two consenting adults.
	\item Cheating or nfidelity -- some characters may not be honest in their relationships.
	\item Harm to animals -- the Avatars of the Gods typically take the form of sentient animals. Some characters may wish to do harm to those specific animals due to their avatar nature.
	\item Immortality and death -- Some characters are grappling with what it means to be immortal.
	\item Tunnel vision -- sine characters are grappling with a single goal that overrides all else.
	\item Character to character competition. Specifically, there is a mechanic in which the students are being ranked in something akin to a popularity contest by the teachers and advisors, (not by their fellow students).

\end{itemize}

\subsubsection{Content NOT Present in Game}
Due to the game design, we can provide the following list of topics and tropes that will \textbf{not} be part of the game material, and shouldn't come up during play. This is not a complete list, and if you have questions or want to check about a particular other topic, please reach out and ask.

\begin{multicols}{2}
\begin{itemize}
 	\item LGBTQIA+ Discrimination.
	\item Fatphobia / fat shaming.
	\item Sexism.
	\item Unwanted sexual attention (harassment, assault, etc.)
	\item Ableism (other than suspicion toward individuals with significant memory loss.)
	\item No characters are soldiers or police, but some characters have positions of authority over other characters.
	\item Medical procedures. All healing is abstracted, and is not meant to support extended graphic roleplay scenes around healing or other medical procedures.
	\item Pregnancy or pregnancy loss.
	\item Substance abuse.
	\item Partner or child abuse.
	\item Pandemics.
\end{itemize}
\end{multicols}

\textbf{A note about sexual content:} This game has no pre-written sexual content, and no sex mechanic. Players are required to obtain explicit, enthusiastic consent from everyone present in a space before engaging in any such play (not just from the players about to engage with the content). Players are allowed to withdraw their consent at any time, and you are required to comply with any request to relocate such play elsewhere.

Once you get your character, please stay within the bounds outlined by that sheet and the game content in general. Do not invent edgelord backstory elements (e.g.: Anything involving the content warnings listed above, any of the content specifically excluded from this game, or trauma, violence or death. This is not an exhaustive list), or introduce new plot points. This is important for the integrity of the game. If you invent something new and abandon content written for the character, you leave the other characters involved in that content out in the cold. This is also crucial for the safety of your fellow players since they didn't get a chance to opt into or out of the new content. If something isn't working for you, get in touch with the GM team as soon as possible, and we'll work together to find a solution.

\section{Safety}

\end{document}
