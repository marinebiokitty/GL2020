\documentclass[char]{GL2020}
\parindent=0pt
\begin{document}
\name{\cAntiChup{}}

You are \cAntiChup{\intro}, citizen of the \pTech{} and at the peak of your career at 63. As \cTechGod{}'s Grace, you are the High \cAntiChup{\Cleric} of the Temple of \cTechGod{}, overseeing the faith and the faithful of your nation, as well as the \pTech{} spy network. You use both overt and covert power to keep the beating heart of the nation under your finely gloved thumb.

Born to a wealthy and religious family, you knew you wanted to become a \cAntiChup{\cleric} from a young age, and your dream was soon realized. As a young \cAntiChup{\cleric}, your somber duty at the temple was approving or rejecting new inventions (a task you have not had to engage with for years at this point). Many ideas were too grandiose to be practical — impossible to mass produce or so revolutionary that they could shake the foundations of your carefully curated utopia to which change must be introduced gradually. The inventors seldom understood your rulings, trapped in their tunnel vision and unable to see the big picture. On one memorable occasion twelve years ago, you were physically attacked over rejecting a magically powered icebox for which it would have taken decades to create the manufacturing capability to produce at scale. Honestly, unless an inventor has the money and connections to understand what the manufacturing capabilities of the \pTech{} are, most of what they create are foolish pipe dreams. True, you did approve a somewhat similar box a few months later, but that inventor had the resources and money to know what was actually feasible to mass produce. As you rose through the ranks of the priesthood, you became increasingly involved in the temple's spy network, seeing it as another essential tool in maintaining social stability. After serving with distinction for decades, you at last rose to the position of \cTechGod{}'s Grace, free to shape Temple policy to best serve the will of \cTechGod{}. The only worldly authority you answer to now is your fellow High Council members.

This weekend you'll have the honor of attending the \pSchool{} as an Advisor, and while you've held this role occasionally before when your \textbf{many} other duties allow, this time feels especially momentous as a personal success. Your own protégé, \cScholarship{\full}, is attending as one of the students honored with the sacred privilege of voting to direct the Storm. It is through your decades of hard work and cutthroat determination that you now find yourself at this most auspicious event, not only as the voice of \cTechGod{} \cTechGod{\themself}, but also in such a position to raise up this most promising young devotee in your own image, empowering \cScholarship{\them} to continue your work and further your power both nationally and abroad.  

\cScholarship{} came from nothing, discovered at an orphanage, but you saw their religious fervor and especially great magical talent. You have made sure that \cScholarship {\they} never wanted for anything, and had the best schooling and best opportunities to meet important people. Of course you asked \cScholarship{\them} to do some things for you that you couldn’t do, but you told yourself this was all part of \cScholarship{\their} training. Besides, you did always watch carefully to make sure nothing truly dangerous ever happened.

Ever since \cScholarship{} began attending the \pSchool{}, you've been obligated to send \cScholarship{\them} only on missions that \cScholarship{\they} could complete on  \cScholarship{\their} own. Now that you are here in person, you can collaborate directly and provide in-person guidance in ways not usually available to you. You'll do \emph{whatever} it takes to ensure that this weekend is a success, and you couldn't be more thrilled at the prospect of such victories.

Your relationship with \cScholarship{} is somewhat complicated, however. \cScholarship{} attends the school by the grace of a Temple scholarship you arranged personally, and so will of course do your bidding. Originally, this was all there was to your relationship, but you uncharacteristically began to develop a certain fondness for your young pupil. \cScholarship{\Theyare} bright, clever, ambitious, and you truly believe \cScholarship{\they} will go far in life. Unfortunately, you only realized this recently, after the latest undertaking you forced \cScholarship{\them} through that you're sure \cScholarship{\they} found unpleasant.  

Technically this particular endeavor started about two years ago, when you decided that, to expand your spy network, you should likely pretend to be an ally to one of the head diplomats of the \pFarm{}, \cEvil{\full}. Though it took some time, you now consider \cEvil{} an actual ally, or at least a person whose brain works in a similar way as yours. You both agree that this war cannot end prematurely, and especially that the \pFarm{} and the \pTech{} must end up on top, and to this end have been brainstorming potential actions. \cEvil{} actually was the one who suggested the mission you sent \cScholarship{} on, though you have not let \cEvil{\them} know you took \cEvil{\their} idea and turned it into reality.  

You had sent \cScholarship{} on a dangerous mission two months ago to ``borrow'' the \iScythe{} from the \pSc{} library and use it to assassinate the Avatar of \cEbbFull{\full}. The mission was a resounding success, giving the \pTech{} the upper hand in the war by eliminating the connection between the \pShip{} and one of their twin Goddesses; the fact that it removed the attunement of the \iScythe{} from the \pFarm{} nation was a small price to pay. Relics can always be reattuned, but the opportunity to deal a crippling blow to one's enemies only comes along so often. Sometimes you feel a small tinge of guilt about this, as you have in the past actually corresponded with one of the main priests of \cEbb{}, \cEbbPriest{\full}, about the creation of the world, but you have assured yourself it was necessary. You still have not told \cEvil{} about the success, but you hope to tell \cEvil{\them} in person this weekend.  

But there was a greater price to you personally, which didn't become apparent until later. After the mission, you intercepted one of \cScholarship{}'s outgoing letters, and therein learned that \cScholarship{\they} want\cScholarship{\verbs} to leave your tutelage. You expected to feel nothing but vague annoyance at another tool breaking — but from the degree of hurt you felt, you realized that you had developed quite a distressing degree of \cAntiChup{\parent}ly feelings toward \cScholarship{\them}. You want to see \cScholarship{} happy. What you can’t understand is how \cScholarship{\they} could be happy with a different mentor — no one can smooth \cScholarship{\their} way to the upper echelons of the church like you. No can could give \cScholarship{\them} half the power that you can. Perhaps \cScholarship{\they} just need\cScholarship{\verbs} a serious talking to, to come to \cScholarship{\their} senses?  

Perhaps you can give them more allowance and encouragement to work on \cScholarship{\their} own projects? In going through the mail, you noticed that \cScholarship{\they} also \cScholarship{\have} become fascinated by the creation of the world research \cEbbPriest{} cares so much about, a thread you mentioned off hand but which has taken hold of \cScholarship{\their} imagination. You could always introduce \cScholarship{\them} to \cEbbPriest{} to win back some favor. Regardless, starting any such conversation would be challenging, as you are loath to admit that you've been going through \cScholarship{\their} mail.  If ultimately \cScholarship{} chooses the suboptimal path, and insists on leaving you, all is not lost. \cScholarship{\Their} new sponsor will need to convince you that \emph{they} are worthy of \cScholarship{} before you will agree to sign away \cScholarship{}'s scholarship documentation.

Regardless of how that resolves, there is work for you and \cScholarship{} to do in the meantime. The first order of business is the all important matter of the vote to direct the Storm. Ideally, \cScholarship{} acquires and casts the most Voting Stones. Your protégé is a compelling candidate, but the \cHeir{\offspring} of the mighty \cHeir{\formal} family, \cHeir{\full}, has generations of influence behind \cHeir{\them}. While it is crucial that the Storm be sent once again to the \pShip{} so that the treaty protecting your nation from the Storm — and starvation — is upheld, having gained the most stones to cast will only benefit \cScholarship{}’s future. It is your sacred duty to protect your people, and \cTechGod{}'s will that you should succeed by any means necessary. And while you're pulling the strings in your own nation, it would also be of great benefit if you could similarly find ways to ensure that the rest of the students from the \pFarm{} also uphold their nation's parts of the shared treaty.

While the \cHeir{\formal} family can be a little inconvenient sometimes, simply due to the amount of influence they wield, you do not like \cTechStar{\full}, full stop. While you have not personally been in charge of approving or denying any individual technology in almost a decade, \cTechStar{}’s current position is related to a troublesome technology. The ``VidCom’’ devices that \cTechStar{\they} created won \cTechStar{\them} the annual TechStar competition this past year. This has given \cTechStar{\them} both a seat on the High Council, and entrance to the \pSchool{}. The council was called upon to vote whether these ``VidCom’’ devices should be allowed to go into mass production, or be reserved for use in the ongoing war. Naturally the council voted to keep it for the war for now. The vote was closer than you would have preferred, the decision being made by only one vote — and \cTechStar{} took quite a bit of umbridge at the decision. It frustrates you that someone so clever cannot think past the end of \cTechStar{\their} own nose to the disastrous consequences of the devices being made publically accessible. They would immediately fall into the hands of your nation’s enemies! 

\cAmbition{\full}, on the other hand, is a student whose acquaintance you would quite like to make. While \cAmbition{\theyhave} no immediate claim to fame, you have reason to believe that \cAmbition{} and \cScholarship{} are becoming friends. You’d like to trust your protégé’s judgment, but since \cScholarship{\theyhave} gotten close with \cTechStar{} as well, you feel the need to meet this person yourself. \cAmbition{\They} could be useful to you, or \cAmbition{\they} could be trying to use \cScholarship{}, and that you simply cannot allow.

In search of allies beyond the borders of the \pTech{}, you wrote to \cPrince{\full} relatively recently. \cPrince{} is the youngest child of \cQueen{\full}. \cPrince{\Theyhave} a burning desire to oust \cPrince{\their} older siblings and take on the title of heir apparent \cPrince{\themself}. You expect \cPrince{\them} to ask you to agree to a formal declaration of support this weekend. You are perfectly prepared to do so, should \cPrince{} be able to make it worth it to you. You are curious what \cPrince{\they} will offer you to try to court your support.

To complicate matters, if you’re here, then surely the other two nations have sent spies as well, if not the spymasters themselves. Finding and neutralizing them is important for maintaining the superiority of the \pTech{}.

Speaking of the will of \cTechGod{}, as a \cAntiChup{\cleric} of \cTechGod{}, you have additional duties to perform — things like spiritual guidance and leading rituals. This task is made more challenging by your counterpart, \cBeetle{\full}. \cBeetle{} is \emph{technically} a \cBeetle{\cleric} of \cTechGod{}, but has clearly gone soft. \cBeetle{\Theyare} slow to act, and slow to pass information back to the church and \cBeetle{\their} home country, to which \cBeetle{\they} owe\cBeetle{\verbs} allegiance. You do not expect to be able to rely on \cBeetle{} for all but the most trivial duties. You expect to need to take on much of the burden related to preparing the Ritual to Control the Storm, even though that task \emph{should} fall to the teachers. Unless it is something that \cScholarship{} would enjoy being in charge of?

This leaves the most important role of a \cAntiChup{\cleric} of \cTechGod{} to you: striking against any and all blasphemy against \cTechGod{}. Being your nation's spymaster helps with this endeavor. The worst form of blasphemy is that which is perpetrated by the \pGoaties{} — an infamous cult that has been insidiously growing in the shadows over the last decade. These heretics would upset the natural balance by elevating their deity \cGenesis{}, one of many insignificant minor gods, to the status of a fifth patron Deity on par with the Gods of the three nations. Their blasphemies do not end there, and include circumventing proper channels of authority, ignoring Intellectual Property rights, scam artistry, and the unregulated magical activation of technology — undermining the pillars of civilization on which the \pTech{} was built. The \pGoaties{} are the power behind the notorious Black Market, an international scourge and the deepest sign of their rejection of \cTechGod{}. More times than you care to count, you've read reports of your own people being cursed or maimed on the battlefield by weapons developed by their countrymen, stolen from the \pTech{} and sold illegally to your nation's enemies. You plan to use the opportunities afforded you this weekend to hunt down and capture the leadership of the \pGoaties{}, and thus remove their unholy influence from your meticulously crafted world.

You have pieced together from various spy reports that at least one of the \pGoaties{}, will be present at the \pSchool{} this very weekend. You've also received intelligence that one of the \pTech{} teachers at the \pSc{} is involved in the Black Market somehow, though this could just be that they are purchasing things from, or supplying the Black Market, not that they are a core part of its operation. You dearly hope that none of your fellow country-folk have turned their back on \cTechGod{} so thoroughly. Could this be why \cBeetle{} is so useless as a \cBeetle{\cleric}? That can't be a coincidence. If you find a member of the \pGoaties{}, it would be best to capture them and bring them back to the \pTech{} for interrogation. Dead people don't talk, and leads released into the custody of other nations have a distressing tendency to escape before you get your turn.

Lastly, as an advisor, you are technically supposed to be involved in the treaty negotiations taking place this weekend between the three nations. Honestly, you are so busy with more important things that the distasteful business of negotiating with the enemy is not a priority for you. \cDiplomat{\full} can probably handle it without you, though you must admit that staying informed on international dealings is useful. And naturally, as a \cAntiChup{\cleric} of \cTechGod{}, you have a responsibility to protect the \pTechies{}. This weekend you suspect this duty may center on \cHeadScientist{} and \cAssistantScientist{}, who are supposed to present the final \pTech{} findings on how to end the Storm. If people don’t like the answers your scientists have, they might try to attack them. As if hurting someone could change facts. 

In order to help you manage this, and honestly, to make sure that everyone can see how brilliant \cScholarship{} is, you have asked \cScholarship{\them} to accompany you and, when you have other pressing matters to attend to, take over some of it. Maybe giving \cScholarship{\them} such an incredible opportunity and displaying such trust in \cScholarship{\them} will help with the whole leaving you situation.

Oh \cTechGod{}, \cDiplomat{} is here. Right, you knew that was going to happen. This is fine; everything is fine. Your history with \cDiplomat{} is a complicated one. Some twenty five odd years ago, the two of you were in a romantic relationship. This was back before either of your careers really took off, and it was those careers that ultimately ended the relationship. The split was mutually agreed upon, though not a happy thing. You both needed the time and energy to focus on climbing your respective ladders. And those ladders were too far apart to be able to assist each other effectively. Still, you sometimes wonder what might have been. And as you look at \cDiplomat{\them} across the room, you see that \cDiplomat{\they} too seem\cDiplomat{\verbs} to carry the weight of \pEarth{} on \cDiplomat{\their} shoulders. You don’t have the time to rekindle a decades-dead romance this weekend. You don’t. You really don’t. But saying ``hi’’ couldn’t hurt anything, right? Besides, you two are crucial allies in protecting the \pTech{} from outside threats. There’s no getting out of working in close proximity this weekend.

A less awkward but still pleasant encounter you expect to have this weekend is with \cHedonist{\full}, a pen pal of yours whom you have never had the opportunity to meet in person. Years ago, as part of your priestly training, you were tasked with striking up a correspondence with a member of the clergy of another nation — \cHedonist{}. The two of you have had many lively theological discussions over the years, and you both deeply value the traditions that form the foundations of the Church. You are looking forward to continuing those discussions in person, and are also pleased to know there is at least \textbf{one} other cleric you can count on this weekend.

This weekend will be a whirlwind. You expect to be kept quite busy by your investigations, negotiations, and supporting \cScholarship{}. But just like in all things before now, you will not fail. You are backed by not only \cTechGod{} — but also the collective will of the entire \pTech{}.

\begin{itemz}[Goals (in roughly descending order of importance)]
    \item Find the \pGoaties{}, stop whatever they currently have planned, accumulate evidence of their myriad past crimes, and bring them to justice for it all. Shut down their unholy Black Market.
    \item Help \cScholarship{} accumulate as many votes to control the Storm as possible. By any means necessary, ensure that the Storm be sent to the \pShip{}, to preserve the treaty between the \pFarm{} and the \pTech{}.
    \item Decide whether you will support \cPrince{}’s bid for the throne, and if so, for what price. 
     \item Find and neutralize (e.g. by unmasking) any enemy spies you can, \textbf{especially} the \pShip{} spy here this weekend.
    \item Convince \cScholarship{} that \cScholarship{\their} best chance of a successful life and career, and therefore happiness, is by remaining under your tutelage. If \cScholarship{\they} will not see reason, at least vet \cScholarship{\their} potential new mentor with great care to make sure they are worthy.
    \item Keep an eye on the preparations for the Ritual to Control the Storm, and assist as necessary given your role as High \cAntiChup{\Cleric} of \cTechGod{}. 
\end{itemz}

\begin{itemz}[Notes]
    \item The last time you were present at the school for the Time of Deciding was 9 years ago.
    \item International espionage is a thriving trade these days, and given the pivotal events happening this weekend, you would not be at all surprised if spies from all three nations are present.
    \item The Order of the Black Crocus is, or rather was, a secret international law enforcement entity with jurisdiction across all three nations. As far as you know, it splintered into factions when the war started, each faction subsumed by the intelligence services of its corresponding nation. You certainly did that with the \pTech{} branch. You have mixed feelings about the group. On the one hand, they were constantly meddling in your affairs as Spymaster, but on the other, they did much to curtail the Black Market.
    \item Any magitech device made by \pTech{} citizens must be empowered by clerics of \cTechGod{} according to Church doctrine. Any device that is activated by another means is both illegal and blasphemous. The much simpler magical items crafted by the \pShippies{} do not need to be empowered in this manner.
\end{itemz}

\begin{contacts}
    \contact{\cScholarship{}} Your protégé, who you think of almost as your \cScholarship{\offspring}. \cScholarship{\Theyare} one of your most valuable tools to accomplish your goals this weekend. You just hope that \cScholarship{\they} can see how the tasks you set \cScholarship{\them} on are as much for \cScholarship{\their} benefit as yours, or your nation’s.
     \contact{\cPrince{}} A powerful potential ally who wants your support for \cPrince{\their} bid for the throne of \pFarm{}. You plan to discuss what \cPrince{\they} can offer you in return this weekend.
    \contact{\cBeetle{}} The other \pTech{} \cBeetle{\cleric} here. \cBeetle{\They} seem\cBeetle{\verbs} to be ignoring \cBeetle{\their} responsibilities toward the \pTech{}. Hopefully it's no more than complacency borne of being too long away from the \pTech{} and working daily with \cTechGod{}'s people.
    \contact{\cDiplomat{}} Your advisor counterpart at the negotiating table. Where you represent the Temple, \cDiplomat{\they} represent\cDiplomat{\verbs} the government. \cDiplomat{} is a highly competent diplomat, having forged the original treaty six years ago, and an old lover.
    \contact{\cEvil{}} An ally of yours for the past two years with whom you brainstormed the plot to assassinate the Avatar of \cEbb{}, though as far as you know, \cEvil{\theydo} not know you actually pulled it off yet.
    \contact{\cTechStar{}} The hopelessly naive inventor of the dangerous “VidCom” technology, which you voted along with a majority of the High Council to restrict to military applications only. You suspect \cTechStar{\they} may resent you for it.
    \contact{\cHedonist{}} An old pen pal of yours, an intelligent and scholarly \cHedonist{\cleric} of \cFarmGod{}. You are looking forward to meeting in person for the first time this weekend.
\end{contacts}

\end{document}






