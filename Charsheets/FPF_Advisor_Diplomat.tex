\documentclass[char]{GL2020}
\parindent=0pt
\begin{document}
\name{\cDiplomat{}}

You are \cDiplomat{\full} (\cDiplomat{\they}/\cDiplomat{\them}), foremost diplomat of the \pTech{}, attending the \pSchool{} on this important weekend as an Advisor at age 58. At the core of your success is a heart hard enough to embrace true utilitarianism. When you speak, others listen, because of your tremendously successful track record, and the weight of the \cDiplomat{\formal} name. This weekend there is nobody attending with a more solid position than your own — or at least that's the impression you'd like to give off. Privately though, you are feeling stretched thin trying to keep everything from falling apart. 

You've accomplished great things over your storied career, but the “End of Suffering” treaty is your crowning glory. Eight years ago, you approached the \pFarm{} and cemented an alliance (to your frustration, it was only signed with a stipulation of secrecy), temporarily turning your two nations against the \pShip{} for the ultimate benefit of all. This new treaty dictated that the Storm would henceforth always be directed towards the \pShip{} until the Storms are ended permanently. The \pFarm{} granted significant trade deals to your own nation on food and goods, and in exchange the \pTech{} has devoted its resources and best scientists, including \cHeadScientist{\full}, to developing technology that will permanently eradicate the Storm itself. Unfortunately, you were unable to convince both governments to make the treaty public until much too close to the Time of Deciding of six years ago.

Worried that the students at the \pSchool{} would not understand the world changing potential of your treaty, you conspired with \cEvil{\full}, a ruthless courtier from the \pFarm{}, to ensure that the Storm would be sent to the \pShip{} that year, no matter how the students voted. According to \cEvil{}, \cHeadScientist{\full} was a crucial part of the plan, though you don’t know why. One advisor slot is always given over to an ``up and coming'' person, to reinforce the dream of upward mobility within the \pTech{}, and so you made sure \cHeadScientist{} came along. Afterwards, on recommendation of \cEvil{}, you rewarded \cHeadScientist{} with a promotion to head scientist of the project to end the Storms; this had the added bonus of placing \cHeadScientist{\them} somewhere you could keep an eye on and protect \cHeadScientist{\them}.

But no silver lining is without its cloud. After the ritual to control the Storm was completed that year, the 12 students adjourned from the Great Hall to the Student Lounge for a private toast without the teachers and advisors breathing down their necks. This was a common occurance at the end of such a stressful weekend, but they usually returned quickly. After nearly an hour of waiting, you, Principal \cPrincipal{\full}, and \cMusic{\full} (one of the teachers) decided that you had better go check on them. What you found was 11 dead bodies. But where was the twelfth student, \cKidScientist{\full}?. While your two companions stared in horror, your mind worked furiously. This must have been \cEvil{}'s doing, to tie up loose ends and make sure the students weren't around to protest that their votes had been altered. Damn \cEvil{\them} for taking such a drastic action without consulting you! Even worse, \cHeirSibling{\full}, your much beloved nibling and heir to the powerful family name, was among the dead. This was both a devastating tragedy for your family and a PR disaster. No one could ever be allowed to find out that you were involved. 

You were shaken from your ruminations when \cPrincipal{} and \cMusic{} proposed keeping the fact that \cKidScientist{} was not among the dead a secret. They believed that \cKidScientist{} must have survived and, not knowing who to trust, fled for \cKidScientist{\their} life. Keeping it out of public knowledge that \cKidScientist{\they} escaped would avoid anyone continuing to hunt \cKidScientist{\them}. Little did they realize that one of those would-be hunters stood in their very midst. Curse \cEvil{} for not being present that weekend! The first chance you got to catch \cEvil{} back at home, you told \cEvil{\them} that \cKidScientist{} had survived. After you chewed \cEvil{\them} out for the huge mess \cEvil{\they} made, the two of you agreed that \cKidScientist{\they} must be found and silenced. Unfortunately, all your efforts to find \cKidScientist{\them} have been fruitless thus far. Perhaps whatever \cEvil{} did just took longer to affect \cKidScientist{}, and \cKidScientist{\they} didn't survive after all? For what little deniability it might provide you, you refused to hear the details of how \cEvil{} managed to rig the vote and kill the students.

Regardless of the lack of proof, the revelation of the “End of Suffering” treaty corresponding approximately with the events of six years ago, and your presence on the scene, has since caused you to fall under suspicion. Not that anyone would dare show their suspicion to your face, but\ldots{} you swear you can \emph{feel} them thinking it. You'll need to keep a sharp eye out for anyone looking into the deaths of the students six years ago. You would surely be implicated by whatever such an investigation uncovered. Of course, you'll want to seemingly support any investigation, as it's the ``right thing to do’’ — but behind the scenes, you'll need to see that it goes nowhere. If the worst should happen, you can always throw \cEvil{} under the mine cart, but \cEvil{\theyare} a useful tool you would prefer not to have to expend if you can avoid it. You'll also have to stay vigilant for the possibility of \cEvil{} betraying you and \cHeadScientist{} to cover up their own guilt if the plot is uncovered. What a mess.

All of this of course begs the question: \emph{why did you do it?} The short answer is ``because it was necessary.’’ The more complicated answer is because \cTechGod{} tasked you with it. Some ten years ago or so, you started having a recurring dream. Not every night, but more nights than not. In the dream, \cTechGod{} sat at \cTechGod{\their} workbench across from you, the pleasant warmth of \cTechGod{\their} forge on your back. The two of you would speak of inconsequential things for a while, but eventually the \cTechGod{\deity} would always grow serious. \cTechGod{} told you that the Storm was not part of the original plan for \pEarth{}, and tasked \textbf{you} with making it stop — with fixing the Gods’ creation. No pressure.

Doable? For someone of your skill, certainly. Easy? Not at all. But \cTechGod{} had to go and make it one layer \textbf{more} difficult. \cTechGod{\They} forbade you to speak to anyone of your dream, until you had the solution in hand. Which meant you were trapped in that oh so enviable position of being tasked with lying for the greater good. There’s precious little more sacred to \cTechGod{}, and you are nothing if not a pious \cDiplomat{\person}. At least you could hold on to the knowledge that \cTechGod{} had chosen \textbf{you} to do this. \cTechGod{\They} believed in you. So you got to work. And ended up with 11 dead students, including your own nibling, and a war. But your scientists are on the cusp of the answer you’ve sought for ten years. And then maybe, just maybe, the world will understand why you had to do it.

You still have the same dream at least once a week.

While your shoulders are heavy with the burden you carry, your heart is as light as it can be for the upcoming nuptials of your \cHeir{\nibling}. \cHeir{\full}, current heir to the \cHeir{\formal} household and younger \cHeir{\sibling} of the late \cHeirSibling{}, is to be married to \cChupStudent{\full}! \cChupStudent{} is the \cChupStudent{\offspring} of a highly regarded noble family from the \pFarm{}. \cChupStudent{}'s family is close to the crown and holds a crucial territory adjoining the border with the \pTech{}. Getting them to agree to marrying off their second born \cChupStudent{\offspring} to the \cHeir{\formal}s was one of your first strokes of political genius some 17 years ago. With this connection forged, the \cHeir{\formal} influence expanded into the \pFarm{}. A true force to be reckoned with. A force for good, regardless of what your naysayers try to convince themselves of.

The only wrinkle is that \cHeir{} has been fighting the betrothal since almost day one. You've known \cHeir{} all \cHeir{\their} life, and \cHeir{\theyare} constantly complaining about having no control over \cHeir{\their} choices or dragging \cHeir{\their} feet around family business. It's exasperating, to say the least! But the time has come at last for the marriage. Both families have agreed to a surprise wedding for Monday, and you have delivered letters just this morning to both students, informing them of such. 

You didn’t expect \cHeir{} to be thrilled about the news, but something about the way \cHeir{} reacted to the news left you with a nagging feeling that you were missing some part of the picture. At the very least, worrying about the wedding should help keep \cHeir{} distracted from digging up any uncomfortable truths about the death of \cHeirSibling{}.

With the wedding imminent, \cHeir{}'s \cFaledonParent{\parent} \cFaledonParent{\full} has asked you, in your frequent role as \cFaledonParent{\their} representative, to perform a brief ritual this weekend. Now that \cHeir{} has come of age and will be graduating soon, you are to pass on the Heir's Seal signet ring so that \cHeir{} can begin conducting family business under \cFaledonParent{}'s guidance once school is finished. This is a task you’ve been dreading for six years. \cHeirSibling{} was always better suited to the role than \cHeir{}. And that stroke of genius, marrying \cHeir{} to a \pFarm{} noble looks a lot less clever when it will give the \pFarm{} such influence in the government of the \pTech{} as the \cChupStudent{\spouse} of one of the members of the High Council.

You are left with three awkward choices. You could do as you’ve been asked, changing nothing. This leads to the \pFarm{} having more influence than you’d like. You could support \cHeir{} in breaking off \cHeir{\their} betrothal, but still pass on the signet ring to \cHeir{\them}. This would upset both the \cChupStudent{\formal} and \cHeir{\formal} families. Or, you could swap in \cAmbition{} for \cHeir{}. 

\cAmbition{} is a \cHeir{\formal} by blood. \cAmbition{} is your child, biologically. But, at the time that \cAmbition{\theywere} born, your older \cFaledonParent{\sibling}, \cFaledonParent{} already had two children of \cFaledonParent{\their} own. As the head of the family, you had to appeal to \cFaledonParent{} for the funds to support \cAmbition{}, which \cFaledonParent{} denied, claiming that a third \cHeir{\formal} for the next generation was unnecessary. Worse still, your romantic partner’s family was working class, poor even. \cFaledonParent{} declared them to be too low class for you to be allowed to marry your child’s other parent. While you were furious with your \cFaledonParent{\sibling}, there was little you could do. The head of the \cHeir{\formal} family has \emph{always} controlled the fate of their family members. 

Reluctantly, you surrendered \cAmbition{} to the adoption system and resigned yourself to never being a direct part of \cAmbition{}’s life. The \cAmbition{\formal} family ultimately adopted \cAmbition{}. (You may have had some influence here.) They were upper middle class, and have done admirably for their adopted \cAmbition{\offspring} given their circumstances. Still, to the best of your ability you have kept an eye on \cAmbition{}, and have supported the \cAmbition{\formal} family where you could. There is a mountain range between the middle class \cAmbition{\formal}s and what \cAmbition{} should have had as \cHeir{\formal} blood. 

Taking care of \cAmbition{} from afar became somewhat easier after you arranged the betrothal between \cHeir{} and \cChupStudent{}, as that crucial alliance brought you back into your \cFaledonParent{\sibling}’s good graces. With restored access to the family’s purse, you’ve anonymously supported \cAmbition{} from afar. You may or may not have pulled some strings to help ensure \cAmbition{\theywere} accepted to the \pSchool{} and that the \cAmbition{\formal} family could afford the tuition. And look what \cAmbition{\theyhave} become.

For better or worse, \cAmbition{} and \cHeir{} have become friends in their time at the school. Neither knows that they are cousins, but they get along like two peas in a pod. \cAmbition{} has even been over for dinner on several occasions. Talk about awkward. Not that \cFaledonParent{} seemed to have any sympathy for your situation, so you suffered through the dinners, and fought about it with \cFaledonParent{} after the youngsters went to bed.

You could tell \cAmbition{} of \cAmbition{\their} connection to the \cHeir{\formal} family, and recognize \cAmbition{\them} instead of \cHeir{} by giving \cAmbition{\them} the signet ring. It would make \cFaledonParent{} furious, but \cFaledonParent{\they} couldn’t actually undo it. It is unclear how the \cChupStudent{\formal} family would feel about it, since they agreed to the betrothal long before \cHeir{} was set to inherit the full influence of the family.

You try to keep arguments with \cFaledonParent{} to a minimum, as you know how much you depend on \cFaledonParent{\their} good graces. But you are growing tired of constantly bowing to \cFaledonParent{}’s suboptimal decisions in public, even if you argue in private. Take a new device created by this year’s TechStar, \cTechStar{\full}. \cTechStar{\Their} invention of the VidCom, a device that allows you to talk and see another person over great distances, would be incredible for both building morale and connection across the \pTech{}.  But \cFaledonParent{} couldn’t see the profit or benefit for \cFaledonParent{\them} and voted against allowing it for broader public use, and nothing you said would move \cFaledonParent{\them}. You know \cTechStar{} will be here this weekend. Hopefully you can avoid the topic - you have no tricks left to try to convince \cFaledonParent{} to change \cFaledonParent{\their} mind.  

As another important matter, you have been honored with bearing one of your nation's Relics, the \iMirror{}. While it may or may not ultimately be used for the Ritual to Control the Storm, as there is an alternate Relic from your nation, the \iLariat{}, in the library, it is still your task to see that the Relic you bear doesn't fall into the wrong hands or be misused. Even more importantly, it is your responsibility as an Advisor to ensure that the right students get the most votes to direct the Storm, and that the Storm is sent to the \pShip{}. Naturally, you hope that \cHeir{} and \cAmbition{} will step up as leaders among the students, as well as charming many teachers and advisors into giving them their voting stones

Returning to the matter of treaties, you've been informed recently that certain parties, in particular \cHeadDiplomat{\full}, premier Diplomat of the \pShip{}, are trying to broker a ceasefire to end the war between the nations. While this effort is commendable in principle, in reality this is likely to result in disaster, as any new treaty ending the hostilities would inevitably be bought at the price of your alliance with the \pFarm{} being abolished — which would mean that the Storm could again be pointed at the \pTech{}. Such an occurrence would be devastating for your nation, as there have been no preparations made to receive the Storm. Such expensive and time consuming work was stopped as part of the alliance you brokered, so that the resources could be diverted towards the more permanent solution of ending the Storm for good — painstakingly constructed technology which would be reduced to slag and rubble if the Storm were to strike your homeland now. You are \emph{so close} at this point. Your scientists are expecting the results of the very last experiment to arrive Friday. You aren’t even sure if they will have the answer when they arrive at the \pSchool{}, or if it will arrive by missive. (Either way, you expect them to fill you in as soon as they know.) The \pFarm{} is similarly unprepared to receive the devastation of the Storm cutting a swath through significant portions of their vital crop land. Such a thing would inevitably cause their nobility to hoard their supplies, and this would in turn bring famine to your own people, as well as the common folk of the \pFarm{}. 

It has been your life's work to take the long view for the betterment of all. The short sighted fools around you cannot see that you are ultimately bringing that protection to all of \pEarth{}; it just takes a little sacrifice along the way. Your people have thrived in the safety you have earned them, and the casualties of war are paltry in comparison. There is no way you are going to let \cHeadDiplomat{} or anyone else broker a ceasefire that would ultimately threaten your people. You cannot afford to let the other two countries sit down and talk without you, though! That way lies only pain and suffering for the \pTech{}. With two scientists and a Cleric as your fellow Advisors, you’ll have to call in any backup you need deliberately. You do not expect any of them to prioritize joining you at the negotiating table (though you would welcome such support, and happily encourage the interest of any \pTech{} students who express it.) If the other negotiators insist on a new treaty, and you cannot prevent it, you can at least ensure that you are named as the Mediator, or are known as the primary author. Ultimately, though, you hope to end the war in other ways. Namely by ending the Storms permanently.

 \cHeadScientist{\full} and \cHeadScientist{\their} assistant \cAssistantScientist{\full} are scheduled to present their research into ending the Storms on Saturday Morning. You scheduled this presentation months ago, to provide the team a little extra motivation to finalize this project that has already dragged on for more years than you hoped. A permanent end to the Storms, and the revelation that the sacrifice of the \pShip{} was justified, will make the need for a new treaty about where to send the Storm obsolete. You intend to do everything you can to make sure the presentation goes flawlessly. As for what they will be presenting, well, their efforts have not been for nothing: Theorizing that the Storms were coming from somewhere outside the known world and could be stopped permanently by reflecting the Storm back to its source, \cHeadScientist{}’s team constructed (at enormous expense) a network of relay stations along the Bones of the World that allows the Storm to be sent to other planes of existence. The user interface is a portable Lightning Rod that must be used in the preparations for the Ritual to Control the Storm. \cHeadScientist{} has also reported that the Storm is drawn to the largest concentrations of magical energy, which explains why it cannot be sent to uninhabited regions of \pEarth{} — most magic resides in people and the things they build, so heavily populated areas are like magnets for the Storm’s wrath. The last piece of the puzzle is the Storm’s point of origin, and you desperately hope your scientists receive that final result in time for their presentation.

But it pays to take a multi-pronged approach, and you suspect that actually ending this war will require eliminating the leader of the \pShip{}’ war efforts. \cLoud{\full}, unfortunately seen as a hero by the \pShippies{}, is a warmongering murderer bent on the destruction of your homeland and the \pFarm{} alike, and must be disposed of immediately. Conveniently enough, your old co-conspirator \cEvil{} is interested in facilitating such a\ldots{} removal. And \cEvil{\they} clearly \cEvil{\have} no qualms about getting \cEvil{\their} hands dirty. As much as you distrust \cEvil{} and are wary of \cEvil{\their} methods, you make it a policy never to set aside a useful tool — and \cEvil{} is definitely that. Ultimate good comes first, whatever the cost, and you'll do whatever you can to aid \cEvil{} in this task and ensure the success of this mission. Although you wouldn't mind doing it in such a way as to keep your own hands clean, should the plot be discovered.

If you weren’t overwhelmed enough yet, \cAntiChup{\full} will be present this weekend too. Your history with \cAntiChup{} is a complicated one. Some twenty five odd years ago, the two of you were in a romantic relationship. This was back before either of your careers took off, and it was those careers that ultimately ended the relationship. The split was mutually agreed upon, though not a happy thing. You both needed the time and energy to focus on climbing your respective ladders. And those ladders were too far apart to be able to assist each other effectively. Still, you sometimes wonder what might have been. And as you look at \cAntiChup{\them} across the room, you see that \cAntiChup{\they} too seem\cAntiChup{\verbs} to carry the weight of \pEarth{} on \cAntiChup{\their} shoulders. You don’t have the time to rekindle a decades-dead romance this weekend. You don’t. You really don’t. But saying ``hi’’ couldn’t hurt anything, right? Besides, you two are crucial allies in protecting the \pTech{} from outside threats. There’s no getting out of working in close proximity this weekend.

They say there is no rest for the weary, and that will certainly be true for you this weekend. So much to do! But you are used to juggling numerous political and social intrigues with ease, and have no doubt that your experience and guile will win out in the end. You'll ensure the \cHeir{\formal} family position remains secure, the Storm is sent to the \pShip{} for the last time ever (if the Storm can’t be stopped this weekend), and \cLoud{} is removed as an obstacle\ldots{} all while ensuring that nobody connects you with the death of the students six years ago.

\begin{itemz}[Goals (in roughly descending order of importance)]
    \item Support the work of your scientists, \cHeadScientist{} and \cAssistantScientist{}, in whatever ways they need. Their research into ending the Storms is the result of everyone's sacrifice, and you want to make sure their hard work is properly recognized and celebrated this weekend. If their research is too late to stop the current Storm, make sure it \textbf{does not} get sent to the \pTech{}, as your nation is wholly unprepared for it, and it would destroy all of the work that has been put into ending the Storms permanently.
    \item Prevent people, and especially \cHeir{}, from digging too deeply into the deaths of the students six years ago, while making it seem like you support such investigations.
    \item Support \cHeir{} and \cAmbition{} as much as possible this weekend, as they grow into their adulthood. Among other things, you’ll need to decide who to give the signet ring to, support both of their efforts in collecting Voting Stones, and possibly ensure \cHeir{} doesn’t weasel out of \cHeir{\their} wedding.
    \item Arrange to assassinate Warlord \cLoud{} with \cEvil{}'s help.
    \item Either ensure the treaty negotiations fail, or position yourself as the designated Mediator (or primary author) in any treaty likely to be ratified by the governments of the three nations.
    \item Ensure that the \iMirror{} remains safe, attuned to the \pTech{}, and under your control.
\end{itemz}

\begin{itemz}[Notes]
    \item International espionage is a thriving trade these days, and given the pivotal events happening this weekend, you would not be at all surprised if spies from all three nations are present.
    \item The Order of the Black Crocus is, or rather was, a secret international law enforcement entity with jurisdiction across all three nations. As far as you know, it splintered into factions when the war started, each faction subsumed by the intelligence services of its corresponding nation. Good riddance, as far as you're concerned. They were a meddlesome bunch.
    \item You were here at the \pSchool{} six years ago. You are one of the four people who know the truth, that only 11 students were actually killed that day, and that the twelfth, \cKidScientist{\full}, remains at large. The other three people who know are \cMusic{\full} and \cPrincipal{\full}, who were there with you, and \cEvil{}, who you told after the fact.
\end{itemz}

\begin{contacts}
    \contact{\cEvil{}} A fellow diplomat from the \pFarm{}, and your ostensible ally — talented at what \cEvil{\theydo}, but with highly questionable morals and methods. Your interests align on certain goals, but you trust \cEvil{\them} about as far as you can throw \cEvil{\them}.
    \contact{\cHeir{}} The current heir of the \cHeir{\formal} family, and your \cHeir{\nibling}. If only \cHeir{\they} had the drive of \cHeir{\their} older \cHeirSibling{\sibling}, \cHeirSibling{}, one of the students killed six years ago.
    \contact{\cChupStudent{}} \cHeir{}'s future spouse. The child of a powerful \pFarm{} family along the border with the \pTech{}. Arranging the match was a major feather in your cap, but with \cHeirSibling{} dead, things are more complicated now.
    \contact{\cAmbition{}} A child you birthed and were obligated to give up for adoption. You have supported \cAmbition{\them} from afar as much as you can. \cAmbition{\Their} increasing interest in the \cHeir{\formal} family has \cFaledonParent{} concerned.
     \contact{\cAntiChup{}} Your advisor counterpart at the negotiating table. Where you represent the government, \cAntiChup{\they} represent\cAntiChup{\verbs} the Temple. A \cAntiChup{\cleric} of considerable reputation and an old lover of yours.
    \contact{\cHeadScientist{}} The lead scientist on the project to end the Storms forever — capable, brilliant, and reliable. Played a role in the covert operation that sealed the alliance with the \pFarm{} six years ago that sent the Storm at the \pShip{} out of turn.
    \contact{\cAssistantScientist{}} \cHeadScientist{}'s research assistant. Seems competent enough.
\end{contacts}

\end{document}




