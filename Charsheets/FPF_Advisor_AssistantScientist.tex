\documentclass[char]{GL2020}
\parindent=0pt
\begin{document}
\name{\cAssistantScientist{}}

You are \cAssistantScientist{\intro}, (\cAssistantScientist{\they}/\cAssistantScientist{\them}), and 25 years old. Six years ago you were \textbf{\cKidScientist{\intro}}, a student at the \pSchool{} — and the only one out of the group of 12 students to make it out alive. Despite your new name and altered appearance, you've still spent the past six years looking over your shoulder, wondering if someone is out there looking for you to finish the job. Being back at the \pSc{} is the best chance you'll have to figure out what happened six years ago — assuming nobody recognizes you while you're at it.

Your family is from a small fiefdom in the \pFarm{} that borders the \pTech{} (squashed between the \cWildCard{\formal} and \cChupStudent{\formal} lands), but you have lived in the \pTech{} most of your life. Born of a midnight romance, your father returned to the \pTech{} when you were still just a promise in your mother's womb. Unlike many families crossing national borders, though, your family encouraged you to maintain close relationships with both sides of your lineage, and while you attended school in \pTech{}, you spent many a happy summer break in the \pFarm{} playing in the vineyards, helping with the harvest, and learning about wine — knowledge that would save your life after you enrolled in the \pSc{} (sponsored by your \pFarm{} family, of course) and participated in the Ritual to Control the Storm.

The Storm was supposed to go to the \pTech{} that year, not the \pShip{}. You, and every student present, knew that they wouldn't have prepared. All the discussions you were privy to suggested students were voting for the \pTech{}. And yet, somehow, the votes came in, and the \pShip{} were the ones targeted, leaving everyone present stunned. Your thoughts swirled. How many would die because of this betrayal? How had such a conspiracy gone undetected? What would happen to you, as a member of the class that so visibly voted to upset the balance between the three nations? But as you looked around and counted in your head, your suspicion grew. You had, of course, followed tradition and voted for the \pTech{}, and you're sure the \pFarm{} student with the most votes had too. Assuming that the students from the \pShip{} hadn't betrayed their own country, it seemed mathematically impossible for this outcome to have occurred\ldots{}

A toast is a common way for students to celebrate the end of the ``Time of Deciding.’’ Someone got it in their head that the students should do so alone — as a sort of quick break from having the teachers and advisors breathing down your necks all weekend. So you all adjourned from the Ritual Space to the Student Lounge. The mulled wine they brought out for the toast interrupted your racing thoughts with the automatic and bone-deep indignation your \pFarm{} relatives had imparted to you. This wine was far too fine to be heated and tainted with spices — this vintage was one of the most expensive and delicate vintages exported by the \pFarm{}! Anyone who could afford to buy such a vintage should know how to properly consume it! Grimacing at the strong odor coming from the carafes, you watched with distaste as your fellow students began to partake.

As you were about to hold your nose and join your friends in drinking from the blasphemously tainted wine, one of the students from the \pFarm{} (who had to be on at \emph{least} their third glass already) coughed, moved as if to put their half-empty glass upon a table, and collapsed to the floor. They began to seize, and white foam spilled from the edges of their mouth as they fought for air. Within moments, another student joined them, then another. One by one, your classmates fell. As you watched in stupefied horror, you realized in a flash of terrible insight what must have happened: Someone had tampered with the votes, and then poisoned the wine to keep the students from telling the secret. You didn't know who, you just knew you had to run. You couldn't trust anyone! You fled that terrible room where so many had just died, and used the ensuing panic in the school to make your escape without being seen.

How you got down from the floating island upon which the school rests, you will never know. The only thought you had was to flee before someone killed you too. In your confused and distraught state, you ran and ran, and knew not where. Eventually, when you recovered your senses, you realized that you were standing within the Storm itself — and that the Storm-ravaged \pShip{} was exactly where you needed to be. After all, you didn't know if anyone was pursuing you, and a country ravaged by the Storm was the last place anyone would come looking. The ferocity of the Storms shook you to your core — but by \cFarmGod{\intro}'s grace it didn't kill you. So you fled — into thunderstorms, where the rain fell so heavily it felt like you were being battered into the ground, into windstorms where the winds near lifted you off your feet. With the winds howling in your ears like the screams of the dying, all you could hear was your classmate's last moments. And the silences, in those brief moments of respite within the tempest, where everything paused for several minutes before the wind and rain and hail and thunder set back in — those silences were filled with their wordless prayers for salvation.

Eventually you wandered up the gangplank of the 4th fleet, Pink Coral Ship. You were rambling, muttering incoherently to yourself, delirious with pneumonia and exposure. The crew nursed you back to health, and when you were well enough, you hesitantly told them who you were and what had happened. They understood immediately that this was some plot, and that you were as much a victim as they. Those months you spent aboard Pink Coral Ship are hazy to you now, filled with odd gaps which you attribute to your delirium and trauma, but you know you felt truly welcomed and cared for.

You weren't born to ship life, though — it fit ill on you. It pained you to leave your rescuers, but you knew that your story didn't end this way. You had to fix it. You had to work your way back into the \pFarm{} or the \pTech{} and find out what had really happened during the Betrayal. Your new friends on the Pink Coral Ship were supportive the whole way. They helped you change your identity, obtain forged documents from the burgeoning Black Market, and remake you into \cAssistantScientist{\full}, a promising young \pTech{} scientist who worked \cAssistantScientist{\their} way up from the middle class through hard work and exceptional scholarship — the identity you still go by to this day. 

You can never repay them for their kindness to you, but you still do what you can. As Assistant Scientist to \cHeadScientist{\intro}, you have access not only to your own research pertaining to magic and the ending of the Storms, but to all kinds of magical tech destined for the war efforts. A little magical nudge here or there, nothing detectable of course, is really all it takes to ensure lengthy and costly repairs once the sabotaged tech gets to its final destination. You even began feeding intelligence to your \pShip{} contacts, who eventually connected you with their nation’s spy network, who you have been working with for two years. One of your secret contacts, \cBunker{\intro}, will be at the Time of Deciding this weekend, so you'll have the opportunity to report to \cBunker{\them} directly.

But your time working and living under your new identity hasn’t all been espionage and sabotage. You’ve made some genuine friendships too. Even though you hate the Betrayal and whoever was responsible for it, you feel some sense of loyalty to \cHeadScientist{} and \cDiplomat{\intro} after all you've worked through together and really do want to see the Storms end permanently. \cHeadScientist{} is truly brilliant, has always treated you with respect and kindness, and seems to want to do what’s best for \pEarth{}. And \cDiplomat{} has been a tireless mentor and supporter to both you and the project as a whole. Of course, it is suspicious that both were there at that terrible Time of Deciding, but you can’t deny that both truly seem to believe in the work and in you. If they end up having been involved, it would break your heart.

You’ve also made a couple of friends closer to your own age, \cHeir{\intro} and \cTechStar{\intro}. The former you met at \cHeir{\their} \cDiplomat{\auncle} \cDiplomat{}’s parties a few years ago, where the two of you immediately hit it off. You know that \cHeir{} is the younger sibling of your friend and classmate \cHeirSibling{\intro} who was brutally murdered along with ten others others six years ago, and \cHeir{\they} reminded you so much of \cHeirSibling{} that you couldn’t help but relax around \cHeir{\them} and drop a bit of the persona you so carefully maintain for your cover identity — especially the few times you have visited \cHeir{\them} on your own. You are looking forward to seeing \cHeir{} this weekend. \cTechStar{} you met more recently, after \cTechStar{\their} victory in the annual TechStar Competition catapulted \cTechStar{\them} to fame and a seat on the High Council. The two of you have met a few times at social functions and exchanged letters, and you’ve been impressed with \cTechStar{\their} idealism, ingenuity, and willingness to question the status quo and how it can be changed for the better. You are also intrigued by \cTechStar{\their} new invention, the VidCom device, which allows long distance communication, and wonder if it could be of use to the \pShip{} if \cTechStar{} could be convinced to share the blueprints for a good cause.

During your four years working on the project to end the Storms, you and \cHeadScientist{} received a lot of half-baked correspondence from well intended folks with truly wild ideas, offering their assistance or asking for help with their own research. Sometimes it was hard to tell whether an idea was ingenious or intractable. One such missive was from one of the foremost Cursemakers in all of \pEarth{}, \cCurse{\intro}, who was working on a secret project to open a portal to the Realm of the Gods, the Divine Realm. \cHeadScientist{} dismissed the idea as both absurd and blasphemous, but you saw true merit in it — if anyone could provide insight into the cause of the Storms, it was the Gods themselves. So you wrote back to \cCurse{} and have been helping with the project ever since. When you arrived this morning at the \pSchool{}, it was the first time that you got to sit down with \cCurse{} and \cCurse{\their} collaborator \cFlowPriest{\intro}. You were surprised to see \cAmbition{\intro} there, who you know by name only as one of \cHeir{}’s best friends, but \cFlowPriest{} vouched for \cAmbition{\them}. You are excited to put the finishing touches on, and finally test, the fruits of your labors.

As for the permanent end to the Storms that you have been researching for the past four years, well\ldots{} it's complicated. Theorizing that the Storms were coming from somewhere outside the known world and could be stopped permanently by reflecting the Storm back to its source, your team was able to construct a network of relay stations along the Bones of the World that allows the Storm to be sent to other planes of existence (yes, those exist). The user interface is a portable Lightning Rod that must be used in the preparations for the Ritual to Control the Storm. You were also able to determine that the Storm is drawn to the largest concentrations of magical energy, which explains why it cannot be sent to uninhabited regions of \pEarth{} — most magic resides in people and the things they create, so heavily populated areas are like magnets for the Storm’s wrath. As for where and what the Storm’s point of origin is, the final test results only just came in as you were on your way to the \pSchool{} for this Time of Deciding, and they are shocking: The Storm originates in the Divine Realm (the Realm of the Gods) at a location containing an enormous concentration of magical energy which you are referring to as Coordinate Zero. Your instruments have also detected another massive concentration of magic in the Divine Realm at a location you are referring to as Coordinate One. Without a way to reach the Divine Realm currently, there is no telling what exactly is at these locations or how Coordinate Zero is causing the Storm, but simulation after simulation has concluded that sending the Storm to either location will destroy whatever is there and end the Storms for good. 

You and \cHeadScientist{} have brought your Lightning Rod with you to this Time of Deciding, and it will need to be incorporated into the preparations for the Ritual to Control the Storm if the goal is to send the Storm to Coordinate Zero or Coordinate One in the Divine Realm. However, you have serious concerns about sending the Storm to either location without any idea what you would be destroying in the process. This whole project has been a huge gamble with so many unknown variables, and a wrong choice could prove catastrophic. It is a good thing you’ve been working with \cCurse{} on a way to reach the Divine Realm; as brazen as it sounds, being able to scout those locations may provide crucial data to inform any decision about where to send the Storm. 

As if you weren't under enough pressure already, you and \cHeadScientist{} are scheduled to present the results of your research first thing Saturday morning! You haven't even had a chance to share your latest findings with your sponsor, \cDiplomat{}, or the members of your nation's High Council who are present, \cAntiChup{\intro} and \cTechStar{}. They’ll surely want to hear your results and strategize about how to frame it to the other nations. Whatever is decided about where to send the Storm this time, you really hope it is not sent to the \pTech{}, as doing so would destroy the relay stations your team has spent the last six years building, and with them, your ability to end the Storms permanently. By that same token, you also desperately want to make sure the Storm doesn’t get sent to the \pShip{} yet again. The best case scenario is probably that you can reach the Divine Realm and determine that one of the two locations there is safe to send the Storm to.

As preoccupied as you are with how to end the Storms, there are other important things to work on this weekend as well. You aim to figure out what really happened six years ago — without blowing your cover to the wrong people — and make sure nothing like it happens again. This is another politically charged cycle, and similar to six years ago, there are important treaties under discussion. You know you are taking an enormous risk by returning to the place where you were almost murdered, but you NEED to make sure history doesn't repeat itself. In order to avoid being recognized by anyone from your old life, you've been using an illusion spell that subtly alters your appearance these past six years (a more drastic alteration is too difficult to maintain long term), and your very life may depend on it holding for the whole weekend.

Anyone from the \pTech{} or the \pFarm{} who was present at the school six years ago, including \cDiplomat{\full}, \cHeadScientist{\full}, \cMusic{\intro}, \cEthics{\intro}, and \cLibrarian{\intro}, is a potential suspect. You don't know who was behind the Betrayal, but both nations have profited greatly from it. Still, some suspects are less likely than others. \cMusic{} was your favorite teacher at school and \cMusic{\they} cared for \cMusic{\their} students like they were \cMusic{\their} own children; you can't imagine \cMusic{\they} could have had anything to do with the atrocity that was committed. And you've worked so closely with \cHeadScientist{} and \cDiplomat{} that it's hard to believe it could have been them either, even if they did technically stand to gain from the Betrayal through it enabling their project to end the Storms. As the Librarian, \cLibrarian{} was in charge of three of the Relics and the preparations for the Ritual to Control the Storm, giving \cLibrarian{\them} plenty of opportunity for foul play. As for \cEthics{}, what more innocent seeming cover identity could there be for a ruthless saboteur than the school’s Morality and Ethics teacher? Regarding the \pShippies{} who were present, you don't believe any of the teachers or advisors would have betrayed their own country. 

To complicate matters further, it is possible that the mastermind was not physically present six years ago, but at the very least their agents must have been here to change the votes and poison the wine. You don't KNOW they'll be here this weekend either, of course\ldots{} but if there was ever a time for a repeat performance of six years ago, it would be now, given all the talk from Ambassador \cHeadDiplomat{\intro} about forging a new peace treaty and restoring balance. As for people who might help you in finding the perpetrators, you feel certain that \cHeir{} would be willing to help you hunt for \cHeir{\their} sibling’s murderer. 

You didn't pay much attention to the preparations for the Ritual to Control the Storm during your last year at the school, but you feel compelled to be involved now. Something during the ritual went catastrophically wrong before, and without your watchful eye on it, something could happen again. And now that you're an advisor, you also have a different role in determining voting authority than you did as a student — you have to decide which student to give your Voting Stone to. The obvious choice would be one of your friends, \cHeir{} or \cTechStar{}, but you need to make sure whoever you give it to won’t vote to send the Storm to the \pShippies{} again. 

Hopefully this weekend you can finally lay your burdens to rest and avenge your fallen classmates, while preventing anything from happening to this year's voting class. Somehow, you will bring their murderers to justice — if you're good, you might even manage it without having your cover blown. And if you’re really good, maybe you can bring your research into how to end the Storms and how to reach the Divine Realm to fruition, use the one to inform the other, and finally end the catastrophe that has plagued \pEarth{} since time immemorial. It won’t make up for all the lives lost because of the Betrayal, but at least it will have given their deaths some semblance of meaning. 

\begin{itemz}[Major Goals]
    \item Decide with \cHeadScientist{}, the Council Members (\cAntiChup{} and \cTechStar{}), and your patron \cDiplomat{} what to do with your research into how to end the Storms, and how to frame the results during your Saturday morning presentation. Should the Storm be sent to Coordinate Zero or Coordinate One in the Divine Realm? If so, you'll need to perform a ritual to prepare the Lightning Rod before the Ritual to Control the Storm is conducted. Whatever happens, you want to keep the Storm from being sent to the \pShip{} again. You also want to make sure it doesn’t get sent to the \pTech{}, as that would destroy all of your and \cHeadScientist{}’s work to end the Storms permanently.
    \item Work with \cCurse{}, \cFlowPriest{}, and \cAmbition{} to create a portal to the Divine Realm — it may be crucial in deciding where to send the Storm.
    \item Find out who was responsible for the Betrayal and murder of your fellow students six years ago, acquire proof, and make it off the island alive with it, so you can bring it to the proper authorities. \cHeir{} can probably be an ally in this.
\end{itemz}

\begin{itemz}[Minor Goals]
    \item Make sure that the Ritual to Control the Storm goes smoothly, including making sure nothing happens to the Relics needed for the ritual. 
    \item Act as an informant for your \pShippie{} spy contact, \cBunker{}. Maybe you can convince your friend \cTechStar{} to share \cTechStar{\their} VidCom technology with the \pShippies{}?
    \item Do not let anyone find out your real identity as \cKidScientist{\full}, unless you are \emph{sure} you can trust them not to hurt you. Maybe \cMusic{} could be one such person? 
\end{itemz}

\begin{itemz}[Notes]
    \item You were not at the school as an advisor three years ago; you and \cHeadScientist{} were too busy with your experiments. Further, you were concerned someone might recognize you if you came back (a concern you still have!)
    \item You never actually converted your patron Deity to \cTechGod{\intro}, Patron Deity of the \pTech{}. Technically you still hold \cFarmGod{} as your patron, but hopefully that won't come up.
\end{itemz}

\begin{contacts}
    \contact{\cHeadScientist{}} Your boss and mentor for the last four years, whom you worked closely with on your research into how to end the Storms. 
    \contact{\cDiplomat{}} \pTech{} advisor and diplomat, and the patron of \cHeadScientist{}'s and your research into ending the Storms.
    \contact{\cHeir{}} Younger \cHeir{\sibling} to \cHeirSibling{}, one of the students murdered six years ago, who had been one of your best friends. As a \cHeir{\formal}, \cHeir{} is a big deal in the \pTech{}, and a good friend to you since you met.
    \contact{\cTechStar{}} This year's TechStar, a student at the school, inventor of the VidCom devices, and a friend of yours. 
    \contact{\cCurse{}} One of the greatest Cursemakers in all of \pEarth{}, and one of your collaborators in creating a portal to the Divine Realm.
    \contact{\cMusic{}} Your favorite teacher from when you were a student. Half of you is afraid \cMusic{\they} will recognize you, the other half hopes \cMusic{\they} will so you can have someone to confide in. But can you trust \cMusic{\them}? After all, \cMusic{\theywere} at the \pSc{} six years ago.
    \contact{\cBunker{}} Your \pShippie{} spy contact, to whom you've occasionally passed secrets about the \pTech{} research initiatives in the past. 
\end{contacts}

\end{document}
