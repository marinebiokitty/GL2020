\documentclass[char]{GL2020}
\parindent=0pt
\begin{document}
\name{\cPrincipal{}}

You are \cPrincipal{\full} (\cPrincipal{\they}/\cPrincipal{\them}). You grew up in the \pShip{} more than 250 years ago and followed your aspirations as a teacher all the way to the \pSchool{}. At age 62, near what you thought was the end of your career, you were chosen to be the next Principal. As the head of this important institution, you have been granted immortality by the Gods until you name your successor and pass on your mantle — something you have finally resolved to do. You are tired — oh so very tired — and it is finally time to lay down your burdens and let younger minds determine the course of the world.

The \pShip{} of your childhood doesn't look all that different from the one of a decade ago. The Council of Storm Watchers was smaller, the rituals less elaborate, and food was a little blander. Still, everything was mostly the same, except for one important thing — the pirates. While it's true that when you took on the Principalship, you foreswore your ties to the \pShippies{}, and accepted the fate that as long as your bore the title you could not leave the island on which the school is built, you have watched from a distance as the scourge of piracy slowly overtook your homeland. Watching their rise has been the hardest thing about remaining impartial and devoting yourself to the \pSc{}, even more so than the recent war. Conflicts come and go — societal rot is harder to root out. And so you worry for the \pShip{} daily.

One of your greatest trials as Principal was the events of six years ago. After the Ritual to Control the Storm was completed, the students were tired but jubilant. Someone proposed a private toast for the students, so they all adjourned from the Great Hall to the Student Lounge, leaving the teachers and advisors twiddling their thumbs. After nearly an hour of waiting, you, \cDiplomat{\full}, and \cMusic{\full} decided that you had better go check on the students. What you found was 11 dead bodies. A quick inspection indicated that it was probably the wine. But where was the twelfth student, \cKidScientist{\full}? \cKidScientist{\They} \cKidScientist{\were} unaccounted for. The three of you stood in that room, the shock washing over you. Was \cKidScientist{} responsible, or \cKidScientist{\were} \cKidScientist{\they} an innocent victim who somehow managed to escape? You spoke quietly, and ultimately agreed to keep the fact that \cKidScientist{} was not among the dead a secret. It was far more likely that someone else had poisoned the students, and \cKidScientist{} had escaped. Keeping it out of public knowledge that \cKidScientist{\they} escaped would avoid anyone continuing to hunt \cKidScientist{\them}. And if \cKidScientist{\they} wanted to surface and return to the public eye, that was \cKidScientist{\their} prerogative.

As if this tragedy weren't enough, the Betrayal of the \pShip{} came hard on its heels, as the Storm was sent to your homeland out of turn. Only a fool would not suspect that the two events are linked, that someone must have somehow rigged the vote under your very nose and killed the students to cover it up, but the official investigation that followed yielded no suspects. You had your hands full with the wave of resignations that followed — teachers who couldn't bear what had happened and chose to leave the school. The school board had to hire quickly to fill their positions, and you had to train all of them before the new school year began. \cPrince{\full}, \cChupInventor{\full}, and \cChupSecond{\full} are all among that wave of new hires. Finding a replacement combat teacher took 2 years and you were most grateful when \cInterpol{\full} applied.

The curse of immortality is watching those you care about pass on without you. The people you grew up with, taught with, laughed with, broke bread with, they are all gone. It's not that you don't value your current colleagues, it's just that with every passing year, you feel more distant from them, more removed from the world. Even your current oldest friend, \cBunker{\full}, who visits the \pSc{} every 6 years to maintain the Bunkers, will pass away long before you, unless you make a change.  

And so, at the beginning of the school year, you announced your intentions to retire, and called for nominations for your successor. Over the past few months, you have evaluated a number of candidates, most of whom were teachers at the school. You believe you've narrowed your decision down to 3 finalists: \cMusic{\full}, \cBeetle{\full}, and \cChupSecond{\full}. You haven't yet spoken with them explicitly about the heavy weight of immortality and responsibility to the school, but you know that conversation will be essential to prepare them for this burden. They must be prepared to watch life pass them by. They must understand that once they become principal, they mustn't leave the island. Your predecessor disregarded this, and it caused the entire Old Wing of the College to collapse. Those ruins have resisted any attempt to clean them up or rebuild them.

As a test, you have also given each of your candidates charge over one of the important activities in the last few days leading up to the Storm. \cMusic{} is in charge of the Ceremony of Excellence — a fitting challenge for someone who has struggled to manage logistics in the past. \cBeetle{} is in charge of the crucial ritual to Renew the Ley Lines that keep the school afloat and channel magic throughout \pEarth{}. You feel it will be a good test of whether \cBeetle{\they} can let go of \cBeetle{\their} ties to the here and now since it involves a sacrifice of power. And \cChupSecond{} is in charge of the most important task of all, executing the Ritual to Control the Storm. As a relatively new teacher with newfangled ideas, you feel this will be an excellent test of \cChupSecond{}’s adherence to indispensable traditions. While you are confident that all three candidates are up to at least the basics of these tasks, you have not discounted the possibility that you'll need to step in to make sure the tasks are completed correctly.

Though all three tasks you have assigned to your candidates are important, the Ritual to Control the Storm is without a doubt the most important event of the weekend. In order for the votes cast by the students to do anything, the ritual must be set up in exacting detail. These preparations traditionally fall to the Librarian, \cLibrarian{\full}. It requires Relics and careful preparation, and cooperation on all sides. The Relics are powerful artifacts in their own right, able to perform miraculous feats when attuned. Several attempts have been made to steal Relics over the decades. Very early in the annals of the \pSc{}, one of the \pFarm{} Relics disappeared. It took years of fevered work by the Priesthood to sanctify a new one. You are committed to safeguarding all six Relics, making sure they are not tampered with or stolen, and that they all end up where they belong at the conclusion of the weekend — the three Relics in the Library should remain at the school, and the three Relics being brought by advisors should go home with those advisors. The three Relics that are stored in the Library are the \iNet{}, the \iLariat{}, and the \iScythe{}. You intend to impress the importance of safeguarding the Relics upon your successor, whomever you should choose. If the Ritual to Control the Storm is not prepared and executed properly, the results would be catastrophic. The Storm would rage unchecked across all of Cengea.

The other half of this process is the voting. The students will each get a certain number of Voting Stones to cast toward where to send the Storm, awarded to them by the teachers and advisors, who each start with one stone to give to a student of their choice. It is your unenviable task to make sure it all happens. You do not have a direct role in the treaty negotiations, but they will influence the direction of the Storm as well, as will the attunement status of the Relics used in the Ritual to Control the Storm. 

Not only are you in charge of making sure all the teachers and advisors dole out their Voting Stones to students on time, but you also have to ensure that the students submit those votes by lunchtime on Sunday. You must strike a careful balance between ensuring the teachers hand out their stones and offer guidance to students who are wavering on the one hand, and reprimanding any teachers who violate their crucial impartiality and exert undue influence on the other. You have not had to fire many teachers over the course of your long career, but you are not afraid to do so should any of them display undue partisanship. 

Among the current crop of teachers, you have found an unexpectedly close friend in \cEthics{\full}, the Morality and Ethics teacher. \cEthics{\They} \cEthics{\have} such a kind and gentle heart, always considering the impacts of \cEthics{\their} actions and what will serve the greater good. You have taken it upon yourself to mentor \cEthics{\them}, though after two decades of friendship, you find yourself learning at least as much from \cEthics{\them} as \cEthics{\they} \cEthics{\do} from you. You also have a close friend in the wise and learned school Librarian, \cLibrarian{}. The two of you have spent many happy hours discussing the finer points of magical theory and the school’s history. \cLibrarian{\They} \cLibrarian{\have} been acting a little strangely the last few days, though, distant and easily confused, and you are increasingly concerned.

After this Time of Deciding is concluded, you just want to retire in peace. No longer immortal, you'll be able to visit each fleet in your homeland, and then live out the rest of your life somewhere, preferably far away from conflict. Once it is safe for you to travel, you want to see the world, experience all of the sights you haven't been able to for hundreds of years, and spend time with friends in a way that will mean more than it has since you took on the mantle of immortality. Even if the war still rages, your time can be spent in quiet reflection, pondering questions you've always found fascinating, but never had time for among all of your other responsibilities — like piecing together the truth about the creation of \pEarth{}. Each country tells it slightly differently, and there are dozens of versions throughout history, so you've always been idly amused by contemplating what really happened. 

All that will take is a smooth transition of responsibility to the next generation. If you can make it out of here with a strong successor picked, the Relics in their proper places, and your friends safe and sound you will be able to finally rest.

\begin{itemz}[Goals (in roughly descending order of importance)]
    \item Make sure that all the voting goes smoothly, the teachers remain impartial, the Voting Stones are all distributed to the students, the students vote on time, and that the ensuing Ritual to Control the Storm goes off without a hitch.
    \item Make sure the six Relics are neither tampered nor absconded with; after the weekend's rituals are complete, the three Relics at the school should remain at the school, and the three being brought by advisors should go home with those advisors.
    \item Decide who should succeed you as the immortal Principal of the \pSchool{}. Enact the transfer of power before the Storm is sent \emph{(OOC: before end of game)}. To that end, keep an eye on \cMusic{}, \cChupSecond{}, and \cBeetle{} as they complete the tasks you've set for them. You can step in if you need to, but these tasks are meant to be tests of their fitness to take your place.
\end{itemz}

\begin{itemz}[Notes]
    \item You know that pirates often identify undercover agents with code phrases. Since many pirates call The Graveyard, off the eastern coast of \pEarth{}, their harbor, you expect these phrases to contain reference to rough or un-navigable waters, or to water untouched by the Goddess' feet (a reference to the \pShippie{} Creation Story).
    \item The Principalship has an emblem representing it, which you will physically pass on to your successor. Please feel free to select a costume component for this purpose that you don't mind lending to another player when you choose your successor. Otherwise the GMs will have a prop for you.
    \item You were here at the \pSchool{} six years ago. You are always here. You are one of the three people who know the truth, that only 11 students were actually killed that day, and that the twelfth, \cKidScientist{\full}, remains at large. The other two people who know are \cDiplomat{\full} and \cMusic{\full}.

\end{itemz}
\begin{contacts}
    \contact{\cMusic{}} One of your three finalists for succession. A little scattered, but seems to be growing at a rapid pace.
    \contact{\cBeetle{}} One of your three finalists for succession. You think \cBeetle{\they} might be too attached to \cBeetle{\their} current role and powers, but \cBeetle{\they} \cBeetle{\are} otherwise extremely competent.
    \contact{\cChupSecond{}} One of your three finalists for succession. Brilliant and full of fresh energy and perspective, but you worry about \cChupSecond{\their} adherence to tradition. 
    \contact{\cBunker{}} Your oldest living friend. A competent engineer who built and maintains the Bunkers at the \pSc{}.
    \contact{\cEthics{}} The Morality and Ethics teacher, a close friend, and someone you've mentored over the years.
    \contact{\cLibrarian{}} The school Librarian and an old friend of yours, who has been acting a little weird lately. 
\end{contacts}

\end{document}






