\documentclass[char]{GL2020}
\parindent=0pt
\begin{document}
\name{\cChupLeader{}}

You are \cChupLeader{\intro} (\cChupLeader{\they}/they), the visionary, charismatic leader of the \pGoaties{}. This is your moment to elevate your patron \cGenesis{\Deity} \cGenesis{\intro} to the head of the Pantheon, liberate the world from the corrupting influence of magic, and end the Storms for good. All you have to do is mess with the Ritual to Control the Storm — and not get caught doing it.

You grew up in the \pShip{}, in 9th Fleet, one of the poorest in the nation. The trade was bad, the fishing was bad — just about everything was bad. You attended 9th Fleet Academy — if you could even call that bottom-of-the-barrel place an academy, worthless as the credentials are. You and your ship scraped by, but just barely. Many of your friends never had enough to eat. So much for a country of balance. Where was the spot at the banquet table for you and yours? One of your few solaces was putting in at port and visiting your mother’s family, the \cCurse{\formal}s; your mother Anemone (Nemy for short) died when you were young, but your \cCurse{\auncle} \cCurse{\intro} always treated you like the \cChupLeader{\offspring} \cCurse{\they} never had, and would sneak you small presents and treats. \cCurse{\They} said you were so like your mother, and you loved hearing about her childhood. The \cCurse{\formal}s are a peasant family from the \pFarm{}, and \cCurse{} ultimately became one of the greatest cursemakers in all of \pEarth{}. You lost touch after the war started six years ago, but still care for each other deeply.

From a young age, you stayed up late into the night, asking any deity who would listen why, \emph{why} was such crushing poverty your lot in life? You were smart, charismatic, and talented! Yet because of where you were born, you were left to languish on the bottom rung of society. Long nights of contemplation brought you to realize that the reason for your pain was distressingly simple. The richer, more prosperous fleets had the resources to attract the best mages, who brought more wealth to the fleets that employed them. The mages strengthened the fleets and their social and financial position, intermingled, and gave birth to more strong mages, who stayed with those fleets. The self perpetuating system gave rise to closed circles of society that could never be breached — with you on the outside. Looking abroad gave you no hope either; this pattern was present the world over, no matter how meritocratic a country claimed to be. And at its very heart was \emph{magic}, the great unbalancer, the force that allowed this unjust system to perpetuate with no hope of revolution. The force with which the world could never be at peace. For years, you stewed on the problem, able to do nothing to change your plight or that of so many others\ldots{}

But that all changed suddenly, just over 10 years ago. Deep in the night, you heard a voice whisper in the dark. It was a quiet voice at first, a voice of something afraid to be caught where it was not welcome. After you listened, you recognized it for what it was — the savior you had been looking for. The \cGenesis{\Deity} called \cGenesis{} was speaking directly in your ear about a better way. \cGenesis{} was a Minor \cGenesis{\Deity} of luck, one hardly anyone had heard of in millenia, one who had been cast aside just like you — but \cGenesis{} had big plans. And \cGenesis{\they} needed your help to make them a reality. One harrowing blood ritual later, and \cGenesis{} was able to free you from the smothering influence of \cEbb{\intro} and \cFlow{\intro}, and make you \cGenesis{\their} first acolyte.

Under the protection of \cGenesis{} alone, you began to move through the poor, lonely, and forgotten of \pEarth{}. You brought what little comforts you could, material and spiritual, using your own personal brand of subtle, persuasive magic to smooth the way and ease the burdens of the downtrodden. (In the subtlety, you are not so different from many \pShippies{}.) Of those you helped, some chose to follow you. Little by little, the group that you would come to call themselves the \pGoaties{} grew. \cChupSecond{\intro} was one of your first followers, and by far the most capable, quickly rising to become your right hand. Embittered by a life of crushing poverty as a fellow member of the 9th Fleet and having little magical talent, \cChupSecond{}'s loyalty to the group was absolute. \cChupSecond{\They} complemented your divine vision and charisma magic with \cChupSecond{\their} ability to turn it into reality through thoughtful planning and considered action. Between the two of you, the \pGoaties{} grew by leaps and bounds — and now finally you are ready to move onto the world stage.

\cChupInventor{\intro} was another of your early recruits, a brilliant inventor from the \pTech{} whose designs had been thwarted by the narrow minded Clerics of \cTechGod{\intro}, who refused to approve \cChupInventor{\their} greatest invention — ``refrigeration coils.'' Having such a skilled engineer in your ranks proved instrumental to the growth of the movement, as you quickly turned the sorry excuse for a Black Market that already existed into a force to be reckoned with. The Black Market, overseen by \cChupSecond{} and \cChupInventor{}, proved to be an invaluable revenue stream, means of moving agents and equipment around discreetly, and above all, a source of exploitable secrets about your enemies. You managed to forge both \cChupSecond{} and \cChupInventor{}’s papers and have had them employed at the school for about five years. The tragedy of 12 dead students not withstanding, the wave of teachers quitting afterwards certainly eased the way.

One of your most socially privileged recruits is \cChupStudent{\intro}, second child of the powerful \cChupStudent{\formal} family of the \pFarm{}. Five years ago, you were in the \pFarm{} working to recruit from their disaffected peasants. As \emph{luck} would have it, a barn caught fire just as you arrived on the scene. This allowed you to rescue the children who had been playing inside and curry favor with the local populace. You began preaching the word of \cGenesis{}, but then some noble brat and \cChupStudent{\their} retainers showed up halfway through, and you thought you were going to be arrested for sedition. But instead, your words (and a touch of charisma magic) got through to \cChupStudent{}, and \cChupStudent{\they} became a secret class traitor and one of your most loyal followers. \cChupStudent{\Their} age and social status have proven invaluable, as it means you finally managed to place a student at the \pSchool{}.

You tried, years ago, to recruit your beloved \cCurse{\auncle} \cCurse{} to the cause, but were saddened to find \cCurse{\they} had grown bitter and jaded with age, believed that such direct social upheaval would inevitably fall hardest on the poor it was meant to help, and rejected your overtures. \cCurse{} does not know about the \pGoaties{} per se, only that you are trying to organize some sort of peasant uprising, and \cCurse{\they} promised to keep your secret even though \cCurse{\theydo}n’t agree with your methods. While the rejection stung, you still love your \cCurse{\auncle}, and a part of you still hopes you can win \cCurse{\them} over to the cause. You may have to take the chance this weekend, as you have heard that \cCurse{\they} will be attending as an advisor.  A surprising move by such a rigid and classist society as \pFarm{}, but a welcome one for you.  

In preparation for your next big move, a little over two years ago you leveraged the luck of \cGenesis{} and a few touches of charisma magic to land a very prestigious position for someone of your background — secretary to \cHeadDiplomat{\intro}, one of the \pShip{}'s leading diplomats. \cHeadDiplomat{\Theywere} working in Fleets 8, 9 and 10 at the time, to try to reduce the pirate threat. Mere days after you began your position as \cHeadDiplomat{}'s secretary, a notorious pirate ship, the Silent Fog, tried (and failed) to kidnap \cHeadDiplomat{\them}; you were shocked to recognize \cPirateChild{\intro} among the pirates, a teenager you had lived with on the same poverty stricken ship for awhile — a fact which you would be able to put to good use later. From your position as \cHeadDiplomat{}'s secretary, you have had a much greater ability to move about in the world, collect information, and position your people where they can have the greatest effect.

The one downside of being \cHeadDiplomat{}’s secretary was that you were forced to work with \cHeadDiplomat{\their} apprentice, \cJuniorStatesman{\intro}. \cJuniorStatesman{} had just returned from a sabbatical fighting on the warfront when you became \cHeadDiplomat{}’s secretary, and you soon learned that \cJuniorStatesman{\theywere} responsible for an attack that killed one uncle and two cousins of yours on the \cCurse{\formal} side of the family; fortunately, \cCurse{} was not there when the attack took place. You have been playing the long game ever since, undermining \cJuniorStatesman{} at every opportunity, and you intend to make \cJuniorStatesman{\them} pay the ultimate price this weekend. It is always the poor and the innocent who suffer the most when the leaders of the three nations decide to go to war, and it is high time those in power answer for their atrocities.

As your plans have grown bolder, one of your most versatile tools has been Avengers — group members who act as assassins, but without the usual drawbacks. Through the power of \cGenesis{}, rather than the Avenger losing their memory, they are forgotten by others in their lives instead. This powerful tool has made assassinations an integral part of your strategy. It was the judicious use of one such Avenger that recently protected your position as an advisor \pSc{} today, and prevented catastrophe. Your diplomat boss, \cHeadDiplomat{}, was on the verge of discovering your association with the \pGoaties{}, and had to be neutralized immediately. It was a win-win — either \cHeadDiplomat{} would be killed, or \cHeadDiplomat{\they} would be forced to kill in self defense and be punished with total amnesia by the Gods. Either way, \cHeadDiplomat{\they} and \cHeadDiplomat{\their} suspicions would be out of the picture, unable to attend the proceedings this weekend, leaving you as the foremost authority on \cHeadDiplomat{}'s plans for the negotiation.

A tiny wrinkle in your plan might be \cEbbPriest{\intro}, a close family friend of \cHeadDiplomat{} whom you have met at some of \cHeadDiplomat{\their} family dinners. \cEbbPriest{\Theywere} present in the house when \cHeadDiplomat{} was attacked, and \cEbbPriest{\were} therefore able to begin investigating all too quickly. You were roused by messenger this very morning, and summoned to \cHeadDiplomat{}’s house, where you spent the next few hours pretending to be devastated at the loss, and happy to help \cEbbPriest{} come up to speed. Apparently \cEbbPriest{\they} had decided that \cEbbPriest{\they} would be taking over \cHeadDiplomat{}’s spot as an Advisor. Whatever, as long as \cEbbPriest{\they} stay\cEbbPriest{\verbs} out of your way, and \cEbbPriest{\do}n’t get too nosy. Your best bet is probably to offer to lead any investigation into who sent an assassin after \cHeadDiplomat{} yourself. \cEbbPriest{} has ordered you and \cJuniorStatesman{} to keep what happened to \cHeadDiplomat{} secret; the only other person who knows is the fourth \pShip{} advisor, \cBunker{\intro}.
 
Now that you and your other operatives are positioned here at the school, it is finally time for the coup de grâce. You will rewrite the rules of the world to be more fair and equitable, with you at the helm to make sure it stays that way. Everything hinges on the Storm. Most people, including most \pGoaties{}, believe that the only valid targets for the Storm are the three nations. But you and a tight circle of other \pGoaties{} have consecrated a new Relic, the \iHorseshoe{}, that is attuned to the \pSchool{}, allowing the Storm to target the \pSc{} if it is used in the Ritual to Control the Storm. Your group's goal this weekend is to use the Storm to destroy the school as a blow against the current Pantheon and political structures of \pEarth{}, permanently ending the Storms in the process. 

In reality, the plan goes two steps further — sending the Storm to the \pSc{} will also drastically weaken all magic on \pEarth{} by destroying the Ley Line Nexus at the center of the island, and serve as an assassination of many of the assembled dignitaries. Those last two items are strictly on a need-to-know basis, and only you and \cChupSecond{} know the full plan; specifically, \cChupInventor{} does \textbf{not} know that magic will be weakened, and \cChupStudent{} does \textbf{not} know that the Storm striking the school will kill many of those assembled here or that \cChupStudent{\their} fellow cultists were responsible for sabotaging the Bunkers. If you want to retain their loyalty, you had best keep it that way. 

An additional secret you have kept from \textbf{all the other} \pGoaties{} is the exact \textbf{reason} that sending the Storm to the school will empower \cGenesis{}: Eons ago, the Patron Gods betrayed Genesis and the other ``Minor’’ Gods and imprisoned them in the Underworld, vastly curtailing their power. The Gates of the Underworld are buried deep within the bowels of the school’s Library, and the full might of the Storm is the only thing that can break them down, thus freeing \cGenesis{} and the other ``Minor’’ Gods. \cGenesis{} has not told you how exactly this will end the Storms, but \cGenesis{\theyhave} assured you it will. And if there is one being you trust, it's \cGenesis{}.

With most magic gone from the world, the followers of the Patron Deities will fall to despair, and many will forsake the gods who have forsaken them. But the \pGoaties{}, having prepared for a near apocalypse scenario, will ride out the turmoil comparably unscathed. Meanwhile, the ``Minor’’ Gods will wage war against the Patron Gods in the Divine Realm. When the dust settles, the Patron Gods will have been defeated both here on \pEarth{} and in the Divine Realm, and a more just and equal society can be ushered forth, with you and \cGenesis{} at the helm.

To make this work, you need to make sure at least one Relic attuned to the school is used in the Ritual to Control the Storm, either by installing the Relic you have constructed — the \iHorseshoe{} — or by re-attuning one of the other Relics. The more Relics attuned to the school, the greater the odds that the Storm will be sent there. In this you have the help of \cChupInventor{}, whose technical expertise should prove useful. You'll have to watch out for the Clerics of each nation, though, as they have the ability to perform rituals to determine if the Relics have been tampered with. 

The second part of your plan requires getting enough \emph{votes} for the Storm to target the school. Only students can vote, but your \pGoatie{} among the students, \cChupStudent{}, is a good start. You can give \cChupStudent{\them} your Voting Stone, but you will also need to recruit more students to the cause. \cInitiate{\intro} is one potential candidate. \cInitiate{\Theyhave} the stink of a Bad Luck Curse upon \cInitiate{\them}. Stupid \pFarm{} cursemakers encroaching on \cGenesis{}' domain! Maybe you can help \cInitiate{\them} with that in exchange for joining the \pGoaties{}? Approaching \cInitiate{\them} will be risky, however, as \cInitiate{\theyare} an Initiate of the Twin Goddesses. 

You know that \cWarlordDaughter{\intro} is also here. As a child of \cLoud{\intro}, you doubt \cWarlordDaughter{\they} could be turned, but you also know \cWarlordDaughter{\theyare} from one of the poorer fleets, and have seen how miserable \cWarlordDaughter{\they} look\cWarlordDaughter{\verbs} at political meetings. Perhaps \cWarlordDaughter{\they} would be open to a different life? Approaching \cWarlordDaughter{} will be risky too, however, as \cWarlordDaughter{\theyare} also an Initiate of the Twin Goddesses. 

And then there is \cPirateChild{}. When \cPirateChild{\theywere} a teenager, the two of you lived on the same poverty stricken ship. \cPirateChild{} eventually went on to become a pirate, and crossed paths with you later in the most unexpected way when \cPirateChild{\they} tried to kidnap \cHeadDiplomat{}. \cPirateChild{\They} reached out to you afterwards, and you found them to be passionate, willful, and deeply angry about the same injustices that you are. You decided to put \cPirateChild{\their} skills to better use, and personally sponsored \cPirateChild{\their} attendance at the \pSchool{}, helping \cPirateChild{\them} forge the necessary paperwork to change \cPirateChild{\their} identity in the process. Your stated reason for doing this was so that \cPirateChild{\they} can help you steal a prototype or blueprints of \cTechStar{\intro}’s VidCom invention (a device which allows long distance communication, and would therefore be invaluable to your cause) ``on behalf of the \pShip{};’’ you know that \cChupInventor{} has also been cultivating a mentor relationship with \cTechStar{} to see if \cTechStar{\they} can be persuaded to share the device willingly. You also plan to try and recruit \cPirateChild{} to the \pGoaties{}. While you try to avoid getting too strongly attached to anyone, you find that you feel protective of \cPirateChild{\them}, and want to make sure nothing bad happens to \cPirateChild{\them}. 

You are extremely confident that you will be victorious this weekend, but you know that your reliable but overly cautious second-in-command, \cChupSecond{}, has doubts. You try not to humor \cChupSecond{\them} too much, but even you have to admit that the risk of failure is real. In order to assuage \cChupSecond{\their} concerns, you've said that if your plans are discovered by your enemies, you will take all the blame, while \cChupSecond{} disavows everything and goes into deep cover as the group's new leader. Pursuant to that goal, \cChupSecond{\they} put in \cChupSecond{\their} bid to become the new Principal of the school when \cPrincipal{\intro} retires, as holding that lofty position would make any future attempts much easier, though it would also mean that \cChupSecond{} would be more limited in movement, as the Principal is confined to the school grounds. You both need to meet this weekend to make sure that things are in place so that this will not disrupt plans for the future. Maybe \cGenesis{} can find a way to overcome this limitation?

This contingency has had the added benefit that the current Principal has tasked \cChupSecond{} with overseeing the execution of the Ritual to Control the Storm in order to prove \cChupSecond{\themself}. This places \cChupSecond{\them} in a uniquely strong position to influence which Relics get used in the ritual, making it far easier to ensure that one of them is attuned to the School. While you have no intention of actually taking the fall if the group is discovered and are confident you can charm your way out of anything, you still intend to support \cChupSecond{}’s bid for Principal because of these benefits. \cChupSecond{\Theyare} a loyal, competent, and extremely useful tool, and you don’t discard your tools lightly.

That is not the only way \cChupSecond{} is serving the cause this weekend. \cChupSecond{\They} recently informed you that \cChupSecond{\theyhave} been having dreams of following a white rabbit through the Library to a marble cavern covered with glowing symbols of \cGenesis{}, where the rabbit tells \cChupSecond{\them} that the time has come to summon the Avatar of Genesis. The dreams sound too vivid to dismiss, and you are offended that \cGenesis{} would send these visions to \cChupSecond{} and not you. Have you done something to earn \cGenesis{\their} disfavor? Still, summoning the Avatar would be a great victory, and increase \cGenesis{}’ power in the mortal world. \cChupSecond{} is convinced that the cavern where the ritual must be performed is hidden below the third tier of the Library, and you have agreed to help find it. The other Followers do not know of this yet, in case it proves to be a fool’s errand. If it does pan out, you intend to lead the ritual yourself in order to regain \cGenesis{}’ favor. 

In the lead-up to this weekend, you also ordered \cChupSecond{} and \cChupInventor{} to sabotage the school's Bunkers, which they succeeded at this morning. The sabotage is intended mainly as a distraction for your enemies, who will focus their energies on repairing the Bunkers instead of your efforts to direct the Storm at the \pSc{}. It will also have the added benefit of reducing the capacity of the Bunkers to protect people from the Storm; you fully intend to make sure that the corrupt, decadent advisors from the three nations are among those who don’t survive, \textbf{especially} \cJuniorStatesman{}. Even if the Bunkers are fully repaired, they will still be unable to protect everyone here from the full might of the Storm when it strikes the school. Meanwhile, the power of \cGenesis{} will protect you and your followers with or without the Bunkers. You have made sure to keep \cChupStudent{} ignorant of both the sabotage and the fact that the Bunkers won't be able to protect everyone; while \cChupStudent{\theyare} deeply loyal to the cause, \cChupStudent{\theyare} also young and naive, and not yet ready to understand the sacrifices that must be made for the greater good.

Still, \cChupStudent{} is useful for some things beyond spearheading the student vote to send the Storm to the school. You know that \cChupSecond{} ordered \cChupStudent{\them} to seek out and destroy any books in the Library that might reveal that sending the Storm to the school will destroy the Ley Line Nexus and weaken magic, in order to make it that much harder for your enemies to discover your plans. \cChupStudent{\They} reported the task complete about two months ago, which means one less thing to worry about. 

Speaking of your followers, you have a worrying issue to resolve with one of the personnel you stationed at the \pSc{}. You were supposed to have placed an Avenger here as a backup plan in case things went wrong, but you don't remember who they are! This is a very bad sign, as it means that your Avenger may have gone rogue and killed someone they were not supposed to. You need to figure out who they are, and how to regain control of them if necessary. You could also swear that you had an informant who was feeding you intelligence on the \pTech{} research into ending the Storms and sabotaging anything that could interfere with the cult's plans, but you can't remember who that is either! Perhaps they are the same person and that's why you can't remember them? Your top three suspects, according to intelligence gathered by \cChupSecond{}, are \cInterpol{\intro}, \cChupAvenger{\intro}, and \cWildCard{\intro}.

This weekend is the culmination of years of work. You will throw down the so-called Patron Gods, elevate \cGenesis{} to \cGenesis{\their} rightful place in the Pantheon, create a more just and equal world here on \pEarth{}, and end the Storms forever. You will accomplish this no matter the cost, by tasks both great and small, the greatest being the destruction of the \pSchool{}, and with it, the weakening of magic in all of \pEarth{}.

\begin{itemz}[Major Goals]
    \item Support the \pGoaties{} agenda:
\begin{itemize}
     \item Send the Storm to the \pSc{} \textbf{at all costs}, in order to weaken all magic in \pEarth{}, free \cGenesis{} from imprisonment, end the Storms forever, and assassinate the corrupt advisors gathered here (including your hated enemy \cJuniorStatesman{}).
     \item Make sure that at least one Relic attuned to the school is used in the Ritual to Control the Storm and enough student votes are cast to send the Storm to the school.
     \item Recruit new members to the \pGoaties{}, especially students. Potential recruits include \cInitiate{} and \cPirateChild{}.
     \item Getting a treaty signed that stipulates sending the Storm to the school would help with your objectives too, but even you have to admit that accomplishing that would be next to impossible, as it would force the \pGoaties{} to go public.
\end{itemize}
    \item Make sure the assassination attempt on \cHeadDiplomat{} is not traced back to you, perhaps by offering to head up the investigation yourself.
    \item Figure out who the Avenger and informant on the Storm research is/are and bring them back into the fold. Your top three suspects are \cInterpol{\full}, \cChupAvenger{\full}, and \cWildCard{\full}.
\end{itemz}

\begin{itemz}[Minor Goals]
    \item Help \cChupSecond{} find the ``Cavern of Fortune’s Shadow’’ in the Library, and lead the Ritual to Summon the Avatar of Genesis there.
     \item Help \cChupSecond{} get appointed by \cPrincipal{} to be the new principal of the \pSchool{}.
    \item Make sure no harm comes to your beloved \cCurse{\auncle} \cCurse{} and your protege \cPirateChild{}, ideally by converting them to the \pGoaties{}.
    \item Work with \cChupInventor{}, who has been grooming \cTechStar{} for recruitment, and \cPirateChild{}, a skilled thief and a pirate, to obtain a prototype or blueprints of the VidCom device from \cTechStar{}.
\end{itemz}

\begin{itemz}[Notes]
    \item Information in the \pGoaties{} is often on a need-to-know basis, and this weekend’s plans are no exception. The other members of the \pGoaties{} do not know that Genesis and the other ``Minor’’ Gods are being imprisoned by the Patron Gods, and you are concerned that if they were to find out, they might see \cGenesis{} as weak and their faith would be shaken. \cChupInventor{} does not know that sending the Storm to the school will weaken magic, and you are concerned that given how important magitech crafting is to \cChupInventor{\them}, \cChupInventor{\they} would abandon the cause if \cChupInventor{\they} found out. Lastly, \cChupStudent{} does not know that \cChupStudent{\their} fellow Followers sabotaged the Bunkers or that the Bunkers will be unable to protect everyone when the Storm strikes the school, and you fear that \cChupStudent{\they} would be overwhelmed with doubts and concerns for the other students if \cChupStudent{\they} found out. It is a heavy burden to protect your followers from knowledge that could destroy them, but you bear it with dignity and resolve.
    \item You have never been to the \pSchool{} before.
\end{itemz}

\subsubsection*{Contacts}
\begin{contacts}[Avenger Suspects]
    \contact{\cInterpol{}}
    \contact{\cChupAvenger{}}
    \contact{\cWildCard{}}
\end{contacts}

\begin{contacts}[Personal Contacts]
    \contact{\cChupSecond{}} Your second in command, a teacher at the \pSc{}, and runner of the Black Market. \cChupSecond{\Theyare} overly cautious and have too little faith in your plan succeeding this weekend, so you are humoring \cChupSecond{\their} bid to be appointed the new principal.
    \contact{\cChupInventor{}} A teacher at the \pSc{} and one of your oldest followers who helps run the Black Market, but is too attached to the existence of magic.     
    \contact{\cChupStudent{}} A student at the school who is fanatically loyal to you as well as your cause, but is too attached to \cChupStudent{\their} fellow students. 
    \contact{\cEbbPriest{}} The \cEbbPriest{\cleric} of \cEbb{} who is now part of the \pShip{} delegation.
    \contact{\cJuniorStatesman{}} \cHeadDiplomat{}'s apprentice, and the one responsible for the deaths of some of your \cCurse{\formal} family members during a \pShippie{} raid on your mother's home village. Does not seem to know or care one bit.
    \contact{\cInitiate{}} Suffers from a bad luck curse. You can help with that, for a price.
    \contact{\cPirateChild{}} A student at the school  who you've known since \cPirateChild{\theywere} a teenager, and who was a crewmember on the Silent Fog, a pirate ship that tried to kidnap \cHeadDiplomat{}. \cPirateChild{\Their} background makes \cPirateChild{\them} ripe for recruitment to the \pGoaties{}, as well as a useful ally for stealing the VidCom technology from \cTechStar{} if simple persuasion fails.
    \contact{\cCurse{}} Your beloved \cCurse{\auncle} from the \pFarm{}, who shares your resentment of the upper classes but rejected your last attempt to recruit \cCurse{\them} to the cause. Maybe \cCurse{\their} views have changed since?
\end{contacts}

\end{document}
