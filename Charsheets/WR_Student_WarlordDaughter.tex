owp\documentclass[char]{GL2020}
\parindent=0pt
\begin{document}
\name{\cWarlordDaughter{}}

You are \cWarlordDaughter{\full} (\cWarlordDaughter{\they}/\cWarlordDaughter{\them}). You are the \cWarlordDaughter{\offspring} of \cLoud{\full}, and that makes life\ldots{} complicated. You love your country and your family, but you are ready to chart your own course out of their shadow.

Your life growing up was simple and full of love. You grew up on an unusually stable ship with very little turnover in the crew. You know that \cLoud{} is your biological \cLoud{\parent} and has been with \cQuiet{\full} since before you were born, a rarity in the \pShip{}. Your parents encouraged your interest in the Priesthood and supported your desire to attend the 1st Fleet academy to study. You had wanted it ever since you were little, and you remember your delight when the Goddess’ mark appeared on you at age 12. But it was not to be.

Everything fell apart when the Storm hit the \pShip{} out of turn. It was a complete disaster. Your ship, Black Crow of the 10th Fleet, nearly sank, and you lost more than half of your crew — your family. Not long after, \cLoud{} began to talk of revenge. Of balancing the scales. Of war. War was a scary prospect, but you understood the need for balance — after all, it is the \pShippies{}’ way of life. What was weird was that \cLoud{} was advocating for it. \cLoud{\They} \cLoud{\were} always the calm, quiet, patient sort. \cLoud{\They} never resorted to violence if there was another way. And then overnight, bloody vengeance was all \cLoud{\they} could talk about. Something was clearly wrong, but you couldn't figure out what amidst the rising tide of a nation mobilizing for war. 

In the wake of this change, your dreams of attending 1st Fleet Religion Academy evaporated. You traveled with the remains of Black Crow, helping \cLoud{} whip up the support \cLoud{\they} needed to convince the Council of Stormwatchers. As time went on, you saw less and less of your parents. Your quiet, happy life was gone.  

One of your few escapes were correspondences you started with a few clerics of \cEbb{} and \cFlow{}, who took pity on your dashed dreams and tried to provide at least a correspondence course when they could, though things came intermittently due to how often you were on the move. Your favorite among them was \cFlowPriest{\full}, who was the mathematics teacher at the \pSchool{}, while also being a priest of \cFlow{}. \cFlowPriest{\They} believed that tradition was good but also needed to be questioned, that war may be necessary, but one always needed to look at cost. You greatly respect \cFlowPriest{\their} advice. 

On the recommendation of \cFlowPriest{}, you also began corresponding with \cEbbPriest[\full}, in order to get a more balanced perspective on the faith and to understand what serving a community full time may look like. You don’t think you ever will get to do such a thing, not with the war on and your responsibilities and perhaps even needing to follow in the footsteps of \cLoud{}, but it has been a nice distraction. \cEbbPriest{} is more traditional and stubborn in \cEbbPriest{\their} theological points, and sometimes you worry with \cEbbPriest{\their} eagerness to discuss how to solve the current political situation that \cEbbPriest{\they} may want something from you. But \cEbbPriest{\they} \cEbbPriest{\have} also listened to your concerns without judgment, helped you think through what you wish would be done about the war, and even helped you recognize your own desire for peace away from the voice of your family. You are very grateful to both of them.  

As the war stretched into its third year, \cQuiet{} convinced \cLoud{} that you would be safer at a school. The fact that nearly everyone in the \pShip{} knew your name as the \cWarlordDaughter{\offspring} of \cLoud{} made you a shoo-in to attend the \pSchool{}. It was supposed to be an honor, but you couldn't help feeling like they had sent you away. At least you knew \cFlowPriest{} would be there, teaching. Without \cFlowPriest{\them}, you don’t know if you could have survived.  

Now you are here, in your second year at the \pSchool{}. Even though you aren't graduating, there are a lot of things weighing on your shoulders. Everyone expects you to be like, speak for, or reject \cLoud{}. They watch your every move like ospreys. Even your fellow students from the \cTech{} and the \cFarm{} watch you closely with distrust in their eyes. It has been hard to make friends outside of the \pShippies{}, though your fellow \pShippie{} students seem to always have your back.  

You will of course comport yourself with all the dignity required of the \cWarlordDaughter{\offspring} of \cLoud{\full}, but you don't actually \textbf{want} to spend all day in stuffy meetings. You are sick of being seen as a stand-in for your \cLoud{\parent}, and are determined to chart your own course. Thankfully, you do have some friends and even a new partner to help you chart this course.  

\cPirateChild{\full} just arrived this year, but has already become one of your favorite people.  \cPirateChild{\They} \cPirateChild{\do} get into a lot of trouble and tend\cPirateChild{\verbs} to speak \cPirateChild{\their} mind regardless of the consequences, and you know you can’t fully join in that, but you so appreciate it. It is like a breath of bracing salt air. Recently, though, \cPirateChild{\they} \cPirateChild{\have} been a bit distant and preoccupied. You hope that \cPirateChild{\they} \cPirateChild{\have} not gotten in more trouble than \cPirateChild{\they} can handle. 

You also are fond of \cInitiate{\full}, a fellow cleric in training, almost in spite of yourself. You admit this fondness is tinged with jealousy, as Initiate{} got to actually study at the academy and \cInitiate{\they} can be a bit clueless about both the world and \cInitiate{\their} own privilege. But \cInitiate{} is just so earnest, helpful and honestly, really cute. You can’t help but be drawn to \cInitiate{\them} even though you want to shake \cInitiate{\them} sometimes! It has been comforting to you to have someone who gets your faith and is on a similar path. Recently \cInitiate{\they} \cInitiate{\have} run into a bit of trouble. \cInitiate{} has family in the \pFarm{} — family going back for generations. What you wouldn’t give to have a family like that. One you could rely on. One that would go to such immense lengths to get you back. One that didn’t switch personalities over night\ldots Is it weird to wish someone wanted you around so badly that they would send a curse across national borders to get you back?  

Okay, so maybe the curse was a little much — or a lot much. You should help \cInitiate{} figure out a way to remove it. You still think \cInitiate{\they} should give \cInitiate{\their} family a chance though. It doesn’t seem like \cInitiate{} understands what it means to be loved like \cInitiate{\they} \cInitiate{\are}. You know \cInitiate{\they} are also getting help from \cInitiate{\their} other good friend \cPresident{\full} on this issue. \cPresident{} and you don’t see eye to eye on many things, and if possible \cPresident{\they} \cPresident{\are} even more privileged than \cInitiate{}. Whether \cInitiate{} should go see \cInitiate{\their} family is one of the many places you argue. But you both agree at least that you want to help \cInitiate{} break this curse.

Of course you don’t get along with every student. \cHeir{\full} for instance, is the heir to one of the most powerful families in the \pTech{}. \cHeir{} drips with privilege, power, and often comes across as unthinking. Yet, despite this, you two often end up together in the student lounge, banging your head against the same problems in mathematics, a discipline with which you both struggle. During these late night sessions, \cHeir{} has made comments that make you believe that \cHeir{\they} are also distancing themselves from a family that controls \cHeir{\their} life. Maybe that is why after an extremely long study session, deeply sleep deprived, you admitted to \cHeir that \cLoud{} is acting out of character by leading the war effort. \cHeir{} agreed to help you investigate this and keep that help a secret, but in return, \cHeir{\they) asked you to help them take a Relic, the  \iMirror{}, off of \cHeir{\their} \cDiplomat{\auncle} this weekend. You are frustrated that \cHeir{\they} won’t tell you why \cHeir{\they} need this mirror, just that it may help end the war and will not hurt your people. You hope you can get \cHeir{\them} to fess up, but you do really need a lot of help with your situation.

Finally, and maybe most importantly, there's \cTechStar{\full}. Your partner, you guess? You were supposed to be studying with \cInitiate{}, but things kept on happening that made it not work out, and \cInitiate{\they} recommended \cTechStar{} instead. At first you were resistant, as \cTechStar{\they} \cTechStar{\were} not only from the \cTech{} but also had a seat on The Council due to \cTechStar{\their} invention. Other \pShippie{} students said \cTechStar{\they} 
\cTechStar{\were} responsible for your suffering. But the more you talked to \cTechStar{\them}, the more you realized how similar you both were: put in positions of perceived power but with little actual voice. Trying to make a difference but thwarted at every turn. You realized you were falling for \cTechStar{\them}, but were terrified of what that meant.

\cInitiate{} thankfully came through, encouraging your interest and even helping you deliver a note expressing your feelings. Hilariously, you received a very similar note from \cTechStar{} the same day. Apparently \cInitiate{} was doing a bit of matchmaking. You two have been together for most of the year, but have decided to keep it secret from everyone but \cInitiate{}. You worry that people won’t understand or will try to break the two of you up. You especially worry that \cPirateChild{} will hate you. You know that \cPirateChild{\they \do}n’t like \cTechStar{}. On top of this, you have realized that, foolish and privileged as \cInitiate{} is, that you also may have a crush on \cInitiate{\them}. While polyamory is fine, and almost expected in the \Ship{}, you know that the \pTech{} does not really practice it. You are unsure if bringing this up would be a good idea, though you sincerely wish you could. What if \cInitiate{} doesn’t feel the same about you? What if \cTechStar{} is staunchly opposed? Will you risk a good thing for so many maybes?

You are glad you have your friends and now a partner, because you feel like you can’t trust many other students. Take \cLibAssist{} of the \pFarm{}, for example. You know that \cLibAssist{\their} family has done some terrible things to your people through not only cruelty with curse trebuchets but also negligence. You see how the rest of the \pFarm{} students protect \cLibAssist{\them} and you wonder if they all are fine with what you all have suffered. You want to believe better, knowing how people judge you, but it's hard to overcome.

While you want to help your friends and are grateful for their support, you know this weekend is the weekend you have step up, make who you are clear to the world, and prove once and for all you are you, yourself, not just a copy of \cLoud{} or at least \cLoud{} as \cLoud{\they} now are.  There are a few ways to do this.  

One way is being chosen as the student speaker for the Ceremony of Excellence. \cMusic{\full} is in charge of organizing the ceremony this year, so you'll need to convince \cMusic{\them} to pick you to give the speech. It is not unusual for an underclassman to be chosen for this and it would make a great first platform to distinguish your views from your \cLoud{\parent}. You are pro justice and fairness. War is one way to accomplish that, but it's complicated, messy, and rife with consequences. That's something that \cLoud{} used to understand — it's something that \cLoud{\they} taught you! You want your \cLoud{\parent} back the way \cLoud{\they} \cLoud{\were}. What ever happened to change \cLoud{\them} so?

In search of the answer, you've searched the upper levels of the Library countless times, to no avail. But during the Time of Deciding, many areas of the Library that are not normally accessible can be reached. It'll be dangerous, and you shouldn't go alone, but maybe, just maybe, you can find something this weekend! You may want to take \cPresident{} as \cPresident{\they} \cPresident{\are} a library assistant (and \cHeir{}, since \Heir{\they} promised to help). The matter is urgent. The looks some of these advisors from other nations give you make your skin crawl, and you're sure some of them mean you harm. Or if not you, then your \cLoud{\parent}, \cLoud{}. You are particularly concerned about the leaders of the \pTech{} and \pFarm{} advisors, \cDiplomat{\full} and \cEvil{\full}, respectively. They both have reputations for being ruthless political players, and were instrumental in the Betrayal of the \pShip{}. You would not be surprised if they are up to something nefarious at this very moment. You have to figure out what happened to \cLoud{} \textbf{this weekend}, and find a way to reverse it. This is so important, and you've been struggling with it for so long!

If only \cHeadDiplomat{\full}, the head diplomat of the \pShip{}, was here! The few times your paths have crossed at official functions, you've found \cHeadDiplomat{\them} to be an incredibly soothing presence. You hope that \cHeadDiplomat{\their} apprentice, \cJuniorStatesman{\full}, is similar. Maybe it is worth approaching \cJuniorStatesman{\them} for help with finding out what happened to your \cLoud{\parent}? And if \cJuniorStatesman{\they} \cJuniorStatesman{\are} one of the many warmongers in charge of your country these days, maybe you can at least talk \cJuniorStatesman{\them} down from it. \cHeadDiplomat{}’s secretary \ChupLeader{\full} is also attending and you know that \cChupLeader{\they} are from a lower fleet and have risen up the ranks. You don’t know if this means \cChupLeader{\they} are more or less in favor of war, but if \ChupLeader{\they} have been in the presence of \cHeadDiplomat{} for a while, you can only hope they will be an ally.  

Or maybe you can turn to one of your favorite teachers, \cEthics{\full}, who teaches Ethics and Morality. In the last few months, you've found \cEthics{\them} more accessible than usual, and \cEthics{\they} might be a good resource, or at least a friendly ear, as you try to work this all out. Besides \cFlowPriest{}, \cEthics{} is one of the best people you have ever met. \cEthics{\They} seem\cEthics{\verbs} to genuinely care about you as a person, and surprisingly for someone from the \pTech{}, for the plight of your country. Any time you feel that you need someone to just talk to, to have a cup of tea with, to ground you with all of the major decisions you constantly must face, \cEthics{} is there. The only thing that puts you a bit on guard is that \cEthic{\they} do ask a bit too many questions about \cLoud{}. Should you trust \cEthics{} to help with your dilemma or will \cEthics{\they} just also want to use you to get to \cLoud{} for some other purpose?  

Maybe \cEbb{} or \cFlow{} themselves could help? You've prayed to them before, but this weekend you will have clerics of both paths at the \pSc{}. You already have corresponded with both, and you regularly debate which Goddess to choose between. You will need to seek both \cFlowPriest{} and \cEbbPriest{}’s council this weekend as you make this decision. You could pick a path, and become a full fledged \cWarlordDaughter{\cleric} yourself. Maybe then the Goddesses would hear your prayer and act. Becoming a \cWarlordDaughter{\cleric} is an involved and life-changing process, but ever since you came up with the idea, you've been committed to this path. Working the will of the Goddesses through your own hands feels right, in a way that nothing else has since \cLoud{} started rabble-rousing.

If the Goddesses will not act based on that, you also have been tasked with a very important mission that once it is accomplished and can be shared will show your dedication not only to the \pShip{} but also to the Gods and prove how different you are from \cLoud{}. \cEbbPriest{}, wrote to you some weeks ago with a grave secret. The Avatar of \cEbb{}, one of the sea serpents, had been killed. You have no idea who could do such a thing, and it makes you shake thinking such blasphemy would happen, even with a war. But you know what you need to do: find out who did this, bring them to justice, and find a way to resurrect the dead Avatar. At first \cEbbPriest{} had promised to pull you out of classes so you could be present for the ritual two months hence, but with \cEbbPriest{} acting as a last minute addition to the advisor roster, the time to act is clearly now. Resurrecting the Avatar will be extremely challenging but you have every confidence that you, \cEbbPriest{}, \cFlowPriest{} and probably \cInitiate{} working together can figure it out. It may be hard, but it will be worth it and doing this would be one more way to step out of \cLoud{}’s shadow.

Your number one priority this weekend is figuring out what happened to your \cLoud{\parent}, \cLoud{}, and fixing it if you can. If \cLoud{\they} were to stop advocating for war, maybe you could get back your simple quiet family life that you miss so much. You refuse to risk the safety of your nation in the meantime, however, and intend to make that very clear in word and deed. Lastly, you do hope to make your own way as more than just your \cLoud{\parent}'s \cWarlordDaughter{\offspring}.

\begin{itemz}[Goals (in roughly descending order of importance)]
    \item Figure out what happened to \cLoud{} and fix it if you can. Foil any plots by the other nations to harm \cLoud{}, yourself, or the \pShip{} at large.
    \item Make sure the Storm does not get sent to the \pShip{}.
    \item Help \cFlowPriest{} and \cEbbPriest{} secretly resurrect the \cEbb{} avatar. Without it, you cannot choose to follow \cEbb{}'s path. Then, choose a path and become a full \cWarlordDaughter{\cleric} of \cEbb{} or \cFlow{}, or try to walk the extremely difficult path of Balance between them.
    \item Convince \cMusic{} to select you as the student speaker for the Ceremony of Excellence — then give the best speech possible to distinguish yourself from your \cLoud{\parent}.
    \item Decide what to do about your relationship with \cTechStar{} and your crush on \cInitiate{}. Do you go public? Do you go polyamorous? It is all so confusing!
    \item Help \cInitiate{} remove \cInitiate{\their} curse, and then convince \cInitiate{\them} to give \cInitiate{\their} family in the \pFarm{} a chance anyway.
    \item Somehow get the \iMirror{} from \cDiplomat{} and give it to \cHeir{}.
\end{itemz}

\begin{itemz}[Notes]
    \item You are in your second of three years at the \pSchool{}.
\end{itemz}

\begin{contacts}
    \contact{\cFlowPriest{}} A \cFlowPriest{\cleric} of \cFlow{} and your main mentor at the school. You will need \cFlowPriest{\their} help to pick a path.
    \contact{\cEbbPriest{}} A Priest of \cEbb{} who follows a more traditional path and has corresponded with you for years. \cEbbPriest{\They} can help you figure out your path, if you can resurrect the Avatar.
    \contact{\cHeir{}} A frenemy, who is helping you out with family business in exchange for a service.
    \contact {\President{}} A fellow student who you don’t always agree with, including about whether \cInitiate{} should go see \cInitiate{\their} family.
    \contact{\cPirateChild{}} Another friend, who gets into the trouble you wish you could, but also understands what it means to grow up on one of the poor ships. Something has been bothering \cPirateChild{} lately.
    \contact{\cInitiate{}} The other \pShippie{} initiate here at the \pSc{} and one of your good friends, despite \cInitiate{\their} privilege. You are jealous of how much \cIntiate{\their} family loves them. You also have a crush on \cInitiate{}.
    \contact{\cTechStar{}} Your new romantic partner, in secret. You really like \cTechStar{\them} but are worried if your relationship goes public.
    \contact{\cEthics{}} Your other favorite teacher, who always has guidance and wise words.  You wonder if \cEthics{\they} can help you with all your conundrums.
\end{contacts}

\end{document}



