\documentclass[char]{GL2020}
\parindent=0pt
\begin{document}
\name{\cEvil{}}

You are \cEvil{\full} (\cEvil{\they}/they), and 50 years old. You are an advisor to \cQueen{\full}, and are here to advance a number of your own plans. If the term ``evil mastermind'' ever applied to anyone, it would apply to you. You may not have started off trying to be ``evil,’’ but after a lifetime of being willing to do whatever it takes to protect yourself and the things you care about, you've embraced the title.

You are the youngest \cEvil{\offspring} of mid-level nobility in \pFarm{}. Your family would be a lot higher ranking if it weren't for the pirates. Your family holds land on the southern coast, near Fleet 7 waters. Every year, the pirates would raid the coast, extorting ridiculous prices in exchange for ``protection,'' and stealing whatever they wanted when their demands weren't met. You hate them with every fiber of your being. Between your family's plight and the fact that as the youngest child you stood to inherit no land, your prospects growing up were modest at best; therefore, you resolved to take your fate into your own hands. You immersed yourself in the brutal politics of the \cQueen{\Monarch}'s court. There, something wonderful happened — and then something terrible.

You met \cPirateChildParent{} and fell in love. Unfortunately, you also met \cEvilNemesis{\full}. \cEvilNemesis{} was incredibly powerful in the \pFarm{}, took an instant dislike to you, and made the pirates that harry your family's coastline look like children. The two of you fought on many fronts for several years. More than once, \cEvilNemesis{\they} nearly succeeded in killing you, or disgracing your family, or some other equally terrible outcome. But one day, \cEvilNemesis{\they} went too far. Not six months after \cPirateChildParent{} had given birth to a baby \cPirateChild{\child}, \cEvilNemesis{} nearly eliminated your family by having your house set on fire. You only just saved your \cPirateChild{\offspring} — but could not save your \cPirateChildParent{\spouse}.

Faced with proof of how merciless your nemesis was, you made a decision you knew was necessary — but that still haunts you every day of your life. To protect your \cPirateChild{\offspring}, you gave \cPirateChild{\them} away. Through a series of contacts, you tracked down a trustworthy connection in \cEthics{\full}, a teacher at the College, and asked \cEthics{\them} to pass the child on to a family somewhere you could never find. Your \cPirateChild{\offspring} was safe, but \cPirateChild{\they} \cPirateChild{\were} also gone from your life forever; that was 18 years ago.

Eight years ago, after 10 years of trying to rid yourself of \cEvilNemesis{} through political maneuvering and backstabbing, you finally, FINALLY, succeeded. You were now one of the most powerful members of the \pFarm{} nobility — at least as far as soft political power goes. You still had no lands, little money (in the grand scheme of things), and only a minor title. But now that your own life was mostly secured, you finally had time to look beyond your own borders, and began collaborating with \cDiplomat{\full}. 

\cDiplomat{} dreams bigger than anyone else you’ve ever worked with, and \cDiplomat{\they} \cDiplomat{\are} a consummate utilitarian. The scientists of the \pTech{} had begun researching how to permanently end the Storms, but their progress was threatened by the prospect of the next Storm being directed at their nation in accordance with the treaty. \cDiplomat{} wanted your help buying their scientists a little time — by directing the Storm at \pShip{} out of turn instead. This act would of course break the ``Time of Peace’’ treaty that had dictated the order of the Storm for over three decades. But the world would be a much better place without the Storms — anyone could see that.  

While \cDiplomat{} was a major piece of the puzzle, you also quietly workshopped a great deal of the \pFarm{} part of the treaty with another contact from court, \cHistory{\full}, a noble who teaches history at the \pSchool{}. \cHistory{} is an expert on the political history of the nations, so on \cHistory{\their} visits to court over school breaks, you two found that you were in the same circles and conversations, and that your minds worked in similar ways. You became confidantes and they have helped support your schemes and diplomatic efforts on occasion. You are cautious about which ones you bring to \cHistory{\them} though, as your moral codes do not always line up - while you will always value what  is good for you, \cHistory{} would not always agree that those things are also good for the nation.

 \cHistory{\They \have} an incredible eye for tying things to precedent, for turning the past and tradition towards one’s own goals, and for giving current action context and heft.  \cHistory{} didn’t know your full plan to make sure things went your way regardless of what actually happened that weekend six years ago, but was and still has been a great support in ensuring that the treaty remains unbroken and supported in court, even though the war has dragged on for six years.  They did suggest signing the treaty in secret as well.

Due to the short-sightedness of other politicians , the ``End of Suffering’’ treaty was signed only secretly initially. It was not until just before the ``Time of Deciding’’ six years ago that it was made public. Out of concern for whether the students would accept the need to send the storm to the \pShippies{}, you and \cDiplomat{} prepared to ensure that the Storm went where it needed to, regardless of whether the students cooperated.

You enlisted a scrappy scientist named \cHeadScientist{\full} as the cat's paw for your plan, since \cHeadScientist{\they} \cHeadScientist{\were} slated to be at the \pSchool{} for that ``Time of Deciding’’ as an advisor from the \pTech{}. You procured a set of fake Voting Stones, had  \cHeadScientist{} swap them in right before the voting stones were passed out to everyone, and then swap the real Voting Stones back in after the Storm's course was set so there would be no hanging thread. The fake stones were a huge source of unregulated magic that \cHeadScientist{} could then do with as \cHeadScientist{\they} pleased. You asked \cDiplomat{} that if the vote went well, to promote \cHeadScientist{} (though \cDiplomat{} still does not know why). The stones and promotion together were ample payment for this minor task. You purposefully kept \cHeadScientist{} in the dark about every other part of the plan.

Phase 2 of the plan was to deal with the 12 students whose votes you had rigged to send the Storm to \pShip{}, who could easily have raised the alarm about the fraudulent outcome — \cDiplomat{} was hesitant to do what was necessary, but you were willing to be the villain if that’s what it took to make the world a better place. So you made the necessary arrangements; a few words in the right ears, and you had yourself an assassin willing to sacrifice their memories in order to poison the wine used for the celebratory toast. Even though you are often one of the advisors for the ``Time of Deciding,’’ you begged off that year in order to maintain plausible deniability, claiming you wanted to ``give someone else a chance.''  You instead spent the weekend with your friend, \cHistory{}, ensuring you had a strong alibi.

Perhaps that was a mistake, because without your guiding hand the assassin failed to definitively eliminate all 12 students; \cDiplomat{} informed you afterwards that only 11 bodies were found, with one student, \cKidScientist{\full}, missing. Despite searching far and wide, you were never able to locate \cKidScientist{\them}. While the most plausible explanation is that \cKidScientist{\they} attempted to flee and died of the poison somewhere in the wilderness, you have not gotten this far by discounting the longshot scenario — that \cKidScientist{} may yet live. The official line from the College has consistently been that all 12 students perished, raising your suspicions that someone on the faculty is covering for \cKidScientist{} and may have helped \cKidScientist{\them} escape. You \emph{have} to keep your eyes peeled this weekend. If \cKidScientist{} resurfaces now, \cKidScientist{\they} could ruin everything. Should \cKidScientist{\they} be so foolish, you will need to find a way to silence them. Maybe you should preemptively seek out someone willing to play the part of assassin? 

After that whole sordid ordeal, you reclaimed your triennial position as an advisor for the ``Time of Deciding,’’ pointing out that the advisor who had stood in for you was incompetent, given that 4 of the best minds in the \pFarm{} perished under their watch.

Fast forward to today, and naturally, you’re facing a whole different host of problems. The pirates from the \pShip{} are more of a nuisance than ever, but even worse, the whole damn nation is at war with yours and the \pTech{}. War is a complicated beast — especially this one. The betrayal of the \pShip{} will ultimately benefit them too — ending the Storms is worth almost any sacrifice. Endless generations without the Storm are surely worth a couple of tough years. But the \pShip{} had to go start a war, and make the loss of life so much higher than it needed to be. There was no need for \pFarm{} blood to be spilt. And how many \pShippies{} have lost their memories through the violence? They are as good as dead to their loved ones too. 

You intend to end this stupid war in your nation's favor, by any means necessary. \cLoud{\full} is the lynchpin on which the \pShip{} war effort hinges. If \cLoud{\they} could be assassinated, the \pShip{} would soon fall under the collective might of the \pFarm{} and the \pTech{}. The key to bringing \cLoud{} low lies in \cLoud{\their} \cWarlordDaughter{\offspring} being at the \pSchool{}. You have begun preparations to use \cWarlordDaughter{\them} as leverage to get the warlord to show \cLoud{\themself} at a known location so you can send a pre-arranged assassin to end \cLoud{\them}. You will probably need the help of your old co-conspirator, \cDiplomat{}, in order to pull this plan off, but fortunately, eliminating the Warlord is in both your interests.

Outside of \cDiplomat{}, you have been speaking with \cAntiChup{\full}, the head of the Church of \cTechGod{} for about two years now. \cAntiChup{\They \have} proven to be just the kind of person you like working with and also know to be careful around: pragmatic, driven, and with an understanding of how cruel the world truly is. Together, you have had some excellent brainstorming sessions about how to cripple the \pShip{}. An idea that occurred to \cAntiChup{\them} that you thought brilliant but likely impossible was to murder an Avatar of one of the \pShip{} Goddesses, \cEbb{} and \cFlow{}. This would require killing a sea serpent that was literally part of a God, but you encouraged \cAntiChup{\them} in their daydream planning of it nonetheless. You have not heard anything about an Avatar dying, so you are pretty sure it didn’t happen, but you want to be sure to help and encourage \cAntiChup{\them} in \cAntiChup{\their} schemes as much as benefits you and the \pFarm{}.  

Alas, \cHeadDiplomat{\full}, from the \pShip{} and a few other delegates have started talking about a new peace treaty. You have no interest in a treaty or ceasefire when total victory is within your grasp. Technically, you are the head \pFarm{} Advisor, so you are supposed to sit in on these negotiations. If you can't convince the other Advisors to disband or boycott the talks, you might be able to pawn the task of pretending to listen off on either \cWildCard{\full} or \cHedonist{\full}, two of the other Advisors. Unless, of course, someone among the \pFarm{} starts advocating for peace, in which case you'll have to bring your authority to bear and make sure they don't influence the proceedings. Again, \cDiplomat{} will be a key ally in this. If neither nation is willing to talk to the \pShip{}, ceasefire talks are dead before they start. You also hope that \cHistory{} will be able to support you, albeit quietly, as you know that as a teacher \cHistory{\they \are} supposed to remain neutral and it would be unfortunate to lose your ally such a beneficial position.

Even after eliminating \cEvilNemesis{}, you still have many enemies, some of whom have followed you for much of your life, and others that are new. \cInterpol{\full} is a pain in your ass as well. \cInterpol{\They} \cInterpol{\are} always nosing into your business, \emph{supposedly} by coincidence. And now, ``coincidentally,’’ \cInterpol{} is employed here at the \pSc{}. As for \cDiplomat{}, the two of you have lots of history of collaborating — but that really just means that you both have dirt on each other. You intend to gain the upper hand in this power struggle, to ensure that if anything you've done together were to come out, you could pin it all on \cDiplomat{} and get away scot free. You don’t trust \cDiplomat{} to actually hold a firm line alongside you. You’ve never had such an ally in your long political career. In the end, everyone is just out for themselves, right? The only exception to this rule may be \cHistory{}, but it is likely because \cHistory{} barely counts as a politician.  

With land on the southern coast of the \pFarm{}, you have a deep, abiding, and systemic dislike of the pirates that harass your people. You even had a run in with the infamous Silent Fog pirate ship once. This leads your thoughts to \cPirateChild{\full}, a student you only just met after arriving on campus. There is something suspicious about \cPirateChild{\them} — a certain swagger, a cocky sneer — that just screams ``pirate'' to you. You even went so far as to examine \cPirateChild{\their} application to the \pSc{}, and the papers looked forged to you. Still, \cPirateChild{\they} \cPirateChild{\are} barely an adult, and you have no proof\ldots{} yet. You will have to keep a close eye on \cPirateChild{\them} this weekend.

Unlike \cPirateChild{}, \cLibAssist{} is a student you have known for a while, and are quite impressed by. The \cLibAssist{\formal} family has had a run of bad luck since one of \cLibAssist{\their} older siblings somehow managed to cause a trebuchet to fail, backfiring the curse into all of their soldiers. The entire family fell from grace. When \cLibAssist{} wrote you out of the blue to strike up a correspondence, you were impressed by the audacity and began to pay attention to \cLibAssist{\them}. \cLibAssist{\They} \cLibAssist{\are} ambitious, ruthless, and cunning. Everything you value in your own work. While you haven’t been willing to provide official patronage to \cLibAssist{} yet, you have provided extensive advice through long letters. You look forward to meeting and supporting \cLibAssist{\them} in person this weekend. If they impress you, you’ll be happy to offer them your patronage.

You are also intrigued by the fact that a little birdy told you \cCurse{\full} (a \textbf{commoner} who somehow managed to worm \cCurse{\their} way into being an Advisor for the \pFarm{}) and \cFlowPriest{\full} (a \cFlowPriest{\cleric}/teacher from the \pShip{}) have been in constant correspondence, working to enchant some kind of plant, of all things — which, as far as you've learned, might manage something as amazing as to open a path to the Realm of the Gods! Obviously, once the enchantments on them are completed, they would hold enormous value. You wonder if you could convince \cCurse{} to bring the plant home to the \pFarm{}? Imagine how much power the \pFarm{} would have if you controlled the only path to the Divine Realm. 

Also close to home is your relationship with \cPrince{\full}. The two of you work together often, and you are even willing to admit that you appreciate \cPrince{\their} company sometimes, and respect \cPrince{\their} scheming mind. You have invested serious resources in facilitating \cPrince{\their} rise to power, knowing full well that \cPrince{\they} intend\cPrince{\verbs} to take the throne one day. You have no interest in seeing the brutal, cutthroat environment you thrive in change and the current heir apparent is anathema to the court of \cQueen{\full}.  

You don't usually take sides before it is obvious which way a conflict will resolve, but you took this risk — in exchange for a promised elevation of your public position, of course (e.g: lands, money, and titles). It should serve both of your ends that you are here at \pSchool{}, where \cPrince{} is currently teaching and able to collaborate in person. While you trust \cPrince{} more than you trust \cDiplomat{}, that isn’t saying a lot. If you come to suspect that \cPrince{} is using you, or not committed to maintaining the political environment you both thrive in (e.g. by discouraging ruthlessness), you are not afraid to bail on \cPrince{\them} and leave \cPrince{\them} high and dry. Or at least make sure that you raise the price of your loyalty as high as you can. You want to get what is needed out of this bargain.

The need to cut your loses make come sooner rather than later, as \cPrince{} appears to have run into some trouble of late. Word among your informants is that \cPrince{\their} signet ring might have been\ldots{} liberated from \cPrince{\their} possession. Probably by pirates — a deep seated hatred of pirates is something that you and \cPrince{} share and the pirate activities have gotten worse and worse since the war started Such artifacts are incredibly politically powerful, and even aside from what you could do with it yourself, the \cQueen{\Monarch} would reward you handsomely for returning it — at the very least with lands and a title of your own, which would secure your legacy. You could of course also give it back to \cPrince{}, if \cPrince{\they} could offer you something more valuable for it (Maybe \cPrince{\they} could be convinced to name you \cPrince{\their} personal spymaster? Or to create a more public role like a chancellor or minister who could preside over the court, leaving \cPrince{} free to attend other matters while you run things?). But it has never been your practice to put the cart before the horse — first you have to find the damn thing. The local Black Market might be the first place to look, though you'll need to be subtle about trying to make contact with its members. Unfortunately, such a powerful artifact is sure to have a protective charm on it that will make it difficult or impossible to recognize. You may need to recruit someone with the proper\ldots{} skills to find the ring without anyone being the wiser. Surely you couldn’t be so lucky that \cLibAssist{} would know who could be discreetly commissioned to the task?

Amazingly, not all of your current problems have to do with politics and intrigue. In the last few months, you've found yourself missing the \cPirateChild{\child} you gave up 18 years ago. You've sent prayers to \cFarmGod{}, asking to know if \cPirateChild{\they} \cPirateChild{\are} at least safe. Those prayers have not been answered — you assume that while it is possible to conceal your involvement in many murders over the years from the mundane authorities, \cFarmGod{} knows and has turned \cFarmGod{\their} back on you. It really is frustrating to have a Deity who cannot see that sometimes it's you or them and murder, betrayal, and death are unavoidable. You have not asked \cEthics{} where \cEthics{\they} sent \cPirateChild{\them} due to fear of endangering yourself and your child.  But now may be the time.  

There's much to accomplish, and little time in which to do it all — but your brilliant mind can handle it. No one can manipulate a situation to their own ends quite like you can, so you expect to be quite successful indeed. By the end of the weekend you'll hopefully have acquired for yourself a few choice items of power and made your enemies’ lives a living hell. You can practically taste the victories.

\begin{itemz}[Goals (in roughly descending order of importance)]
    \item Assassinate the \pShippies{} Warlord, \cLoud{\full} with help from \cDiplomat{}.
    \item Make sure the treaty negotiations with the \pShip{} fail, and that the Storm gets sent to the \pShip{} per your nation's alliance with the \pTech{}.
    \item Make sure no one finds out you arranged the assassination of the students six years ago.
    \item Acquire the \pFarm{} signet ring that \cPrince{} lost, and decide how to best use it to your advantage.
    \item Convince  \cCurse{} to bring the magic plant they are working on back to the \pFarm{} rather than doing anything with it here at the \pSchool{} or letting it go to another nation. If \cCurse{} cannot be convinced, you are not above using violence or theft.    \item Sniff out and destroy any filthy pirates you find. They've always been a thorn in your side as a leader in the \pFarm{}, and you'll be damned if you let them infest the \pSc{}. \cPirateChild{} and \cJuniorStatesman{} have particularly drawn your suspicion.
\end{itemz}

\begin{itemz}[Notes]
    \item Due to your dislike of pirates, and frequent clashes with them, you've picked up a little about their code phrases. Pirates often use code phrases that refer to the graveyard where their ships hide as ``untouched waters,'' and tools related to navigation and ``charting courses.'' Anyone using such language is to be viewed with deep suspicion of being a pirate.
    \item You are almost always one of the advisors for the Time of Deciding — this is unusual and a sign of your unique position; advisors usually rotate regularly. You begged off six years ago, saying you wanted to ``give someone else a chance,’’ and then promptly took the position back afterwards, claiming, ``clearly they were incompetent, given we lost four of the best minds in the \pFarm{} under their watch.’’
    \item International espionage is a thriving trade these days, and given the pivotal events happening this weekend, you would not be at all surprised if spies from all three nations are present.
    \item The Order of the Black Crocus is, or rather was, a secret international law enforcement entity with jurisdiction across all three nations. As far as you know, it splintered into factions when the war started, each faction subsumed by the intelligence services of its corresponding nation. Good riddance, as far as you're concerned. They were a meddlesome bunch.
     \item Feel free to make up a name for your \cPirateChild{\offspring}. You gave \cPirateChild{\them} up before \cPirateChild{\they} were one year old though, so you know nothing of how \cPirateChild{\they} might be now.
\end{itemz}

\begin{contacts}
    \contact{\cDiplomat{}} A powerful diplomat from the \pTech{}, and your co-conspirator on the covert operation that redirected the Storm to the \pShip{} six years ago. You were the one that covered your tracks. While \cDiplomat{\they} is a useful tool, you trust \cDiplomat{\them} about as far as you can throw \cDiplomat{\them}.
    \contact{\cPrince{}} Your generalized ally, in addition to being a \cPrince{\Heir} of your country, angling to inherit the throne. You aren’t afraid to increase the price of your aid if \cPrince{\they} try to take actions that don’t serve your agenda.
    \contact{\cInterpol{}} Has a nasty habit of knowing too much of your business (and meddling in it), and you're sure there's more to \cInterpol{\them} than meets the eye.
    \contact{\cHeadScientist{}} A scientist you've manipulated in the past. You've mostly kept your distance, as \cHeadScientist{\their} value lies primarily in \cHeadScientist{\them} not being easily traced back to you.
    \contact{\cPirateChild{}} A student who has drawn your attention as behaving suspiciously\ldots pirate-like. \cPirateChild{\Their} paperwork looks forged to you, but you have no definitive proof yet, just a feeling\ldots
     \contact{\cLibAssist{}} A student who has drawn your attention as an excellent candidate for you to mentor.
    \contact{\cCurse{}} A \textbf{commoner} and begrudgingly renowned Curse Maker from the \pFarm{}, who is secretly creating a means of accessing the Divine Realm.
    \contact{\cAntiChup{}} An ally of yours for the past two years with whom you brainstormed a plot to assassinate the Avatar of \cEbb{}; as far as you are aware, \cAntiChup{\they} never followed through on the admittedly impossible scheme.

\end{contacts}

\end{document}





