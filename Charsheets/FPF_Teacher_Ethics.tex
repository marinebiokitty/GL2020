\documentclass[char]{GL2020}
\parindent=0pt
\begin{document}
\name{\cEthics{}}

You are \cEthics{\intro}, 44 years old, and a teacher at the \pSchool{}. You teach Ethics and Morality, which is quite the load to bear right now, given how unethically so many powerful people on \pEarth{} are behaving. You've found yourself bending more and more rules to stay true to your moral compass, and you only hope you can hold it together through this weekend; with a possible end to a senseless war, now is no time to fall apart.

You grew up in the \pTech{}. Your family lived in a poor neighborhood, but scraped together enough money to send you and your younger sibling to good schools. The merit scholarships you collected along the way certainly didn't hurt. Even from a young age, the disparities were obvious between your family and others from your school. It was enough to keep you up at night, thinking about what made someone \emph{good}. For years you thought, wrote, debated, and advocated for change in the \pTech{}. After getting fed up with your well marshaled arguments falling on the deaf ears of the Council, you decided to turn to teaching instead.

You started teaching at the \pSc{} 20 years ago; your career as an activist and writer helped land you the coveted position at such a young age. Two years later, you were approached by one of the advisors from the \pFarm{}. \cEvil{\They} indicated a need for utmost secrecy to help save a life, and that \cEvil{\they} heard from all sources that you were one of the most ethical and trustworthy people in \pEarth{}. Once you understood the situation — someone was hunting \cEvil{\full}'s family for political gains — you agreed to take \cEvil{\their} baby \cPirateChild{\child} and hide \cPirateChild{\them}. You reached out to \cPirate{\full}, a teacher brand new to the school at the time, and a person you'd grown to be fast friends with. \cPirate{\They} still had a lot of connections in the \pShip{}, whereas yours from the \pTech{} had already started to atrophy. \cPirate{} agreed to take the \cPirateChild{\child} and find \cPirateChild{\them} a good home. It hurt your heart to just send the \cPirateChild{\child} off to an unknown fate, but there was nothing for it — the best way to keep the \cPirateChild{\child} safe was to hide \cPirateChild{\them} as well as possible.

In honor of that wayward child, you do your best to shelter and support every student that comes to the school. You are firm with your boundaries, but are as kind and understanding as possible. You relish helping students work through their moral quandaries and have become somewhat of a popular person to ask for advice. You try to instill in each of your students the idea that ethics should be subservient to morality and that tradition is no reason to withhold justice. You wish that you had more good help in giving the students the support they need. One of your longtime colleagues, \cMusic{\full}, means well, but seems more interested in feeding them soothing platitudes than concrete advice and much needed perspective. You understand the value in tending to emotions, not just intellect, but you feel that \cMusic{} takes it too far, and you clash frequently, albeit quietly.

Indeed, the first big test of morality your students face is the vote to direct the Storm. You thought you were doing pretty well with the students six years ago — but when the votes were revealed, and the \pShip{} betrayed, you were horrified. And then to lose every one of them to an unsolved murder? You had your hands full in the immediate aftermath, working closely with \cFlowPriest{\full} and \cMusic{\full} to provide support for everyone. Many of your colleagues ended up resigning after that. You nearly quit yourself out of depression at your own failure. But no — you had to stand firm. You had to stay, and try harder for the next group. While you loathe whoever orchestrated this betrayal, your primary concern right now is to help your current students find the light — and you have the strong hope that the student's this year could be made of stern enough stuff to right the wrong wrought six years ago. If the students refuse to support the ongoing treaty between the \pFarm{} and the \pTech{}, the advisors may be persuaded to make reparations to the \pShip{} and re-establish some form of the original treaty that governed the Storm. You just have to convince the students to take that stand without being accused of orchestrating it yourself.

But even if the Storm is directed away from the \pShip{} this year, there is still the war to consider. To that end, you have begun engaging with one of your \pShippie{} students, \cWarlordDaughter{\full}, who happens to be the Warlord's \cWarlordDaughter{\offspring}. \cWarlordDaughter{\They} seem\cWarlordDaughter{\verbs} to really enjoy your class, which makes a great starting point for conversation. \cWarlordDaughter{} is key to ending the war, in your opinion. There are critical treaty negotiations going on this weekend that have the potential to end the war. But even if the \pFarm{} and the \pTech{} are ready to discuss reparations for the \pShippies{}, the \pShip{} must also be willing to come to the negotiating table. Your best chance for insight into the warlike \cLoud{\full} is \cLoud{\their} \cWarlordDaughter{\offspring}, \cWarlordDaughter{}. What information you can find on Black Crow ship suggests that \cLoud{} was quite the calm and patient co-captain with their partner \cQuiet{\full}. You suspect something must have happened to shift \cLoud{\their} personality so drastically, and hope that if it can somehow be reversed, the \pShip{} may be willing to negotiate a truce. You want \cWarlordDaughter{} to trust you as you can tell just by talking to \cWarlordDaughter{\them} that \cWarlordDaughter{\they} also feel\cWarlordDaughter{\verbs} that something is wrong and \cWarlordDaughter{\are} yearning for \cWarlordDaughter{\their} own voice. Maybe giving \cWarlordDaughter{} your Voting Stone would cement this trust, and you both can work together on figuring out what exactly happened to \cLoud{}. Besides, you can be almost certain that \cWarlordDaughter{} would not send the Storm to the \pShip{}, making \cWarlordDaughter{} the safest bet for the outcome you want.  

But as long as the war is still on, you intend to at least do what you can to alleviate the suffering it causes. In pursuit of this, some years back you stumbled backwards into the black market. You never meant to get so involved in illegal activity, but when all the legal avenues are blocked, it is a moral obligation to use an illegal one to make necessary change. You originally got involved dealing information, feeding bits and pieces into the \pShip{} spy network, in hopes of giving the \pShippies{} the upper hand in the war for long enough that the \pFarm{} and the \pTech{} would agree to a ceasefire, and ultimately a just treaty that makes up for the wrongs done these past six years. Now you are involved in many more things — helping pass information, technology, and even food to people in desperate need of it. You are pretty sure that you have kept your involvement a secret so far, since you have been using \cLibrarian{\full}'s assistant, \cLibAssist{\full}, to pass information back and forth between you and the black market. Lately, though, you have begun to question whether \cLibAssist{\they} can be trusted, and where the items and information you pass on to \cLibAssist{\them} are really ending up. You have been increasingly troubled by the shifty nature of the market and what else they might be dealing in. 

Fortunately, you were contacted a few months back by someone in the \pShip{} spy network, who gave you a way to occasionally pass them information directly, rather than going through the black market. They have an agent here this weekend among the \pShip{} advisors whom you can pass information and items to. That agent is none other than the engineer charged with maintaining the Bunkers, \cBunker{\full}. You are nervous about making direct contact with a \pShip{} spy, but it will certainly speed up communications. 

Things have become complicated, though. Ethics is a lonely discipline, but you have been fortunate to make a few friends among the staff. One of them is \cHistory{\full}, the history teacher who arrived about five years after you. Though from very different backgrounds, you two quickly became friends, delighting in spirited arguments over the ethical actions of the past, commiserating over the students you were trying to help, and finding in each other a wonderful sounding board for your own beliefs, different as they were (and are). You don’t share everything, of course — \cHistory{\they} certainly doesn’t know about your spy activities, though you may have let slip some of your feelings towards \cBeetle{}, which is harmless, right? But you respect each other and care deeply for each other. Unfortunately, while you were in the teacher’s lounge one day, you noticed a small compartment under the table where \cHistory{}, the History teacher at the \pSchool{},  normally sits. You were worried that someone may be tampering with a fellow teacher and so you worked at it for awhile and finally got it open. What was in there was not what you expected — a coded letter signed with the personal seal of the \cQueen{\Monarch} of the \pFarm{}! You hastily, and somewhat ashamedly, put the letter back, so you don’t know what it said (not that you would necessarily have been able to decode it anyway), but the existence of that letter makes it pretty clear that \cHistory{} is a spy for the \cQueen{\Monarch}.

This places you in an awkward position. You know this information would be important to the \pShip{}, but you have moral qualms about sharing it, as \cHistory{} is a good friend, a colleague, and excellent teacher; the information could completely ruin \cHistory{\their} career, and maybe even jeopardize \cHistory{\their} safety. Besides...it's not that you have the moral high ground of not being a spy yourself. But if it turns out to be the key to ending the war, you're morally obligated to share it\ldots{} aren't you? It is tough to care so deeply about what is right. Normally, you would seek advice from your friend and mentor at the school, Principal \cPrincipal{\full}, but in this case that is out of the question. \cPrincipal{} is the first one \cHistory{} would be in trouble with.

With this new revelation, you are so grateful you have your best friend at the school, \cBeetle{\full}, the Religion teacher and a \cBeetle{\cleric} of \cTechGod{}. You have both been teaching at the college for many years, and have developed a deep connection based on mutual care, trust, and intellectual rigor. You have found \cBeetle{\them} to be thoughtful, intelligent, and possessed of a robust moral philosophy. You two even agree on how unjust the situation has been to the \pShippies{}, though you don’t think \cBeetle{} is taking as drastic measures as you are to right wrongs. 

Things got a bit\ldots complicated three years ago when you met \cJuniorStatesman{\full}, \cBeetle{}’s romantic partner and apprentice to the head \pShippie{} diplomat. \cJuniorStatesman{} was charming, passionate, intelligent, and kind, and you found yourself strongly attracted to \cJuniorStatesman{\them}. \cJuniorStatesman{} and \cBeetle{} were and are in a polyamorous relationship (something unheard of in your homeland, but you’ve had a long time at the college to learn about other relationship structures), and you decided to take the risk and talk to both of them about your feelings. To your surprise and delight, \cJuniorStatesman{} was attracted to you too and \cBeetle{} was okay with it if the two of you wanted to explore that. That weekend fling was one of the most passionate and heady experiences of your life. Unfortunately, the war complicates things greatly, and \cJuniorStatesman{} has made no move over the past few years to deepen your relationship. You hope this is purely due to the potential optics of the situation (even though teachers at the \pSc{} are supposed to be neutral, you still hail from the \pTech{}) and not due to some decrease in interest — surely \cJuniorStatesman{} would have told you outright if that were the case? You are both excited and nervous that \cJuniorStatesman{} will be here this weekend. You really need to speak with \cJuniorStatesman{\them} about what exactly your relationship status is, and where you two want to see it go in the future.

Your weekend fling with \cJuniorStatesman{} also got you wondering about your friendship with \cBeetle{}. The two of you have had such a strong connection for so long, a deep and abiding love that a part of you yearns to explore further. Another part is terrified of complicating things or harming the friendship if it doesn’t work out. With \cJuniorStatesman{} here this weekend, you will have the opportunity to talk with both of them about your feelings, if you can work up the courage to share them as you did three years ago. Back home in the \pTech{}, polyamory may be considered scandalous, but the heart wants what it wants…doesn’t it?

This weekend, there is a war to end, and countless injustices to set on a path to healing. As one of the school's most beloved teachers, you have the ability to influence the students to make the right choice, and advocate for an end to the war. You have espionage work to do and hard choices to make about which secrets to share. And on a more personal note, you need to figure out what you are going to do about your feelings for \cJuniorStatesman{} and \cBeetle{}.

\begin{itemz}[Major Goals]
    \item Convince the students to vote to send the Storm somewhere other than the \pShip{}.  Giving your Voting Stone to \cWarlordDaughter{} would help, but \cWarlordDaughter{\theyare}n’t the only student you could give your stone to.
    \item Find a way to end the war. The treaty negotiations this weekend are one avenue. To aid those negotiations, you are hoping that you can speak to \cWarlordDaughter{} about how to get \cLoud{} to stop the relentless \pShip{} assault, or at least come to the bargaining table.
    \item Figure out what exactly the ``situationship’’ is between you and \cJuniorStatesman{}. Related, decide how and whether to approach \cBeetle{} about starting a polyamorous relationship.
\end{itemz}

\begin{itemz}[Minor Goals]
    \item Find information on the \pFarm{} and the \pTech{} to feed to the \pShip{} via your spy contact, \cBunker{}, or through the Black Market (via \cLibAssist{}). That includes deciding whether or not to reveal that \cHistory{} is a spy for the \cQueen{\Monarch} of \pFarm{}.
    \item Keep an eye out for \cMusic{} and make sure that students seeking \cMusic{\their} advice have the opportunity to get more well rounded advice from you as well.
\end{itemz}

\begin{itemz}[Notes]
    \item You teach Ethics and Morality at the \pSchool{}.
    \item You were present at the school for the Time of Deciding six years ago, but don't know anything beyond the official story. You were too busy providing emotional support to the advisors and your fellow teachers, and trying to cobble together some form of self care, to want to get underfoot with the investigation.
\end{itemz}

\begin{contacts}
    \contact{\cMusic{}} A \pFarm{} teacher who coddles the students a little too much, trying to shield them from realities that they would be better off facing in a safe environment like the \pSc{}.
    \contact{\cLibAssist{}} The Librarian's assistant, and your one point of contact with the black market. You are beginning to question whether you can really trust \cLibAssist{\them}.
    \contact{\cWarlordDaughter{}} A student from \pShip{}, the \cWarlordDaughter{\offspring} of the Warlord, and, you hope, the key to ending this senseless war.
    \contact{\cHistory{}} The history teacher at the \pSc{}, a good friend of yours, and apparently a spy for the \cQueen{\Monarch} of the \pFarm{}?!
    \contact{\cBunker{}} The advisor charged with maintaining the Bunkers at the school. Also, as it happens, your \pShip{} spy contact.
    \contact{\cBeetle{}} Your compassionate, idealistic, and charmingly awkward best friend. A part of you yearns to start up a romantic relationship with \cBeetle{\them}, but another part is terrified it could backfire and destroy your friendship.
    \contact{\cJuniorStatesman{}} Attractive, intelligent, kind, and complicatedly enough, \cBeetle{}'s polyamorous, long-distance partner. The two of you had a passionate fling three years ago and you haven’t been able to get \cJuniorStatesman{\them} out of your mind since.
    \contact{\cEvil{}} A \pFarm{} advisor who asked for your help saving a life 18 years ago.
    \contact{\cPirate{}} A fellow teacher, from the \pShip{}, to whom you entrusted that life. You grew to be fast friends 18 years ago, and that friendship has endured.
\end{contacts}

\end{document}
