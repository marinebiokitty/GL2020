\documentclass[char]{GL2020}
\parindent=0pt
\begin{document}
\name{\cHistory{}}

You are \cHistory{\full} (\cHistory{\they}/\cHistory{\them}) and comfortably middle aged. You teach History at the \pSchool{}, helping to shape new minds while simultaneously secretly managing \cQueen{\Their} Majesty's personal spy network. It's not \emph{exactly} accurate to say that your loyalties are split, but it would be wrong to say that after over fifteen years at the \pSchool{} you have completely avoided becoming a little attached to the place. The students here need guidance, the \pSc{} needs maintenance, and sometimes international politics needs a swift kick in the rear. It is a busy life, but you are good at it, you enjoy it, and the duplicity feels so natural at this point that you hardly notice the qualms.

You are the firstborn \cHistory{\offspring} of a middlingly powerful noble family in the \pFarm{}. You and your siblings wanted little growing up, and you navigated the viper's nest of Court with aplomb. While your younger \cWildCardFriend{\sibling} \cWildCardFriend{} attended the \pSchool{}, you attended a university in the \pFarm{} and majored in history, which you genuinely enjoyed. Your thesis was a treatise on the creation story. Your life's ambition was to teach at a \pFarm{} university for a few years before marrying well and settling down on your family's estate to raise a family. Sure enough, you earned a history professorship at the same prestigious university you attended; you even landed a side gig as a history tutor for the children of the powerful \cChupStudent{\formal} family after your \cWildCardFriend{\sibling} married into the family. (Tutoring is an unusual passtime for the firstborn of a family, but you have always loved every aspect of teaching). \cChupStudent{\full} was a delight to teach, a curious and intelligent youngster who never shied away from a challenge.

During this time, you also became close to \cWildCard{\full}, a good friend of your \cWildCardFriend{\sibling}’s, as \cWildCard{\they} regularly visited the \cChupStudent{\formal} family. You both took delight in helping \cChupStudent{} while talking late into the night about your scholarship and debating theories about the creation of the world, the nature of the Gods, and the best way to quickly read through archival material. Sometimes \cWildCard{\they} even talked about the Divine Realm, wondering about its true nature. You two have kept in touch,discussing the latest discoveries and theories in long letters while visiting each other when you can. 

Everything in your life seemed to be going according to plan — until your teaching credentials and skill at courtly intrigue brought you to the attention of the most powerful people in the country, including \cQueen{\full} \cQueen{\themself}. In a heartbeat, your perfect life was swept aside in favor of the needs of the nation. A position at the \pSchool{} was quickly arranged for you. At first, it was simply in exchange for small reports on the activities of your fellow teachers, sent home clandestinely. Then came a request for a verbal report, made to \cQueen{\Their} Majesty directly. And then it was private conferences with the \cQueen{\Monarch}, time spent advising \cQueen{\them} on matters of national importance. It was always in private and unofficial, but it occurred with increasing frequency. Before you really knew how it happened, you were your \cQueen{\Monarch}'s personal spymaster. Sure, there is an official state spymaster — but you answer to the \cQueen{\Monarch} alone. If this became known, at best you would lose your job. Teachers are supposed to be neutral entities, teaching all students equally, and to the benefit of all three nations. At worst, you would be struck down in secret by the nobles of your own nation, or by jealous rivals in other nations who could finally strike at you with your mask removed.

You therefore took it very seriously when recent intelligence reports suggested the presence of two other clandestine operatives at the school this weekend. If the reports are correct, a spy from the \pTech{} will be here amongst the advisors, and another spy of unknown nationality has been operating here amongst the teachers for some time. You've long suspected the latter, as you recognize the fingerprints of another operative at work even if you can't pin down who it is. Given the pivotal events happening this weekend, you would not be surprised if the unknown spy is from the \pShip{}, but that is just a hunch. You'll have to stay on your toes, and make sure these other operatives don't find you or do anything to interfere with your objectives.

That is not to say you have not made friends, even though you always stay on your guard. Many have come and gone, but your friend \cEthics{\full}, the ethics teacher, has stayed. \cEthics{} started out at the \pSchool{} five years before you, and as such was a bit of a mentor. You two come from very different backgrounds, and have quite different views of the world, but find common ground in your delight over looking at what is ethical in history, in debate, in challenging each other’s viewpoints and raising difficult questions. \cEthics{} is an intellectual equal, though \cEthics{\theywere} raised poor in the \pTech{}. You do know that \cEthics{} has sympathies towards the \pShip{}, and so you are careful when discussing current politics. It hurts you a bit that you cannot be fully open with a friend, especially because \cEthics{\they} let it slip that \cEthics{\theyare} romantically interested in \cEthics{\their} close friend \cBeetle{\full} to you, and you wish you could be equally open.  

In service to your cover as a teacher, and in honor of your first love (History), you do your best with the students. Not all young minds start out eager to learn, but every class has one or two who really take to it. While your expertise is in the history of the \pFarm{}, you teach the history of all of \pEarth{}. You have a lot of hope for \cLibAssist{\full}, but otherwise the current crop of students is a little disappointing. \cLibAssist{} has a cunning mind that is quite skilled at seeing flaws in plans. You have been delighted to support \cLibAssist{\them} as best you can without seeming to play favorites. 

But your influence extends well beyond the school, and you can potentially offer \cLibAssist{} much more. Realizing that you needed an agent amongst the students and recognizing \cLibAssist{}'s potential, you recruited \cLibAssist{\them} a year ago to be your spy and informant, and \cLibAssist{\theyhave} not disappointed. The more valuable intelligence \cLibAssist{\they} bring\cLibAssist{\verbs} you, the more strings you will be able to pull with the \cQueen{\Monarch} to see that \cLibAssist{} goes far in life, \cLibAssist{\their} lowborn status notwithstanding. The most recent tidbit \cLibAssist{\they} shared with you is the mysterious disappearance and reappearance of the \iScythe{} from the library a few weeks ago, seemingly without anyone else noticing. Troubling, and intriguing.

Of course, not all students are excellent, and honestly some are just deeply bothersome. Take \cTechStar{\full}, for example. While no one can doubt \cTechStar{\their} magitech abilities, as \cTechStar{\theywere} named TechStar in the 0\pTech{}, \cTechStar{\they} constantly push\cTechStar{\verbes} back on your lessons, either questioning the use of studying history or trying to make arguments against traditional wisdom in some effort to prove \cTechStar{\their} radical credentials or intelligence. It all is very tiresome and when \cTechStar{\they} skip\cTechStar{\verbs} class to do “important” things you feel, to your shame, a sense of relief. You are keeping a close eye on \cTechStar{\them}, though, in case the radical ideas become more than just ideas.

The students aside, the \pSc{} is a marvel unto itself. The floating island on which it is built is unique — it is sustained by the Ley Line Nexus, which needs to be maintained with the utmost care. You've watched that ritual before, always led by the principal, and have contemplated participating in the past — but always decided against it, due to the price (sacrificing some magical energy). Still, some part of you wonders if this might be the year to help out. It is obvious to you, and anyone else who has studied the phenomenon, that the Ley Lines need the island to stay together as much as the island needs the Ley Lines to float. If something catastrophic were to happen to the island, the proximity of so much intense magical energy, unshielded by the rocks and dirt, would shred the fragile Ley Lines in a matter of hours. With the Ley Lines severed, the flow of magic throughout Cengea would slow to a trickle. Only the Deities themselves could repair the damage. Worse, with so little magic, people would starve and die in droves — no food, no medicine, no protection from sea serpents, no technology. You shudder to even think of it.

The Library at the \pSc{} is a wonder of architecture and magic. In it are stored some of the most rare and valuable tomes in all of \pEarth{}, which you count yourself lucky for the opportunity to study whenever you want, as well as many magical artifacts. These priceless pieces are stored here because, while the Storm rages around the island as its power builds, it never directly strikes the school — so items stored here are protected in a way that no other place on \pEarth{} is. For this weekend, the most important such items are the Relics. The Relics can be attuned to one of the three nations, and three of them must be used in the Ritual to Control the Storm. The integrity of the Relics is the purview of the clergy, and the design of the ritual is the responsibility of the students, but the necessary preparations for the Ritual fall to the teachers. You've helped \cLibrarian{\full} with this in the past, and found \cLibrarian{\them} quite pleasant to work with. You look forward to helping with this duty again this weekend.

The third noteworthy part of the school is the Bunkers. These marvels of engineering provide a way to protect the people who stay at the school during the Storm Surges. An advisor named \cBunker{\full} led the team that built them several decades ago, and usually comes by every six years to look in on them (though \cBunker{\they} also came three years ago to perform some extra maintenance that was needed). While you haven't met \cBunker{\them} in person too many times, you've maintained a casual correspondence, and found quite the kindred spirit. \cBunker{} is a gifted engineer, and has a keen mind for current affairs.

But with a war on, maybe anyone with any sense of self preservation has a keen interest in current affairs. Six years ago, the \pFarm{} and \pTech{} signed a treaty that completely screwed over the \pShippies{}. The Storm was directed to the \pShip{} instead of the \pTech{} that year in the most momentous stab in the back of the modern era. You knew about this plan, of course.  You had personally discussed it not only with the \cQueen{\Monarch} but with one of the top diplomats in \pFarm{} who had become a reliable contact, \cEvil{\full}. While \cEvil{they} did not know your true role, as a regular visitor to the court and an expert on the political history of the nations, you two found that you were in the same circles and conversations, and that your minds worked in similar ways. You became confidantes and you have helped support \cEvil{}’s schemes and diplomatic efforts on occasion, as long as they serve the good of the nation. It is not known to many, but the \pFarm{} part of the treaty was brainstormed over many a dinner and glass of wine between you and \cEvil{}.

Everything went as expected at first, but then the students who were voting to direct that Storm were killed. That detail troubled you deeply. How could you have not been told that was part of the plan? Obviously you would have tried to persuade people to find another way, but that was not sufficient grounds to cut you out of that particular piece of intelligence. The \cQueen{\Monarch} had even gone so far as to order you to not stay at the \pSc{} for the ``Time of Deciding’’ that year, so you were not present for the incident. Though \cQueen{\they} claimed it was to avoid any possible suspicion landing on you, a part of you peevishly wonders if it was also to keep you from interfering. That same part also wonders if \cEvil{} knew, but you know \cEvil{} was held back from the school for that Time of Deciding too. In fact \cEvil{}, had spent it visiting you! Regardless, this incident convinced you that you needed to turn more attention to what was going on at home. You recalled some of your agents and began to place them within the \pFarm{} in hopes of never being caught unaware like that again. One need not be a diligent student of history to predict the war that followed the Betrayal. And thus you were confronted with the question: Is there ever such a thing as a just war?

At first the question seemed moot. Your nation was at war. The \cQueen{\Monarch} was for it. Therefore, you were obligated to support it. Until one day, only a few weeks ago, you got a letter from \cQueen{\Monarch} \cQueen{} \cQueen{\themself}. It was written in your own personal code that only the two of you know, and in it was the most unexpected news: The \cQueen{\Monarch} has changed \cQueen{\their} mind and is now against continuing the war! \cQueen{\They} had thought long and hard, seen the toll the war had taken, and finally decided to change \cQueen{\their} stance. \cQueen{\They} ordered you to keep this information secret for both national and international reasons for now. To show \cQueen{\their} hand too soon could alienate the \pTech{} before a defensive alliance could be formed with the \pShip{} (if a treaty between all three nations proves out of reach). To make it publicly known that \cQueen{\they} changed \cQueen{\their} mind without anything to show for it would also be seen as weakness in the eyes of various \pFarm{} noble families, which could easily lead to a civil war and the loss of even more lives.

The \cQueen{\Monarch} wants a treaty, ideally one without too many painful reparations, but as long as it involves protection from retaliation by the \pTech{}, \cQueen{\they} support\cQueen{\verbs} it. Naturally, this comes with the caveat that the \pFarm{} are not remotely prepared to receive the Storm this year. They might be willing to do so in the future, but not this year; the cost of life and destruction to the harvest (and consequently most of the food supply for the whole continent) would be tremendous. It falls to you to influence the negotiations in favor of a treaty without revealing \cQueen{\Their} Majesty's position on the matter. \cEvil{\full} carries the official public responsibility for negotiating at the treaty table, but as an absolute last resort, you have a letter of authority from the \cQueen{\Monarch}. This would allow you to reveal your authority and rank and allow you to speak on \cQueen{\their} Majesty's behalf — but this is only to be used if you cannot accomplish your ends in a more subtle way. You hope your friendship with \cEvil{} will allow you to accomplish this, though you also know that \cEvil{} is quite in favor of the war continuing. You just hope you don’t have to choose between a friend and your \cQueen{\monarch}.

The \cQueen{\Monarch} also tasked you some time ago with keeping an eye on \cQueen{\their} youngest \cPrince{\offspring} and (previously secret) favored heir to the throne, \cPrince{\full}, who teaches Cursebreaking at the \pSchool{}. It's really quite unfortunate that the \cPrince{\Heir} is mixed up with commoner magic, as it contributes to \cPrince{\their} lack of popularity among many of the nobles. The young \cPrince{\heir} generally keeps \cPrince{\their} nose clean, and in general you have found \cPrince{\them} to be a talented, shrewd and charismatic person. However, there are two items which have caught your attention. The first is \cPrince{}'s romance with \cPirate{\full}, the Shop and Crafting teacher at the \pSchool{}. \cPirate{} hails from the \pShip{}, an enemy nation, is not a noble, and records of \cPirate{\their} activities prior to arriving at the college 18 years ago are not corroborated anywhere outside of \cPirate{\their} application to teach at the college. You suspect the records are forged, but have been unable to verify this. In short, \cPirate{\theyare} not a suitable match for a \cPrince{\heir}, and \cPrince{} doesn't seem to care. 

Of equally great concern, one month ago, the \cQueen{\Monarch} lent \cPrince{} one of the two royal signet rings and tasked \cPrince{\them} with bringing it to Duke \cChupStudent{\formal}, father to \cChupStudent{}, in order to use it to approve a piece of legislation the Duke had drafted. The assignment should only have taken a week or two at most, but \cPrince{} has yet to return the signed and sealed legislation — and the signet ring — to the \cQueen{\Monarch}. The prince is obviously stalling, and it falls to you to find out why. Hopefully the \cPrince{\Heir} at least has the sense to stay out of treaty negotiations — it would ruin the illusion of impartiality teachers are supposed to maintain and very likely cost \cPrince{\them \their} job.

As if you didn't have enough to deal with, you are also tasked with investigating a radical religious cult that has been rabble-rousing and stirring up outright sedition among your nation's lowborn peasants, malcontents, and youth. The symbol of this enigmatic cult, who call themselves the \pGoaties{}, is a four-leaf clover. While most of your knowledge comes from the \pFarm{}, you have reason to believe that the cult has been operating internationally for at least a few years. Given how quickly the cult has spread and the destabilizing effect it is having on your nation, you suspect that it is backed by the \pShip{} in an effort to tip the balance of the war, though you have no hard evidence yet to support that theory. 

What you do have evidence of, in the form of a recently intercepted letter, is that the \pGoaties{} have an agent among the \pFarm{} students at the \pSchool{}. Given the delicate treaty negotiations, the Ritual to Control the Storm, and the vulnerability of the Ley Lines, it doesn't take much imagination to see that they could cause a great deal of damage if they so choose. You'd best figure out who they are and what they're up to, and fast; if only there had been time to assemble full dossiers on all four suspects. \cLibAssist{}, at least, is beyond suspicion; you vetted \cLibAssist{\them} carefully when you first took \cLibAssist{\them} on as an informant. That is not to say that \cLibAssist{\their} hands are entirely clean — \cLibAssist{\theyare} clearly a messenger for the Black Market, a detail \cLibAssist{\theyhave} yet to admit to you — but you're confident \cLibAssist{\theyare} not a \pGoatie{}. 

You also know \cChupStudent{\full} from your days tutoring \cChupStudent{\them} before you began teaching at the \pSchool{}. While it is disappointing that \cChupStudent{\they} seem\cChupStudent{\verbs} content to coast on \cChupStudent{\their} raw talent these days instead of putting in the effort you remember from their childhood, you have a very hard time picturing \cChupStudent{\them} joining a radical religious cult. Your prime suspects are therefore the other two \pFarm{} students: \cAdopted{\full}, who hails from a commoner background and clearly resents being adopted by a noble family, and \cDisney{\full}, who has no traceable background whatsoever, seems to have simply appeared nine months ago, and is being sponsored for mysterious reasons by \cWildCard{}. When you pressed \cWildCard{} on the matter, \cWildCard{\they} said simply that \cDisney{\they \were} a talented orphan who \cWildCard{\they} want to help. While you trust \cWildCard{}, you wonder if \cWildCard{\they \have} been led astray, or worse, involved somehow. That seems deeply out of character.  

Of more personal interest to you is a history research project you have caught wind of through your fellow teacher \cBeetle{\full}. Apparently, \cEbbPriest{\full}, \cHeadScientist{\full}, and \cScholarship{\full} are doing a deep dive into the history and creation of \pEarth{}— a subject you are deeply passionate about. You hope that you can find the time to provide the insights of a real historian to their efforts this weekend.

This weekend you expect to be quite busy doing all the things you do best. You have truths to uncover, both for an accurate historical record, and to the advantage of your \cQueen{\Monarch}. You have students, Relics, and priceless tomes to protect. And you have a war to end without your \cQueen{\Monarch}’s involvement being noticed. Good thing you are up to the challenge.

\begin{itemz}[Goals (in roughly descending order of importance)]
    \item Be the \cQueen{\Monarch}'s voice in the treaty negotiations, advocating for a treaty and end to the war, \textbf{without} revealing \cQueen{\Their} Majesty's position. Any treaty should ensure that the Storm not be sent to your nation at this ``Time of Deciding,’’ as the consequences would be devastating.
    \item Act as the \cQueen{\Monarch}'s spymaster and collect as much information as you can that might benefit \cQueen{\them} and your nation. Intelligence reports suggest that the \pTech{} has a spy here this weekend among the advisors, and that one of your fellow teachers has been operating as a spy of unknown nationality right under your nose.
    \item Investigate the \pGoaties{} and disrupt anything they might be plotting. The letter you intercepted suggests one of their operatives is a student from the \pFarm{}.
    \item Keep an eye on \cPrince{}. Find a way to disrupt \cPrince{\their} romance with \cPirate{}, who is an unsuitable match for the \cPrince{\heir}. Find out why \cPrince{\theyhave} not returned the \cQueen{\Monarch}'s signet ring and documents from Duke \cChupStudent{\formal}.
    \item Work with \cLibrarian{} to prepare the Ritual to Control the Storm and make sure nothing happens to the Relics needed for the ritual.
    \item Support and guide \cLibAssist{}, and in exchange continue to leverage \cLibAssist{\their} abilities as your spy amongst the students.
    \item Investigate the deaths of the students who directed the Storm six years ago. You don't know who ordered their assassination, but you don't approve of their methods, nor do you appreciate being cut out of the planning.
    \item Assist \cEbbPriest{}, \cHeadScientist{}, and \cScholarship{} in their research into the history and creation of Cengea. *Sigh* Before you became a spy for the \cQueen{\Monarch}, this would have been your top priority, but duty must come first…
\end{itemz}

\begin{itemz}[Notes]
    \item You are the history teacher at the \pSchool{}.
    \item You were ordered by the \cQueen{\Monarch} not to be here six years ago when the students who directed the Storm at the \pShippies{} out of turn all died suddenly. You suspect a coverup on the part of your nation and the \pTech{}.
    \item Pirates are a frequent concern for your nation, and as the \cQueen{\Monarch}'s personal spymaster, you naturally know some of their goings-on. Pirates often use code phrases that refer to to the graveyard where their ships hide as ``untouched waters,'' tools related to navigation, and ``charting courses.'' Anyone using such language is to be viewed with deep suspicion of being a pirate.
    \item The Order of the Black Crocus is, or rather was, a secret international law enforcement entity with jurisdiction across all three nations. As far as you know, it splintered into factions when the war started, each faction subsumed by the intelligence services of its corresponding nation. You are sometimes nostalgic for the spirit of international collaboration that fueled the group and the contributions it made to peace and security, but on the other hand, you certainly don't miss their meddling in your and the \cQueen{\Monarch}'s affairs.
\end{itemz}

\begin{contacts}
    \contact{\cWildCard{}} An old friend from back in your tutoring days, when your dreams were very different. You still keep in regular communication, though \cWildCard{} has been cagey recently about \cWildCard{\their} new protege.  
    \contact{\cLibrarian{}} The ultimate authority on the Ritual to Control the Storm.
    \contact{\cEthics{}} The Morality and Ethics teacher at the school, and a good friend of yours with whom you share spirited discussions. Apparently has a crush on \cBeetle{}.
    \contact{\cBunker{}} The genius behind the Bunkers, and an excellent conversationalist with whom you've kept up a casual correspondence.
    \contact{\cEvil{}} \cEvil{} is the public voice of the \cQueen{\Monarch} at the treaty talks, and also happens to have been your friend and political ally for many years.
    \contact{\cLibAssist{}} Your most promising student — clever, ruthless, discreet, and hungry for opportunities to advance. You might make a protégé of \cLibAssist{\them} yet. Currently acting as your spy and informant among the students.
    \contact{\cChupStudent{}} Secondborn child of the powerful \cChupStudent{\formal} family, whom you tutored years ago. \cChupStudent{\They} seem\cChupStudent{\verbs} to have unfortunately grown into a lackadaisical young adult coasting on raw talent alone.
    \contact{\cAdopted{}} A fiery student from a commoner background who seems to bear a great deal of resentment towards the noble \cAdopted{\formal} family who adopted \cAdopted{\them}. One of your prime suspects for being a \pGoatie{}.
    \contact{\cDisney{}} An entirely too innocent seeming student with no traceable background whatsoever, who seems to have simply appeared one day and been sponsored for mysterious reasons by \cWildCard{}. \cDisney{} claims to be a Cleric of Luminary, but what better cover for being a \pGoatie{} than to pose as a Cleric of one of the Patron Gods?
    \contact{\cPrince{}} The \cQueen{\Monarch}'s youngest \cPrince{\offspring} and favored heir, whom you are charged with monitoring.
    \contact{\cBeetle{}} A colleague who teaches Religion at the school. The two of you share a passion for the mythology and ancient history of \pEarth{}.
    \contact{\cTechStar{}} Your least favorite student, argumentative, rebellious, and often full of mistaken beliefs about how history should be taught.  
\end{contacts}

\end{document}

