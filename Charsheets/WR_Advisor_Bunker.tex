\documentclass[char]{GL2020}
\parindent=0pt
\begin{document}
\name{\cBunker{}}

You are \cBunker{\full} (\cBunker{\they}/\cBunker{\them}), master engineer and Spymaster for the \pShip{} Council of Storm Watchers. You became a master at a young age -- but by now, you have left that youth far behind. This will be your 6th trip to the \pSchool{}, a trip you have come to make every other storm cycle. The purpose is ostensibly to check the integrity of the Bunkers you helped to create and build, which makes for great cover as the Council of Storm Watcher's spymaster. Though you spend more time looking over spy reports than drafting blueprints nowadays, you remain one of the best engineers of your generation. It frustrates you that everyone credits \cFlowFull{\full} for your incredible work -- if only people were more willing to give credit where it is due!

Sure \cFlow{} is the goddess of beginnings, new things, and growth, but what are these abstract concepts without the application of a human mind? Nothing. You carve \cFlow{\their} symbol on the things you make because you must -- no one would buy from you otherwise -- but it hurts your soul to do it. Humans are more than just pawns of the Deities, and you are more than just an instrument of \cFlow{}. 

You grew up in 6th fleet, but went to 3rd fleet academy for their engineering and shipwright curriculum. Math, science, and tactile skills all came easily to you, and you excelled. You flew through your apprenticeship in record time, and were a practicing master architect by age 26. When you were approached about the Bunker Revitalization Project by a \cFlow{} cleric claiming to have been given your name by the \cFlow{\God} \cFlow{\themself}, you gritted your teeth at the saccharine religiousness of it all, but agreed - you'd never get a chance to work on something so important otherwise. 

So you set to work, alongside architects from the other two nations, and soon rebuilt the aging architecture into a set of 3 brand-new magical bunkers, able to withstand the swelling power of the Storm as it swirls about the island on which the \pSchool{} is built. Inside the bunkers, the individuals who must remain at the \pSc{} to execute the Ritual to Control the Storm are protected from various unpleasant side-effects, both temporary and permanent, of being exposed to the ferocity of these Storm Surges.

That project brought you international attention, and an unusual degree of ability to pass freely between the three nations. This in turn made you a valuable asset for the spymaster of the \pShip{}, who soon recruited you for various missions. At first you didn't like all the sneaking around, but you soon fell in love with deciphering enemy messages. After all, what are ciphers if not applied math -- the very language you've spoken all your life? The previous spymaster often thanked \cEbb{} for finding someone as good at taking codes apart as you. You just rolled your eyes and kept working.

And so, like clockwork, you return to the \pSchool{} every 6 years. At first it was just to keep an eye on and maintain the bunker project. Once you started spy work in earnest though, it became an important source of information as well. Over the years, a gaggle of helpers has coalesced around you. Some are more helpful than others. You'd probably finish faster with fewer untrained helpers bumbling around, but people gossip while they ``help,'' and you hear all kinds of things while they think your attention is entirely on whether this nut is tight enough. You also have ongoing correspondences with Principal \cPrincipal{\full}, an old friend with whom you've worked for many years to help maintain the school, and \cHistory{\full}, the school's History teacher. Neither of them is aware they are your informants, of course, so they sometimes let useful information slip. As the immortal principal, \cPrincipal{} has 200 years of experiences and observations to draw from, and \cHistory{} is extremely knowledgeable about the Relics, the school's history, and current events. 

In addition to all your unwitting informants, you also have two knowing informants whom you have recruited over the years. One is established inside the school itself and has been reporting to you for two years – the Morality and Ethics teacher, \cEthics{\full}. The other is \cAssistantScientist{\full}, a highly placed research assistant in the \pTech{}, who has been passing you information on secret research initiatives and weapons technology for a few years. Both are invaluable assets, and both will be here this weekend. They already know that you are a \pShip{} spy, but not the spymaster, and you would rather keep it that way. You will have to find a way to make contact and learn any information they have for you without revealing your role within the organization. That will prove especially important this weekend, as intelligence reports suggest that a spy from the \pFarm{} is here at the \pSc{} too, so it's important that you stay ahead of them and unmask them if you can. Given the pivotal events happening this weekend, you would not be surprised if the \pTech{} has a spy here as well, but you have no specific intelligence to that effect.

All things must come to an end, however. You are feeling the weight of your years -- it is past time you took on an apprentice who can handle one or both sets of duties. Someone quick and clever. Someone young, but committed. Someone who loves building things, and taking them apart, as much as you do. You figure your two best options are \cTechStar{\full} and \cInitiate{\full}. \cTechStar{} is from the \pTech{}, which is a little disappointing - \cTechStar{\they} would have made a fine shipwright - but no one is perfect. \cTechStar{\They} \cTechStar{\are} quite innovative with technology, and as bold as brass, as evidenced by \cTechStar{\their} writing to you out of the blue to request you bring a modest quantity of several rather hard to come by pieces of misc. technology with you to the school. You were happy to oblige, driven primarily by your curiosity about what \cTechStar{\they} could want the parts for\ldots{} Of course, to take on your spy work, \cTechStar{} would need to renounce loyalty to the \pTech{}, and swear to the \pShip{} instead. \emph{Sigh.} To maintain appearances, this will also mean forswearing \cTechGod{} for \cEbb{} and \cFlow{}. You'll have to speak with those blasted Clerics about it. Still, it is an open question how much you could really trust \cTechStar{} even if \cTechStar{\theyare} willing to take those steps. You can't possibly turn your country over to someone who doesn't have the interests of the \pShip{} at heart, especially when it so often comes at the expense of the other nations these days.

You have slightly higher hopes for \cInitiate{}, who has proven most eager to work with you, and \cInitiate{\they} \cInitiate{\have} one of the best minds you've encountered for the work you do. \cInitiate{\Theyare} a \pShippie{} and \cInitiate{\have} unimpeachable loyalty to the nation, which means you could actually trust \cInitiate{\them} to take over your spy work without much trouble. Your main hesitation with \cInitiate{} is that, while \cInitiate{\theyare} a very clever young \cInitiate{\person}, \cInitiate{\theyare} also an initiate of \cEbb{} and \cFlow{}. If you were to appoint \cInitiate{\them} as your replacement, \cInitiate{\they} might devote too much time to \cInitiate{\their} priestly duties and neglect the Bunkers and the spy work. What's more, by giving up all credit for personal accomplishments to the Goddesses, \cInitiate{\they} \cInitiate{\have} never built a proper sense of self confidence. Thirdly, \cInitiate{} may find your religious views intolerable and refuse to work with you. Why does everything in this world come down to religion?

You had actually already picked out a promising apprentice 6 years ago to learn the ins and outs of Bunker maintenance, and introduce to spy work -- but due to the massacre of 6 years ago, well\ldots{} that didn't work out. With all of the kids dead, and nothing but questions over how the vote to direct the storm could have gone the way it did, you sulked your way home. You were supposed to be \textbf{good} at your job as spymaster. How could something like this have happened right under your nose without you having nary a whiff of it ahead of time? 

While you immediately put an entire task force on finding out what happened in the wake of the event, you have spent six fruitless years trying to track down what happened. In the end, though, internal political pressure, the start of the war, and what you suspect was a masterfully orchestrated coverup prevented you from finding anything solid beyond the very basic cause of death -- a mass poisoning. More than once you nearly sent your prayers on the wind to the Goddesses. For surely the only way your enemies could elude you so completely would be if they had the support of other deities? But your sensible side prevented you - what could \cEbb{} and \cFlow{} really do anyway? And would your enemies, mass murderers as they are, really have the personal protection of \cTechGod{} or \cFarmGod{}? If all of this were somehow ordained by the Gods, it was above your pay grade, and you frankly wanted nothing more to do with it. But at the same time, it is in your nature to be inquisitive. As you are returning to the \pSchool{} for what may be the last time, maybe, just maybe, you can finally find some answers.

Of more pressing concern to you right now, though, is an imminent threat to the safety of the school, your nation, and possibly the world -- the \pGoaties{}. When they first appeared on your radar 7 years ago, you didn't take them too seriously – just a fringe group concerned with stirring up trouble among \pEarth{}'s poorest. It happens every decade or so, and fizzles out on it's own. While you dutifully formed a task force to monitor them, you didn't give it much, if any, personal oversight. It was only yesterday that you realized how fatal of a mistake that was. An agent outside of the taskforce personally proved that yesterday's assassination attempt on \cHeadDiplomat{\full}, the head diplomat who was meant to meet you at the school as head of the \pShip{} delegation, was the work of a cult assassin.  \cHeadDiplomat{} has been a close ally and correspondent of yours for years, so this came as quite a shock. While the assassination attempt failed, the result was almost as bad -- \cHeadDiplomat{} was forced to kill in self-defense, resulting in instant amnesia. Meanwhile, your taskforce's latest report indicated that the \pGoaties{} were YEARS away from developing the kind of power necessary for that strike. There were no two ways about it -- your taskforce on the \pGoaties{} was clearly compromised. You ordered every agent on it detained for further questioning, and discarded everything you thought you knew about the \pGoaties{} as deliberate misinformation. It was time to start from scratch.

What you could verify with agents you trusted and pull together from other sources on less than a day's notice was sparse -- but it wasn't nothing. Your intelligence suggests that one of your three fellow \pShip{} advisors is in fact a Follower of Genesis and may be behind the assassination attempt on \cHeadDiplomat{}. You have not yet been able to narrow it down further, as all three had the necessary means and motive. \cChupLeader{\full} and \cJuniorStatesman{\full} are both assistants to \cHeadDiplomat{}, and both can hope to gain social advancement by helping fill \cHeadDiplomat{\their} role in the negotiations this weekend. As for \cEbbPriest{\full}, \cEbbPriest{\they} \cEbbPriest{\were} the one who so conveniently conducted the investigation of the assassination attempt, and then happened to declare \cEbbPriest{\themself} \cHeadDiplomat{}'s replacement for the negotiations. The whole thing stinks, and you intend to find the traitor in your midst. 

The \pGoaties{} seem to have spread like wildfire since the start of the war, and have become much more powerful than one could reasonably expect in such a short time frame. You suspect that means they are backed by a foreign government, but you have no hard evidence of this fact. One suppressed and abandoned report caught your eye in your frantic scramble. This report claims that the true goal of the \pGoaties{} is to eliminate magic from the face of \pEarth{}. While you can see the appeal on to an amateurish, hot-headed, and desperate for change person, it's perfectly clear to you as a professional how much death and destruction this would cause, even beyond the evil that must be wrought to get to that point. The last piece of information that you can trust is that the \pGoaties{} seem to want to sabotage the Ritual to Control the Storm somehow. This sabotage will likely center on the Relics — but exactly how, you do not know. Could they be trying to create fake Relics? Damage or break existing ones? Steal them outright? You have to plan to counter it all. As such, anyone who brings an unknown Relic, tries to tamper with an existing one, or even just spends too much time with one is a suspect for being a cultist. And any requests to change which Relics are used in the ritual for any reason should be met with the utmost scrutiny for potential foul play. 

What's more, you now know for sure that the \pGoaties{} are here at the school. Within an hour of arriving, you realized that you had already been stolen from. Documents with much of the precious little intelligence you can trust on the Followers of Genesis, have vanished from your briefcase, which you had stored in a secure location in your room. (These documents are represented in game by ``\iFolderOfNotes{}''). You don't know for sure what the Followers are up to, but it would seem that they know your identity and are one step ahead of you yet again. You are determined to NOT underestimate them again. If and when you do catch the Followers, you should bring them back to the Wave Riders for interrogation. Dead people don't talk, and leads released into the custody of other nations have a distressing tendency to escape before you get your turn. If you can't capture them, then foiling any plans they have and uncovering their identities would still go a long way towards neutralizing the group.

The Ritual to Control the Storm and the Bunkers aren't the only thing that needs to be safeguarded at the school. The cluster of Ley Lines that rise from the ground to the sky here, the very reason this island floats, needs maintenance every 3 years as well. Organizing this rather unpleasant task falls to the immortal principal of the school, your old friend \cPrincipal{}, and you have come to treat participating in that process as part of your duty whenever you are here. Unlike the students who are so enamored of their magical power, you don't mind paying the temporary price that the ritual exacts. You should check with \cPrincipal{} to see if the plan is as usual -- to do the ritual Friday night before things get too hectic.

Of course, even if you can prevent the Followers of Genesis from interfering with the weekend's rituals, you still have to make sure the Storm doesn't get sent to the \pShip{} yet again. And whatever may come out of the treaty negotiations, the ultimate arbiters of that decision are the students themselves. You need to make sure the right student wins the most votes to direct the Storm. You believe that student to be \cWarlordDaughter{\full} – not just because \cWarlordDaughter{\theyare} the Warlord's daughter, but because \cWarlordDaughter{\their} loyalty to the nation is unimpeachable and \cWarlordDaughter{\they} \cWarlordDaughter{\have} a good head on \cWarlordDaughter{\their} shoulders. Your best bet for helping \cWarlordDaughter{} win the most votes is to convince the other \pShip{} advisors and the \pFarm{} teachers to rank \cWarlordDaughter{\them} the highest.

You plan to make this weekend one of uncovering past secrets and preparing for the future. Your job is to protect people, and the \pShippies{} in particular. There are Bunkers to inspect, a new apprentice to pick and begin training, old mysteries to solve, and a few friends to catch up with along the way. As long as you don't smother under all the bowing and scraping to the gods, and no one accuses you of being blasphemous over it, it should be a welcome challenge.

\begin{itemz}[Goals (in roughly descending order of importance)]
	\item Check the Bunkers over carefully throughout the weekend, make sure there is no chance of a catastrophic failure, and repair any normal wear and tear you find.
	\item Collect as much reliable intelligence on the \pGoaties{} as you can. Keep them from sabotaging the Ritual to Control the Storm or the Ritual to Renew the Leylines, including safeguarding the six Relics. Bring them to justice for their many misdeeds, including the assassination attempt on \cHeadDiplomat{}. Your three main suspects are: \cChupLeader{}, \cJuniorStatesman{}, and \cEbbPriest{}.
	\item Make sure the Storm does not get sent to the \pShip{}. Help \cWarlordDaughter{} win the most votes to direct the Storm by convincing the other \pShip{} advisors and the \pFarm{} teachers to rank \cWarlordDaughter{\them} the highest.
	\item Pick either \cTechStar{} or \cInitiate{} as your apprentice, convince them to agree to it (if necessary), and begin the process of training them to look after the Bunkers and take over as spymaster for the \pShippies{}.
	\item Find whoever stole \iFolderOfNotes{} from your briefcase, and recover your notes. Then you may be able to use them to find the rest of the \pGoaties{}.
	\item Make contact with your two informants, \cEthics{} and \cAssistantScientist{}, and see if they have any useful information for you. Intelligence reports suggest that a spy from the \pFarm{} is here this weekend, so it's important that you stay ahead of them and unmask them if you can.
	\item Learn anything you can about the truth behind the Betrayal that sent the Storm to the \pShip{} and murdered 12 students 6 years ago.
\end{itemz}

\begin{itemz}[Notes]
	\item You were present at the school 6 years ago, but had never bothered really engaging with the students; what you overheard from the advisors was that nothing was out of the ordinary, the students were generally on board with sending the storm to the \pTech{}, and that was that. The students going elsewhere for their toast after directing the Storm was a little unusual, but you didn't think anything of it, until the students didn't come back. You didn't dare leave the room where the other advisors were – you couldn't risk missing a twitch or a facial expression that might lead you to someone who knew what was going on. But no one blinked. And then \cPrincipal{}, \cDiplomat{\full}, and \cMusic{\full} came back from checking on the students and reported that they were all dead.
	\item The Order of the Black Crocus is, or rather was, a secret international law enforcement entity with jurisdiction across all three nations. As far as you know, it splintered into factions when the war started, each faction subsumed by the intelligence services of its corresponding nation. You certainly did that with your branch. As meddlesome and irritating as it was, you grudgingly admit that the group did a lot of good for international peace and security while it existed.
	\item You designed the Bunkers to withstand Storm Surges — brief, intense discharges of magical and kinetic energy as the Storm gathers. You are not certain how they would hold up against the full fury of the Storm, but your best estimate is that they would be able to protect about half as many people as they do against the surges. Fortunately, it is extremely unlikely that they would ever be put to that test — the only way you know of that the school could ever be struck by the Storm in such a manner would be if the Ritual to Control the Storm were not performed or failed entirely. In that event, the school would be the least of your concerns, as the Storm would rage unchecked across all of Cengea.
	\item You have stored the 3 ``Misc. Magitech Parts'’ requested by \cTechStar{} in your guest quarters for now. Since each item is 1 hand bulky, you didn't feel like hauling them around with you until you could arrange to pass them off.
\end{itemz}

\begin{contacts}
	\contact{\cInitiate{}} A young \cInitiate{\person} from \pShip{}, and one of the candidates you are considering to be your apprentice. \cInitiate{} is very enthusiastic to learn engineering from you, but \cInitiate{\their} level of commitment to religion concerns you.
	\contact{\cTechStar{}} A young \cTechStar{\person} from \pTech{}. \cTechStar{} is quite the inventor if what you've heard from the Tech Star Competition is right, and might make a good apprentice for you, if you could switch \cTechStar{\their} loyalties to the \pShip{}.
\contact{\cPrincipal{}} An old friend originally hailing from \pShip{} (long before your time, though). You have helped \cPrincipal{\them} maintain the school for several decades.
	\contact{\cHistory{}} The History teacher at the school and a good friend of yours, learned in lore and with a great appreciation for the school's history.
	\contact{\cEthics{}} The Morality and Ethics teacher at the College of the Gods, and more importantly, your secret informant at the school.
	\contact{\cAssistantScientist{}} A highly placed research assistant in the \pTech{} who has been passing you secrets for a few years now.
	\contact{\cChupLeader{}} \cHeadDiplomat{}'s personal assistant, now playing a more prominent role in the treaty negotiations with \cHeadDiplomat{} out of the picture. Therefore, one of your prime suspects in orchestrating the assassination attempt and being one of the \pGoaties{}.
	\contact{\cJuniorStatesman{}} \cHeadDiplomat{}'s apprentice, now playing a more prominent role in the treaty negotiations with \cHeadDiplomat{} out of the picture. Therefore, one of your prime suspects in orchestrating the assassination attempt and being one of the \pGoaties{}.
	\contact{\cEbbPriest{}} A \cEbbPriest{\cleric} of \cEbb{}, the one who investigated the assassination attempt on \cHeadDiplomat{}, and who then conveniently declared \cEbbPriest{\themself} \cHeadDiplomat{}'s replacement for the negotiations. Therefore, one of your prime suspects in orchestrating the assassination attempt and being one of the \pGoaties{}. - Wouldn't that be ironic for a supposed \cEbbPriest{\cleric} of \cEbb{}.
		\contact{\cWarlordDaughter{}} The Warlord's \cWarlordDaughter{\child}, bright and dedicated, and the student whom you support for winning the most votes to direct the Storm.
\end{contacts}

\end{document}


