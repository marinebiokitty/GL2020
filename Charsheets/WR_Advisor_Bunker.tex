\documentclass[char]{GL2020}
\usepackage[normalem]{ulem}
\parindent=0pt
\begin{document}
\name{\cBunker{}}

You are \cBunker{\full} (\cBunker{\they}/\cBunker{\them}), master engineer and Spymaster for the \pShip{} Council of Storm Watchers. You became a master engineer at a young age, but by now, you have left that youth far behind. This will be your 6th trip to the \pSchool{}, a trip you have come to make every other storm cycle. The purpose is ostensibly to check the integrity of the Bunkers you helped to create and build, which makes for great cover as the Council of Storm Watchers’ Spymaster. Though you spend more time looking over spy reports than drafting blueprints nowadays, you remain one of the best engineers of your generation. It frustrates you that everyone credits \cEbb{} for your incredible work — if only people were more willing to give credit where it is due!

Sure, \cEbb{} is the Goddess of ships, completed things, and trimming away the unneeded, but what are these abstract concepts without the application of a human mind? Nothing. You carve \cEbb{\their} symbol on the things you make because you must — no one would buy from you otherwise — but it hurts your soul to do it. Humans are more than just pawns of the Deities, and you are more than just an instrument of \cEbb{\them}.

You grew up in 6th Fleet, but went to 3rd Fleet Academy for their engineering and shipwright curriculum. Math, science, and tactile skills all came easily to you and you flew through your apprenticeship in record time, becoming a master engineer by age 26. When you were approached about the Bunker Revitalization Project by a Cleric of \cEbb{} claiming to have been given your name by the Goddess \cEbb{\themself}, you gritted your teeth at the saccharine religiousness of it all, but agreed — you'd never get a chance to work on something so important otherwise. 

So you set to work, alongside engineers and architects from the other two nations, and soon rebuilt the aging architecture into a set of 3 brand new magical Bunkers, able to withstand the swelling power of the Storm as it swirled about the island on which the \pSchool{} is built. Inside the Bunkers, the individuals who must remain at the \pSc{} to execute the Ritual to Control the Storm are protected from the various unpleasant side-effects, both temporary and permanent, of being exposed to the ferocity of these Storm Surges.

That project brought you international attention, and an unusual degree of ability to pass freely between the three nations. This in turn made you a valuable asset for the spymaster of the \pShip{}, who soon recruited you for various missions. At first you didn't like all the sneaking around, but you soon fell in love with deciphering enemy messages. After all, what are ciphers if not applied math — the very language you've spoken all your life? The previous spymaster often thanked \cEbb{} for finding someone as good at taking codes apart as you. You just rolled your eyes and kept working.

And so, like clockwork, you return to the \pSchool{} every six years (though you also came three years ago as part of your clandestine investigation into who was responsible for the Betrayal). At first it was just to keep an eye on and maintain the Bunkers. Once you started spy work in earnest, though, it became an important source of information as well. Over the years, a gaggle of helpers has coalesced around you. You'd probably finish faster with fewer untrained helpers bumbling around, but people gossip while they ``help,'' and you hear all kinds of things while they think your attention is entirely on whether this nut is tight enough. 

In addition to these helpers, you have cultivated ongoing correspondences with Principal \cPrincipal{\full}, an old friend with whom you've worked for many years to help maintain the school, \cHistory{\full}, the school's History teacher, and \cChupInventor{\full}, the school’s Diplomacy and Political Science teacher. None of them are aware they are your informants, of course, so they sometimes let useful information slip. As the immortal principal, \cPrincipal{} has 200 years of experiences and observations to draw from. \cHistory{} is extremely knowledgeable about the Relics, the school's history, and current events. And \cChupInventor{} is a skilled inventor in \cChupInventor{\their} spare time, and has provided \textbf{actual} assistance in maintaining the school’s Bunkers. The two of you enjoy talking shop, and it turns out that \cChupInventor{} shares your disdain for organized religion, so you can speak more openly with each other about your frustrations. You are genuinely fond of \cPrincipal{}, \cHistory{}, and \cChupInventor{}, and consider them friends, not just unwitting intelligence assets. 

In addition to all your unwitting informants, you also have two knowing informants whom you have recruited over the years. One is established inside the school itself and has been reporting to you for a few months — the Morality and Ethics teacher, \cEthics{\full}. The other is \cAssistantScientist{\full}, a highly placed research assistant in the \pTech{}, who has been passing you information on secret research initiatives and weapons technology for a few years. Both have let you know that they feel betrayed by the direction the \pTech{} has taken and want to right such wrongs as best they can. You were wary at first, but \cEthics{} has a sterling reputation and literally teaches morality and ethics. Careful investigation confirmed for you that \cEthics{} ascribes to a personal morality that is distinct, and often in conflict with the ethics touted by \cEthics{\their} country, and those expected of \cEthics{\their} role as a teacher at the \pSc{}. \cAssistantScientist{} was personally vouched for by the crew of Pink Coral Ship, 4th Fleet. At first you were surprised by a \pShippie{} ship vouching for a \pTech{} scientist, but multiple reports corroborate that \cAssistantScientist{\they} spent a few months on the ship in the immediate aftermath of the Betrayal, helping with repairs alongside the rest of the crew.

Both of them have proven to be invaluable assets, and both will be here this weekend. They already know that you are a \pShip{} spy, but not the Spymaster, and you would rather keep it that way. You will have to find a way to make contact and learn any information they have for you without revealing your precise role within the organization. That will prove especially important this weekend, as intelligence reports suggest that a spy from the \pFarm{} is here at the \pSc{} too, so it's important that you stay ahead of them and unmask them if you can. Given the pivotal events happening this weekend, you would not be surprised if the \pTech{} has a spy here as well, but you have no specific intelligence to that effect.

One of the teachers at the \pSc{}, being a \pShippie{}, is not even an informant, just a friend, and that is \cPirate{\full}, the crafting teacher. Like \cChupInventor{}, \cPirate{\theyare} a genuine asset to Bunker repair, and you hope to get \cPirate{\their} help again. You two have spent a few ``Times of Deciding’’ enjoying talking about the joys and sorrows of building. Though you learned that \cPirate{\they} recently started dating \cPrince{\full} almost three years ago! It was nice having one friend who you were not also plying for information, but this development may force that to change. You can’t really justify leaving potentially crucial information about the \pFarm{} royal family on the table. This year that may have to change.

All things must come to an end, however. You are feeling the weight of your years — it is past time you took on an apprentice who can handle one or both sets of duties. Someone quick and clever. Someone young, but committed. Someone who loves building things, and taking them apart, as much as you do. You figure your two best options are \cTechStar{\full} and \cInitiate{\full}. \cTechStar{} is from the \pTech{}, which is a little disappointing — \cTechStar{\they} would have made a fine shipwright — but no one is perfect. \cTechStar{\Theyare} quite innovative with technology, and as bold as brass, as evidenced by \cTechStar{\their} writing to you out of the blue to request you bring a modest quantity of rather hard to come by spare parts with you to the school. 

You were happy to oblige, driven primarily by your curiosity about what \cTechStar{\they} could want the parts for\ldots{} It certainly helped that \cChupInventor{} also recommended \cTechStar{\them} to you as a protege around the same time. Of course, to take on your spy work, \cTechStar{} would need to renounce loyalty to the \pTech{}, and swear to the \pShip{} instead. \emph{Sigh.} To maintain appearances, this will also mean forswearing \cTechGod{} for \cEbb{} and \cFlow{}. You'll have to speak with those blasted Clerics about it. Still, it is an open question how much you could really trust \cTechStar{} even if \cTechStar{\theyare} willing to take those steps. You can't possibly turn your country over to someone who doesn't have the interests of the \pShip{} at heart, especially when it so often comes at the expense of the other nations these days.

You have higher hopes for \cInitiate{}, who has proven most eager to work with you, and \cInitiate{\theyhave} one of the best minds you've encountered for engineering. \cInitiate{\Theyare} a \pShippie{} and \cInitiate{\have} unimpeachable loyalty to the nation, which means you could actually trust \cInitiate{\them} to take over your spy work without much trouble. And perhaps most importantly, you’ve known \cInitiate{} since \cInitiate{\they} \cInitiate{were} a small child attending your good friend \cHeadDiplomat{\full}’s dinner parties; you watched the kid grow up, and can personally vouch for \cInitiate{\their} character. Your only hesitation with \cInitiate{} is that, while \cInitiate{\theyare} a very clever young \cInitiate{\person}, \cInitiate{\theyare} also an Initiate of \cEbb{} and \cFlow{}. If you were to appoint \cInitiate{\them} as your replacement, you worry that \cInitiate{\they} might devote too much time to \cInitiate{\their} priestly duties at the expense of the Bunkers and the spy work. You also quietly admit to a certain sense of disappointment that \cInitiate{\theyhave} has chosen to waste \cInitiate{\their} talents on the Priesthood, and wonder if \cInitiate{\theyhave} the priorities you need in an apprentice.

There is also \cInitiate{}’s best friend and \cHeadDiplomat{}’s \cPresident{\nibling}, \cPresident{\full}. You watched \cPresident{\them} grow up too and \cPresident{\theyare} a natural born leader. You’re very fond of the kid, but you're not seriously considering \cPresident{\them} to be your apprentice — \cPresident{\theyare} too earnest and forthright for spywork, and just not mathematically minded enough for engineering. 

You had actually already picked out a promising apprentice six years ago to learn the ins and outs of Bunker maintenance, and introduce to spy work, but due to the massacre six years ago, well\ldots{} that didn't work out. With all of the students dead, and nothing but questions over how the vote to direct the Storm could have gone the way it did, you sulked your way home. You were supposed to be \textbf{good} at your job as spymaster. How could something like this have happened right under your nose without you having nary a whiff of it ahead of time? You immediately put an entire task force on unraveling the conspiracy that masterminded the Betrayal, but their efforts over the past six years have proven maddeningly fruitless. In the end, internal political pressure, the start of the war, and what you suspect was a masterfully orchestrated coverup prevented you from finding anything solid beyond the basic cause of death — a mass poisoning via mulled wine. That and an abiding certainty that the Voting Stones must have been tampered with somehow after the votes were cast. You even risked drawing attention to yourself and broke your normal pattern. You attended the ``Time of Deciding’’ three years ago, claiming paranoia over the Bunkers as your cover. But even investigating personally led to nothing. As you return to the \pSchool{} for what may be the last time, maybe, just maybe, you can finally find some answers this time.

Of more pressing concern to you right now, though, is a looming threat to the safety of the school, your nation, and possibly the world — the \pGoaties{}. When they first came to your attention 7 years ago, you didn't take them too seriously — just a fringe cult concerned with stirring up trouble among the poorer \pShippie{} fleets. It happens every decade or so, and usually fizzles out on its own. While you dutifully formed a task force to monitor them, you didn't give it much personal oversight. It was only yesterday that you realized how fatal of a mistake that was. 

An agent outside of the taskforce proved that yesterday's assassination attempt on \cHeadDiplomat{\full}, the diplomat who was meant to meet you at the school as head of the \pShip{} delegation, was the work of a cult assassin. \cHeadDiplomat{} has been a close friend and ally of yours for years, so this came as quite a shock. While the assassination attempt failed, the result was almost as bad — \cHeadDiplomat{} was forced to kill in self defense, resulting in total, instant amnesia. Meanwhile, your task force's latest report indicated that the \pGoaties{} were \emph{years} away from developing the kind of power necessary for that strike. There were no two ways about it — your taskforce on the \pGoaties{} was clearly compromised. You ordered every agent on it detained for further questioning, and discarded everything you thought you knew about the group as deliberate misinformation. It was time to start from scratch.

What you could verify with agents you trusted and pull together from other sources on less than a day's notice was sparse — but it wasn't nothing. Your intelligence suggests that one of your three fellow \pShip{} advisors is in fact a \pGoatie{} and may be behind the assassination attempt on \cHeadDiplomat{}. You have not yet been able to narrow it down further, as all three had the necessary means and motive. \cChupLeader{\full} and \cJuniorStatesman{\full} are both assistants to \cHeadDiplomat{}, and both can hope to gain social advancement by helping fill \cHeadDiplomat{\their} role in the negotiations this weekend. As for \cEbbPriest{\full}, \cEbbPriest{\theywere} literally in \cHeadDiplomat{}’s house ``visiting’’ the night the assassin struck, ready to so conveniently conduct the investigation of the assassination attempt, and then put \cEbbPriest{\themself} forward as \cHeadDiplomat{}'s replacement for the negotiations. Supposedly, \cEbbPriest{} is a lifelong friend of \cHeadDiplomat{}'s spouse, but could it just be a long con? The whole thing stinks like rotten fish, and you intend to find the traitor in your midst.

The \pGoaties{} seem to have spread like wildfire since the start of the war, and have become much more powerful than one would expect in such a short time frame. You suspect that means they are backed by a foreign government, but you have no hard evidence of that. The only other piece of information that you can trust is that the \pGoaties{} seem to want to sabotage the Ritual to Control the Storm somehow. This sabotage will likely center on the Relics, the Voting Stones, or both — but exactly how, you do not know. Could they be trying to create fake Relics or Voting Stones? Steal or destroy existing ones? You have to plan to counter it all. As such, anyone who brings an unknown Relic, tries to tamper with the Relics or Voting Stones, or even just spends too much time with them is a suspect for being a cultist. It is not lost on you that this threat bears an alarming similarity to the Betrayal six years ago — perhaps the \pGoaties{} are the culprits you’ve been seeking all along, and you’ll be able to catch two fish with one line.

What's more, you now know for sure that the \pGoaties{} are here at the school. Within an hour of arriving, you realized that you had already been stolen from. Documents with much of the precious little intelligence you can trust on the \pGoaties{} have vanished from your briefcase, which you had stored in a secure location in your room. (These documents are represented in game by ``\iFolderOfNotes{}''). You don't know for sure what the Followers are up to, but it would seem that they know your identity and are one step ahead of you yet again. You are determined to \emph{not} underestimate them again. If and when you do catch them, you should bring them back to the \pShip{} for interrogation. Dead people don't talk, and leads released into the custody of other nations have a distressing tendency to escape before you get your turn. If you can't capture them, then foiling any plans they have and uncovering their identities would still go a long way towards neutralizing the group.

The Ritual to Control the Storm and the Bunkers aren't the only thing that needs to be safeguarded at the school. The cluster of Ley Lines that rise from the ground to the sky here, the very reason this island floats, needs maintenance every 3 years as well. Organizing this rather unpleasant task normally falls to the immortal principal of the school, your old friend \cPrincipal{}, and you have come to treat participating in that process as part of your duty whenever you are here. Unlike the students who are so enamored of their magical power, you don't mind paying the temporary price that the ritual exacts. This year, because of \cPrincipal{\their} pending retirement, \cPrincipal{} has delegated the task to one of the candidates for the principalship, \cBeetle{\full}, but you still feel you should help out. You should check with \cBeetle{} to see if the plan is as usual — to do the ritual Friday night before things get too hectic.

Of course, even if you can prevent the \pGoaties{} from interfering with the weekend's rituals, you still have to make sure the Storm doesn't get sent to the \pShip{} yet again. Whatever may come out of the treaty negotiations — and you intend to make sure they strongly favor your nation — the strongest determinant of the Storm’s destination is the student vote. You need to make sure the right students get the most Voting Stones to direct the Storm. Both \pShippie{} students you know — \cPresident{} and \cInitiate{} — can be relied upon to make the right decision. \cWarlordDaughter{\full}, as the child of the Warlord, presumably can as well, but you don’t know \cWarlordDaughter{\them} personally. At the very least, you can give \cPresident{} or \cInitiate{} your own Voting Stone, and try to convince others to do the same.

You plan to make this weekend one of uncovering past secrets and preparing for the future. Your job is to protect people, and the \pShippies{} in particular. There are Bunkers to inspect, a new apprentice to pick and begin training, conspiracies to unravel, and a few friends to catch up with along the way. As long as you don't smother under all the bowing and scraping to the Gods, and no one accuses you of being blasphemous over it, it should be a welcome challenge.

\begin{itemz}[Goals (in roughly descending order of importance)]
    \item Check the Bunkers over carefully throughout the weekend, make sure there is no chance of a catastrophic failure, and repair any normal wear and tear you find.
    \item Collect as much reliable intelligence on the \pGoaties{} as you can. Keep them from sabotaging the Ritual to Control the Storm or the Ritual to Renew the Ley Lines, including safeguarding the six Relics. Bring them to justice for their many misdeeds, including the assassination attempt on \cHeadDiplomat{}. Your three main suspects are \cChupLeader{}, \cJuniorStatesman{}, and \cEbbPriest{}.
    \item Make sure the Storm does not get sent to the \pShip{}. To that end, make sure the treaty negotiations go in your favor, and help the \pShippie{} students collect the most Voting Stones to direct the Storm.
    \item Choose an apprentice (your current candidates are \cTechStar{} and \cInitiate{}) and begin the process of training them to look after the Bunkers and take over as Spymaster for the \pShippies{}.
    \item Find whoever stole \iFolderOfNotes{} from your briefcase, and recover your notes. Then you may be able to use them to find the rest of the \pGoaties{}.
    \item Make contact with your two informants, \cEthics{} and \cAssistantScientist{}, and see if they have any useful information for you. Intelligence reports suggest that a spy from the \pFarm{} is here this weekend, so it's important that you stay ahead of them and unmask them if you can.
    \item Learn anything you can about the truth behind the Betrayal that sent the Storm to the \pShip{} and murdered 12 students six years ago.
\end{itemz}

\begin{itemz}[Notes]
    \item You were present at the school 18, 12, and six years ago, but had never bothered really engaging with the students. What you overheard from the advisors six years ago was that nothing was out of the ordinary, the students were generally on board with sending the storm to the \pTech{}, and that was that. The students going elsewhere for their toast after directing the Storm was a little unusual, but you didn't think anything of it, until the students didn't come back. You didn't dare leave the room where the other advisors were — you couldn't risk missing a twitch or a facial expression that might lead you to someone who knew what was going on. But no one blinked. And then \cPrincipal{}, \cDiplomat{\full}, and \cMusic{\full} came back from checking on the students and reported that they were all dead.
    \item You would not normally have come to the \pSc{} three years ago, but in a desperate attempt to find out what happened to the students, you insisted on being on the Advisor roster. To your great frustration, even by investigating personally you turned up nothing.
    \item The Order of the Black Crocus is, or rather was, a secret international law enforcement entity with jurisdiction across all three nations. As far as you know, it splintered into factions when the war started, each faction subsumed by the intelligence services of its corresponding nation. You certainly did that with your branch. As meddlesome and irritating as it was, you grudgingly admit that the group did a lot of good for international peace and security while it existed.
    \item You designed the Bunkers to withstand Storm Surges — brief, intense discharges of magical and kinetic energy as the Storm gathers. You are not certain how they would hold up against the full fury of the Storm, but your best estimate is that they would be able to protect about half as many people as they do against the surges. Fortunately, it is extremely unlikely that they would ever be put to that test — the only way you know of that the school could ever be struck by the Storm in such a manner would be if the Ritual to Control the Storm were not performed or failed entirely. In that event, the school would be the least of your concerns, as the Storm would rage unchecked across all of Cengea.
    \item You have 3 ``\iMagitechParts{}'' requested by \cTechStar{} on your person. 
\end{itemz}

\begin{contacts}
    \contact{\cInitiate{}} A young \cInitiate{\person} from \pShip{}, and one of the candidates you are considering to be your apprentice. \cInitiate{} is very enthusiastic to learn engineering from you, but \cInitiate{\their} level of commitment to religion concerns you.
    \contact{\cTechStar{}} A young \cTechStar{\person} from the \pTech{}. \cTechStar{} is quite the inventor if what you've heard from the TechStar Competition is right, and might make a good apprentice for you, if you could switch \cTechStar{\their} loyalties to the \pShip{}.
    \contact{\cPrincipal{}} An old friend originally hailing from \pShip{} (long before your time, though). You have helped \cPrincipal{\them} maintain the school for decades.
    \contact{\cHistory{}} The History teacher at the school and a good friend of yours, learned in lore and with a great appreciation for the school's history.
    \contact{\cChupInventor{}} The school’s Diplomacy and Political Science teacher, as well as a skilled amateur inventor who helps you maintain the Bunkers. 
    \contact{\cEthics{}} The Morality and Ethics teacher at the \pSchool{}, and more importantly, your secret informant at the school.
    \contact{\cAssistantScientist{}} A highly placed research assistant in the \pTech{} who has been passing you secrets for a few years now.
    \contact{\cPirate{}} The crafting teacher at the \pSchool{}, hailing from the \pShip{}. Originally a friend you enjoyed having no ulterior motives with, but now dating \cPrince{}, and therefore a potential source of crucial information.
    \contact{\cChupLeader{}} \cHeadDiplomat{}'s personal assistant, now playing a more prominent role in the treaty negotiations with \cHeadDiplomat{} out of the picture. Therefore, one of your prime suspects in orchestrating the assassination attempt and being one of the \pGoaties{}.
    \contact{\cJuniorStatesman{}} \cHeadDiplomat{}'s apprentice, now playing a more prominent role in the treaty negotiations with \cHeadDiplomat{} out of the picture. Therefore, one of your prime suspects in orchestrating the assassination attempt and being one of the \pGoaties{}.
    \contact{\cEbbPriest{}} A \cEbbPriest{\cleric} of \cEbb{}, the one who investigated the assassination attempt on \cHeadDiplomat{}, and who then conveniently declared \cEbbPriest{\themself} \cHeadDiplomat{}'s replacement for the negotiations. Therefore, one of your prime suspects in orchestrating the assassination attempt and being one of the \pGoaties{} — wouldn't that be delightfully ironic for a supposed \cEbbPriest{\cleric} of \cEbb{}?
    \contact{\cPresident{}} \cPresident{\Nibling} of your close friend and ally \cHeadDiplomat{}, and a natural born leader among the students.
\end{contacts}

\end{document}

