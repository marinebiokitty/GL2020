\documentclass[char]{GL2020}
\usepackage[normalem]{ulem}
\parindent=0pt
\begin{document}
\name{\cFlowPriest{}}

You are \cFlowPriest{\full} (\cFlowPriest{\they}/\cFlowPriest{\them}), and a teacher at the \pSchool{}. You teach Mathematics, and have done so for the last 15 years. Mathematics is the language of nature, and understanding that language allows you to understand and create \emph{beautiful} things, and use that understanding to make balanced decisions. It is through your love of patterns, the predictions they allow, and the creation of new things that you serve \cFlow{\full}.

You grew up in the \pShip{}, the \cFlowPriest{\offspring} of shipwrights in the 4th Fleet. You built your first model boat when you were five. In your teenage years, you moved around among a number of ships, ultimately settling in a woodworking compound in 6th Fleet Shell Bay. From there you went to 2nd Fleet Academy to study Religion. You were drawn particularly to \cFlow{}, and soon completed your initiation to \cFlow{\their} Path.

You ended up serving in 5th Fleet, Coral Sands Harbor town. While the work was nice, you aspired to more. Eventually a position came open at the \pSchool{}, to which you applied and were selected. Ever since then, you have worked at the \pSc{}, shaping the new minds that come through, and sending them out into the world to do \cFlow{}'s work.

Your time at the \pSchool{} shaped you as well. Over the years you bonded with a number of your fellow teachers, but perhaps none more than the music teacher, \cMusic{\full}. The two of you would have long, heady discussions about religion, values, and how the Gods speak through the language of mathematics and music, two sides of the same coin. Compassionate, wise, and with a deep spirituality and reverence for the divine, \cMusic{} quickly earned your deepest respect and friendship. You even found yourself developing the kind of intense crush normally reserved only for the young and foolish in those early years — now over a decade ago. But you had no reason to believe \cMusic{} returned your feelings, and you were both so busy with your many responsibilities. You let it go; to ebb away like so many things do in life — and thus make room for other things.

Your greatest test came six years ago, when the infamous Betrayal of your people occurred — the Storm was sent to the \pShip{} out of turn, and the 12 students who performed the ritual were murdered by unknown agents. You had your hands full in the immediate aftermath, working closely with \cEthics{\full} and \cMusic{} to provide emotional support for everyone. Many of your colleagues ended up resigning after that. You nearly quit yourself out of horror and disgust at what had transpired. But no. You had to stay, and be a safe harbor in this ongoing storm. 

But as you gradually adjusted to the new normal, you were reminded that life is full of peril and uncertainty, and no one can say what the future will bring. You do not fear death and know when your time comes you will find your place in the stars and the Goddesses’ embrace, but you do fear reaching the end of your days with a heart full of regrets. As one of only a handful of teachers who stayed on after the loss, you’ve worked closely with \cMusic{} to support your new colleagues. The increased proximity has unearthed your previous infatuation. It seems that rather than washing away, it has rested and grown like a precious pearl. Perhaps it is time to put fear and doubts aside and tell \cMusic{\them} how you feel? But everything is complicated, especially now that \cMusic{} has decided to run for Principal of the \pSchool{}. You know the opportunity means a lot to \cMusic{\them}, but also means \cMusic{\they} will be immortal if they win. \cMusic{} has been coming to you for counsel, asking about the tradeoffs and seeking balance in this decision. You want to be impartial, but you must admit that you have let your sadness infiltrate the advice you have given. Perhaps you have been highlighting what one gives up too much? Is it right to tell \cMusic{} now, in case your feelings toward \cMusic{\them} would influence \cMusic{\their} decision? Or is it better to wait until after the matter of the principalship has been decided, so as not to unduly influence \cMusic{\them} into maybe giving up something for your sake, instead of acting for \cMusic{\them}self?

\cFlow{} is about growth and change. You feel that it is past time to branch out and take risks, try bold new ideas, and rebuild sorely needed bridges with people from other nations. In the past few years, you have struck up correspondences with several like minded individuals around \pEarth{}. One of your favorites is \cCurse{\full}. Over time the two of you have gone from just chatting about magical theory to actually creating something completely new. Humans have never in recorded history set foot in the Realm of the Gods. But where there is a will, there is a way. You and \cCurse{} have been working for years on the ``\iBeansMB{}'' which, once you complete them, can be planted and will grow into a portal that will connect \pEarth{} to the Realm of the Gods — you think. You won't know for sure until you try it! The two of you have had little chance to work together in person until this morning, and you expect to be able to finish them this weekend! 

To those who would naysay such endeavors (or point out that they would be considered blasphemous by all 3 Churches), you ask why they should fear the Deities so. Just because no one has done something yet does not mean it shouldn't be done. Just because a worthy endeavor requires a few taboo rituals doesn't mean you shouldn't try. By the teachings of \cFlow{}, it is quite the opposite — the new and creative is particularly beloved of your Goddess, and what is legal is not always what is beneficial for humanity. 

You and \cCurse{} are aided in this endeavor by \cAmbition{\full}, a particularly bright and keen student from the \pTech{} who often turns to you for guidance. It was \cWarlordDaughter{} who originally encouraged \cAmbition{} to speak to you outside of class, and now you have tea weekly. \cAmbition{} is looking for a place to belong and has taken a keen interest in your nation’s way of life, so you decided to give \cAmbition{\them} this opportunity to demonstrate \cAmbition{\their} skills and contribute to something truly extraordinary. If \cAmbition{\they} succeed\cAmbition{\verbs}, you would be glad to help \cAmbition{\them} gain a place of prominence in the \pShip{}. Your little group of conspirators is rounded out by \cAssistantScientist{\full}, whom \cCurse{} wrote to a few years ago for \cAssistantScientist{\their} technical expertise and who jumped at the chance to be a part of something this revolutionary. 

Even as you contemplate visiting the Divine Realm, you desperately wish that things on \pEarth{} were in better shape. The \pFarm{} and the \pTech{} were fools to upset the fragile balance, and the war that \cLoud{\full} brings to bear on them is their just reward, as terrible as it has been. But you fear that someone has meddled further. The \cEbb{} Avatar on \pEarth{} is dead. You know from the lips of the Goddesses to your own ears. Avatars don't die of natural causes. Someone killed it. Someone very strong in magic, who had some way to lure the Avatar to shore, or found a ship treacherous enough to carry them to deep waters, and a weapon capable of the deed. You have little doubt that pirates were involved. While pirates should be on the \pShip{}’ side, they seem to only want to destroy and you can’t think of who else would have this capability or care this little for the consequences.

Still, all hope is not lost. It can't be just fortune that \cEbbPriest{\full}, \cEbbPriest{\cleric} of \cEbb{}, is coming to the \pSchool{} as part of the \pShip{} Advisor contingent. You know \cEbbPriest{} through correspondence that you have shared throughout the years, and you respect \cEbbPriest{\their} theological knowledge and deep intellectual curiosity. You even recommended that \cWarlordDaughter{\full} (now one of your favorite students, but back then just a disappointed youth whose parents were in no position to support \cWarlordDaughter{\their} attending 1st or 7th Fleet Academies for religious studies), begin corresponding with \cEbbPriest{\them} as part of \cWarlordsDaughter{}’s ad hoc initiate training. 

You expect to take your cues on resurrecting the \emph{\cEbb{}} Avatar from the \cEbbPriest{\cleric} of \emph{\cEbb{}}. Of course, there are complications of which the answers have yet to reveal themselves. Chief among them is the central problem of where you are supposed to get a live sea serpent to act as a host. The cost of the ritual also will be quite steep (OOC: see details in greensheet). It is just so hard to concentrate enough divine energy on the mortal plane. You are pretty sure that if you resurrected the Avatar in the Divine Realm it would be much less taxing, which makes your plan to get there this weekend even more pressing. Not far behind, though, is the matter of keeping things quiet. If word should get out as to what you are trying to do, the other nations would surely try to stop you, as being short an Avatar puts the \pShippies{} at a grave disadvantage in the war.

Thankfully, you trust that your initiates will rise to such a serious task. \cWarlordDaughter{} is the child of \cLoud{\full}, but clearly ready to sail \cWarlordDaughter{\their} own ship. Your heart goes out to \cWarlordsDaugther{\them}, as \cWarlordDaughter{\they} desperately wanted to attend 1st Fleet’s theological seminary, but the war has obligated duties to \cWarlordDaughter{\their} ship that precluded this. You two have corresponded since before \cWarlordDaughter{\they} attended the \pSchool{}. You have tried to help \cWarlordDaughter{\them} \cWarlordDaughter{\have} a strong theological underpinning even without formal training at the academy. Ever since \cWarlordDaughter{\they} 
\cWarlordDaughter{\have} started to attend the \pSchool{}, \cWarlordDaughter{\they} \cWarlordDaughter{\have} have shown quite the religious aptitude, as well as a deep desire to do the right thing, to speak for peace, and to chart \cWarlordDaughter{\their} own course. You know \cWarlordDaughter{\they} \cWarlordDaughter{\have} a lot of challenges ahead.

While you have known \cWarlordDaughter{} via letters for about five years, you only met \cInitiate{} when \cInitiate{\they} arrived at the \pSchool{} three years ago. You quickly found yourself in a mentorship role here as well, a role you enjoy immensely. \cInitiate{} has had the privilege of schooling, but is very naive about how the world works. Yet \cInitiate{\they} \cInitiate{\are} immensely kind, caring and deeply curious about the world, especially about mathematics and engineering. You have been counseling \cInitiate{\them} through some trouble with \cInitiate{\their} grandparents, who are from the \pFarm{} and who seem to really want to see \cInitiate{}. Sometimes \cInitiate{\they} seem\cInitiate{\verbs} to lack confidence, so you have asked \cInitiate{\them} to help with the Avatar situation. You believe this will not only help the faith, but \cInitiate{\them} as well. And maybe through this work \cInitiate{\they} will feel confident choosing a Path.

\cPirateChild{\full}, on the other hand, tests your patience at every turn. While you can sympathize with some of the concerns \cPirateChild{\they} \cPirateChild{\are} wont to express about the plight of the poorer fleets and the inequalities in \pShippie{} society, \cPirateChild{\their} irreverence, disrespect, and obnoxious attitude only serve to disrupt the the classroom and the other students. You have tried everything from gently reasoning with \cPirateChild{} to giving \cPirateChild{\them} detention and have been met only with sneers and defiance. It troubles you that you have so little ability to get through to one of your students, from your own nation no less.

But certain problematic individuals notwithstanding, both you and \cEbbPriest{} still have responsibilities as the best source of comfort and support for the people of your nation this weekend. If anyone should express a sincere desire to join the \pShip{} religion, and forsake their own, helping them do that falls to you and \cEbbPriest{} as well. You must impress upon anyone you induct, be it as a cleric or worshiper, to always bear in mind that worship does not diminish the worshiper. It's not a chore, or a burden to celebrate your Patron Goddesses, and the things they enable you to create. Sharing this message is not a task you take lightly. You have special hope for \cAmbition{} in this.  

It is also your job to guide the two initiates of the faith here, \cInitiate{} and \cWarlordDaughter{}, as they decide which path to pick and, if they prove worthy, to induct them as full fledged clerics. You especially think \cFlow{} would be the right choice for \cInitiate{}. Their right to try for the priesthood shines from the mark of the Goddesses upon their persons \emph{(OOC: represented as a temporary tattoo either on the player's body, or tucked into their badge holder)}. You aren't sure of the reasoning the Goddesses use in who they mark and who they don't, but since time immemorial, none without the mark have ever succeeded in joining the priesthood. In the past century, the priesthood has stopped even letting people try if they don't have the mark. Sometimes you wonder if that rule is short-sighted. You know your view is the minority in the church, however, and that most people think like \cEbbPriest{} does — no mark means no chance at ever becoming a cleric, end of discussion. 

You have been quietly flouting that rule with \cPirate{\full}, a fellow teacher who is deeply devoted to becoming a \cPirate{\cleric}. You are convinced of \cPirate{\their} faith and dedication and have been training \cPirate{\them} as an initiate in secret. \cPirate{} DOES NOT bear the mark of the Goddesses, and it has never been known to appear past puberty. At some point you know you will have to come to terms with this fact, decide whether to allow \cPirate{\them} to undertake the ritual, and if so, somehow convince \cEbbPriest{} to go along with it.

If only all of the advisors were as pious as \cEbbPriest{}. \cBunker{\full} in particular concerns you. \cBunker{\They} \cBunker{\are} clearly begrudging in \cBunker{\their} faith. You don't understand why. Just look at all of the things \cBunker{} has created — there are few more prolific engineers in all of \pEarth{}. So why \cBunker{\are} \cBunker{\they} reluctant to celebrate \cBunker{\their} gifts? You do hope that \cBunker{\their} bad attitude does not rub off on any of the students this weekend.

Perhaps the most important thing going on this weekend will be the Ritual to Control the Storm. As a teacher, it's important for you to ensure that the ritual goes off without a hitch, that the Relics are being treated respectfully, and that the students (especially the four \pShip{} students) are taking a proper interest in the proceedings. If the ritual were to fail to complete, an uncontrolled disaster of the Storm would rage across all of \pEarth{}, not just one nation. This is obviously something to be avoided at all costs. As part of this as well, you must decide who to give your Voting Stone too. You want to make sure it goes to someone who understands the depths of the situation in the world and who would not vote to send it to the \pShip{}. Obviously a \pShippie{} would be the best for this, and based on your experience, you feel that \cWarlordDaughter{} understands the situation the best out of all of the students, though \cInitiate{} may prove \cInitiate{\them}self this weekend. But there is also \cAmbition{}. It is a gamble, but you feel that if \cAmbition{\they} can be trusted for the Divine Realm and if \cAmbition{\they} keep\cAmbition{\verbs} on going the way \cAmbition{\they} \cAmbition{\are} going, \cAmbition{\they} may find a better home among the \pShippies{} than the \pTech{}. You will need to watch \cAmbition{\them} closely to see if the gamble is worth it.  

With so many weighty responsibilities, you really hope that you won't be too busy for the Ceremony of Excellence. It is traditional that a teacher hailing from each of the different nations gives a speech once every three years. Well, this year, the \pShip{} are up, and you are of the opinion that you are the best choice. You know that \cPirate{} has expressed an interest in it too, but really, who could be better to set the students up to create a better world than a current \cFlowPriest{\cleric} of \cFlowFull{\full}? This speech will help shape their mindset during the Time of Deciding, so you need to make sure that you do everything you can to influence that in a positive direction. This speech is also a chance to influence your fellow teachers, and the advisors as well. It's not an opportunity you intend to pass up lightly.

This weekend will be extremely busy. You need to make sure that the ritual goes off without a hitch, resurrect an Avatar, and pave the way to meet your deities, all while performing your normal duties as a \cFlowPriest{\cleric} of \cFlow{} and a teacher at the \pSchool{}. You may have a lot to accomplish — but as always, your faith in the balance that defines your people will see you through.

\begin{itemz}[Goals (in roughly descending order of importance)]
    \item Make sure the Storm does not get sent to the \pShip{}. Pursuant to that, decide which student to give your Voting Stone to for the student vote for where to send the Storm.
    \item Resurrect the \cEbb{} Avatar and keep the process a secret from the other nations.
    \item Help guide \cInitiate{}, \cWarlordDaughter{}, and \cPirate{} in their path to becoming clerics.
    \item Finish the \iBeansMB{}, plant them, and visit the Realm of the Gods.
    \item Ensure the Ritual to Control the Storm is properly set up, goes smoothly, and that none of the six Relics are tampered with. Encourage as many students as possible to take an active hand in the preparations, and educate them on the functions of all the steps so they can understand the system.
    \item Decide what to do about your romantic feelings for your dear friend \cMusic{}.
    \item Convince \cMusic{} to pick you to give the speech in the Ceremony of Excellence and then give a moving and inspiring speech.
\end{itemz}

\begin{itemz}[Notes]
    \item Until the \cEbb{} Avatar is resurrected, both the path of \cEbb{} and the Path of Balance are locked to new initiates.
    \item You teach mathematics at the \pSchool{}.
    \item You were present at the school six years ago, but too busy trying to console everyone, including yourself, to think it wise to get underfoot with the investigation into the deaths of the students.
    \item In your work as a \cFlowPriest{\cleric}, you have heard whisperings of a heretical cult among the \pShip{}, known as the \pGoaties{}. You have few solid facts, but you know they have lured some of the faithful away from the path of the Twin Goddesses, and committed violent acts in the name of their false god, \cGenesis{}.
\end{itemz}

\begin{contacts}
    \contact{\cCurse{}} A kindred spirit in the \pFarm{}, \cCurse{} has been working with you on the \iBeansMB{} to reach the Divine Realm.
    \contact{\cEbbPriest{}} A surprise addition to the Advisor delegation from \pShip{}, but you will be grateful to have \cEbbPriest{\their} support this weekend. \cEbbPriest{\They} \cEbbPriest{\are} more strict about rules than you are, and the two of you don't always see eye-to-eye, even on matters of theology.
    \contact{\cInitiate{}} A trustworthy initiate of the \pShippies{} religion. As \cInitiate{\they} \cInitiate{\are} graduating this year, it is about time \cInitiate{} picks a Path and comes into \cInitiate{\their} full power.
    \contact{\cWarlordDaughter{}} An initiate of the \pShippies{} religion. \cWarlordDaughter{} is in \cWarlordDaughter{\their} second year at the \pSc{}, and still has some time to choose a Path if \cWarlordDaughter{\they} wish\cWarlordDaughter{\verbes} to delay.
    \contact{\cPirate{}} Your friendly rival for the honor of giving the teacher's speech during the Ceremony of Excellence. More importantly, a devotee to the \pShip{} Goddesses whom you are training in secret to become a \cPirate{\cleric}, despite \cPirate{\their} lack of the mark of the Goddesses.
    \contact{\cBunker{}} An engineer and follower of \cEbb{}. You worry about \cBunker{\their} lack of enthusiasm for giving the Goddesses their due, and intend to shield your initiates \cInitiate{} and \cWarlordDaughter{} from any undue influence \cBunker{\they} might try to exert.
    \contact{\cMusic{}} Your dear friend and fellow veteran teacher, with whom you have been through so much and had so many deep discussions. You have long had romantic feelings for \cMusic{\them} and wonder if now might be the time to risk sharing how you feel.
    \contact{\cAmbition{}} A bright and eager student from the \pTech{} who is looking for a place to belong. You have been providing spiritual guidance and \cAmbition{\they} \cAmbition{\are} helping with your and \cCurse{}’s project to reach the Divine Realm. 
    \contact{\cAssistantScientist{}} A brilliant scientist from the \pTech{} who has been helping you and \cCurse{} with your project to reach the Divine Realm.
    \contact{\cPirateChild{}} An irreverent and obnoxious student from your own nation whom you just can’t seem to get through to.
\end{contacts}

\end{document}



