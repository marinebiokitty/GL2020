\documentclass[char]{GL2020}
\parindent=0pt
\begin{document}
\name{\cPirateChild{}}

You are pretending to be \cPirateChild{\intro}, and 18 years old. That's not your real identity, though. You are really a \pShippie{} pirate hailing from Silent Fog ship in the 8th Fleet, and your mission here has been years in the making. You were determined to stick it to ``civilized society'' and had no intention of actually sticking around this stupid school, but you may have accidentally made some friends here that make your mission\ldots{} more complicated.

You grew up on the high seas, and your loyalty lies first with your crew on the Silent Fog and second with your seafaring nation's people, if not with her laws. The Silent Fog is a notorious pirate ship, terrorizing the coast of the \pFarm{} well beyond the normal extortion racket. They are your family, and you are a good pirate, and a damn fine pickpocket. And if your fear of fire is a little stronger than normal for a pirate, no one makes a big deal of it. You weren't born on the Silent Fog, though. One of her former crew brought you to the ship, who in turn passed you off to another ship, the Oyster Bay of 12th Fleet. You spent your first 13 years on the Oyster Bay before hopping between multiple ports and ships throughout 8 and 9th Fleets. Finally you wormed your way back aboard the Silent Fog driven by curiosity about where you originally came from. But the crew of the Silent Fog at the time didn't know, and you fit right in, so you stayed. The crew member who had brought you aboard when you were a baby was away on a secret mission long in the making, one which you now get to be a part of.

Life on the Silent Fog is hard. Infamy does not put fish in the nets or materialize supplies for repairs. Your ship regularly takes daring action to take what you need to survive. Your first foray into all of this failed miserably. Two and a half years ago the Silent Fog decided that capturing and holding \cHeadDiplomat{\intro} for ransom was a risk worth taking. \cHeadDiplomat{}, the main diplomat of the \pShippies{} did not represent you all, and if you succeeded, you would surely force the hand of the Council of Storm Watchers. But the security was too tight, and you were thrown off your game by seeing an old shipmate, from your days of ship hopping around fleets 8 and 9, sitting taking notes as \cHeadDiplomat{} intoned. 

You escaped that disaster with your life, but decided to seek out \cChupLeader{\intro}. Turns out that \cChupLeader{\they} had made good and now \cChupLeader{\were} acting as \cHeadDiplomat{}’s secretary. Part of you envied \cChupLeader{\them}. Part of you thought \cChupLeader{\them} a sell out — abandoning an honest living for a cushy desk job. Still, you reached out and made contact. Now you two kept in touch, frequently discussing the distress you felt about the inequality in the \pShip{}. \cChupLeader{} encouraged you to write, to share your ideas. Keeping this contact proved useful for your next mission, which was to infiltrate the \pSchool{}. 

At the \pSchool{} is housed the ``\iNet{}''. This object is a Relic that according to legend was built by artisans from all 12 fleets under the direction of the Goddesses themselves. And while it can certainly be used in the rituals at the \pSc{}, and to entangle someone, it can also be used to incredible effect by an individual ship. If the Silent Fog could get the Net aboard, it would be able to control the waters in the \pWod{} — it would be uncatchable. You have been tasked with stealing the \iNet{}, still attuned to the \pShip{} (which means it can’t be used for the Ritual to Control the Storm) and bringing it home.

Quite a big task, but you don't have to do this alone. This plan isn’t new. In fact it may be older than you are. The former crew member of the Silent Fog is here. Somewhere. Their coded messages say so. But it was agreed upon, long before you could read those codes, that it was crucial to wait until the last possible minute for you to make contact. This is the last minute, and you need to find someone who will respond to the phrase, ``The Waters untouched are not abandoned,'' with, ``by those brave enough chart the course.'' The two of you will then need to devise a way to steal the \iNet{} and get it home. You'll only get one shot at this as their cover will be blown after this — so make it count, and make sure the two of you get home safely. The undercover pirate has to be one of the teachers. But which one? Is it one of the \pShip{} teachers, or are they deeper undercover, pretending to be from one of the other nations? During the Time of Deciding, there are many fewer people at the \pSc{}, so your pool of possibilities is much smaller, so making contact should be easier and your odds of getting caught much lower.

But now you're getting ahead of yourself. When your mysterious crewmate wrote to the ship that they would need a second agent to liberate the \iNet{}, there was somewhat of a scramble to find a way to get a student into the school. It was great luck that \cChupLeader{} reached out just then. It seems that the \pShip{} may need your specific skills to liberate some blueprints for a device called the VidCom, which allows people to communicate over long distances. You care about patriotism about as far as you could throw an anchor, but you would be glad to screw over the \pTech{}, and these devices could be invaluable to the Silent Fog. The blueprint is with its inventor, \cTechStar{\intro}, a first year student at the \pSchool{}. \cChupLeader{} offered to pay your tuition to the \pSc{} in return for stealing the blueprints. You of course plan to find a way to make a copy before you have to turn them over to \cChupLeader{} (OOC: You will need to find a mechanic to copy blueprints.)

Not long after that disastrous attempt to kidnap \cChupLeader{}, you also began to converse with \cJuniorStatesman{\intro}, the assistant diplomat to \cHeadDiplomat{}. It started as a dare — your shipmates did not believe that some goody goody 2nd Fleet diplomat would bother answering your letter. Joke’s on them, \cJuniorStatesman{} wrote you back, \emph{and} takes your concerns about inequality seriously. You have been surprised by \cJuniorStatesman{\their} willingness to listen to your concerns about peace, the lower fleets’ precarious positions, and how much justice and reform are needed within the \pShip{}. Maybe it is because your cover is one of the 5th Fleet, so \cJuniorStatesman{\they} trust\cJuniorStatesman{\verbs} you more than \cJuniorStatesman{\they} otherwise might, but you hope it is because \cJuniorStatesman{\they} actually care\cJuniorStatesman{\verbs}. Which still catches you by surprise and is really messing with your whole concept of how people with power act in the world. \cJuniorStatesman{} is going to be here this weekend, so you hope that you can keep up the ruse and maybe press the case even more.

In the meantime, you’ve had to maintain your cover as a student. In the process, you have become very fond of two of the \pShippie{} teachers. \cPirate{\intro} is the crafting teacher, and while so much of what \pSchool{} teaches you is simply not practical, knowing how to repair ships and use magical crafting are incredibly important skills. You are glad that you are getting something out of this year-long undercover mission that you can bring back home. Besides, \cPirate{} listens and seems to understand your frustration and even encourages some of your anger. You can’t trust anyone, but you want to trust \cPirate{\them}.

And then there is \cChupAvenger{\intro}, the art teacher. Art is totally impractical in your life, but you have found that you have a deep passion for it as well as the talent. It is another thing that frustrates you — if you lived in the 2nd Fleet or something you would have been able to explore this side of you so much more. But you are grateful you can do it a bit here. You have not talked about your anger about the system to \cChupAvenger{} as much, but on the occasions when you have expressed it, you have seen subtle hints of \cChupAvenger{\their} approval. 

You don’t know nearly as much about the other \pShippie{} teachers. You have gone on a few serious rants about class inequality in \cChupSecond{\intro}’s class and sometimes \cChupSecond{\theyhave} nodded along and encouraged you. Other times \cChupSecond{\theyhave} landed you in detention before you could get three words out. \cFlowPriest{\intro} is even worse. \cFlowPriest{\Theyare} on your case all the time for not taking your education seriously enough, for being a troublemaker and rabble rouser, for sleeping in class, for not bothering to turn in your math homework — seriously, what use is advanced mathematics going to be once you go home? You avoid \cFlowPriest{\them} whenever possible, and annoy \cFlowPriest{\them} until \cFlowPriest{\they} leave\cFlowPriest{\verbs} when you can’t.

Conning a clueless landlubber out of some blueprints all seemed well and good, until you arrived at the \pSchool{} and actually met \cTechStar{} and \cTechStar{\their} best friend \cDisney{\intro}. You don't like \cTechStar{}. You two just seem to be opposites about everything. \cTechStar{\They} sit\cTechStar{\verbs} on the High Council of the \pTech{} for the Goddesses’ sake and are a know-it-all who speaks idealism but knows little hardship. And boy is the feeling mutual. \cTechStar{} makes no bones about disliking you. While this helps ease any potential guilt over stealing from \cTechStar{\them}, it does make your mission more difficult. Somehow you have to overcome \cTechStar{\their} disdain and your own disgust and trick \cTechStar{\them} or get \cTechStar{\them} to trust you with the blueprints. Or get \cDisney{} to convince \cTechStar{\them} \cTechStar{\they} should.

The problem is that you actually like \cDisney{}! While you initially scoped \cDisney{} out as an easy in to get close to \cTechStar{}, pretending to be friends very quickly melted into being \emph{actual} friends. \cDisney{\Theywere} asleep for 2 centuries — everyone \cDisney{\they} knew and loved is gone. You aren't sentimental or anything, but ugh, \cDisney{\they} grew on you like a limpet. \cDisney{\Theyhave} even told you (in very strict confidence!) about what happened to \cDisney{\them} before \cDisney{\theywere} put to sleep, that \cDisney{\theywere} part of a religious group called the \cDisneySect{\intro} which the \pFarm{} did not like due to something about visiting the Divine Realm. You don’t know all of the specifics, but it sounds very cool, very serious, and very dangerous. You don’t know if \cDisney{} wants to go to the Divine Realm now, but you think it would be amazing, and you are hoping to encourage \cDisney{} to try again with your help! You may need to work with \cTechStar{} to do so, you fear. You don’t like \cTechStar{\them} but \cTechStar{\theyare} very smart and working together on this may endear you to \cTechStar{\them}. 

You and \cDisney{} get along so well that when you learned it was going to be \cDisney{}'s birthday this Sunday, you knew you had to do something big. At first the idea seemed ridiculous and over the top. But the more you thought about it, the more enamored of it you became. You snuck a sea serpent egg into the \pSc{} with you about a month ago. Your patience ran out though, and you showed it to \cDisney{} early. Unfortunately, \cDisney{\theywere}n't as keen on the idea as you, pointing out things like how big sea serpents get. That is definitely a problem, but one to solve\ldots{} later.

So you brought the question to another friend of yours, \cAdopted{\intro}. When did you end up with so many friends? You need to stop making so many, since you’re about to blow your cover and leave them all forever. But really, \cAdopted{\they} would make a wonderful pirate, and you have half a mind to try and convince \cAdopted{\them} to come with you when you leave. \cAdopted{} was born poor but happy, but when \cAdopted{\their} magical talent came to light \cAdopted{\theywere} forcibly adopted into a noble house, the same one as one of the teachers here, \cMusic{\intro}. Of course \cAdopted{\they} hate\cAdopted{\verbs} it and \cAdopted{\their} anger, \cAdopted{\their} passion, and \cAdopted{\their} clear-eyed view of how absolutely screwed up the system of class is across the world makes you natural allies, and, well, friends. So even though \cAdopted{} shook \cAdopted{\their} head at you both, \cAdopted{\theywere} still willing to help. With \cAdopted{\their} help, the two of you disguised the egg as a stuffed animal while you tried to figure out what to do. And now that it's hatched, the disguise spell from \cAdopted{} has broken, and the precious little thing needs to be taken care of.

It's not like it would have made a very convincing stuffed animal when it keeps wiggling and sneezing, anyway. Hmm, you hope it isn't sick\ldots{} And who knew they ate so much? (For now you and your friends have hidden the sea serpent in a hidden compartment behind one of the bookshelves in the student lounge.) Maybe trying to gift your friend a wild creature that will grow up into a ravenous monster wasn't the best idea. Still, it would be a shame to have the cute little thing taken away, and \cDisney{} does seem very attached to it now that it's hatched. Your biggest concern is the clerics. The Church of \cEbb{\intro} and \cFlow{\intro} has a complicated relationship with sea serpents, as the Avatars of the Goddesses manifest in the serpents, but full grown serpents are incredibly dangerous. And the clerics and their initiates have been talking a lot about Avatars in hushed tones in the last 24 hours\ldots{} Maybe you can convince \cTechStar{} to throw a surprise party for \cDisney{} to make up for all the stress of keeping the sea serpent hidden? You will probably need \cAdopted{} to mediate between the two of you, though, since it would spoil the surprise if \cDisney{} had to do the mediating.

Even though you are making friends, that does not mean your distrust of the \pFarm{} and the \pTech{} is gone. There are students here who really represent to you the harm done by their nations, and you just can’t get over that reminder. It makes you wary of many of the students from outside of the \pShip{}. You already don’t like \cTechStar{}. And then there is \cLibAssist{\intro}. \cLibAssist{}’s family did some terrible things both to your people with curse trebuchets and also to their own allies through equipment malfunctions (serves them right for trusting the \pTech{} though). Just seeing \cLibAssist{\them} here reminds you of all of the crimes against your people. 

And then there is \cPresident{\intro}. \cPresident{\They} may be from the \pShip{}, but represents to you the inequality and sheer wrongness of the system that perpetuates \pShippie{} culture. \cPresident{\Theyare} from 2nd Fleet, had the best education, and are of course the beloved student body president. Oh, and \cPresident{\their} \cPresident{\auncle} is \cHeadDiplomat{}, the self same diplomat that you tried to kidnap all those years ago and failed. Not only does \cPresident{} represent the system of entitlement that means powerful families stay powerful while the rest of the \pShip{} and especially the Silent Fog struggles to survive, but \cPresident{\their} very presence reminds you of how you failed at your first mission. You hope that if you steal the \iNet{} before you go you can wave it in \cPresident{}’s face. 

But honestly, the biggest temptation to blow your cover is how many excellent things there are to steal at the \pSc{}. You were always an expert pickpocket, having learned from the best, and the \pSc{} is just full of deep pockets and valuable trinkets that aren't nailed down. You've even managed to steal small things from some of the teachers' offices. \cChupInventor{\intro} was the hardest. Why \cChupInventor{\do} \cChupInventor{\theyhave} so much security on \cChupInventor{\their} office anyway? Obviously none of the teachers would \emph{approve} of you stealing things (except your mysterious undercover pirate contact), but there are a few you’d particularly prefer to avoid getting caught by: \cFlowPriest{}, \cInterpol{\intro} and \cPrincipal{\intro} foremost among them. As for what to do with the stolen items, you've heard there is a black market at the school — the perfect place to fence them, and see if there is any interesting contraband for sale while you're at it! You just need to find out who to talk to.

Lastly, there is the matter of the Ceremony of Excellence. Those stuck-up 1st Fleet kids are graduating, but that doesn't mean one of them has to be the ones to give the speech — any student can do it. But you don't particularly want to hear from \cWarlordDaughter{\intro} either. While you like \cWarlordDaughter{} alright, \cWarlordDaughter{\their} \cLoud{\parent}, \cLoud{\intro}, is the one leading the war effort. No one has suffered from the curses and trebuchets of the \pFarm{} and the \pTech{} more than the pirates and others left on the margins of the \pShip{} society. You know that only you can really talk about the true nature of the world, and maybe shock a few of the stuck up types in the process. All you have to do is convince \cMusic{\full} that you are the best choice. Not that public speaking is your favorite, but it would be glorious to have a platform to give a lot of people a piece of your mind, both within the \pShippies{} and beyond.

You were \emph{not} expecting to tolerate \cWarlordDaughter{}, much less enjoy spending time with \cWarlordDaughter{\them}. \cWarlordDaughter{} gets treated poorly by students of the \pFarm{} and the \pTech{}, and only you get to give \cWarlordDaughter{\them} a hard time for being the the \cWarlordDaughter{\offspring} of \cLoud{}. People from other nations can shove it. You regularly give a piece of your mind to anyone bothering \cWarlordDaughter{\them}. Besides, weirdly enough \cWarlordDaughter{\they} actually really seem to want to be different from \cLoud{}. You always thought \cWarlordDaughter{\they} would be stuck up, but \cWarlordDaughter{\theyare} the opposite — kind, humble, and acutely aware of \cWarlordDaughter{\their} shortcomings. 

Maybe it is just the stress of your mission, but your dreams have been restless of late too. The past three nights you've had dreams about the parent who gave you up all those 17 years ago. Last night, the message was most emphatic — you are supposed to find them. Apparently they will be at the \pSc{} this weekend? You would probably have ignored the dream, except that it had divine weight behind it. Too bad the Deity couldn't be bothered to reveal themself. Still, when a Deity says climb the rigging, you climb first and ask ``how high?'' later. The pirate who is here undercover is the self-same one who brought you to the Silent Fog all those years ago. They have to know something about who gave you to them, so you can start your investigation there. Just one more reason to find your undercover contact as soon as possible.

Though this weekend is intense, you have your friends, your favorite teachers, \cChupAvenger{} and \cPirate{}, and many advisors to help you make your decisions. You know you can be open about anything with \cChupLeader{}, who is attending this weekend and you think you are starting to trust \cJuniorStatesman{} too. You don’t personally know \cBunker{\intro}, who is one of the greatest engineers in the world, but you have heard that \cBunker{\theyare} practical and probably give better advice than any ivory tower shut-in. Maybe it is for the best that \cHeadDiplomat{} has been replaced by \cEbbPriest{\intro} among the advisors. What if \cHeadDiplomat{\they} had recognized you? But having a second cleric around means you need to be extra careful to hide that serpent when \cEbbPriest{\theyare} around.

This weekend will be a high stakes gamble, but if it pays off, you and the rest of the Silent Fog will be set for generations. You just have to figure out if you are ready to say goodbye to all of your new friends, and discover the identity of your birth parent.

\begin{itemz}[Major Goals]
    \item Find the undercover pirate, help them steal the properly attuned ``\iNet{}'', and bring both the pirate and the Relic home with you. Don't let the \iNet{} be used in the Ritual to Control the Storm. Avoid blowing either of your covers until the last possible second.
    \item Work with \cDisney{} and \cAdopted{} to keep the baby sea serpent a secret — more importantly, take care of it and keep it alive, as it would be really awkward if the birthday gift for your friend were to die.
    \item Get the VidCom blueprints, or at least a prototype, from \cTechStar{} and then give them to \cChupLeader{}.
\end{itemz}

\begin{itemz}[Minor Goals]
    \item Get chosen to give the student speech at the Ceremony of Excellence. You can figure out what to actually say later. Maybe \cJuniorStatesman{}, \cChupAvenger{} or \cPirate{} have some ideas?
    \item Find the parent that gave you up as a baby. Step one is finding the undercover pirate, who should know where to start looking.
    \item Steal anything valuable that isn’t nailed down, and sell it on the Black Market. Is \cChupInventor{}’s hiding something worth stealing?
    \item Help and support \cWarlordDaughter{}. Those of you from higher fleets have to stick together!
\end{itemz}

\begin{itemz}[Notes]
    \item You are supposed to find the undercover pirate by asking around using the phrase: ``The Waters untouched are not abandoned.'’ You expect the following response from the right person: ``by those brave enough chart the course.'' You will have to be careful though, many people do not like pirates, or even outright hate them. If the wrong person somehow knew the code phrases, or overheard you, it could blow everything. You have \emph{deliberately} made no attempt to find your contact before today.
    \item You have a strong phobia of fire, particularly large open flames, but torches and even candles make you nervous.
        \item There is a greensheet associated with the Bookcase in the Student Lounge where the baby sea serpent is hidden. This greensheet will explain how to take care of the baby over the course of the weekend. Your S-Score (1) allows you to freely view the sign hidden by the Bookcase sign, and access the greensheet.
\end{itemz}

\begin{contacts}
    \contact{\cChupLeader{}} Your sponsor at the \pSc{}, who knows you are a pirate due to catching you attempting to kidnap \cChupLeader{\their} employer a few years back. \cChupLeader{} has asked you to get close to \cTechStar{} and \cTechStar{\their} friends in order to steal the blueprints for the VidCom device.
    \contact{\cJuniorStatesman{}} A diplomat you have been writing on behalf of the 9th Fleet and all others who suffer due to the injustices of \pShippie{} culture.
    \contact{\cPirate{}} The crafting teacher whose class you love and who encourages your anger.
    \contact{\cChupAvenger{}} The art teacher whose class is a guilty pleasure and who responds unpredictably to your rants about injustice.
    \contact{\cTechStar{}} A truly obnoxious know-it-all you are trying to steal the VidCom blueprints from. Pretending to be \cTechStar{\their} friend would probably be a good idea, but unfortunately you really can’t stand \cTechStar{\them}.
    \contact{\cDisney{}} You first scoped \cDisney{\them} out to try to get close to \cTechStar{}, but you soon became actual friends. You brought \cDisney{\them} a sea serpent egg for \cDisney{\their} birthday.
    \contact{\cAdopted{}} Another friend who is helping you hide the sea serpent you brought \cDisney{} for \cDisney{\their} birthday.
    \contact{\cChupInventor{}} A teacher from the \pTech{} with a suspicious amount of security on \cChupInventor{\their} office.
    \contact{\cWarlordDaughter{}} the child of \cLoud{} who you really should not like, but has become a friend due to \cWarlordDaughter{\their} forthrightness and honesty.
\end{contacts}
\end{document}
