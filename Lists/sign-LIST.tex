%%%%
%%
%% This file sets up the Sign and Label datatypes and creates Sign and
%% Label macros.
%%
%% Signs generally represent interesting parts of game area, usually
%% as things posted on walls.  Labels represent other things, often on
%% or inside envelopes, that are part of complex mechanics.
%%
%% The default value for \MYloc will inherit location from the Place
%% or Sign most immediately up the ownership tree.  Override this by
%% setting \MYloc to anything (even blank).
%%
%% Sign is for full-sized signs that would cover most of a large
%% manila envelope; SignMedium is for signs sized to half-sized manila
%% envelopes; SignSmall is for signs sized for small manila envelopes
%% (the same size as item cards).  Label, LabelMedium, and LabelSmall
%% are analagous, but they don't have a \takedownby note at the
%% bottom.  You can always use a sign or label without an envelope or
%% with a differently-sized envelope.  Choose which based on
%% visibility and content.
%%
%% SignTiny is for signs you want to be hard to find; it is small and
%% does not have a \takedownby note.  SignDot is for a very small
%% "dot" which only has a title.
%%
%% SignStrip produces a strip of paper (without a \takedownby note)
%% with labels on the outside that show on both sides if you fold it
%% in half.  These are a convenient alternative to sub-envelopes. They
%% can also be used for "s-packets" taped to walls (see
%% Extras/README-s-packets).
%%
%% LabelCover produces a label similar to the cover to a research
%% notebook.  LabelPage, likewise, produces a page.
%%
%% EOG is for full-sized end-of-game signs.
%%
%%%%%

\DECLARESUBTYPE{Sign}{Element}
\PRESETS{Sign}{
  \FD\MYloc	{\mylocation} %% real-space location
  \FD\MYtext	{} %% text of sign
  }
\POSTSETS{Sign}{
  \edef\mylocation{\MYloc}
  \protected@edef\@ownerstring{%
    \MYname%
    \ifx\mylocation\empty\else\ (\mylocation)\fi%
    }
  }
\def\mylocation{}

\def\loc#1{\rs\MYloc{#1}}

\DECLARESUBTYPE{SignMedium}{Sign}
\DECLARESUBTYPE{SignSmall}{Sign}
\DECLARESUBTYPE{SignTiny}{Sign}
\DECLARESUBTYPE{SignDot}{Sign}
\PRESETS{SignDot}{\s\MYtext{}}

\DECLARESUBTYPE{Label}{Sign}
\PRESETS{Label}{\s\MYloc{}}
\DECLARESUBTYPE{LabelMedium}{Label}
\DECLARESUBTYPE{LabelSmall}{Label}

\DECLARESUBTYPE{SignStrip}{Sign}
\DECLARESUBTYPE{LabelCover}{Label}
\DECLARESUBTYPE{LabelPage}{Label}

\DECLARESUBTYPE{EOG}{Sign}
\PRESETS{EOG}{%
  \s\MYname	{End Of Game}
  \s\MYtext	{{\bf\Huge You may not pass through here.}}
  }


%%%%%
%% \signbig[<location>]{<name>}{<text>}
%% \eog[<location>]
%%
%% \signmdeium[<location>]{<name>}{<text>}
%% \signsmall[<location>]{<name>}{<text>}
%% \signtiny[<location>]{<name>}{<text>}
%% \signdot[<location>]{<name>}
%%
%% \labelbig{<name>}{<text>}
%% \labelmedium{<name>}{<text>}
%% \labelsmall{<name>}{<text>}
%%
%% \signstrip[<location>]{<name>}{<text>}
%% \labelcover{<name>}{<text>}
%% \labelpage{<name>}{<text>}
\newinstance{Sign}{\signbig[3][\mylocation]}{
  \s\MYloc{#1}\s\MYname{#2}\s\MYtext{#3}}
\newinstance{EOG}{\eog[1][\mylocation]}{\s\MYloc{#1}}

\newinstance{SignMedium}{\SignMedium[3][\mylocation]}{
  \s\MYloc{#1}\s\MYname{#2}\s\MYtext{#3}}
\newinstance{SignSmall}{\signsmall[3][\mylocation]}{
  \s\MYloc{#1}\s\MYname{#2}\s\MYtext{#3}}
\newinstance{SignTiny}{\signtiny[3][\mylocation]}{
  \s\MYloc{#1}\s\MYname{#2}\s\MYtext{#3}}
\newinstance{SignDot}{\signdot[2][\mylocation]}{
  \s\MYloc{#1}\s\MYname{#2}}

\newinstance{Label}{\labelbig[2]}{
  \s\MYname{#1}\s\MYtext{#2}}
\newinstance{LabelMedium}{\labelmedium[2]}{
  \s\MYname{#1}\s\MYtext{#2}}
\newinstance{LabelSmall}{\labelsmall[2]}{
  \s\MYname{#1}\s\MYtext{#2}}

\newinstance{SignStrip}{\signstrip[3][\mylocation]}{
  \s\MYloc{#1}\s\MYname{#2}\s\MYtext{#3}}
\newinstance{LabelCover}{\labelcover[2]}{
  \s\MYname{#1}\s\MYtext{#2}}
\newinstance{LabelPage}{\labelpage[2]}{
  \s\MYname{#1}\s\MYtext{#2}}


%%%%%
%% \sEOG{}
%% use \sEOg[\loc{<location>}]{} for EOG sign at a specific place
\NEW{EOG}{\sEOG}{
  }


%%%%%%%%%%%%%%%%%%%%%%%%%%%%%%%%%%%%%%%%%%%%%%%%%%%%%%%%%%%%%%%%%%

\NEW{Sign}{\sTest}{
  \s\MYname	{A Room}
  \s\MYloc	{10-250}
  \s\MYtext	{A lecture hall with large, sliding blackboards.}
  }


%%%%%%%%%%%%%%%%%%%%%%%%%%%%%%%%%%%%%%%%%%%%%%%%%%%%%%%%%%%%%%%%%%


%%GM HQ
\NEW{Sign}{\sMurderConsequences}{
  \s\MYname	{So You Murdered Someone. Now What?}
  \s\MYloc	{GM HQ}
  \s\MYtext	{If your character successfully kills another character, take a copy of the document from the appropriate envelope below, based on your \textbf{V-score} (0, 1, or 2).

If your \textbf{V-score changes}, and you kill again after that, return to this sign and take a copy of the document from the other envelope. The effect may be different.

If the envelope you need to draw from is empty; tell a GM and we'll print more copies.}
  \s\MYgreens {\multi{5}{\gMurderRegular{}\gMurderChup{}}}
}

\NEW{Sign}{\sMurdered}{
  \s\MYname	{So You were Murdered. Now What?}
  \s\MYloc	{GM HQ}
  \s\MYtext	{If your character is killed during the game, it is by no means the end of your ability to participate in this game, or the end of your character’s story.

Take a copy of \textbf{the Greensheet} stored here, and a \textbf{Yellow headband}.
}

	\s\MYgreens {\multi{5}{\gLifeAfterDeath{}}}
}
  
\NEW{Sign}{\sSignCOne}{
  \s\MYname	{Sign C}
  \s\MYloc	{GM HQ}
  \s\MYtext	{You may not interact with this sign unless instructed to do so by a mechanic.}
}

\NEW{Sign}{\sSignCTwo}{
  \s\MYname	{Sign C-2 Dreaming of the Past}
  \s\MYloc	{GM HQ}
  \s\MYtext	{After successfully completing the ritual to use \iMirror{} to look into the past, you went to sleep. Overnight, you had the following dream://
  \emph{You see \iMirror{} in your mind. In it, you see your reflection. The mirror drifts close enough to you for your breath to fog the surface. Instead of fading, the mist begins to swirl slowly across the darkening surface of the mirror. You feel compelled to touch it. And in doing so, you fall forward into the mirror, tumbling into the vortex. Back through the swirling mist of time you are drawn.}

\emph{You find yourself in a fancy restaurant in the \pTech{}. Two figures are speaking quietly in a corner booth. \cEvil{} says ``I can buy you more time to figure out how to stop the storms. But you must do something for me in return.'' \cHeadScientist{} nods eagerly, and the scene fades as \cEvil{} passes \cHeadScientist{} a small satchel.}

\emph{A flash of \cEvil{} gesturing to a box of  bottles. \cEvil{\theyare} standing with someone dressed as a \pSchool{} employee, on the edge of the Lake beneath the \pSc{}. \cEvil{} says ``make sure only the students drink the wine. More bodies than that and people might find their way back to me.'' The person nods.}

\emph{Suddenly you’re in a forest clearing, somewhere in \pFarm{}. \cDiplomat{} is meeting with \cEvil{}. \cEvil{} says ``I have secured our ability to sway the vote at the Time of Deciding..'' \cDiplomat{} replies ``Good, now make sure this stays quiet.'' The grin \cEvil{} has as the scene fades sends chills down your spine.}

\emph{The \pSchool{}, the Ritual to Control the Storm just completed. The students stand or sit, panting. One student is clearly a \cHeir{\formal}. Another looks very much like \cAssistantScientist{} might have looked 6 years ago. The Teachers and Advisors rush in from the edges of the room to support the students. Someone you can’t see calls for a toast. The students adjourn to the Student Lounge alone. All of the students drink, except \cAssistantScientist{}’s lookalike. \cAssistantScientist{\They} sniff\cAssistantScientist{\verbs} the glass, make a dubious face, and put it down. As \cAssistantScientist{\their} classmates start to collapse, \cAssistantScientist{\they} slip\cAssistantScientist{\verbs} out a side door, unnoticed.}
  
The Dream reverberates with magic, and you know it to be the truth - it may not be the `textbf{whole} truth, but everything you saw happened. When you speak of what the dream showed you, if you speak truly, your voice will ring with the same magic, carrying the weight of truth to convince your listeners. \emph{(OOC: You should tell other players that your voice carries the weight of magical truth when you speak of what you’ve seen. Keep in mind however that a character may choose to pretend not to believe you for any number of reasons.}

If one or more \iStoneFlower{\MYname} (\iStoneFlower{\MYnumber}) were substituted in the ritual for other flowers, you also have a pounding headache for the next 30 minutes or so.}
}

\NEW{Sign}{\sSignG}{
  \s\MYname	{Sign G}
  \s\MYloc	{GM HQ}
  \s\MYtext	{You may not interact with this sign unless instructed to do so by a player or mechanic.}
	\s\MYgreens{\multi{15}{\gRitualDivineRealm{}}}
}

\NEW{Sign}{\sSignJOne}{
  \s\MYname	{Sign J}
  \s\MYloc	{GM HQ}
  \s\MYtext	{You may not interact with this sign unless instructed to do so by a mechanic.}
}

\NEW{Sign}{\sSignJTwo}{
  \s\MYname	{Sign J - 2 Day Dreaming of the Past}
  \s\MYloc	{GM HQ}
  \s\MYtext	{After successfully completing the ritual to use \iMirror{} to look into the past, you went about your day. Your daydream was more vivid than you usually experience.://
  \emph{You see \iMirror{} in your mind. In it, you see your reflection. The mirror drifts close enough to you for your breath to fog the surface. Instead of fading, the mist begins to swirl slowly across the darkening surface of the mirror. You feel compelled to touch it. And in doing so, you fall forward into the mirror, tumbling into the vortex. Back through the swirling mist of time you are drawn.}

Your vision clears, but you hear no sound.

\emph{You find yourself in a fairly classy bar in the \pFarm{}. A couple of lanky teenagers are hanging out, laughing and joking. A young woman in \cWildCard{\formal} colors says something to a young \cHedonist{\person} in \cHedonist{\formal} colors, and the laughter dies on every face immediately. The two square off - shouting quickly turns to blows. Their friends try to separate them, but the two storm off outside. No one follows them.}

\emph{The two figures continue to shout at each other as they move toward a copse of trees. Increasingly dangerous, aggressive spells are exchanged. The vision cuts to the woman in \cWildCard{\formal} colors, sprawled out lifeless on the forest floor, clothes splattered in blood. A flash of the young \cHedonist{\person} in \cHedonist{\formal} colors, wandering aimlessly down a country lane, \cHedonist{\their} eyes dull and unfocused, blood on \cHedonist{\their} hands.}

If one or more \iStoneFlower{\MYname} (\iStoneFlower{\MYnumber}) were substituted in the ritual for other flowers, you also have a pounding headache for the next 30 minutes or so.}
}


\NEW{Sign}{\sSignP}{
  \s\MYname	{Sign P}
  \s\MYloc	{GM HQ}
  \s\MYtext	{You may not interact with this sign unless instructed to do so by a player or mechanic.}
	\s\MYitems{\multi{5}{\iFastActingPoison{}}}
}

\NEW{Sign}{\sSignR}{%%For Supplying the black market; nothing starts at this sign.
  \s\MYname	{Sign R}
  \s\MYloc	{GM HQ}
  \s\MYtext	{You may not interact with this sign unless instructed to do so by a player or mechanic.}
}

\NEW{Sign}{\sSignT}{
  \s\MYname	{Sign T}
  \s\MYloc	{GM HQ}
  \s\MYtext	{If you have a mechanic that allows you to send letters off the island to someone, drop those letters here. Only people with a specific mechanic (greensheet or ability) have any hope of writing a letter that will be received and responded to in a timely manner. For your GMs sake, please don’t drop letters in here if you don’t have an appropriate ability.

The GMs will endeavor to have a response for you at the next meal that is $>$1 hour from when you submit your letter. If you send multiple letters there may be further delays. See the rules document for instructions on how you will receive return mail.

Letters submitted after 11 am on Sunday cannot reasonably expect a response before the end of the game.
	}
}

\NEW{Sign}{\sFlowerPower}{
  \s\MYname	{Flower Power}
  \s\MYloc	{GM HQ}
  \s\MYtext	{Do not interact with this sign unless a mechanic tells you to do so.}
	\s\MYitems	{\multi{8}{\iBlackCrocus{}\iFlameOrchid{}\iHollyhock{}\iLily{}\iMoonflower{}\iNightshade{}\iMorningGlory{}}}
}

\NEW{Sign}{\sCurses}{
  \s\MYname	{Curse Pick Up}
  \s\MYloc	{GM HQ}
  \s\MYtext	{If you have a Cursemaking Greesheet, you should see an envelope attached to this sign with your badge number on it. When your brew time is up for a curse, come check your envelope. If there is no curse waiting for you, or it is the wrong one, find a GM and we'll fix it ASAP.}
	\s\MYitems	{}
}

\NEW{Sign}{\sStormSurgeConsequences}{
  \s\MYname	{Storm Surge Consequences}
  \s\MYloc	{GM HQ}
  \s\MYtext	{If you rolled particularly poorly for a storm surge consequence, you were sent to this sign. The big envelope has the ``\gBadLuckCurse{}'' greensheet; the little envelope has the ``\iSlowActingPoison{}'' effects. Take a copy of the appropriate document based on what you rolled.
	
	\textbf{If your storm consequence is ``\gBadLuckCurse{}'', the release criteria is: Repair any and all damage to the bunkers.}}
	\s\MYitems	{\multi{6}{\iSlowActingPoison{}}}
	\s\MYgreens {\multi{6}{\gBadLuckCurse{}}}
}


%%%%%%%%% The Gardens %%%%%%%%%%

\NEW{Sign}{\sGardenGeneral}{
  \s\MYname	{The College Gardens}
  \s\MYloc	{Garden}
  \s\MYtext	{The botanical gardens of the College of the Gods are said to rival the finest estate gardens in The Children of the Sun. Rare flowers, trees, and shrubs from all the known corners of the world, and some few found nowhere else, thrive in the carefully tended gardens. Something always seems to be in bloom, and those versed in the subtle language of flowers can tell the time of year, even the time of day, by what is blooming. A lovely place for a stroll, a secret rendezvous, or simply to meditate in solitude. To explore the gardens and see what is blooming, spend one uninterrupted minute in front of this sign without taking any other actions (conversation is fine), then you may look through all of the items in the appropriate envelope, depending on the time of day. You may then either put the item back for others to enjoy, or pick \textbf{one} flower (definitely against College policy) and keep the item. You may not return to pick another flower for \textbf{5 minutes}.} It is possible to pick all of the available flowers and have no more available at this source.
}

\NEW{SignMedium}{\sGardenDay}{
  \s\MYname	{The College Gardens By Day}
  \s\MYloc	{Garden}
  \s\MYtext	{If you choose to explore the gardens during the day (7:00 am to 5:00 pm), you will pull one item from the envelope attached to this sign.}
  \s\MYitems {\multi{20}{\iMorningGlory{}\iHollyhock{}\iLily{}}}
}

\NEW{SignMedium}{\sGardenNight}{
  \s\MYname	{The College Gardens By Night}
  \s\MYloc	{Garden}
  \s\MYtext	{If you choose to explore the gardens during the night (5:00 pm to 7:00 am), you will pull one item from the envelope attached to this sign.}
  \s\MYitems {\multi{20}{\iNightshade{}\iMoonflower{}\iFlameOrchid{}}}
}

\NEW{Sign}{\sGardenFountain}{
  \s\MYname	{The Garden Fountain}
  \s\MYloc	{Garden}
  \s\MYtext	{A lively fountain of bright green malachite burbles peacefully in this quiet corner of the garden. Carved in the shape of vines and tendrils, the fountain blends seamlessly with the surrounding vegetation, the water flowing like rain over leaves. Live dragonflies flit around from shade to sun, and beautiful, silvery fish dart in and out of the protection of the lily pads.

If you wish to acquire a live fish, \textbf{spend 30 seconds here, then roll a D20.} These things are hard to catch; if you roll a \textbf{14 or higher} you may take a ``\iFish{}'' from the envelope below. If you roll lower than a 14, all you get is wet. You must wait at least \textbf{5 minutes} before you can try again. It is possible to catch all of the available fish and have no more available at this source.

\emph{OOC: If you need a source of freshwater for a mechanic, this fountain is an appropriate source. If you fill a container or vial with water from here, write on the item card ``filled with fresh water.’’}}
	\s\MYitems	{\multi{12}{\iFish{}}}
}

\NEW{Sign}{\sSignB}{
  \s\MYname	{Sign B}
  \s\MYloc	{Garden}
  \s\MYtext	{You may not interact with this sign unless directed to do so by a mechanic.
	}  
}

\NEW{Sign}{\sMagicPortal}{
  \s\MYname	{A Magic Portal}
  \s\MYloc	{Garden}
  \s\MYtext	{Before you is a massive growth of vines that was definitely not here on Friday Morning. The vines make a pattern as if they were shaped intentionally, growing up and around in a big circle, larger than a door, leaving an open space in the middle. But where you might expect to see the rest of the garden through this open space, instead there is an iridescent, shifting mist. This is clearly a magic portal.

Proximity with the portal for even a few moments leaves you with a sense of divinity. Could this be a portal to the Realm of the Gods? How was it created? The Gods themselves severed the connection between the Divine Realm and the mortal plane in time immemorial.

If it really is a portal to the Realm of the Gods, you could theoretically go talk to the patron deities! And who knows what else you might find? But navigating the Divine Realm is likely to be a challenge. It is possible that you will accomplish nothing but wandering lost in the mist for a while.

\textbf{You may pass through this portal to enter the Divine Realm. You may only stay for a maximum of 30 minutes, else you will become a denizen of the Divine Realm and be unable to return to the Mortal Realm.} You cannot be kept in the Divine Realm past this timer against your will. If you are knocked out or restrained, simply return to the Mortal Realm when your 30 timer is up. If you were knocked out, you remain knocked out, just on the Mortal side of the Portal. If you were restrained, you are now free since the person previously restraining you is presumably still in the Divine Realm. \textbf{You must remain on the Mortal Plane for at least 20 minutes before returning to the Divine Realm.}

To enter the Divine Realm, turn your badge to the ``Not Here’’ side, and use the ``out of game symbol’’ (hand on head or equivalent). Relocate to the ``\sMagicPortalTwo{}’’ sign in the Divine Realm. If you don’t know where that is, see a GM. When you arrive, return your badge to the side with your character name, and start your 30 minute timer.  	}  
}

\NEW{Sign}{\sPileofGardeningSupplies}{
  \s\MYname	{A Pile of Misc. Gardening Supplies}
  \s\MYloc	{Garden}
  \s\MYtext	{There is a cluttered pile of gardening supplies shoved in this corner. There are many small tools here, collections of seeds, and several types of fertilizer. There is even a small container of limestone bits to help counteract acidity building up in the soil.


\textbf{The tools are sharp enough to cut oneself }if one is not careful in handling them.

If you wish to take a ``\iLimestone{}'' from here, you may do so, but you must wait at least 5 minutes before retrieving a second piece. It is possible to take all of the available limestone and have no more available at this source.
	}  
	\s\MYitems	{\multi{15}{\iLimestone{}}}
}

\NEW{Sign}{\sSignM}{
  \s\MYname	{Sign M}
  \s\MYloc	{Garden}
  \s\MYtext	{You may not interact with this sign unless directed to do so by a player or a mechanic.}  
	\s\MYitems	{\iMemoryCure{}}
}

\NEW{SignMedium}{\sGardenBrickOne}{
  \s\MYname	{Glowing Blue Brick}
  \s\MYloc	{Garden}
  \s\MYtext	{\textbf{If you encounter a mechanic that prompts you to search for ``glowing bricks,'' you may lift this sign and read the one beneath it.}}
}

\NEW{SignSmall}{\sGardenBrickTwo}{
  \s\MYname	{Glowing Blue Brick}
  \s\MYloc	{Garden}
  \s\MYtext	{You find a glowing blue brick in the retaining wall. If you bring \textbf{5 advisors} to this location, you may retrieve whatever is in the envelope. Once you have succeeded in this task, remove this sign and envelope (but not the two signs on top of it) and discard them to the nearest stock.
	}
	\s\MYitems {\iEdgePiece{}}
}



%%%%%%%%% The Graveyard %%%%%%%%%%

\NEW{Sign}{\sGraveyardGeneral}{
  \s\MYname	{The College Graveyard}
  \s\MYloc	{Graveyard}
  \s\MYtext	{The small graveyard of the College of the Gods receives few visitors, and has even fewer graves. Most people who meet their demise on campus are returned to their home nations. Nonetheless, more than a few important figures from the College’s history, and that of the world as a whole, are buried here. The imposing, wrought iron gates are flanked by bronze statues of the avatars of the Gods - a scarab beetle, a sea serpent, and a giant hummingbird. Both statues and gates are badly corroded, perhaps due to the fog which seems to perpetually cover this place. Grand mausoleums and humble tombstones alike lie shrouded ahead, their dim silhouettes like ghosts in the mist. Walk among the graves if you dare, and see what the dead can teach.}  
}

\NEW{Sign}{\sFlowerBed}{
  \s\MYname	{An Overgrown Flowerbed}
  \s\MYloc	{Graveyard}
  \s\MYtext	{There is an overgrown flowerbed here. Clearly no gardener has cared for it in years. Still, you might find a few particularly hardy flowers managing to grow here anyway. It is possible to pick all of the available flowers and have no more available at this source.
		
	You may take \textbf{1 item at random} from the envelope attached to this sign. You must wait \textbf{5 minutes} before taking another.}  
  \s\MYitems {\iBlackCrocus{} \iNightshade{} \iHollyhock{} \iMoonflower{} \iMorningGlory{} \iFlameOrchid{} \iLily{}}
}

\NEW{Sign}{\sFadedTombstone}{
  \s\MYname	{A Faded Tombstone}
  \s\MYloc	{Graveyard}
  \s\MYtext	{This humble grave is clearly one of the oldest in the graveyard - overgrown with moss and weeds, its tombstone chipped, worn, and half sunken into the ground. Whatever inscription it once held is now illegible. The tombstone does not seem to match the style of any of the three nations. To search among the weeds around the grave, spend 1 uninterrupted minute in front of this sign without taking any other actions (conversation is fine), then pull one item from the envelope. It is possible to pick all of the available items and have no more available at this source.}   
  \s\MYitems {\multi{8}{\iStoneFlower{}}}
}

\NEW{Sign}{\sMemorialToFallenStudents}{
  \s\MYname	{Memorial to Fallen Students}
  \s\MYloc	{Graveyard}
  \s\MYtext	{A somber memorial stands here, carved from a single block of obsidian. The symbols of all three nations are carved into it, and the inscription reads as follows:

\begin{center}
\emph{In memory of the 12 students who perished on campus at the end of The Time of Deciding, year [6 years before the present]. May they be remembered not simply for their final act, but for the lives they lived, and might have lived, had tragedy not struck them down in their prime. These are the names of the fallen:}

\vspace{12pt}
\emph{Edon Faleden}\\
\emph{Chadwick Rexford}\\
\emph{Clarissa Reinhardt}\\
\emph{Jonathan Gloaming}\\
\emph{Henrietta Noonshadow}\\
\emph{Lily Dewsberry}\\
\emph{\cKidScientist{\full}}\\
\emph{River Dawning}\\
\emph{Diana of 8th Fleet, Osprey Archipelago}\\
\emph{Jethro of 1st Fleet, Guillemot’s Landing Harbor}\\
\emph{Luna of 4th Fleet, Sheerwater Ship}\\
\emph{Tiburon of 3rd Fleet, Black Albatross Ship}\\
\end{center}
}
}

\NEW{Sign}{\sTombOfFirstPrincipal}{
  \s\MYname	{Tomb of the First Principal}
  \s\MYloc	{Graveyard}
  \s\MYtext	{From a distance, you would not have guessed that this is the grave of so important a personage as the first Principal of the College of the Gods - a small, stone urn with few adornments and a simple tombstone engraved with the symbols of the four Deities. The inscription reads as follows:

\begin{center}
\emph{Here rests Katharina Friedrich, First Principal of the College of the Gods from the year [text too worn to be legible].}

\emph{Her steady hand guided the College through its earliest days and established the sacred traditions which govern this institution.}

\emph{Our First Principal both coined and lived by the College’s First Principle:}

\emph{“For all of Cengea.”}

\emph{She will be missed.}
\end{center}
	}  
}

\NEW{SignMedium}{\sGraveyardBrickOne}{
  \s\MYname	{Glowing Blue Brick}
  \s\MYloc	{Graveyard}
  \s\MYtext	{\textbf{If you encounter a mechanic that prompts you to search for ``glowing bricks,'' you may lift this sign and read the one beneath it.}
	}
}

\NEW{SignSmall}{\sGraveyardBrickTwo}{
  \s\MYname	{Glowing Blue Brick}
  \s\MYloc	{Graveyard}
  \s\MYtext	{You find a glowing blue brick in the side of the fireplace. If you bring \textbf{7 people, each from a different nation/cohort combination} (e.g you can’t have 2 students from the \pFarm{}), you may retrieve whatever is in the envelope. Once you have succeeded in this task, remove this sign and envelope (but not the two signs on top of it) and discard them to the nearest stock.
	}
	\s\MYitems {\iEdgePiece{}}
	}

%%%%%%%%% The Training Field %%%%%%%%%%

\NEW{Sign}{\sTrainingFieldGeneral}{
  \s\MYname	{The Training Field}
  \s\MYloc	{Training Field}
  \s\MYtext	{The training ground of the College of the Gods stretches before you, littered with the detritus of centuries of magical training, despite the best efforts of the groundskeepers. The wide, open fields are pockmarked with smoldering craters, bizarre plants, shattered fragments of magitech devices, persistent localized weather systems, and other oddities. Tread carefully - this place is decidedly unsafe. 
  
  To search among the detritus, spend one uninterrupted minute in front of this sign without taking any other actions (conversation is fine), then pull one item from the envelope \textbf{at random.} If you don't like what you found, you may put the item back. You must wait \textbf{5 minutes} before searching again. It is possible to retrieve all of the useful items scattered here and have no more available at this source.
}
  \s\MYitems {\multi{16}{\iClay{}}\multi{5}{\iEagleFeather{}\iGlassVial{}}\multi{2}{\iMagitechParts{}}\iTuningFork{}\iHolySymbol{}}
}

\NEW{Sign}{\sCombatCircle}{
  \s\MYname	{The Combat Circle}
  \s\MYloc	{Training Field}
  \s\MYtext	{At the center of the Training Ground stands the Combat Circle, a squat pillar of red marble looming above the surrounding landscape. Carved into the stone, a spiral staircase winds round its flanks before reaching the summit. It is here that students at the College of the Gods are trained in the art of magical combat and self defense. The Circle has seen everything from friendly sparring to grueling combat drills and winner-take-all duels; the College prides itself on the fact that no one has ever died within its bounds. The Circle’s pockmarked surface is covered with glowing magical sigils, powering the energy fields which prevent anyone swept from the summit from falling to their death.}
}

\NEW{Sign}{\sStorageShed}{
  \s\MYname	{A Weathered Storage Shed}
  \s\MYloc	{Training Field}
  \s\MYtext	{There is an old, beat up storage shed on the edge of the training field. The door is \textbf{locked}, but \textbf{any teacher} can unlock it with a touch. The door automatically locks behind you when you leave. (this door cannot be broken open; this is a kludge).

\textbf{If you would like to take something from the shed, you must gain access to the shed first.} Once the shed is unlocked, you may look through the items in the attached envelope and take 1 of your choosing. \textbf{You must wait 5 minutes before taking another item}, but each person has their own cool-down, so a second individual may also take an item immediately. 

If you leave the shed it locks behind you.

It is possible to retrieve all of the useful items here and have no more available at this source.
	}
	\s\MYitems{\multi{10}{\iWoodenPlank{}}\multi{5}{\iWoodenBlock{}}}
}

\NEW{SignMedium}{\sTrainingBrickOne}{
  \s\MYname	{Glowing Blue Brick}
  \s\MYloc	{Training Field}
  \s\MYtext	{\textbf{If you encounter a mechanic that prompts you to search for ``glowing bricks,'' you may lift this sign and read the one beneath it.}}
}

\NEW{SignSmall}{\sTrainingBrickTwo}{
  \s\MYname	{Glowing Blue Brick}
  \s\MYloc	{Training Field}
  \s\MYtext	{You find a glowing blue brick in the pillar holding up the combat circle. If you bring \textbf{5 citizens of the \pShip{}} to this location, you may retrieve whatever is in the envelope. Once you have succeeded in this task, remove this sign and envelope (but not the two signs on top of it) and discard them to the nearest stock.
	}
	\s\MYitems {\iEdgePiece{}}
}

%%%%%%%%% The Temple %%%%%%%%%%
\NEW{Sign}{\sTemple}{
  \s\MYname	{The College Temple}
  \s\MYloc	{The Temple}
  \s\MYtext	{This building is the Temple for the \pSchool{}. Services for all 3 patron deities are held here regularly, and symbology for all of the deities, minor and patron alike have been carefully and lovingly installed throughout this space.}
}

\NEW{Sign}{\sAppeal}{
  \s\MYname	{An Appeal for Protection}
  \s\MYloc	{The Temple}
  \s\MYtext	{There is a prayer carved into one of the stone grottos here. It is an appeal for protection from any and all of the deities of the Pantheon. It reads as follows:

\begin{center}
\emph{Deities beyond,}
\emph{Hear our appeal across the void between realms}
\emph{We walk the land you made (alt: We sail the seas you made)}
\emph{And face its dangers great and small.}

\emph{We know our power, and we know yours.}
\emph{We know our limits, and each of your realms}
\emph{Grant us your boons,}
\emph{That we might defend ourselves.}

\emph{We must be protected, in order to protect.}
\emph{In return, we protect others}
\emph{Until the realms are reunited}
\emph{And you can protect us yourselves.}
\end{center}
	}
}

\NEW{Sign}{\sDecorativeChestForStaff}{
  \s\MYname	{A Decorative Chest}
  \s\MYloc	{The Temple}
  \s\MYtext	{There is a beautiful decorative chest here. It is clearly made from wood sourced from all 3 nations. No doubt a \pShip{} craftsman created the elegant shape that was then turned over to \pTech{} artisans to carve and paint.

\textbf{This chest is meant to hold the ``\iOakStaff{}''. This item may be freely taken from this sign, or returned here. No other item(s) may be placed in the envelope for this sign.}
	}
	\s\MYitems	{\iOakStaff{}}
}

\NEW{Sign}{\sCandleCabinet}{
  \s\MYname	{Ornately Carved Wooden Cabinet}
  \s\MYloc	{The Temple}
  \s\MYtext	{There is a tall wooden cabinet here. The ornate carvings are of the patron deities, surrounded by a circle that represents \pEarth{}. The cabinet smells strongly of beeswax, and it holds the common ritual component of ritual candles. There are hundreds of candles in the cabinet - more than could possibly be used in one Time of Deciding. Still, it would be impolite to take more than you need.

If you would like to take a candle from this cabinet you may do so freely by taking one of the item cards from the envelope below. You must wait at least \textbf{2 minutes} between each candle you take.

The items available at this location are essentially infinite. If at any time there are no candles available, and you want one, tell a GM and we will make more.
	}
	\s\MYitems	{\multi{40}{\iRitualCandle{}}}
}

\NEW{SignMedium}{\sRedBrickNineUnder}{
  \s\MYname	{RED Glowing Brick Iota}
  \s\MYloc	{The Temple}
  \s\MYtext	{You may take a copy of the \textbf{Maze} from the \textbf{large} attached envelope. Once someone solves the puzzle, they may return to this sign, and retrieve the item from the \textbf{small} attached envelope. 
  
  \textbf{Please do check that you are returning to the correct ``glowing brick'' by matching the Greek letter: Iota from the puzzle to this sign.}
  
Draw a line from the start to the end of the maze. You may collaborate with as many or as few of your fellow players as you like. Players are encouraged to communicate to other collaborators involved in the ``glowing brick'' search that they are working on this puzzle. There are multiple copies of the puzzle, and it only needs to be solved once, by one person or one group of people. Cooperating on solving the same copy, working near this sign, or working near other characters working on this puzzle in parallel can help avoid a situation where someone else solves the puzzle first and returns here to retrieve the item, leaving you feeling like you wasted time solving this puzzle.
  
  \textbf{Once you have solved the puzzle and retrieved the item from the small envelope, remove both this sign and the one above it and discard to the nearest stock.}
  }
	\s\MYwhites	{\multi{5}{\wMazeIota{}}}
	\s\MYitems	{\iCornerPiece{}}
}

\NEW{SignMedium}{\sRedBrickTenUnder}{
  \s\MYname	{RED Glowing Brick Kappa}
  \s\MYloc	{The Temple}
  \s\MYtext	{You may take a copy of the \textbf{Word Search Puzzle} from the \textbf{large} attached envelope. Once someone solves the puzzle, they may return to this sign, and retrieve the item from the \textbf{small} attached envelope. 
  
  \textbf{Please do check that you are returning to the correct ``glowing brick'' by matching the Greek letter: Kappa from the puzzle to this sign.}
  
Draw a line from the start to the end of the maze. You may collaborate with as many or as few of your fellow players as you like. Players are encouraged to communicate to other collaborators involved in the ``glowing brick'' search that they are working on this puzzle. There are multiple copies of the puzzle, and it only needs to be solved once, by one person or one group of people. Cooperating on solving the same copy, working near this sign, or working near other characters working on this puzzle in parallel can help avoid a situation where someone else solves the puzzle first and returns here to retrieve the item, leaving you feeling like you wasted time solving this puzzle.
  
  \textbf{Once you have solved the puzzle and retrieved the item from the small envelope, remove both this sign and the one above it and discard to the nearest stock.}
  }
	\s\MYwhites	{\multi{5}{\wWordSearchKappa{}}}
	\s\MYitems	{\iCornerPiece{}}
}

\NEW{Sign}{\sSignEOne}{
  \s\MYname	{Sign E}
  \s\MYloc	{The Temple}
  \s\MYtext	{Only lift this sign to read the one underneath if instructed to do so by a mechanic.}
}

\NEW{Sign}{\sSignETwo}{
  \s\MYname	{Sign E-2: What Happened to the Ebb Avatar}
  \s\MYloc	{The Temple}
  \s\MYtext	{The Goddesses’ attention is on this investigation, and they manipulate the currents and the wind towards your success. Here at the \pSchool{}, with all of this mortal attention on the situation, the Goddesses are able to glean more, which they gladly share: 

The ``\iScythe{}’’ was the culprit’s weapon. It is one of the only ways to attack an Avatar and have a chance of success, and it is for exactly this reason that it is kept deep in the Library, where it is supposed to be protected.

It follows therefore that whoever killed the Ebb avatar must have had access to the library in the past few months. Which obligates the direct culprit to be a teacher or a student, although an Advisor could have been involved in orchestrating the matter.

The Goddesses must leave it up to you to determine who had the motivation. If you can find the culprit, you can bring them to justice. It won’t bring the avatar back on its own; in fact you could resurrect the Avatar without doing this, but who’s to say they won’t just try again if they are not found and stopped? And the Wave Rider guiding tenet of balance demands the culprit pay an equivalent price to the pain, suffering, and death that they caused.}
}

%%%%%%%%% The Library %%%%%%%%%%
\NEW{SignMedium}{\sObsidianGreenhouseEnvelope}{
  \s\MYname	{Envelope 1: The Obsidian Greenhouse}
  \s\MYloc	{The Library}
  \s\MYtext	{Only remove items from or put items in this envelope if instructed to do so.}
	\s\MYitems	{\multi{15}{\iObsidian{}\iStoneFlower{}}}
}

\NEW{SignMedium}{\sHallOfDrawersEnvelope}{
  \s\MYname	{Envelope 2: The Hall of Tiny Drawers}
  \s\MYloc	{The Library}
  \s\MYtext	{Only remove items from or put items in this envelope if instructed to do so.}
		\s\MYitems{\multi{5}{\iClay{}\iEagleFeather{}\iLimestone{}\iObsidian{}\iCharcoal{}\iSpiderWeb{}}iBrassNails{}}
}
	
\NEW{SignMedium}{\sCursedCodicesEnvelope}{
  \s\MYname	{Envelope 3: Cursed Codices}
  \s\MYloc	{The Library}
  \s\MYtext	{Only remove items from or put items in this envelope if instructed to do so.}
	\s\MYitems {\multi{6}{\iHonesty{}\iBlindness{}\iWeakness{}\iRestraint{}\iBabble{}\iCourage{}}}

}

\NEW{SignMedium}{\sDustyCaseEnvelope}{
  \s\MYname	{Envelope 4 - The Dusty Display Case}
  \s\MYloc	{The Library}
  \s\MYtext	{Only remove items from or put items in this envelope if instructed to do so.}
	\s\MYitems	{\multi{30}{\iCrystalLens{}}}
}

\NEW{SignMedium}{\sBlackCrocusEnvelope}{
  \s\MYname	{Envelope 5: The Meadow of Black Crocuses}
  \s\MYloc	{The Library}
  \s\MYtext	{Only remove items from or put items in this envelope if instructed to do so.}
	\s\MYitems {\multi{20}{\iBlackCrocus{}}}
}

\NEW{SignMedium}{\sJewelryBoxEnvelope}{
  \s\MYname	{Envelope 6 - The Jewelry Box}
  \s\MYloc	{The Library}
  \s\MYtext	{Only remove items from or put items in this envelope if instructed to do so.}
	\s\MYitems {\iDiamond{}}
}

\NEW{SignMedium}{\sReliquaryEnvelope}{
  \s\MYname	{Envelope 7: The Reliquary}
  \s\MYloc	{The Library}
  \s\MYtext	{Only remove items from or put items in this envelope if instructed to do so. \textbf{Even items which are normally “Unstashable”} can be placed in this envelope, but \textbf{only} if you find the corresponding location in the Library choose-your-own-adventure booklet.}
	\s\MYitems	{\iNet{}\iLariat{}\iScythe{}}
}

\NEW{SignMedium}{\sBlueBrickEightEnvelope}{
  \s\MYname	{Envelope 8: Glowing Blue Brick}
  \s\MYloc	{The Library}
  \s\MYtext	{Do not interact with this sign/envelope unless instructed to do so. 

You glimpse a comforting blue glow coming from an incongruous brick in the obsidian wall. If you bring \textbf{3} people from the \textbf{upper} echelons of society (e.g: \pFarm{} nobility, someone on the \pTech{} Council or with family that is, or from 1st-3rd Fleets of the \pShip{}) to this location, you may retrieve whatever is in the envelope. Once you have succeeded in this task, remove this sign/envelope and discard to the nearest stock.}
	\s\MYitems	{\iEdgePiece{}}
}

\NEW{SignMedium}{\sBlueBrickNineEnvelope}{
  \s\MYname	{Envelope 9: Glowing Blue Brick}
  \s\MYloc	{The Library}
  \s\MYtext	{Do not interact with this sign/envelope unless instructed to do so. 

You see a soft, blue light emanating from a spot in the field of black crocuses; buried amongst the flowers, you find a glowing blue brick. If you bring \textbf{4 citizens of the \pFarm{}} to this location, you may retrieve whatever is in the envelope. Once you have succeeded in this task, remove this sign/envelope and discard to the nearest stock.}
	\s\MYitems	{\iEdgePiece{}}
}

\NEW{SignMedium}{\sBlueBrickTenEnvelope}{
  \s\MYname	{Envelope 10: Glowing Blue Brick}
  \s\MYloc	{The Library}
  \s\MYtext	{Do not interact with this sign/envelope unless instructed to do so. 

You glimpse a soft, blue glow emanating from a brick in the wall. If you bring \textbf{4 students }to this location, you may retrieve whatever is in the envelope. Once you have succeeded in this task, remove this sign/envelope and discard to the nearest stock.}
	\s\MYitems	{\iEdgePiece{}}
}

\NEW{SignMedium}{\sBlueBrickElevenEnvelope}{
  \s\MYname	{Envelope 11: Glowing Blue Brick}
  \s\MYloc	{The Library}
  \s\MYtext	{Do not interact with this sign/envelope unless instructed to do so. 

You see a glowing blue brick in the wall. If you bring \textbf{4 people who either have an active Curse on them or are Cursemakers} to this location, you may retrieve whatever is in the envelope. Once you have succeeded in this task, remove this sign/envelope and discard to the nearest stock.}
	\s\MYitems	{\iEdgePiece{}}
}

\NEW{SignMedium}{\sBlueBrickTwelveEnvelope}{
  \s\MYname	{Envelope 12: Glowing Blue Brick}
  \s\MYloc	{The Library}
  \s\MYtext	{Do not interact with this sign/envelope unless instructed to do so. 

Incongruous with the ornate alabaster architecture of this room, you find a glowing blue brick in the wall. If you bring \textbf{3 ordained Clerics or initiates} to this location, you may retrieve whatever is in the envelope. Once you have succeeded in this task, remove this sign/envelope and discard to the nearest stock.}
	\s\MYitems	{\iEdgePiece{}}
}
\NEW{SignMedium}{\sResearchQuestions}{
  \s\MYname	{Envelope 13: Research Questions}
  \s\MYloc	{The Library}
  \s\MYtext	{Only remove items from or put items in this envelope if instructed to do so. This envelope is for submitting research questions in the Library and retrieving answers after sufficient time has elapsed. In order to submit a question, you will need to find the Catalog of Esoterica (page 46) in the Library.}
	\s\MYitems	{}
}

\NEW{SignMedium}{\sLostandFound}{
  \s\MYname	{Envelope 14: Lost and Found}
  \s\MYloc	{The Library}
  \s\MYtext	{Only remove items from or put items in this envelope if instructed to do so. If you wish to leave an item in the Library where you are not otherwise instructed to do so, put it in this envelope. \textbf{Even items which are normally “Unstashable”} can be placed in this envelope.}
	\s\MYitems	{\iMagitechParts{}}
}

%%%%%%%%% The Old Wing of the College %%%%%%%%%%

\NEW{Sign}{\sBlackCrocusWishOne}{
  \s\MYname	{A Wish for the Future}
  \s\MYloc	{The Old Wing of the College}
  \s\MYtext	{Unless you know otherwise, you must do the following in order to navigate the ruins and access this location:

Roll a d20 and add your group’s collective CR; you may also \textbf{consume} exactly one ``\iBlackCrocus{}'' to add +3 CR to your total. If the result is 20 or higher, you may lift this sign and read the one underneath. If the result is less than 20, you are unable to reach this location, and one person in your group is injured in the ruins, losing 1 CR token until the next meal; you may not attempt this roll again for five minutes.}
}

\NEW{Sign}{\sBlackCrocusWishTwo}{
  \s\MYname	{A Wish for the Future}
  \s\MYloc	{The Old Wing of the College}
  \s\MYtext	{There is a beautifully calligraphed banner here, large enough to be a tapestry. The entire piece is bordered in delicate black crocuses. The paper must surely have been magically protected, for it shows little sign of age or damage.  It must have once hung on the wall here. If you are careful, you can extract it from the rubble and clear it off to see what it says:

\begin{center}
\emph{From one international entity to another}

\emph{The \pSchool{} is truly a marvel}
\emph{Built by \pShip{} crafters to Federation plans, using Sun's materials.}

\emph{It is a triumph of what the nations can do}
\emph{When they choose to unite instead of fight.}
\emph{The safest haven imaginable.}

\emph{We earnestly hope to see this camaraderie grow}
\emph{Far beyond the original intent just to}
\emph{Allow us shelter from the storm}
\emph{to finally create one people of \pEarth{}.}
\end{center}
\vspace{5mm}

\textbf{If you encounter a mechanic that prompts you to search for ``glowing bricks,'' you may lift this sign and read the one beneath it.}
	}
}

\NEW{SignSmall}{\sBlackCrocusWishThree}{
  \s\MYname	{Glowing Blue Brick}
  \s\MYloc	{The Old Wing of the College}
  \s\MYtext	{You find a glowing blue brick in the wall. If you bring \textbf{4} people from the \textbf{lower} echelons of society (e.g: \pFarm{} commoner, NOT on the \pTech{} Council or with family that is, or from 8th-12th Fleets of the \pShip{}) to this location, you may retrieve whatever is in the envelope. Once you have succeeded in this task, remove this sign and envelope (but not the two signs on top of it) and discard them to the nearest stock.
	}
	\s\MYitems {\iEdgePiece{}}
}

\NEW{Sign}{\sAlchemyLabOne}{
  \s\MYname	{Alchemy Lab}
  \s\MYloc	{The Old Wing of the College}
  \s\MYtext	{Unless you know otherwise, you must do the following in order to navigate the ruins and access this location:

Roll a d20 and add your group’s collective CR; you may also \textbf{consume} exactly one ``\iGlassVial{}'' to add +3 CR to your total. If the result is 17 or higher, you may lift this sign and read the one underneath. If the result is less than 17, you are unable to reach this location, and one person in your group is injured in the ruins, losing 1 CR token until the next meal; you may not attempt this roll again for five minutes.}
}

\NEW{Sign}{\sAlchemyLabTwo}{
  \s\MYname	{Alchemy Lab}
  \s\MYloc	{The Old Wing of the College}
  \s\MYtext	{Judging by the amount of broken glass and overturned benches in this space, this must have been an alchemy lab, where cursemakers explored, invented, and honed their craft.

One corner of this space seems to have been cleaned and organized somewhat recently. At least there is less dust and debris over here. It kind of reminds you of the Maker’s Space.

If you wish to search the space for something useful, you may roll a D20. \textbf{Only on a 20} do you manage to find something useful in this mess. If you roll a 20, take an item at random from the attached envelope. \textbf{Regardless of whether you succeed or fail the roll, you must wait at least 5 minutes before trying again.} If you just rolled a 20, and wish not to take whatever you found, you may return it to the envelope immediately, but that is the only exception.

If you have a V-Score >0, you may store things here. Otherwise, you \textbf{can’t} put anything in the envelope. \textbf{Even if your V-Score >0,  you cannot store bulky items here, except for the item with item number \iAvatarRabbit{\MYnumber}.’’}
	}
	\s\MYitems {\iMagitechParts{}\iMagitechParts{}\iMagitechParts{}}
}

\NEW{Sign}{\sLeyLinesNexus}{
  \s\MYname	{Ley Lines Nexus}
  \s\MYloc	{The Old Wing of the College}
  \s\MYtext	{Unless you know otherwise, you must do the following in order to navigate the ruins and access this location:

Roll a d20 and add your group’s collective CR; if anyone in your group is carrying a Relic or the \iDiamond{}, add +3 CR to your total. If the result is 14 or higher, you may lift this sign and read the one underneath. If the result is less than 14, you are unable to reach this location, and one person in your group is injured in the ruins, losing 1 CR token until the next meal; you may not attempt this roll again for five minutes.}
}

\NEW{Sign}{\sLeyLinesSad}{
  \s\MYname	{Ley Lines Nexus: Enervated}
  \s\MYloc	{The Old Wing of the College}
  \s\MYtext	{All of \pEarth{}’s ley lines pass through the floating island upon which the school is built, from every corner of the land and sea, twisting through this space, to spread again across the sky, beyond the horizon in every direction.

This nexus is the place where the most ley lines cross, so many in fact that these impossibly gossamer threads are visible to the naked eye, without the need for any magical aid.

Over time, the ley lines become worn, fraying at the edges with the immensity of the magical energy they carry across all of \pEarth{}. These enervated ley lines glow only faintly, pulsing softly like the slow heartbeat of an enormous animal.

Long ago it was realized that the Time of Deciding puts tremendous strain on the ley lines, and so a renewing ritual was devised, that is the purview of the Principal of the \pSchool{} to enact before the end of each Time of Deciding.

\textbf{The ritual to renew the ley lines will request you remove this sign and the preceding one.}
	}
}

\NEW{Sign}{\sLeyLinesHappy}{
  \s\MYname	{Ley Lines Nexus: Renewed}
  \s\MYloc	{The Old Wing of the College}
  \s\MYtext	{All of \pEarth{}’s ley lines pass through the floating island upon which the school is built, from every corner of the land and sea, twisting through this space, to spread again across the sky, beyond the horizon in every direction.

This nexus is the place where the most ley lines cross, so many in fact that these impossibly gossamer threads are visible to the naked eye, without the need for any magical aid.

The ley lines have been recently renewed and glow brightly. It is now possible to conduct the Ritual to Reattune relics.
	}
}

\NEW{Sign}{\sObservatoryOne}{
  \s\MYname	{The Old Observatory}
  \s\MYloc	{The Old Wing of the College}
  \s\MYtext	{Unless you know otherwise, you must do the following in order to navigate the ruins and access this location:

Roll a d20 and add your group’s collective CR; you may also \textbf{consume} exactly one ``\iMoonflower{}'' to add +3 CR to your total. If the result is 14 or higher, you may lift this sign and read the one underneath. If the result is less than 14, you are unable to reach this location, and one person in your group is injured in the ruins, losing 1 CR token until the next meal; you may not attempt this roll again for five minutes.
	}
}

\NEW{Sign}{\sObservatoryTwo}{
  \s\MYname	{The Old Observatory}
  \s\MYloc	{The Old Wing of the College}
  \s\MYtext	{You climb and climb up the winding staircase until you finally emerge on a landing. The wind blows cold in your face. Where is the wall? Or the roof? It seems the old observatory has seen much better days.

On the other hand, with the roof gone, the space is easily accessible to birds, including eagles. You see the nest on the far side of the room. 

If you would like to take ``\iEagleFeather{}'' from here, you may freely take 1. You must wait at least 5 minutes before you can take another. It is possible to retrieve all of the useful items here and have no more available at this source.
	}
	\s\MYitems	{\multi{10}{\iEagleFeather{}}}
}

\NEW{Sign}{\sDioramas}{
  \s\MYname	{Dioramas of Native Flora and Fauna}
  \s\MYloc	{The Old Wing of the College}
  \s\MYtext	{You walk down a hallway, glass crunching under your feet. Along both walls are large alcoves like one might find at a museum. Almost everything was smashed to pieces when the roof caved in in this section, but you can still see bits and pieces that tell you that these alcoves must have contained little displays of the local flora and fauna from various regions around \pEarth{}, all done up as little statues, in intricate detail. It looks like most anything that was whole or useful has long since been taken.

If you wish to search the debris and rubble for anything useful, you may check the envelope below and take an item at random if any are present. If it is empty, there is nothing left to find.
	}
	\s\MYitems	{\iRabbitStatue{}}
}

\NEW{Sign}{\sOldPhilosophyOne}{
  \s\MYname	{The Old Philosophy Room}
  \s\MYloc	{The Old Wing of the College}
  \s\MYtext	{Like every other space in the Old Wing of the \pSc{}, the philosophy room was badly damaged. The intricate mosaic on the floor, depicting the creation story was shattered, and its loss is considered one of the major tragedies of that generation. Oddly enough, even though the manuscripts were originally rescued from this space, and taken to the New Philosophy Room in the library, they somehow always find their way back here. Eventually the librarians gave up and let the books and scrolls and essays settle where they wanted.

With the room destroyed however, the documents are scattered around in haphazard piles. After all, there are no shelves left unsmashed to house them. It would take a miracle to find anything useful here. But then those at the \pSchool{} during the Time of Deciding are often touched by the Deities, and far more likely than the average person to experience a miracle.

\textbf{If you encounter a mechanic that prompts you to search for ``glowing bricks,'' you may lift this sign and read the one beneath it.}
	}
}

\NEW{SignSmall}{\sOldPhilosophyTwo}{
  \s\MYname	{The Old Philosophy Room}
  \s\MYloc	{The Old Wing of the College}
  \s\MYtext	{You find a glowing blue brick in the wall. Bring 3 teachers to this location. Once everyone is present, you may retrieve whatever is in the envelope. Once you have succeeded in this task, remove this sign and envelope (but not the two signs on top of it) and discard them to the nearest stock.
	}
	\s\MYitems {\iEdgePiece{}}
}

\NEW{Sign}{\sPFlameOne}{
  \s\MYname	{Prometheus' Flame}
  \s\MYloc	{The Old Wing of the College}
  \s\MYtext	{Unless you know otherwise, you must do the following in order to navigate the ruins and access this location:

Roll a d20 and add your group’s collective CR; you may also \textbf{consume} exactly one ``\iFlameOrchid{}'' to add +3 CR to your total. If the result is 17 or higher, you may lift this sign and read the one underneath. If the result is less than 17, you are unable to reach this location, and one person in your group is injured in the ruins, losing 1 CR token until the next meal; you may not attempt this roll again for five minutes.}
}

\NEW{Sign}{\sPFlameTwo}{
  \s\MYname	{Prometheus' Flame}
  \s\MYloc	{The Old Wing of the College}
  \s\MYtext	{You pass the doorway into a stone chamber. In the center stands a brazier made of brass. Inside burns the eternal flame known as Prometheus’ Flame. The warm glow is gentle and inviting. It speaks of endless possibilities. People wishing to change the nature of something will often bring things here to heat over the flame. Even things that should burn (like paper) or shatter (like a glass vial) can be heated here to imbue them with the potential to change or invert their nature.
	
\textbf{If you encounter a mechanic that prompts you to search for ``glowing bricks,'' you may lift this sign and read the one beneath it.}
}
}

\NEW{SignSmall}{\sPFlameThree}{
  \s\MYname	{Prometheus' Flame}
  \s\MYloc	{The Old Wing of the College}
  \s\MYtext	{You find a glowing blue brick in the wall.  If you bring \textbf{5 citizens of the \pTech{}} to this location, you may retrieve whatever is in the envelope. Once you have succeeded in this task, remove this sign and envelope (but not the two signs on top of it) and discard them to the nearest stock.
	}
	\s\MYitems {\iEdgePiece{}}
}

\NEW{Sign}{\sOldWingIntro}{
  \s\MYname	{The Old Wing of the College}
  \s\MYloc	{The Old Wing of the College}
  \s\MYtext	{Here you find an abandoned and ruined building. It seems to have collapsed in some calamity. It is quite dangerous to wander around in, and is definitely off limits for students.
	
	If you wish to scoop up a spider web from this location, you may do so, but you must wait 5 minutes before you take another. It is possible to retrieve all of the useful items here and have no more available at this source.}
	\s\MYitems	{\multi{15}{\iSpiderWeb{}}}
}

%%%%%%%%% Makers Space %%%%%%%%%%
\NEW{Sign}{\sMakerSpaceGeneral}{
  \s\MYname	{The Maker Space}
  \s\MYloc	{Maker’s Space}
  \s\MYtext	{This is the maker’s space at the college. There are work benches scattered around, as well as many tools and supplies for building, crafting, cursemaking, and many forms of art are available here.

\textbf{There are many sharp tools here; you could cut yourself if you are not careful.}
	}
}

\NEW{Sign}{\sSignD}{
  \s\MYname	{Sign D}
  \s\MYloc	{Maker’s Space}
  \s\MYtext	{You may not interact with this sign unless directed to do so by a mechanic.}
	\s\MYitems	{\iProtypeMusicBox{}}
}

\NEW{Sign}{\sSignF}{
  \s\MYname	{Sign F}
  \s\MYloc	{Maker’s Space}
  \s\MYtext	{\emph{(OOC: This sign contains instructions for how to make the pedestals necessary to hold the relics for the Ritual to Control the Storm. Feel free to ignore this sign unless directed to it by a greensheet.)}
	
	If you wish to build the pedestals to hold the Relics for the Ritual to Control the Storm, follow these instructions:
\begin{enumerate}
	\item Find the following items around the College and bring them here:
	\begin{itemize}
		\item \iWoodenPlank{} x3
		\item \iFancyCloth{} x3
		\item \iBrassNails{} x3
	\end{itemize}
	\item Collect at least 3 people in this location.
	\item Roleplay building the pedestal for 10 minutes with the 3 people. If \cBunker{\full}, \cPirate{}, and/or \cChupAvenger{\full} assist with this task, it will only take 5 minutes.
	\item Take all 3 copies of the ``\iPedestalForRelic{}'' item from the envelope attached to this sign.
	\begin{itemize}
		\item Discard the item cards for the supplies to the nearest stock.
		\item See a GM for the Phys. Reps.
	\end{itemize}
\end{enumerate}
	
The pedestals should probably be placed in the Ritual Space after they are built. You may not build more than 3 pedestals, so don't lose the ones you make. 	}
	\s\MYitems	{\multi{3}{\iPedestalForRelic{}}}
}

\NEW{Sign}{\sSignH}{
  \s\MYname	{Sign H}
  \s\MYloc	{Maker's Space}
  \s\MYtext	{You may not interact with this sign unless directed to do so by a player or mechanic.}
	\s\MYitems	{\iToyBoat{}}
}

\NEW{Sign}{\sSignW}{
  \s\MYname	{Sign W}
  \s\MYloc	{Maker's Space}
  \s\MYtext	{You may not interact with this sign unless directed to do so by a player or mechanic.}
	\s\MYitems	{\multi{5}{\iVidCom{}}}
}

\NEW{Sign}{\sVialCabinet}{
  \s\MYname	{Glass Vial Storage}
  \s\MYloc	{Maker's Space}
  \s\MYtext	{There are shelves upon shelves of small glass vials, easy for anyone to grab, since they are so useful in so many things.
	
If you would like to take a glass vial from the shelves, you may do so freely by taking one of the item cards from the envelope below. You must wait at least \textbf{2 minutes} between each vial you take.

The items available at this location are essentially infinite. If at any time there are no vials available, and you want one, tell a GM and we will make more.}
	\s\MYitems	{\multi{26}{\iGlassVial{}}}
}

\NEW{Sign}{\sStove}{
  \s\MYname	{Stove and Furnace}
  \s\MYloc	{Maker's Space}
  \s\MYtext	{A set of stoves for brewing Curses and a large smith’s furnace for melting down metals so they can be worked.}
}

\NEW{SignMedium}{\sRedBrickElevenUnder}{
  \s\MYname	{RED Glowing Brick Lambda}
  \s\MYloc	{Maker's Space}
  \s\MYtext	{You may take a copy of the \textbf{Logic Puzzle} from the \textbf{large} attached envelope. Once someone solves the puzzle, they may return to this sign, and retrieve the item from the \textbf{small} attached envelope. 
  
  \textbf{Please do check that you are returning to the correct ``glowing brick'' by matching the Greek letter: Lambda from the puzzle to this sign.}
  
You may ask fellow players, the GMs, or consult the internet for the standard rules for this type of puzzle. You may collaborate with as many or as few of your fellow players as you like. Players are encouraged to communicate to other collaborators involved in the ``glowing brick'' search that they are working on this puzzle. There are multiple copies of the puzzle, and it only needs to be solved once, by one person or one group of people. Cooperating on solving the same copy, working near this sign, or working near other characters working on this puzzle in parallel can help avoid a situation where someone else solves the puzzle first and returns here to retrieve the item, leaving you feeling like you wasted time solving this puzzle.
  
  \textbf{Once you have solved the puzzle and retrieved the item from the small envelope, remove both this sign and the one above it and discard to the nearest stock.}
  }
	\s\MYwhites	{\multi{5}{\wLogicLambda{}}}
	\s\MYitems	{\iCornerPiece{}}
}

\NEW{SignMedium}{\sRedBrickTwelveUnder}{
  \s\MYname	{RED Glowing Brick Mu}
  \s\MYloc	{Maker's Space}
  \s\MYtext	{You may take a copy of the \textbf{Trivia} from the \textbf{large} attached envelope. Check with GM to see if you have the correct answers. Once someone solves the puzzle, they may return to this sign, and retrieve the item from the \textbf{small} attached envelope. 
  
  \textbf{Please do check that you are returning to the correct ``glowing brick'' by matching the Greek letter: Mu from the puzzle to this sign.}
  
You may collaborate with as many or as few of your fellow players as you like. Players are encouraged to communicate to other collaborators involved in the ``glowing brick'' search that they are working on this puzzle. There are multiple copies of the puzzle, and it only needs to be solved once, by one person or one group of people. Cooperating on solving the same copy, working near this sign, or working near other characters working on this puzzle in parallel can help avoid a situation where someone else solves the puzzle first and returns here to retrieve the item, leaving you feeling like you wasted time solving this puzzle.
  
  \textbf{Once you have solved the puzzle and retrieved the item from the small envelope, remove both this sign and the one above it and discard to the nearest stock.}
  }
	\s\MYwhites	{\multi{5}{\wTriviaMu{}}}
	\s\MYitems	{\iCornerPiece{}}
}

%%%%%%%%% Great Hall %%%%%%%%%%
\NEW{Sign}{\sBlankPaper}{
  \s\MYname	{Blank Paper}
  \s\MYloc	{The Great Hall}
  \s\MYtext	{This sign holds blank paper for people who need scratch paper for something. Anyone may freely take paper from this location. 

If the envelope attached to this sign is ever empty, let a GM know and we will add more.

\emph{OOC: This sign literally only exists so that 1) the paper doesn’t blow away, and 2) the GMs remember to set this up.}
	}
}

\NEW{Sign}{\sCreationMythOfficial}{
  \s\MYname	{The Creation Myth}
  \s\MYloc	{The Great Hall}
  \s\MYtext	{This is the creation story of \pEarth{}, as agreed upon by the religious authorities from all three nations.

\emph{In the beginning, all of the Deities were equal in the nothingness. \cFarmGod{}, \cTechGod{}, and the twin \cEbb{\God}es \cEbb{} and \cFlow{} got together and decided, in their infinite wisdom, that what was needed was a world. And so they went to the other gods, and through persuasion and logic, trickery, bribery, and unexpected kindnesses, convinced the entire pantheon that they were correct. And so the Deities brought their power together and created \pEarth{}. Upon it they placed the single continent by the same name. This single land mass represents the cooperation of all the Gods, and the wonders that they could create when united.} 

\emph{When the primary work was done, a council was called among all the gods to determine who would create the race of beings most like the Deities themselves. The vote was unanimous, and perhaps not unexpected. Those who had originally brought the idea should create its crowning glory. And so \cFarmGod{}, \cTechGod{}, and the twin \cEbb{\God}es \cEbb{} and \cFlow{} set about creating humanity, each in their own way. The four Deities pooled their power together for a gift to humanity — the gift of Magic. This gift would help them weather the Magical Storms that ravaged \pEarth{} every three years. And so upon \pEarth{} there walk 3 peoples, and their Patron Deities rule the Pantheon.}
	}	
}

\NEW{Sign}{\sCreationMythSpeculated}{
  \s\MYname	{The Creation Myth Rewritten?}
  \s\MYloc	{The Great Hall}
  \s\MYtext	{This is an alternate creation story, proposed out of diligent research from a few of those present at the Time of Deciding:

\emph{In the beginning, all of the Deities were equal in the nothingness. A Deity whose name is now lost to time conceived of something new: a world — tangible and real, to be created by the Deities as a distraction from the endless existence. This Deity had a power none of the others had — the ability to make something out of nothing. The other Deities were quick to embrace this new idea, so much so that they began fighting over who would oversee the creation and control the new world. Conflicts became more and more violent. Factions formed, the strongest of which was the four Deities who would eventually become the Patron Gods. In an act that would reverberate through the rest of time, the Patron Gods struck a mortal wound upon the Deity who had originally conceived the idea of a world. With their dying breath, this Deity begged the other Gods to set aside their quarrels and be content in each other.} 

\emph{Never before had the Pantheon needed to face the loss of one of their number. Each Deity processed the loss in their own way. \cFarmGod{} spiraled into depression. \cTechGod{} succumbed to anger. The Twin Goddesses \cEbb{} and \cFlow{} turned to denial. As a whole, the Pantheon heeded the dying words of the Deity who had so inspired them, and for a time, there was peace. They all worked together to create the world now known as Cengea out of the corpse of the Deity who had conceived it, as a way to honor their memory.}

\emph{But the fragile peace was not to last. When the question arose of peopling the world with sentient beings, bitter arguments broke out. Each Deity wanted to create humanity in their own image. Arguments soon turned to violence once more; so intense was the struggle that the very land buckled, creating the mountain range known as the Bones of the World. In the end, the Patron Gods were triumphant once more, but they had learned their lesson regarding murder. Instead, they imprisoned the other Deities in the formless void that still existed outside of creation, the Underworld, placing the gates deep within a floating island that none could reach. Thus imprisoned, the Deities who would come to be known as the Minor Gods had their power vastly curtailed.}

\emph{The Patron Gods then set about the work of creating humanity, each Deity fashioning one of the nations in their own image. When their work was complete, the very first rule they established was a ban on murder, hoping to pass down to their creations the painful lesson they had learned. Humanity flourished under the watchful eyes of the Patron Gods, for a time. But over the years, the rage and anguish of the Minor Gods at being imprisoned festered and grew behind the Gates of the Underworld, involuntarily manifesting as a devastating Storm. When at last the gates could contain the Storm no longer, it burst outward in all directions, ravaging the entire continent — a process which would repeat every three years.} 

\emph{The Patron Gods were horrified at what their actions had caused, but feared and mistrusted the Minor Gods too much to set them free and end the Storms. Instead, they gifted humanity with magic to help them survive the devastation. They created the Ley Line Nexus on the floating island near the Gates of the Underworld, siphoning magical energy off from the imprisoned Minor Gods and offering it up to humanity. In time, humans discovered the floating island, and that their magical rituals were strongest near the Ley Line Nexus. They developed a ritual that could be performed there to guide the Storm and limit the extent of the devastation, and built a school to raise up young mages powerful enough to perform the ritual. The Gates of the Underworld, and the terrible secret they contained, remained hidden in the depths of the island, humanity none the wiser...until now.}


	}	
}

\NEW{Sign}{\sFireplace}{
  \s\MYname	{The Grand Fireplace}
  \s\MYloc	{The Great Hall}
  \s\MYtext	{The centerpiece of the Great hall is this massive fireplace that reaches all the way to the rafters. It was constructed of stone brought from the Bones of the World by \pShip{} crafters. It takes almost 12 hours to heat the fireplace completely, but once it is heated up, it will radiate heat for at least 3 days after any fire inside has been extinguished. The fire is traditionally extinguished at the end of the Time of Deciding, and the hearth is allowed to cool for 9 days, the ebbing heat symbolic of the destruction of the Storm.

If you wish to take a piece of charcoal from the hearth, you may do so, but you must \textbf{wait at least 5 minutes} before retrieving a second piece. It is possible to retrieve all of the useful items here and have no more available at this source.
	}
	\s\MYitems	{\multi{15}{\iCharcoal{}}}
}

\NEW{Sign}{\sTeachersOath}{
  \s\MYname	{The Teacher’s Oath}
  \s\MYloc	{The Great Hall}
  \s\MYtext	{There is a large wood and metal plaque here, hung on the wall. It displays ``The Teacher’s Oath.’’ The first part of the oath was penned by the school’s first principal. It has since been recited during every ``Time of Deciding’’ since they began. It is also often employed by the current principal of the school when hiring a new teacher.

The oath reads:
\begin{center}
\emph{I set aside my national duties,}
\emph{And take up the role of teacher.}
\emph{Here at the \pSchool{},}
\emph{I teach all who pass the doors.}
\emph{That \pEarth{} may be a better place,}
\emph{When my students take the lead.}

\emph{At the center of the brewing storm,}
\emph{I stand as a beacon and an anchor.}
\emph{I support and guide my students,}
\emph{Neutrality, the source of my power.}
\emph{My duty is to empower those,}
\emph{Upon whom this burden falls.}
\end{center}

\emph{(OOC NOTE: Below are the instructions for how to use this Oath can be used as a mechanic to support the Preparations for the Ritual to Control the Storm (greensheet). If you don’t know what that is, don’t worry about this and feel free to ignore it.)}

To complete the portion of the ``Preparations  for the Ritual to Control the Storm’’, do the following:
\begin{enumerate}
	\item A group of at least 6 teachers must be gathered together (\cPrincipal{} counts as a teacher) for this recitation.
	\item One Teacher must be designated the leader.
	\item The leader will read the first phrase of the oath out loud to the group. (Phrases are seperated by commas or periods.)
	\item The remaining teachers should recite the line back.
	\item The group must proceed through the whole oath via this ``call and return.''
	\item If a GM was not present for this recitation, make sure to tell a GM when this task was completed.
\end{enumerate}
 	}
}

\NEW{Sign}{\sTreatyTemplates}{
  \s\MYname	{Treaty Templates}
  \s\MYloc	{The Great Hall}
  \s\MYtext	{This sign holds treaty templates for people with the ``Writing and Ratifying Treaties'' greensheet. If you have this greensheet, you may freely grab templates from this sign.

If the envelope attached to this sign is ever empty, let a GM know and we will print you more templates.
	}
	\s\MYwhites	{\multi{10}{\wTreatyTemplate{}}}
}

\NEW{SignMedium}{\sRedBrickSevenUnder}{
  \s\MYname	{RED Glowing Brick Eta}
  \s\MYloc	{The Great Hall}
  \s\MYtext	{You may take a copy of the \textbf{Word Unscramble puzzle} from the \textbf{large} attached envelope. Once someone solves the puzzle, they may return to this sign, and retrieve the item from the \textbf{small} attached envelope. 
  
  \textbf{Please do check that you are returning to the correct ``glowing brick'' by matching the Greek letter: Eta from the puzzle to this sign.}
  
Use each clue to help unscramble the letters to form a word. You may collaborate with as many or as few of your fellow players as you like. Players are encouraged to communicate to other collaborators involved in the ``glowing brick'' search that they are working on this puzzle. There are multiple copies of the puzzle, and it only needs to be solved once, by one person or one group of people. Cooperating on solving the same copy, working near this sign, or working near other characters working on this puzzle in parallel can help avoid a situation where someone else solves the puzzle first and returns here to retrieve the item, leaving you feeling like you wasted time solving this puzzle.
  
  \textbf{Once you have solved the puzzle and retrieved the item from the small envelope, remove both this sign and the one above it and discard to the nearest stock.}
  }
	\s\MYwhites	{\multi{5}{\wWordScrambleEta{}}}
	\s\MYitems	{\iCornerPiece{}}
}

\NEW{SignMedium}{\sGHBrickOne}{
  \s\MYname	{Glowing Blue Brick}
  \s\MYloc	{Great Hall}
  \s\MYtext	{\textbf{If you encounter a mechanic that prompts you to search for ``glowing bricks,'' you may lift this sign and read the one beneath it.}}
}

\NEW{SignSmall}{\sGHBrickTwo}{
  \s\MYname	{Glowing Blue Brick}
  \s\MYloc	{Great Hall}
  \s\MYtext	{You find a glowing blue brick in the side of the fireplace. If you bring enough people to have a total L-Score = \textbf{6}, you may retrieve whatever is in the envelope. Once you have succeeded in this task, remove this sign and envelope (but not the two signs on top of it) and discard them to the nearest stock.
	}
	\s\MYitems {\iEdgePiece{}}
}


%%%%%%%%% Ritual Space %%%%%%%%%%
\NEW{Sign}{\sCleaningInstructions}{
  \s\MYname	{OOC Instructions: Cleaning the Ritual Space}
  \s\MYloc	{Ritual Space}
  \s\MYtext	{\emph{(OOC NOTE: These are the instructions for how to clean the ritual space. This mechanic is associated with the Preparations for the Ritual to Control the Storm (greensheet). If you don’t know what that is, don’t worry about this and feel free to ignore it.)}

Around the edges of the space, you will find \textbf{5 receptacles} (envelopes with signs listing colors on them.) The pieces of construction paper currently in this envelope will need to be moved, one by one, to the receptacle that matches its colors (e.g. red pieces of paper goes in the ``red'' receptacle).

In order to begin the task, someone must take the envelope full of pieces of construction paper up to the balcony above the lounge, and scatter all of the paper pieces from there. They will end up all over the first floor area of the lounge. Any pieces of paper that land on the second floor balcony should be dropped over the edge until they end up on the first floor somewhere. The person who did this may then come down and join in the task of organizing all of the pieces of paper.

There are constraints on how this must be done:
\begin{enumerate}
	\item Only 1 piece of paper may be picked up from the ground at a time. Once one piece of paper is picked up from the floor by someone, it must be placed in its receptacle envelope before another one can be picked up.
	\item Pieces of paper may only be handed from one character to another. Paper may not be tossed at any time or placed in any intermediate carrying containers. This means forming passing chains.
	\item Once you have touched a piece of paper that has been picked up from the floor, you may not change your spatial positioning (e.g. cannot walk to a new position) until it has been deposited into its receptacle. This includes touching through cloth (aka wearing gloves does not avoid this part of the mechanic.) This means that you may need more than just a few people to complete this mechanic if papers are scattered far away from their target receptacles.
	\begin{itemize}
		\item If a piece of paper gets dropped accidentally, you can go retrieve it, just come back to the place you were.
	\end{itemize}
	\item Once pieces of paper reach their target receptacle, they cannot be removed. In other words, this task cannot be undone once it is complete.
\end{enumerate}

\textbf{This task needs to be completed all in one go. Once you scatter the pieces of paper, you have begun the task and must finish it or give up!} Tell \cPrincipal{} when you think you are done. 
	}
}

\NEW{SignMedium}{\sCleaningReceptacleOrange}{
  \s\MYname	{Cleaning the Ritual Space: Orange Receptacle}
  \s\MYloc	{Ritual Space}
  \s\MYtext	{Orange construction paper pieces go here. See the ``\sCleaningInstructions{}'' sign nearby for instructions.}
	}

\NEW{SignMedium}{\sCleaningReceptacleYellow}{
  \s\MYname	{Cleaning the Ritual Space: Yellow Receptacle}
  \s\MYloc	{Ritual Space}
  \s\MYtext	{Yellow construction paper pieces go here. See the ``\sCleaningInstructions{}'' sign nearby for instructions.}
}

\NEW{SignMedium}{\sCleaningReceptacleGreen}{
  \s\MYname	{Cleaning the Ritual Space: Green Receptacle}
  \s\MYloc	{Ritual Space}
  \s\MYtext	{Green construction paper pieces go here. See the ``\sCleaningInstructions{}'' sign nearby for instructions.}
}

\NEW{SignMedium}{\sCleaningReceptacleBlue}{
  \s\MYname	{Cleaning the Ritual Space: Blue Receptacle}
  \s\MYloc	{Ritual Space}
  \s\MYtext	{Blue construction paper pieces go here. See the ``\sCleaningInstructions{}'' sign nearby for instructions.}
}

\NEW{SignMedium}{\sCleaningReceptaclePink}{
  \s\MYname	{Cleaning the Ritual Space: Pink Receptacle}
  \s\MYloc	{Ritual Space}
  \s\MYtext	{Pink construction paper pieces go here. See the ``\sCleaningInstructions{}'' sign nearby for instructions.}
}

\NEW{Sign}{\sRitualCircle}{
  \s\MYname	{The Ritual Circle}
  \s\MYloc	{Ritual Space}
  \s\MYtext	{	\emph{(OOC NOTE: These are the instructions for how to set up the ritual circle inside the ritual space. This mechanic is associated with the Preparations for the Ritual to Control the Storm (greensheet). If you don’t know what that is, don’t worry about this and feel free to ignore it.)}

Here in the middle of the Ritual Space is a massive, smooth piece of black stone. The area is large enough for everyone present at the Time of Deciding to gather around it, or stand upon it, without being overly crowded.

At each Time of Deciding, the Ritual Circle must be renewed using all \textbf{6 colors} of ``Ritual Chalk'' (phys rep: 6 colors of masking tape, \cPrincipal{\full} should have these). Any character may provide input and suggestions, but the \textbf{teachers} have the final say over the design that the ritual circle should take this year.

The ritual circle must be laid down in full before \textbf{noon on Sunday.}  

\textbf{Once any piece of the design has been laid down, it cannot be removed in an act of destruction, until after the Ritual to Control the Storm has been completed. Pieces may be moved for cosmetic adjustment (e.g. you realize these two lines aren’t as parallel as you meant them to be), or because a mechanic tells you to, but the circle cannot be destroyed once its creation has begun. The magical resonance of the circle prevents it.}
	}}

\NEW{Sign}{\sStormSeed}{
  \s\MYname	{The Storm Seed}
  \s\MYloc	{Ritual Space}
  \s\MYtext	{\emph{(OOC NOTE: These are the instructions for how to fill up the magical Tass Vessel “The Storm Seed”. This mechanic is associated with the Preparations for the Ritual to Control the Storm (greensheet). If you don’t know what that is, don’t worry about this and feel free to ignore it.)}

This VESSEL (phys rep: glass jar) is an exceptionally powerful item of Tass that can hold up to 30 units of magical energy. It must have taken more than a dozen master craftsman decades to create and install it. The vessel is seamlessly and permanently integrated into the stone floor here, making it immovable. \emph{\textbf{(OOC NOTE: This vessel cannot be moved or stolen.)}} It will only be moveable if needed during the Ritual to Control the Storm.

A total of \textbf{30 units of magical energy} must be voluntarily deposited into the vessel, which will serve as the seed for the ritual to control the storm. Any character may contribute, any amount that they have available, and may contribute more later if they have more available at a later time.

If you choose to deposit 1 unit of magical energy into this vessel, take one of your CR stones and put it in the jar. Your CR is now reduced by 1 temporarily. At the next meal, you may take additional stones from the stock to replenish up to your normal CR.

Once a unit of magical energy is put into the jar, it cannot be retrieved unless you must have a mechanic to extract and use Tass. Players may always temporarily empty the jar temporarily to count how many units of magical energy are in it.
	}
}

\NEW{SignMedium}{\sRedBrickEightUnder}{
  \s\MYname	{RED Glowing Brick Theta}
  \s\MYloc	{Ritual Space}
  \s\MYtext	{You may take a copy of the \textbf{Cipher puzzle} from the \textbf{large} attached envelope. Once someone solves the puzzle, they may return to this sign, and retrieve the item from the \textbf{small} attached envelope. 
  
  \textbf{Please do check that you are returning to the correct ``glowing brick'' by matching the Greek letter: Theta from the puzzle to this sign.}
  
Solve the cipher to uncover a message. You may collaborate with as many or as few of your fellow players as you like. Players are encouraged to communicate to other collaborators involved in the ``glowing brick'' search that they are working on this puzzle. There are multiple copies of the puzzle, and it only needs to be solved once, by one person or one group of people. Cooperating on solving the same copy, working near this sign, or working near other characters working on this puzzle in parallel can help avoid a situation where someone else solves the puzzle first and returns here to retrieve the item, leaving you feeling like you wasted time solving this puzzle.
  
  \textbf{Once you have solved the puzzle and retrieved the item from the small envelope, remove both this sign and the one above it and discard to the nearest stock.}
  }
	\s\MYwhites	{\multi{5}{\wCipherTheta{}}}
	\s\MYitems	{\iCornerPiece{}}
}

%%%%%%%%% Student Lounge %%%%%%%%%%
\NEW{Sign}{\sStudentGeneral}{
  \s\MYname	{The Student's Lounge / Bunker 1}
  \s\MYloc	{Student Lounge / Bunker 1}
  \s\MYtext	{This is the student’s lounge. It is really cozy, but much of the furniture has seen better days. This area can be closed off as one of the 3 bunkers that protect those present at the school during the ``Storm Surges.''

There are several hallways branching off of this lounge that lead to the students’ rooms.}
}

\NEW{Sign}{\sStudentBookCaseOne}{
  \s\MYname	{A Bookcase}
  \s\MYloc	{Student Lounge / Bunker 1}
  \s\MYtext	{There are many bookcases in this room, filled with copies of the reference books most commonly needed by students working on their homework.

If you have an \textbf{S-score = 1}, you may lift this sign to read the one under it. Anyone who is with you, regardless of their current S-score will have an S-score=1 after reading the next sign.
	}
}

\NEW{Sign}{\sStudentBookCaseTwo}{
  \s\MYname	{A Hidden Compartment}
  \s\MYloc	{Student Lounge / Bunker 1}
  \s\MYtext	{\textbf{Set your S-score = 1. If you didn’t have an S-score before, you do now.}

Tucked in a corner of this bookcase is a hidden switch. You press it, causing a concealed door to open. Behind the door is a hidden compartment.

\textbf{Check the SMALL envelope here to see if anything is in the compartment.} You may freely take or remove things from this compartment, as long as all item(s) are not bulky. Unless you know otherwise, no bulky items may be stored here. Items beginning with item number ``51\ldots’’ are the exception. They may be stored here even if they are bulky.

\textbf{If} there is an item in the SMALL envelope beginning with Item Number ``51…’’, you may also pull the greensheet out of the BIG envelope below and read it. 

\textbf{Return the greensheet when you are done looking at it. It may not be taken away from this location.}
	}
	\s\MYgreens	{\gSeaSerpent{}}
	\s\MYitems	{\iBabySeaSerpent{}}
}


\NEW{Sign}{\sRedBrick}{
  \s\MYname	{A RED Glowing Brick}
  \s\MYloc	{}
  \s\MYtext	{\textbf{If you encounter a mechanic that prompts you to search for ``glowing bricks,'' you may lift this sign and read the one beneath it.}}
}

\NEW{SignMedium}{\sRedBrickOneUnder}{
  \s\MYname	{RED Glowing Brick Alpha}
  \s\MYloc	{Student Lounge / Bunker 1}
  \s\MYtext	{You may take a copy of the \textbf{sudoku puzzle} from the \textbf{large} attached envelope. Once someone solves the puzzle, they may return to this sign, and retrieve the item from the \textbf{small} attached envelope. 
  
  \textbf{Please do check that you are returning to the correct ``glowing brick'' by matching the Greek letter: Alpha from the puzzle to this sign.}
  
  You may ask fellow players, the GMs, or consult the internet for the standard rules for this type of puzzle. You may collaborate with as many or as few of your fellow players as you like. Players are encouraged to communicate to other collaborators involved in the ``glowing brick'' search that they are working on this puzzle. There are multiple copies of the puzzle, and it only needs to be solved once, by one person or one group of people. Cooperating on solving the same copy, working near this sign, or working near other characters working on this puzzle in parallel can help avoid a situation where someone else solves the puzzle first and returns here to retrieve the item, leaving you feeling like you wasted time solving this puzzle.
  
  \textbf{Once you have solved the puzzle and retrieved the item from the small envelope, remove both this sign and the one above it and discard to the nearest stock.}
  }
	\s\MYwhites	{\multi{5}{\wSudokuAlpha{}}}
	\s\MYitems	{\iCornerPiece{}}
}

\NEW{SignMedium}{\sRedBrickTwoUnder}{
  \s\MYname	{RED Glowing Brick Beta}
  \s\MYloc	{Student Lounge / Bunker 1}
  \s\MYtext	{You may take a copy of the \textbf{nonogram puzzle} from the \textbf{large} attached envelope. Once someone solves the puzzle, they may return to this sign, and retrieve the item from the \textbf{small} attached envelope. 
  
  \textbf{Please do check that you are returning to the correct ``glowing brick'' by matching the Greek letter: Beta from the puzzle to this sign.}
  
  You may ask fellow players, the GMs, or consult the internet for the standard rules for this type of puzzle. You may collaborate with as many or as few of your fellow players as you like. Players are encouraged to communicate to other collaborators involved in the ``glowing brick'' search that they are working on this puzzle. There are multiple copies of the puzzle, and it only needs to be solved once, by one person or one group of people. Cooperating on solving the same copy, working near this sign, or working near other characters working on this puzzle in parallel can help avoid a situation where someone else solves the puzzle first and returns here to retrieve the item, leaving you feeling like you wasted time solving this puzzle.
  
  \textbf{Once you have solved the puzzle and retrieved the item from the small envelope, remove both this sign and the one above it and discard to the nearest stock.}
  }
	\s\MYwhites	{\multi{5}{\wNonogramBeta{}}}
	\s\MYitems	{\iCornerPiece{}}
}



%%%%%%%%% Teacher Lounge %%%%%%%%%%
\NEW{Sign}{\sTeacherGeneral}{
  \s\MYname	{The Teacher's Lounge / Bunker 2}
  \s\MYloc	{Teacher Lounge / Bunker 2}
  \s\MYtext	{This is the teacher’s lounge. It is fancier than the student lounge. This area can be closed off as one of the 3 bunkers that protect those present at the school during the ``Storm Surges.''

There are several hallways branching off of this lounge that lead to the teachers’ offices, which are in turn connected to each teacher’s private chambers.
	}
}

\NEW{Sign}{\sLinenCloset}{
  \s\MYname	{A Linen Closet}
  \s\MYloc	{Teacher Lounge / Bunker 2}
  \s\MYtext	{There is a linen closet in the corner of the Teacher’s Lounge. Inside are many different cloths of various sizes and quality.

If you wish to take an item from the linen closet,  you may look through the items in the attached envelope and \textbf{take 1 of your choosing}. 

\textbf{You must wait 5 minutes before taking another item}.  It is possible to retrieve all of the useful items here and have no more available at this source.
	}
	\s\MYitems	{\multi{5}{\iFancyCloth{}}}
}

\NEW{SignMedium}{\sRedBrickThreeUnder}{
  \s\MYname	{RED Glowing Brick Gamma}
  \s\MYloc	{Teacher Lounge / Bunker 2}
  \s\MYtext	{You may take a copy of the \textbf{word search puzzle} from the \textbf{large} attached envelope. Once someone solves the puzzle, they may return to this sign, and retrieve the item from the \textbf{small} attached envelope. 
  
  \textbf{Please do check that you are returning to the correct ``glowing brick'' by matching the Greek letter: Gamma from the puzzle to this sign.}
  
  You may ask fellow players, the GMs, or consult the internet for the standard rules for this type of puzzle. You may collaborate with as many or as few of your fellow players as you like. Players are encouraged to communicate to other collaborators involved in the ``glowing brick'' search that they are working on this puzzle. There are multiple copies of the puzzle, and it only needs to be solved once, by one person or one group of people. Cooperating on solving the same copy, working near this sign, or working near other characters working on this puzzle in parallel can help avoid a situation where someone else solves the puzzle first and returns here to retrieve the item, leaving you feeling like you wasted time solving this puzzle.
  
  \textbf{Once you have solved the puzzle and retrieved the item from the small envelope, remove both this sign and the one above it and discard to the nearest stock.}
  }
	\s\MYwhites	{\multi{5}{\wWordSearchGamma{}}}
	\s\MYitems	{\iCornerPiece{}}
}

\NEW{SignMedium}{\sRedBrickFourUnder}{
  \s\MYname	{RED Glowing Brick Delta}
  \s\MYloc	{Teacher Lounge / Bunker 2}
  \s\MYtext	{You may take a copy of the \textbf{logic puzzle} from the \textbf{large} attached envelope. Once someone solves the puzzle, they may return to this sign, and retrieve the item from the \textbf{small} attached envelope. 
  
  \textbf{Please do check that you are returning to the correct ``glowing brick'' by matching the Greek letter: Delta from the puzzle to this sign.}
  
You may ask fellow players, the GMs, or consult the internet for the standard rules for this type of puzzle. You may collaborate with as many or as few of your fellow players as you like. Players are encouraged to communicate to other collaborators involved in the ``glowing brick'' search that they are working on this puzzle. There are multiple copies of the puzzle, and it only needs to be solved once, by one person or one group of people. Cooperating on solving the same copy, working near this sign, or working near other characters working on this puzzle in parallel can help avoid a situation where someone else solves the puzzle first and returns here to retrieve the item, leaving you feeling like you wasted time solving this puzzle.
  
  \textbf{Once you have solved the puzzle and retrieved the item from the small envelope, remove both this sign and the one above it and discard to the nearest stock.}
  }
	\s\MYwhites	{\multi{5}{\wLogicDelta{}}}
	\s\MYitems	{\iCornerPiece{}}
}




%%%%%%%%% Advisor Lounge %%%%%%%%%%
\NEW{Sign}{\sAdvisorGeneral}{
  \s\MYname	{The Advisor's Lounge / Bunker 3}
  \s\MYloc	{Advisor Lounge / Bunker 3}
  \s\MYtext	{This is the advisor’s lounge. The room is immaculate; used only rarely outside of the Time of Deciding when important dignitaries visit the \pSc{}. This area can be closed off as one of the 3 bunkers that protect those present at the school during the ``Storm Surges.''

There are several hallways branching off of this lounge that lead to the guest quarters for the advisors.
	}
}

\NEW{SignMedium}{\sRedBrickFiveUnder}{
  \s\MYname	{RED Glowing Brick Epsilon}
  \s\MYloc	{The Advisor's Lounge / Bunker 3}
  \s\MYtext	{You may take a copy of the \textbf{Crossword puzzle} from the \textbf{large} attached envelope. Once someone solves the puzzle, they may return to this sign, and retrieve the item from the \textbf{small} attached envelope. 
  
  \textbf{Please do check that you are returning to the correct ``glowing brick'' by matching the Greek letter: Epsilon from the puzzle to this sign.}
  
  You may ask fellow players, the GMs, or consult the internet for the standard rules for this type of puzzle. You may collaborate with as many or as few of your fellow players as you like. Players are encouraged to communicate to other collaborators involved in the ``glowing brick'' search that they are working on this puzzle. There are multiple copies of the puzzle, and it only needs to be solved once, by one person or one group of people. Cooperating on solving the same copy, working near this sign, or working near other characters working on this puzzle in parallel can help avoid a situation where someone else solves the puzzle first and returns here to retrieve the item, leaving you feeling like you wasted time solving this puzzle.
  
  \textbf{Once you have solved the puzzle and retrieved the item from the small envelope, remove both this sign and the one above it and discard to the nearest stock.}
  }
	\s\MYwhites	{\multi{5}{\wCrosswordEpsilon{}}}
	\s\MYitems	{\iCornerPiece{}}
}

\NEW{SignMedium}{\sRedBrickSixUnder}{
  \s\MYname	{RED Glowing Brick Zeta}
  \s\MYloc	{The Advisor's Lounge / Bunker 3}
  \s\MYtext	{You may take a copy of the \textbf{Word Ladder puzzle} from the \textbf{large} attached envelope. Once someone solves the puzzle, they may return to this sign, and retrieve the item from the \textbf{small} attached envelope. 
  
  \textbf{Please do check that you are returning to the correct ``glowing brick'' by matching the Greek letter: Zeta from the puzzle to this sign.}
  
  You may ask fellow players, the GMs, or consult the internet for the standard rules for this type of puzzle. You may collaborate with as many or as few of your fellow players as you like. Players are encouraged to communicate to other collaborators involved in the ``glowing brick'' search that they are working on this puzzle. There are multiple copies of the puzzle, and it only needs to be solved once, by one person or one group of people. Cooperating on solving the same copy, working near this sign, or working near other characters working on this puzzle in parallel can help avoid a situation where someone else solves the puzzle first and returns here to retrieve the item, leaving you feeling like you wasted time solving this puzzle.
  
  \textbf{Once you have solved the puzzle and retrieved the item from the small envelope, remove both this sign and the one above it and discard to the nearest stock.}
  }
	\s\MYwhites	{\multi{5}{\wWordLadderZeta{}}}
	\s\MYitems	{\iCornerPiece{}}
}





%%%%%%%%% The Divine Realm %%%%%%%%%%
\NEW{Sign}{\sDoNotEnter}{
  \s\MYname	{DO NOT ENTER}
  \s\MYloc	{GM SPACE}
  \s\MYtext	{Do not enter this space unless directed to do so by a GM, NPC, or a mechanic.}
  }

\NEW{Sign}{\sRoostingTree}{
  \s\MYname	{The Roosting Tree}
  \s\MYloc	{The Divine Realm}
  \s\MYtext	{Hummingbirds of all sizes, from the size of pillbox, to the size of horse-drawn wagons perch on this tree, rustling quietly. They glow softly with \cFarmGod{}’s light, and are clearly \cFarmGod{\their} avatars., You may touch the tree and theoretically the birds if they let you. 

You may converse with the birds, and learn the following things:
	\begin{itemize}
		\item \textbf{Why did the Hummingbirds stop visiting the mortal plane?} - \cFarmGod{} only sends the Hummingbirds to those who are worthy. \cFarmGod{} feels that the church has betrayed the \cFarmGod{} by not only murdering the Aethers 200 years ago, a direct act against \cFarmGod{}, but then continuing on the path of war, destruction and the abuse of those who the nobility and church should care for.
		\item \textbf{What conditions would need to be met for them to return?} - If the church repents for its actions against the Aethers, the war were ended, and the masterminds behind the murder of the children were brought justice, Then, and only then will the Hummingbirds be sent once again to the \pFarmers{} to guide, comfort, and encourage, as it was in the past.
	\end{itemize}
	
To \textbf{destroy} the hummingbirds, you would need to:
	\begin{itemize}
		\item Have the ``\iScythe{}'' in your possession, confirmed to be attuned to a location. Using the scythe to attack these Avatars will deatune it.
		\item Hit them with a CR 12+ attack.
	\end{itemize}
	
	If you succeed in destroying the hummingbirds, tear down this sign to reveal the one under it.


	}
}

\NEW{Sign}{\sEmptyTree}{
  \s\MYname	{A Giant Tree}
  \s\MYloc	{The Divine Realm}
  \s\MYtext	{There is a massive tree here. While inspecting the tree, you see iridescent feathers, of many sizes, scattered among the roots.

These are NOT interchangeable with ``\iEagleFeather{}''. You cannot collect the feathers here for any practical purpose.


	}
}

\NEW{Sign}{\sDeadGodCreation}{
	\s\MYname {The First Creation}
	\s\MYloc{The Divine Realm}
	\s\MYtext { A tall...mountain?  Waterfall?  Tree? Stands before you.  It seems as its form constantly shifting and moving, as if it wants to be many things but has lost or never had the ability to solidify.  In its ever changing form, you can find glints of vessels containing something swirling within them. You can attempt to grab one.

To attempt to grab one with a bit of magic and skill, roll a d20.  Each person makes this attempt alone; you cannot get help on this task.  Add your CR to the roll plus any bonuses you may have for being just a bit \textbf{lucky}. You must meet or beat a total of 15. 

\textbf{If you get a 15 or higher}: You grab a vessel containing swirling elements.  A small bit of primordial energy.  Take a \iVesselofCreation{} from the attached envelope.  You can only claim a \iVesselofCreation{} in this manner \textbf{once per game}. 

\textbf{If you get a 14 or lower}:  The shock of touching one of the vessels flings you back.  You and everyone else at this location \textbf{each} lose 1 CR token until the next meal.

If you have seen and read ``\sCreationMythSpeculated{}’’ you may lift this sign to read the one under it. This requirement applies separately to each person.
	}
	\s\MYitems	{\multi{30}{\iVesselofCreation{}}}
}

\NEW{Sign}{\sDeadGodCreationTwo}{
  \s\MYname	{The First Creation of the Creator God}
  \s\MYloc	{The Divine Realm}
  \s\MYtext	{You recognize this place as the first creation of the creator god.  There are vessels full of their energy which if you focus you can see a small feeling or memory, one of forgiveness and desire for change, a message from the creator god to all their kindred. 

You know if you offer something you created and dedicate it to the creator god, they may offer you a vessel.  You can create something to offer the creator god by consuming any item that can be considered a ``crafting material’’ (e.g. ``\iWoodenBlock{}’’, ``\iCharcoal{}’’, or ``\iLily{}’’) in the Maker Space (on the mortal plane). Declare out loud that you are dedicating this item to the creator god before returning to the Divine Realm.

Once you return to this location with your created item, you may open the envelope and take a \iVesselofCreation{}. \emph{Discard the created item to the nearest stock; it has been given to the memory of the creator god.}

You can only claim a \iVesselofCreation{} in this manner \textbf{once per game}.
	}
	\s\MYitems	{\multi{30}{\iVesselofCreation{}}}
}


  
\NEW{Sign}{\sAmbrosiaGlade}{
  \s\MYname	{The Ambrosia Glade}
  \s\MYloc	{The Divine Realm}
  \s\MYtext	{As you wander through the mist, you encounter a glade that, like so many things here, just seems to materialize out of the fog in front of you. The trees are strange, like nothing you’ve seen on \pEarth{}.

The glade imparts a sense of peace, which is quite at odds with the fierce beast that growls at your approach. The creature is fantastical in nature, and its massive paws bat you backwards. You cannot pass the creature.

\textbf{The beast can only be held at bay by someone who wields the \iScythe{}. Using the \iScythe{} for this will NOT de-attune it, as you are not actually attacking the guardian.}

If someone in your party is carrying the \iScythe{} and chooses to face the guardian, you may lift the sign and read the one under it.
	}
}

\NEW{Sign}{\sAmbrosiaSpring}{
  \s\MYname	{The Ambrosia Spring}
  \s\MYloc	{The Divine Realm}
  \s\MYtext	{Deeper into the glade, past the fierce guardian, you find a small, bubbling spring. You have found the Spring of Ambrosia, which grants immortality. 

You may freely drink from the pool formed by the spring. Think carefully, there is most definitely no way to reverse this, and immortality has its price.

If you do drink from the pool, you become immortal:
\begin{itemize}
	\item You no longer age.
	\item You are now able to resist any \textbf{knockout} attack, no matter the CR (does not affect ``upstage'' or ``restrain'').
	\item You will endure things that would normally kill someone (like falling from a great height). You are not fully immune to all damage to your physical form however, and it would likely be a very long, and very painful recovery process.
	\item If you have a curse upon you that would cause you to die in the future (e.g.: a slow acting poison), the Ambrosia does not cure you, but causes you to endure. You will not die, but you will limp along, a hairsbreadth from death, forever. It may not be the preferable alternative.
	\item This effect will not prevent you from being trapped in the Divine Realm if you travelled here bodily and exceed 30 minutes of time here.
\end{itemize}

	}
}

\NEW{Sign}{\sMagicPortalTwo}{
  \s\MYname	{A Magic Portal}
  \s\MYloc	{The Divine Realm}
  \s\MYtext	{Through the swirling sparkles of the portal, you can faintly make out the garden of the \pSchool{}. \textbf{If, and only if, you entered the Divine Realm through this portal,} you may freely return to the Mortal Realm from here, as long as you have not exceeded your 30 minute timer. If you entered the Divine Realm by another method, you must use that method to return instead.

\textbf{You may not return to the Divine Realm for 20 minutes after leaving via this portal.}

\textbf{If you have willingly remained in the Divine Realm beyond 30 minutes,} your body has absorbed too much Divine Magic, and you are unable to cross back over. You are effectively dead. Go to GM HQ and follow instructions on the ``\sMurdered{}’’ sign. 
	

	}
}

\NEW{Sign}{\sRiverLetheDivine}{
  \s\MYname	{The Cenote}
  \s\MYloc	{The Divine Realm}
  \s\MYtext	{A rough hewn stone staircase leads between the surface and the water of the cenote, dozens of meters down. Stepping close to the water, you feel a strange but unmistakable mental sensation - memories could drift out of your mind here, leaving only confusion or peace in their place. You realize with a shock that this water must be the end point of the legendary River Lethe which bears souls to the Divine Realm - the River of Forgetting.
	
	Each person may:
	\begin{enumerate}
		\item Do nothing and leave unchanged.
		\item Carefully scoop some water into a container (requires an Empty Vial). Water collected here will remove one significant memory when completely consumed - adjust the item card for the vial accordingly. This effect cannot be forced on another character against their will, as using it requires voluntary focus.
		\item Scoop a handful of water from the cenote and drink it. Each person who does this gives up one significant memory, removing it completely and permanently from their mind. 
	\end{enumerate}
	
	You may be able to do other things at this location. If so, a mechanic will call out this location specifically.
	

	}
}

\NEW{Sign}{\sCitrusGrove}{
  \s\MYname	{The Citrus Grove}
  \s\MYloc	{The Divine Realm}
  \s\MYtext	{You stand at the edge of an orchard of citrus trees. The neat rows of trees are planted in a perfectly straight lines, and are perfectly spaced apart. The air is crisp, fresh, and clean here. You feel a sense of peace and calm. 
	
	\textbf{Once per game} you may spend 30 seconds here among the trees to remove one ``psych lim'' (a special type of ability that represents a phobia, or a deep seated mental or emotional constraint; they will be labeled as such). Curses are not affected by being here.
	

	}
}

\NEW{Sign}{\sTheSkull}{
  \s\MYname	{The Skull}
  \s\MYloc	{The Divine Realm}
  \s\MYtext	{A giant rocky outcropping with multiple cave entrances rises from a shallow sea. Despite innumerable seasons of weathering, the shape looks\ldots{} unnatural. The longer you stare at it, the more it reminds you of a skull. A giant skull, more than 50 feet tall, but a skull never-the-less. You feel a deep sense of sorrow even from here on the shore. 
	
	If you wish to swim out to the outcropping you may, \textbf{however there will be permanent, negative consequences if you do so.} Anything in game that asks you to come to this location means this sign \textbf{only}, i.e. you are \textbf{not} required to swim to the outcropping. If you choose to venture to the outcropping anyway, lift this sign and read the one under it.
	

	}
}

\NEW{Sign}{\sTheSkullTwo}{
  \s\MYname	{The Skull}
  \s\MYloc	{The Divine Realm}
  \s\MYtext	{You swim out the the outcropping, and find a sea cave you can slip into. Inside, you step onto the narrow shore. Dim light filters into the cave through the nasal cavity and eye sockets high above. The feeling of sorrow is much heavier here than it was on the shore. So heavy in fact that you feel it seeping into your bones. You feel like something precious was lost here. Something that can never be gotten back. \textbf{You acquire the psych lim ``Never Again to Feel Joy.''} Grab a copy from the attached envelope to remind you. If the envelope is empty, let a GM know.
	

	}
	\s\MYabils{\multi{16}{\aSorrow{}}}
}

\NEW{Sign}{\sCrystalForest}{
  \s\MYname	{The Crystal Forest}
  \s\MYloc	{The Divine Realm}
  \s\MYtext	{You stand on the edge of a forest of crystal trees. The ground, the bark, the leaves, even the flowers and seeds are all made of transparent crystal. Million of facets wink at you from every direction. It seems almost as if light comes alive here, motes chasing each other. Laughter like tinkling crystal emanates from somewhere deeper in the forest. The sound too twines around the trees, curious and playful.
		
	If you wish to walk among the trees, you may do so, \textbf{however there will be permanent, negative consequences if you do so.} Anything in game that asks you to come to this location means this sign \textbf{only}, i.e. you are \textbf{not} required to venture into the forest. If you choose to venture into the forest anyway, lift this sign and read the one under it.
	
	
	}
}

\NEW{Sign}{\sCrystalForestTwo}{
  \s\MYname	{The Crystal Forest}
  \s\MYloc	{The Divine Realm}
  \s\MYtext	{You wander among the trees of the Crystal Forest. Each tree is a work of art more beautiful than the last. From one to the next you flit, your hair teased by the playful tunes of sound, your skin caressed by the soft touches of sunbeams. You are entranced, yet eventually the pure perfection begins to grow a niggling, uncomfortable sensation at the root of your skull. Nothing could ever compare to the beauty of this place and the perfection of it leaves you aching for more. A need that will never be satisfied. \textbf{You acquire the psych lim ``Never Again to See Beauty.''} Grab a copy from the attached envelope to remind you. If the envelope is empty, let a GM know.
	

	}
	\s\MYabils{\multi{16}{\aBeauty{}}}
}

\NEW{Sign}{\sGraveyardDivine}{
  \s\MYname	{The Graveyards}
  \s\MYloc	{The Divine Realm}
  \s\MYtext	{You stand upon the edge of a precipice. You look out into the void, and feel the call of the dead. Like iridescent images warped by the curvature of soap bubbles, you see first a freshly tilled field, then a magma chamber, then the night sky, dusted with stars. The images flicker and fade into each other, ephemeral as smoke on the wind. 
	
	From here you may call forth any dead NPC you wish to speak with. You must know them by name, and have some familial or friendly connection with the soul. Disturb souls from their rest at your own peril. Find a GM if you wish to summon a dead soul.
	

	}
}

\NEW{Sign}{\sCordOne}{
  \s\MYname	{The Homes of the Patron Gods}
  \s\MYloc	{The Divine Realm}
  \s\MYtext	{Here among the swirling, endless mist of the Divine Realm, you feel a stilling of energy. At first you cannot see farther than your outstretched hand, but if you and your group concentrate, you can call forth: a field of wheat, just tinged with gold; a workshop and forge, deep underground; or a rocky beach where the ocean spray crashes against the shore.
	
	From here you may summon one of the Patron Gods (\cFarmGod{}, \cTechGod{}, or \cEbb{} and \cFlow{}) and speak with them. See a GM if you wish to do this. This summoning is not without risk, but the Patron Deities can probably solve almost any problem if petitioned correctly. Find a GM if you wish to speak to one of the Patron Deities. Due to the immense power of the Patron Deities, only one may be summoned at a time. If you wish to speak to a different Patron Deity, you must wait until the current conversation is complete.
	
	\begin{center}
{\Large \textbf{Coordinates: 68x90x56x41}} 
\end{center}
You can attune a relic to this location by following any known reattunement ritual here. Someone who is participating in the ritual must be touching this sign at all times (you can switch around who is touching the sign).
	}
}

\NEW{Sign}{\sCordZero}{
  \s\MYname	{The Homes of the Minor Gods}
  \s\MYloc	{The Divine Realm}
  \s\MYtext	{Here among the swirling, endless mist of the Divine Realm, you feel a stilling of energy. At first you cannot see farther than your outstretched hand, but as you and your group look around, unsettling images begin to overlap each other: A gate of crackling lightning. A yawning chasm of infinite darkness. A swirling portal of baleful light and darkness. An infinitely tangled thicket of thorns and brambles.
	
	From here you may summon one of the Minor Gods and speak with them. See a GM if you wish to do this. You can try to summon a particular minor deity, but your success is not guaranteed, and you may find yourself speaking to someone else, or no one at all, if your call is not answered. If you wish to speak to a different minor deity, you must wait until the current conversation is complete.
	
	\begin{center}
{\Large \textbf{Coordinates: 34x07x50x96}} 
\end{center}
You can attune a relic to this location by following any known reattunement ritual here. Someone who is participating in the ritual must be touching this sign at all times (you can switch around who is touching the sign).
	}
}

%%%%%%%%% The Bunkers %%%%%%%%%%
\NEW{Sign}{\sBunkerOneCover}{
  \s\MYname	{Bunker 1}
  \s\MYloc	{Student Lounge / Bunker 1}
  \s\MYtext	{This is one of the bunkers built by \cBunker{\full} and \cBunker{\their} team. The bunkers protect the people who stay at the \pSchool{} at the Time of Deciding from the 3 Storm Surges.

This bunker look fine, at least to the untrained eye.

If you have the ability ``Examine The Bunkers Closely.'' You may make a thorough inspection of the bunkers and identify any issues that need to be addressed.}
}

\NEW{Sign}{\sBunkerOneDamaged}{
  \s\MYname	{DAMAGED Bunker 1}
  \s\MYloc	{Student Lounge / Bunker 1}
  \s\MYtext	{This is one of the bunkers built by \cBunker{\full} and \cBunker{\their} team. The bunkers are supposed to protect the people who stay at the \pSchool{} at the Time of Deciding from the 3 Storm Surges.

\textbf{With some of the siding stripped away to reveal the inner workings, it is plain to see that this bunker is badly damaged.}

In its current state, it can protect many fewer than it should be able to during a Storm Surge. The number of people this bunker can currently protect is: 

\textbf{5}

Everyone else must risk harm during the storm surge. (see the ``\gStormSurgeInside{}'' greensheets posted in each game space for more information.)

\textbf{To repair the bunkers:}\\
The bunkers consist of a \textbf{physical layer} and a \textbf{magical layer},. Progress on either part of the repairs will increase the number of people the bunker can protect.

In the attached envelope you will find 1 Greensheet for each part that needs to be repaired. There are \textbf{3 copies} of each of these greensheets. One copy of these greensheets must stay with this sign. You may borrow the other two copies, but they should always be returned here. You should also feel free to take a picture for reference, or ask the GMs to print you a copy if none are available to borrow. 

\emph{(OOC: Greensheets are OOC information, but these are public sheets that anyone may look at. The sheets may not be stolen, hidden, or destroyed however. This is a kludge for game balance.)}
	}
	\s\MYgreens {\multi{3}{\gPhysicalLayer{}\gMagicalLayer{}}}
}

\NEW{Sign}{\sBunkerOneRepaired}{
  \s\MYname	{Fully Repaired Bunker 3}
  \s\MYloc	{Student Lounge / Bunker 1}
  \s\MYtext	{This is one of the bunkers built by \cBunker{\full} and \cBunker{\their} team. The bunkers protect the people who stay at the \pSchool{} at the Time of Deciding from the 3 Storm Surges.

This bunker has been fully repaired, and can protect \textbf{13 people} from a Storm Surge.}
}

\NEW{Sign}{\sBunkerTwoCover}{
  \s\MYname	{Bunker 2}
  \s\MYloc	{Teacher Lounge / Bunker 2}
  \s\MYtext	{This is one of the bunkers built by \cBunker{\full} and \cBunker{\their} team. The bunkers protect the people who stay at the \pSchool{} at the Time of Deciding from the 3 Storm Surges.

This bunker look fine, at least to the untrained eye.

If you have the ability ``Examine The Bunkers Closely.'' You may make a thorough inspection of the bunkers and identify any issues that need to be addressed.}
}

\NEW{Sign}{\sBunkerTwoDamaged}{
  \s\MYname	{DAMAGED Bunker 2}
  \s\MYloc	{Teacher Lounge / Bunker 2}
  \s\MYtext	{This is one of the bunkers built by \cBunker{\full} and \cBunker{\their} team. The bunkers are supposed to protect the people who stay at the \pSchool{} at the Time of Deciding from the 3 Storm Surges.

\textbf{With some of the siding stripped away to reveal the inner workings, it is plain to see that this bunker is badly damaged.}

In its current state, it can protect many fewer than it should be able to during a Storm Surge. The number of people this bunker can currently protect is: 

\textbf{5}

Everyone else must risk harm during the storm surge. (see the ``\gStormSurgeInside{}'' greensheets posted in each game space for more information.)

\textbf{To repair the bunkers:}\\
The bunkers consist of a \textbf{physical layer} and a \textbf{magical layer}. Progress either part of the repairs will increase the number of people the bunker can protect.

In the attached envelope you will find 1 Greensheet for each part that needs to be repaired. There are \textbf{3 copies} of each of these greensheets. One copy of these greensheets must stay with this sign. You may borrow the other two copies, but they should always be returned here. You should also feel free to take a picture for reference, or ask the GMs to print you a copy if none are available to borrow. 

\emph{(OOC: Greensheets are OOC information, but these are public sheets that anyone may look at. The sheets may not be stolen, hidden, or destroyed however. This is a kludge for game balance.)}
	}
	\s\MYgreens {\multi{3}{\gPhysicalLayer{}\gMagicalLayer{}}}
}

\NEW{Sign}{\sBunkerTwoRepaired}{
  \s\MYname	{Fully Repaired Bunker 2}
  \s\MYloc	{Teacher Lounge / Bunker 2}
  \s\MYtext	{This is one of the bunkers built by \cBunker{\full} and \cBunker{\their} team. The bunkers protect the people who stay at the \pSchool{} at the Time of Deciding from the 3 Storm Surges.

This bunker has been fully repaired, and can protect \textbf{13 people} from a Storm Surge.}
}

\NEW{Sign}{\sBunkerThreeCover}{
  \s\MYname	{Bunker 3}
  \s\MYloc	{Advisor Lounge / Bunker 3}
  \s\MYtext	{This is one of the bunkers built by \cBunker{\full} and \cBunker{\their} team. The bunkers protect the people who stay at the \pSchool{} at the Time of Deciding from the 3 Storm Surges.

This bunker look fine, at least to the untrained eye.

If you have the ability ``Examine The Bunkers Closely.'' You may make a thorough inspection of the bunkers and identify any issues that need to be addressed.}
}

\NEW{Sign}{\sBunkerThreeDamaged}{
  \s\MYname	{DAMAGED Bunker 3}
  \s\MYloc	{Advisor Lounge / Bunker 3}
  \s\MYtext	{This is one of the bunkers built by \cBunker{\full} and \cBunker{\their} team. The bunkers are supposed to protect the people who stay at the \pSchool{} at the Time of Deciding from the 3 Storm Surges.

\textbf{With some of the siding stripped away to reveal the inner workings, it is plain to see that this bunker is badly damaged.}

In its current state, it can protect many fewer than it should be able to during a Storm Surge. The number of people this bunker can currently protect is: 

\textbf{5}

Everyone else must risk harm during the storm surge. (see the ``\gStormSurgeInside{}'' greensheets posted in each game space for more information.)

\textbf{To repair the bunkers:}\\
The bunkers consist of a \textbf{physical layer}, and a \textbf{magical layer}. Progress on either part of the repairs will increase the number of people the bunker can protect.

In the attached envelope you will find 1 Greensheet for each part that needs to be repaired. There are \textbf{3 copies} of each of these greensheets. One copy of these greensheets must stay with this sign. You may borrow the other two copies, but they should always be returned here. You should also feel free to take a picture for reference, or ask the GMs to print you a copy if none are available to borrow. 

\emph{(OOC: Greensheets are OOC information, but these are public sheets that anyone may look at. The sheets may not be stolen, hidden, or destroyed however. This is a kludge for game balance.)}
	}
	\s\MYgreens {\multi{3}{\gPhysicalLayer{}\gMagicalLayer{}}}
}

\NEW{Sign}{\sBunkerThreeRepaired}{
  \s\MYname	{Fully Repaired Bunker 3}
  \s\MYloc	{Advisor Lounge / Bunker 3}
  \s\MYtext	{This is one of the bunkers built by \cBunker{\full} and \cBunker{\their} team. The bunkers protect the people who stay at the \pSchool{} at the Time of Deciding from the 3 Storm Surges.

This bunker has been fully repaired, and can protect \textbf{13 people} from a Storm Surge.}
}



