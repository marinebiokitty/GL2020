%%%%%
%%
%% This file sets up the Abil datatype and creates Abil macros.  These
%% are for abilities that characters may have.
%%
%%%%%

\DECLARESUBTYPE{Abil}{Element}
\PRESETS{Abil}{
  \F\MYtext	%% text of ability, read by user
  \F\MYeffect	%% effect text of ability, read by recipient(s)
  }


%%%%%
%% \ability{<name>}{<text>}{<effect>}
%%
%% \ability is a wrapper around \INSTANCE, useful for 1-shot abilities,
%% etc.
\newinstance{Abil}{\ability[3]}{
  \s\MYname{#1}\s\MYtext{#2}\s\MYeffect{#3}}


%%%%%%%%%%%%%%%%%%%%%%%%%%%%%%%%%%%%%%%%%%%%%%%%%%%%%%%%%%%%%%%%%%

\NEW{Abil}{\aTest}{
  \s\MYname	{Test Ability}
  \s\MYtext	{You are a test.}
  \s\MYeffect	{This is a Test.}
  }

\NEW{Abil}{\aSpecial}{
  \s\MYname	{Special Powers}
  \s\MYtext	{You have special powers, as detailed in your \gTest{}
		greensheet.}
  \s\MYeffect	{I have special powers!}
  \s\MYgreens	{\gTest{}}
  \suite
  }

\NEW{Abil}{\aFiremansCarry}{
  \s\MYname	{Fireman's Carry}
  \s\MYtext	{You can carry a body as if it were two hands bulky.}
  \s\MYeffect	{I can carry this body well.}
  }

%%%%%%%%%%%%%%%%%%%%%%%%%%%%%%%%%%%%%%%%%%%%%%%%%%%%%%%%%%%%%%%%


  
%% Basic DarkWater-style Martial Attack abilities


%% Everyone has these 3
\NEW{Abil}{\aAssist}{
  \s\MYname	{Assist}
  \s\MYtext	{You can assist someone else's attack.  You must be
		within ZoC of both the attacker and target.  Within two
		seconds of an attack, direct this at the attacker,
		saying ``\MYname'' and your CR.}
  \s\MYeffect	{I assist your attack.}
  }

\NEW{Abil}{\aKnockOut}{
  \s\MYname	{Knock Out}
  \s\MYtext	{You can knock someone out as an attack.  This requires
		a {\bf blunt} weapon.  Say ``\MYname'' and your CR.}
  \s\MYeffect	{I knock you out.}
  }

\NEW{Abil}{\aWound}{
  \s\MYname	{Wound}
  \s\MYtext	{You can wound someone as an attack.  This requires an
		{\bf edged} weapon, such as a knife.  Say ``\MYname'' and
		your CR.}
  \s\MYeffect	{I wound you.}
  }


%% the \basecombat macro can be prepended to the Char abils list
%% (in char-LIST.tex)
\def\basecombat{\aKnockOut{}\aWound{}\aAssist{}}

%% only some people have these
\NEW{Abil}{\aDisarm}{
  \s\MYname	{Disarm}
  \s\MYtext	{You can disarm one item from someone as an attack.  Say
		``\MYname'' and your CR.  Point at the item you want to
		disarm.  If the attack works, they must drop that item.}
  \s\MYeffect	{I disarm that item.}
  }

\NEW{Abil}{\aRestrain}{
  \s\MYname	{Restrain}
  \s\MYtext	{You can restrain someone as an attack.  Say ``\MYname''
		and your CR.  You may freely drag, attack, or (if you have
		a weapon) killing-blow them.  To do anything else, or if
		your health state changes, incant ``release'' and let them
		go.}
  \s\MYeffect	{I restrain you.  You are restrained until I incant
		``release.''}
  }
\NEW{Abil}{\aThrow}{
  \s\MYname	{Throw}
  \s\MYtext	{You can throw someone as an attack.  Say ``\MYname'' and
		your CR.  Point in the direction you want to throw them.}
  \s\MYeffect	{I throw you.  Go in the direction I point ten full steps
		or until you hit a wall or similar.}
  }

%%%%%%%%%%%%%%%%%%%%%%%%%%%%%%%%%%%%%%%%%%%


%%% Abilities Related to Voting
\NEW{Abil}{\aPreviewVoteOne}{
  \s\MYname	{Viewing Submitted Votes}
  \s\MYtext	{\emph{Uses: 1.} \textbf{This ability is only useable Sunday from “Game On” to Noon.} You may open one “Voting Stone” item to see how that vote is cast. You may not alter the vote in any way and must replace it exactly as you found it, where you found it. You may search the ballot box for a particular vote, or examine a voting stone someone is still carrying (with their permission). Be careful - a voting stone without a NAME on it may not have a vote inside, but if you use this ability, its use is expended regardless of there being a vote inside.
}
  \s\MYeffect	{You see me messing with the ballot box, and/or the voting stones. This is an inherently suspicious activity.}
}
  
\NEW{Abil}{\aPreviewVoteThree}{
  \s\MYname	{Viewing Submitted Votes}
  \s\MYtext	{\emph{Uses: 3.} \textbf{This ability is only useable Sunday from “Game On” to Noon.} You may open one “Voting Stone” item to see how that vote is cast. You may not alter the vote in any way and must replace it exactly as you found it, where you found it. You may search the ballot box for a particular vote, or examine a voting stone someone is still carrying (with their permission). Be careful - a voting stone without a NAME on it may not have a vote inside, but if you use this ability, its use is expended regardless of there being a vote inside.
		}
  \s\MYeffect	{You see me messing with the ballot box, and/or the voting stones. This is an inherently suspicious activity.}
}

%%%%%%%%%%%%%%%%%%%%%%%%%%%%%%%%%%%%%%%%%%%

\NEW{Abil}{\aExamineBunkers}{
  \s\MYname	{Examine The Bunkers Closely}
  \s\MYtext	{\emph{Uses: Unlimited.} Since you helped build the Bunkers, you know what you are looking for in the way of wear and tear better than any normal person. You may spend \textbf{2 minutes} within 1 ZoC of one of the 3 bunker signs (one in each of the Student lounge, teacher’s lounge, and advisor’s lounge). You are encouraged to technobabble at anyone nearby during this time. After that time, you may \textbf{remove} the top sign (not just lift it) and read the next sign. Your investigation will allow anyone to see the state of the bunker after you are done.}
  \s\MYeffect	{You see me checking over the Bunkers carefully.}
}
