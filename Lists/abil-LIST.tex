%%%%%
%%
%% This file sets up the Abil datatype and creates Abil macros.  These
%% are for abilities that characters may have.
%%
%%%%%

\DECLARESUBTYPE{Abil}{Element}
\PRESETS{Abil}{
  \F\MYtext	%% text of ability, read by user
  \F\MYeffect	%% effect text of ability, read by recipient(s)
  }


%%%%%
%% \ability{<name>}{<text>}{<effect>}
%%
%% \ability is a wrapper around \INSTANCE, useful for 1-shot abilities,
%% etc.
\newinstance{Abil}{\ability[3]}{
  \s\MYname{#1}\s\MYtext{#2}\s\MYeffect{#3}}


%%%%%%%%%%%%%%%%%%%%%%%%%%%%%%%%%%%%%%%%%%%%%%%%%%%%%%%%%%%%%%%%%%

\NEW{Abil}{\aTest}{
  \s\MYname	{Test Ability}
  \s\MYtext	{You are a test.}
  \s\MYeffect	{This is a Test.}
  }

\NEW{Abil}{\aSpecial}{
  \s\MYname	{Special Powers}
  \s\MYtext	{You have special powers, as detailed in your Test
		greensheet.}
  \s\MYeffect	{I have special powers!}
  %\s\MYgreens	{\gTest{}}
  \suite
  }

\NEW{Abil}{\aFiremansCarry}{
  \s\MYname	{Fireman's Carry}
  \s\MYtext	{You can carry a body as if it were two hands bulky.}
  \s\MYeffect	{I can carry this body well.}
  }

%%%%%%%%%%%%%%%%%%%%%%%%%%%%%%%%%%%%%%%%%%%%%%%%%%%%%%%%%%%%%%%%


  
%% Basic DarkWater-style Martial Attack abilities


%% Everyone has these 3
\NEW{Abil}{\aAssist}{
  \s\MYname	{Assist}
  \s\MYtext	{You can assist someone else's attack.  You must be
		within ZoC of both the attacker and target.  Within two
		seconds of an attack, direct this at the attacker,
		saying ``\MYname'' and your CR.}
  \s\MYeffect	{I assist your attack.}
  }

\NEW{Abil}{\aKnockOut}{
  \s\MYname	{Knock Out}
  \s\MYtext	{You can knock someone out as an attack.  This requires
		a {\bf blunt} weapon.  Say ``\MYname'' and your CR.}
  \s\MYeffect	{I knock you out.}
  }

\NEW{Abil}{\aWound}{
  \s\MYname	{Wound}
  \s\MYtext	{You can wound someone as an attack.  This requires an
		{\bf edged} weapon, such as a knife.  Say ``\MYname'' and
		your CR.}
  \s\MYeffect	{I wound you.}
  }


%% the \basecombat macro can be prepended to the Char abils list
%% (in char-LIST.tex)
\def\basecombat{\aKnockOut{}\aWound{}\aAssist{}}

%% only some people have these
\NEW{Abil}{\aDisarm}{
  \s\MYname	{Disarm}
  \s\MYtext	{You can disarm one item from someone as an attack.  Say
		``\MYname'' and your CR.  Point at the item you want to
		disarm.  If the attack works, they must drop that item.}
  \s\MYeffect	{I disarm that item.}
  }

\NEW{Abil}{\aRestrain}{
  \s\MYname	{Restrain}
  \s\MYtext	{You can restrain someone as an attack.  Say ``\MYname''
		and your CR.  You may freely drag, attack, or (if you have
		a weapon) killing-blow them.  To do anything else, or if
		your health state changes, incant ``release'' and let them
		go.}
  \s\MYeffect	{I restrain you.  You are restrained until I incant
		``release.''}
  }
\NEW{Abil}{\aThrow}{
  \s\MYname	{Throw}
  \s\MYtext	{You can throw someone as an attack.  Say ``\MYname'' and
		your CR.  Point in the direction you want to throw them.}
  \s\MYeffect	{I throw you.  Go in the direction I point ten full steps
		or until you hit a wall or similar.}
  }

%%%%%%%%%%%%%%%%%%%%%%%%%%%%%%%%%%%%%%%%%%%

%%%%%%%%% Abilities in GM HQ %%%%%%%%%%%%



%%%%%%%%%%%%%%%%%%%%%%% Abilities Related to Voting %%%%%%%%%%%%%

\NEW{Abil}{\aPreviewVoteOne}{
  \s\MYname	{Viewing Submitted Votes}
  \s\MYtext	{\emph{Uses: 1.} \textbf{This ability is only usable Sunday from “Game On” to Noon.} You may open one “Voting Stone” item to see how that vote is cast. You may not alter the vote in any way and must replace it exactly as you found it, where you found it. You may search the ballot box for a particular vote, or examine a voting stone someone is still carrying (with their permission). Be careful - a voting stone without a NAME on it may not have a vote inside, but if you use this ability, its use is expended regardless of there being a vote inside.
}
  \s\MYeffect	{You see me messing with the ballot box, and/or the voting stones. This is an inherently suspicious activity.}
}
  
\NEW{Abil}{\aPreviewVoteThree}{
  \s\MYname	{Viewing Submitted Votes}
  \s\MYtext	{\emph{Uses: 3.} \textbf{This ability is only usable Sunday from “Game On” to Noon.} You may open one “Voting Stone” item to see how that vote is cast. You may not alter the vote in any way and must replace it exactly as you found it, where you found it. You may search the ballot box for a particular vote, or examine a voting stone someone is still carrying (with their permission). Be careful - a voting stone without a NAME on it may not have a vote inside, but if you use this ability, its use is expended regardless of there being a vote inside.
		}
  \s\MYeffect	{You see me messing with the ballot box, and/or the voting stones. This is an inherently suspicious activity.}
}

\NEW{Abil}{\aStealVoteOne}{
  \s\MYname	{Stealing Voting Stones}
  \s\MYtext	{\emph{Uses: 1.} This is a stealth ability (See Rules \& Scenario Section 8.4). Allows you to steal 1 Voting Stone from the target using the waylay mechanic and a 10-count. The ability is thwarted if anyone attacks you, you lose line of sight to the target, or the target notices the symbol before the count is complete. A failed attempt does not exhaust the ability, but you must start the count over.
		}
  \s\MYeffect	{Do not show the effect to the target. Instead, find a GM and notify them you have used the ability and on whom; the GM will transfer the Voting Stone to you.}
}

\NEW{Abil}{\aStealVoteThree}{
  \s\MYname	{Stealing Voting Stones}
  \s\MYtext	{\emph{Uses: 3x.} This is a stealth ability (See Rules \& Scenario Section 8.4). Allows you to steal 1 Voting Stone from the target using the waylay mechanic and a 10-count. The ability is thwarted if anyone attacks you, you lose line of sight to the target, or the target notices the symbol before the count is complete. A failed attempt does not exhaust the ability, but you must start the count over.
		}
  \s\MYeffect	{Do not show the effect to the target. Instead, find a GM and notify them you have used the ability and on whom; the GM will transfer the Voting Stone to you.}
}

\NEW{Abil}{\aTransferVoteOne}{
  \s\MYname	{Transferring Voting Stones}
  \s\MYtext	{\emph{Uses: 1.} This ability allows a \textbf{willing} student to transfer one of their voting stones to another student of their choice. \textbf{Only stones that have not had a vote cast with them can be transferred.} The student receiving the voting stone does not need to be willing, but both giver and receiver must be within 1 ZoC of you.

		}
  \s\MYeffect	{\textbf{Giver:} Give one of your voting stones to a student of your choice within 1 ZoC. \textbf{Receiver}: Another student gives you one of their voting stones.}
}

\NEW{Abil}{\aTransferVoteThree}{
  \s\MYname	{Transferring Voting Stones}
  \s\MYtext	{\emph{Uses: 3.} This ability allows a \textbf{willing} student to transfer one of their voting stones to another student of their choice. \textbf{Only stones that have not had a vote cast with them can be transferred.} The student receiving the voting stone does not need to be willing, but both giver and receiver must be within 1 ZoC of you.

		}
  \s\MYeffect	{\textbf{Giver:} Give one of your voting stones to a student of your choice within 1 ZoC. \textbf{Receiver}: Another student gives you one of their voting stones.}
}

\NEW{Abil}{\aManipulateVoteOne}{
  \s\MYname	{Manipulating the Vote}
  \s\MYtext	{\emph{Uses: 1.} \textbf{This ability is only useable Sunday from “Game On” to Noon.} You may use this ability to do \textbf{1} of 2 options:
	
	\begin{itemize}
		\item \textbf{Submit an anonymous vote:} Tell a GM you wish to use this ability for this purpose. They’ll handle the rest.
		\item \textbf{Replace an existing vote:} Search the ballot box for an already submitted vote from a specific individual. Open the envelope, remove the paper present, and \textbf{discard it without reading it}. Write your vote on a \textbf{new} piece of paper, put it in the envelope, and return the envelope/stone to the ballot box.
	\end{itemize}
	}
  \s\MYeffect	{You see me messing with the ballot box, and/or the voting stones. This is an inherently suspicious activity.}
}

\NEW{Abil}{\aManipulateVoteThree}{
  \s\MYname	{Manipulating the Vote}
  \s\MYtext	{\emph{Uses: 3.} \textbf{This ability is only useable Sunday from “Game On” to Noon.} You may use this ability to do \textbf{1} of 2 options:
	
	\begin{itemize}
		\item \textbf{Submit an anonymous vote:} Tell a GM you wish to use this ability for this purpose. They’ll handle the rest.
		\item \textbf{Replace an existing vote:} Search the ballot box for an already submitted vote from a specific individual. Open the envelope, remove the paper present, and \textbf{discard it without reading it}. Write your vote on a \textbf{new} piece of paper, put it in the envelope, and return the envelope/stone to the ballot box.
	\end{itemize}
	}
  \s\MYeffect	{You see me messing with the ballot box, and/or the voting stones. This is an inherently suspicious activity.}
}

%%%%%%%%%%%%%%%%%%  Other Player Abilities  %%%%%%%%%%%%%%%%%%%%%%%%%

\NEW{Abil}{\aExamineBunkers}{
  \s\MYname	{Examine The Bunkers Closely}
  \s\MYtext	{\emph{Uses: Unlimited.} Since you helped build the Bunkers, you know what you are looking for in the way of wear and tear better than any normal person. You may spend \textbf{2 minutes} within 1 ZoC of one of the 3 bunker signs (one in each of the Student lounge, teacher’s lounge, and advisor’s lounge). You are encouraged to technobabble at anyone nearby during this time. After that time, you may \textbf{remove} the top sign (not just lift it) and read the next sign. Your investigation will allow anyone to see the state of the bunker after you are done.}
  \s\MYeffect	{You see me checking over the Bunkers carefully.}
}

\NEW{Abil}{\aHealingMusicBox}{
  \s\MYname	{Magical Healing Through Music}
  \s\MYtext	{\emph{Uses: Unlimited; 20 min cool down after use.} You may use this music box to cast healing magic normally reserved exclusively for the clerics of \cFarmGod{}. To do so requires:
\begin{enumerate}
	\item You must be within ZoC of the person to be healed. You cannot target yourself.
	\item You must declare your desired effect out loud. Your options are:
	\begin{enumerate}
		\item Restore 1 magical unit to an individual who has given away some of the magical essence, rather than waiting for it to regenerate on its own. You cannot increase someone’s CR past their maximum this way.
		\item Immediately revive an unconscious target.
	\end{enumerate}
	\item Play the music box for at least 30 seconds.
	\item At the end of the 30 seconds, the effect occurs if applicable to the situation.
\end{enumerate}
}
  \s\MYeffect	{You see me holding a music box that is playing music.}
}

\NEW{Abil}{\aBuildVCD}{
  \s\MYname	{Build Vid-Com Device}
  \s\MYtext	{\emph{Uses: Unlimited.} You’ve been supplying the war effort with Vid-Com devices because you must, but that’s not what you wanted your technology used for. You want them for people to communicate with their families far away, and you are increasingly willing to turn to illegal means to make it happen.

\textbf{If you want to build a Vid-Com device in game:}
\begin{enumerate}
	\item You must have the ``\iVCDBlueprint{}’’ as well as 1 ``\iMagitechParts{}’’
	\item Spend 2 minutes in a Maker’s Space roleplaying building the device. Then discard the ``Misc. Magitech Parts’’ to the nearest stock.
	\item Grab a ``\iVidCom{}'' item card from ``\sSignW{}'' in the Maker’s Space.
	\item Get the Device activated. The church has thus far refused to do it, except in direct service of the war, so you may need the Black Market to do it.
\end{enumerate}
}
  \s\MYeffect	{You see me fiddling with some sort of magitech.}
}

\NEW{Abil}{\aUseTass}{
  \s\MYname	{Using Tass}
  \s\MYtext	{\emph{Uses: Unlimited.} You are able to access the magical energy inside of any item with the ``TASS’’ tag. You can either:

\begin{enumerate}
	\item Extract the energy in the moment of need to increase the CR of your \textbf{next} magical attack by +1. You can never combine multiple units of Tass into a single attack.
	\item Extract the energy in the moment of need to expend for a task that requires the input of magical energy (these are very rare; the most common would be transferring this energy to another item of Tass for some reason.)
\end{enumerate}

In either case, take the CR stone out of the item of Tass and return it to the nearest stock (or place inside a different item of Tass). Do not destroy the Tass item, it can be recharged and reused.
}
  \s\MYeffect	{You see me messing with this item of Tass.}
}

\NEW{Abil}{\aLuck}{
  \s\MYname	{Luck of Genesis}
  \s\MYtext	{\emph{Uses: 1x/day.} Once per day, you may \textbf{EITHER} gain advantage (roll twice and keep the higher result) on one mechanic that asks you to roll a die  \textbf{OR} you may increase your CR by +3 for \textbf{resisting} 1 attack.
	}
  \s\MYeffect	{Just a Lucky Break.}
}

\NEW{Abil}{\aLettersHome}{
  \s\MYname	{Letters Home}
  \s\MYtext	{\emph{Uses: Unlimited} You can address letters to your parents (\cLoud{} and/or \cQuiet{}) and drop them in the envelope associated with ``Sign T’’ in GM HQ. These letters are \textbf{in game items}, and you must \textbf{sign honestly them as yourself}. If they are able to respond to you, they will do so. You should expect responses at the next meal $>$1 hour from when you drop the letter off at ``Sign T’’. 11am on Sunday is the cut off for last letters that you can reasonably expect a response to. Unless a mechanic tells you otherwise, you may not send game items, only written letters, via this mechanic.

	}
  \s\MYeffect	{You see me sending a letter off.}
}

\NEW{Abil}{\aLying}{
  \s\MYname	{Expert Lying and Expert Lie Detection}
  \s\MYtext	{\emph{Uses: 1x/day} You are an expert at lying and at detecting lies. \textbf{Once per day,} you may \textbf{either} determine if someone is lying to you, or resist a lie detection ability used on you.
  
To determine if someone is lying to you, go OOC and ask someone whether to the best of their knowledge the last statement they just made is truthful. No one else gets to hear the answer, so it would be your word against theirs.

To resist a lie detection ability used on you, simply lie either IC or OOC as the mechanic demands. \textbf{Do not show this ability to anyone if you use it for this purpose}. This ability only works for 1 statement or in answer to 1 question. \textbf{This ability does not work against Curses which compel honesty or against the Lariat of Truth.}	}
  \s\MYeffect	{Tell me OOC whether you believe the last statement you just made to be truthful and complete. Aka is your character lying to my character, or concealing information?}
}

\NEW{Abil}{\aBeetle}{
  \s\MYname	{Ask the Beetle}
  \s\MYtext	{\emph{Uses: 3x}  You may use the beetle to ask a question of \cTechGod{}. You must have the \iAvatarBeetle{} in your possession, and it must be alive to use this ability. Ask your question to a GM and we will do our best to answer you in a way that gives you strong direction without necessarily revealing things that can be uncovered in game through character interactions or exploration.}
  \s\MYeffect	{You see me talking to something cupped in my hand.}
}

\NEW{Abil}{\aHaveYouConsidered}{
  \s\MYname	{Have you Considered?}
  \s\MYtext	{\emph{Uses: 2x}  Use this ability to get a group to talk about a specific topic. Everyone in the conversation must share at least one thought about the topic before they may talk about something else. People under the influence of another ability to NOT talk about a particular topic cannot be compelled with this ability. Instead, they must indicate a disinterest in the topic.}
  \s\MYeffect	{Have you Considered…? \emph{(OOC: Everyone in this conversation must give their opinion on a topic of my choice before the conversation may move on. If you are under the influence of another ability to NOT talk about this topic, you must indicate a disinterest in the topic at hand.)}}
}

\NEW{Abil}{\aImpassionedSpeech}{
  \s\MYname	{Give an Impassioned Speech}
  \s\MYtext	{\emph{Uses: 2x}  You can make an impassioned speech that is imbued with your personal brand of magic. You \textbf{can} use this ability mid combat to interrupt it. When you wish to use this ability, go OOC and read the ``Ability Effect'' side of this ability out loud to the room.
  
 \textbf{You must then speak IC for at least 1 minute about something (presumably the thing you are passionate about).} For the 5 minute duration, everyone who hears you is enraptured by you. If you are trying to persuade people to do something, or not do something, players may choose whether their character is persuaded. Even if the player determines their character is not persuaded, they must roll a D20. On a 16 or higher (25% chance) they are able to tear themselves away from your mesmerizing oration and act against your wishes.
 
\textbf{You may not use this ability during a ``vote of no confidence'' called by any Follower of Genesis,} as doing so would undermine the democratic ideals Genesis teaches.}
  \s\MYeffect	{I am using an ability. This ability interrupts combat (if any is currently happening). If you can hear me right now, you may not leave the area, interrupt me if I am talking, or take any combat actions unless I direct you to do so. If your character \textbf{strongly} disagrees with something I say or ask you to do, and therefore wishes to resist this effect, roll a D20. On a 16 or higher you are able to tear yourself away from my mesmerizing oration and act on your own. You may not roll ``just because you don’t like being controlled,'' you must have an IC reason to disagree with the specific things I am saying. This effect will last for 5 minutes, starting now.}
}

\NEW{Abil}{\aOration}{
  \s\MYname	{Oration}
  \s\MYtext	{\emph{Uses: 2x}You can make an impassioned speech that is infused with a little bit of magic. You \textbf{can} use this ability mid combat to interrupt it. When you wish to use this ability, go OOC and read the ``Ability Effect’’ side of this ability card out loud to the room.
  
  \textbf{You must then speak IC for up to 5 minutes (presumably about the thing you are passionate about).} For the duration, everyone who hears you is captivated by what I have to say. If a character wishes to leave the area, or take actions other than listening to you, they must roll a D20. On a 14 or higher (35\% chance) they are able to tear themselves away from your mesmerizing oration and leave the area.
}
  \s\MYeffect	{I am using an ability. This ability interrupts combat (if any is currently happening). As long as I am talking, or for 5 minutes (whichever is shorter), everyone who hears me is captivated by what I have to say. If your character wishes to leave the area, or take actions other than listening to me, you must roll a D20. On a 14 or higher you are able to tear themselves away from your mesmerizing oration and leave the area.}
 }

\NEW{Abil}{\aJewelry}{
  \s\MYname	{Jewelry Appraisal}
  \s\MYtext	{\emph{Uses: Unlimited} If an item of jewelry (necklace, earrings, bracelet, ring, etc) is an item-envelope, you may appraise it (aka learn more about it) by spending 30 seconds examining the object. During this time, you can do nothing else; no talking, no other mechanics, etc. Even noticing a waylay disrupts this and you would have to start over. Once 30 seconds have elapsed, open the envelope and read the note inside. You may show others standing with you at the time what the note inside says. Replace the information in the envelope when you are done. }
  \s\MYeffect	{You see me examining this piece of jewelry.}
 }


\NEW{Abil}{\aPickPocket}{
  \s\MYname	{Pickpocket}
  \s\MYtext	{\emph{Uses: Unlimited; Notify a GM when you want to use this ability. They will help adjudicate to try to avoid arousing suspicion toward you.} You may attempt to pick someone’s pocket by using a waylay. This will \textbf{only} work with non-bulky items. 
  \begin{itemize}
  	\item A 5-count waylay will get you a random item from them, or allow you to place an item of your choice on their person
	\item A 10-count waylay count will get you a specific item.
	\item Voting Stones may not be stolen via this mechanic.
  \end{itemize}
  
  If someone catches you in the waylay, you must tell them ``you see me reaching for $<$Target’s$>$ pockets or bag.''}
  \s\MYeffect	{You see me reaching for their pockets or bag.}
}

\NEW{Abil}{\aFirePhobia}{
  \s\MYname	{Phobia of Fire}
  \s\MYtext	{You have a phobia of fires. No one who lives on ships likes fires on account of the risk they pose, but you break out in a cold sweat any time you see so much as a candle. No one knows why you react so strongly, but you do. 

For any mechanic that requires you to interact with an open flame, even a candle, you are highly encouraged to roleplay being as afraid / avoidant as is fun for you as a player. \textbf{Optional:} When you encounter a source of fire, you may choose to roll a D20. On a 1 or a 2, you cannot bring yourself to approach or handle the source of fire at all, and must leave the area as quickly as is possible to do so safely.}
  \s\MYeffect	{I look distressed.}
}

\NEW{Abil}{\aPrivate}{
  \s\MYname	{This Conversation is Private}
  \s\MYtext	{\emph{Uses: 1x/day} You may have a single \textbf{5 minute} or less conversation with the \textbf{one person} who is your spy network contact. If you are not sure who this is, ask a GM. You may not activate this ability purely in response to someone approaching who is OOC (since your character doesn’t know that anything has changed). Instead you should plan to activate this ability preemptively when you intend to initiate a sensitive conversation.

If anyone approaches you while this ability is active, \textbf{even if they are out of character,} show them the ``Ability Effect’’ side of this card. If they refuse to leave to talk to a GM, you may call a game halt and fetch a GM yourself. Your timer for this ability does not run down during a game halt.}
  \s\MYeffect	{This conversation is magically private. No one may listen in, even if you are invisible or using magic to eavesdrop. If you have a question about it, see a GM.}
}

\NEW{Abil}{\aStay}{
  \s\MYname	{``We’ll Stay''}
  \s\MYtext	{\emph{Uses: 3x} When one or more teachers and/or advisors try to exclude you or any other student(s) from a conversation or dismiss your views \textbf{because of your youth,} you may use this ability to force them to let you stay for the duration of the conversation and consider your perspectives. They do not have to agree with you, but they must treat you as equals until the conversation ends. When you use the ability, you must call the teachers and/or advisors out on trying to exclude you and argue that you should be allowed to contribute.
}
  \s\MYeffect	{If you are trying to exclude me or other student(s) from this conversation or dismiss our views \textbf{because of our youth} (as opposed to other primary reasons), you are impressed by how I stand up for myself and the other students; you must let us stay, hear our arguments out, and treat us as equals for the duration of the conversation.}
}

\NEW{Abil}{\aNewTopic}{
  \s\MYname	{Let’s Talk about Something Else}
  \s\MYtext	{\emph{Uses: 2x} You may use this ability to cause everyone involved in the current conversation to stop the current line of inquiry, and focus on something else. You may not dictate the topic, but no one in the group may return to the previous topic for at least 5 minutes, even if they walk away to have a conversation with someone else. This ability cannot be used to stop combat.}
  \s\MYeffect	{Let’s talk about something else. (Everyone here must abandon this line of inquiry for 5 minutes and talk about something else, even if you walk away from this conversation.)}
}

\NEW{Abil}{\aDisguise}{
  \s\MYname	{Disguise Self}
  \s\MYtext	{\emph{Uses: Unlimited} Part of how you avoid being recognized is a disguise spell. Disguise spells are \textbf{incredibly} tricky to maintain, and bigger changes are more taxing. So you’ve only modified your features some. It does pose the risk that *somebody* might recognize you, but it’s better than risking it catastrophically failing in the middle of the Time of Deciding and *everybody* recognizing you.

You can turn the disguise spell off at will, by putting on a green headband. Tell folks you look different now when you have the green headband on. You may choose to don your other badge as well, but that is entirely up to you. You turn the disguise spell back on, by taking the headband off (and switching the badge back if you chose to wear the other one). Show anyone who sees you turning the disguise on/off the ``ability effect’’ side of this card.
}
  \s\MYeffect	{My features shimmer briefly and now I look somewhat different.}
}

\NEW{Abil}{\aWarlordDaughter}{
  \s\MYname	{Child of the Warlord}
  \s\MYtext	{\emph{Uses: 1x/day} As desperate as you are to distance yourself from your Warlord \cLoud{parent}'s bloodthirsty ways, \cLoud{\they} cast\cLoud{\plural} a long shadow, and there are times when your only recourse is to draw upon that shadow. By invoking \cLoud{}’s name and calling upon your own resolve and buried anger and frustration, you can intimidate almost any opponent. 

Make an Upstage attack at +1 CR against the person you are trying to intimidate and accompany it with dire threats and invoking \cLoud{} by name; no one can assist you in this attack. If your attack is successful, your target is thoroughly intimidated by you and cannot interfere with your actions for the next five minutes. They are not strictly obligated to assist you or answer your questions, but they are afraid of your wrath. There may be social consequences for using this ability.
}
  \s\MYeffect	{You are intimidated by me for the next five minutes and cannot interfere with my actions. You are not strictly obligated to assist me or answer my questions, but you are afraid of my wrath.}
}

\NEW{Abil}{\aSonata}{
  \s\MYname	{Soul Stirring Sonata}
  \s\MYtext	{\emph{Uses: 1x/day.} You are an extremely talented musician and have special dispensation from the Church to put those talents to use. While your performances lack the magical healing powers of the Clerics, you can stir powerful emotions in your listeners. When you activate this ability, for the next five minutes, everyone within earshot is enraptured by your performance and can take no action except listening; this \textbf{can} be used to interrupt combat. Choose an emotion you are aiming to elicit; your audience will be flooded with that emotion at least until the five minutes are up. If you feel comfortable doing so, you are welcome to actually sing or perform for the duration, or play music from your phone, but regardless, you cannot take any other action until the five minutes are up. If the music stops, the effect ends early.}
  \s\MYeffect	{For the next five minutes, everyone within earshot is enraptured by my musical performance and can take no action except listening; this interrupts combat. You are flooded with an emotion of my choice at least until the five minutes are up.}
}

\NEW{Abil}{\aUnobtrusive}{
  \s\MYname	{Unobtrusive}
  \s\MYtext	{\emph{Uses: 3x.}You have a tendency to overhear things that people say in private. Whether it’s your unobtrusive presence, your earnest desire to help others putting people at ease, or a gift of the Goddesses, words not intended for your ears tend to find their way to them nonetheless. 

When you activate this ability, for the next five minutes, other characters do not react to your presence unless you speak or take an action, in which case the ability immediately ends. They notice you are there, but you are so unobtrusive that no one bothers to change the subject or ask you to leave, even if the conversation is secret. Some conversations may be magically protected, preventing this ability from being used.  Players in such a conversation will tell you as much.
}
  \s\MYeffect	{For the next five minutes, you cannot react to my presence unless I speak or take an action. You notice I am here, but I am so unobtrusive that you see no reason to change the subject or ask me to leave, even if the conversation is secret. If the conversation is magically protected by an ability, tell me, as this ability will not work.}
}

\NEW{Abil}{\aGenesisGift}{
  \s\MYname	{Gifts of Genesis}
  \s\MYtext	{\emph{Uses: 7x (you have 8 eligible abilities; you must keep at least 1 for yourself).} As a devoted Follower of \cGenesis{}, you have been granted the authority to \textbf{give away} some of your abilities to new recruits, helping them feel more welcome and connected to \cGenesis{}' power. Anyone who hears \cGenesis{}'s call will gain ``\aLuck{\MYname}'' but  you may give away (unspent instances) of the abilities: ``\aPreviewVoteOne{\MYname},'' ``\aStealVoteOne{\MYname},'' and ``\aManipulateVoteOne{\MYname}''. You may only give abilities to people with a \textbf{V-score = 1 or 2} (you may freely ask someone their V-score). Once given away, you cannot reclaim abilities.}
  \s\MYeffect	{You feel a soft pulse of divine, lucky energy. You now have the ability I am giving you. Normally abilities are non-transferable; this mechanic overrides that.}
}

