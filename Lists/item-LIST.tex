
%%
%% This file creates the Item, ItemPacket, ItemFold, ItemEnvelope, and
%% ItemLabel datatypes, and creates macros for each.  These are for
%% various types of in-game items.
%%
%%%%%


%%%%%
%% Item macros are for normal item cards.
\DECLARESUBTYPE{Item}{TransElement}
\PRESETS{Item}{
  \FD\MYtext	{} %% longer text of item
  \FD\MYmark	{} %% possible contents of shaded ``mark'' on card
  \FD\MYbulky	{0} %% potential bulkiness
  \FD\MYcapacity{N/A} %% potential capacity
  \sd\MYlistmap	{\item\MYname\ifx\MYnumber\empty\else\ (\MYnumber)\fi}
  }


%%%%%
%% \prop
%% \unstash
%% \bulky{<number>}
%% \contain{<number>}
%%
%% \prop inside an Item macro labels the card as a prop.  \unstash
%% labels the card as unstashable.  \bulky{n} labels the card as
%% n-hands bulky.  \contain{n} labels the card with n-hands capacity.
\def\prop{%
  \append\MYmark{ ~PROP~ }}
\def\tass{%
  \append\MYmark{ ~TASS~ }}
\def\unstash{%
  \append\MYmark{ ~UNSTASHABLE~ }}
\def\bulky#1{%
  \s\MYbulky{#1}%
  \append\MYmark{\mbox{ ~\MYbulky-Hand(s)~Bulky~ }}}
\def\contain#1{%
  \s\MYcapacity{#1}%
  \append\MYmark{\mbox{ ~\MYcapacity-Hand~Capacity~ }}}


%%%%%
%% ItemPacket macros are for item cards with an attached packet.
%% They are a subtype of Item.
\DECLARESUBTYPE{ItemPacket}{Item}
\PRESETS{ItemPacket}{
  \F\MYcontents
  }


%%%%%
%% ItemFold macros are for items represented by just a folded packet.
%% They are a subtype of ItemPacket, with the longer text and ``mark''
%% left blank, since they have no actual item card.
\DECLARESUBTYPE{ItemFold}{ItemPacket}
\PRESETS{ItemFold}{
  \s\MYmark{}
  }


%%%%%
%% ItemEnvelope macros are for items represented by just an envelope.
%% They are a subtype of ItemPacket, with the longer text and ``mark''
%% left blank, since they have no actual item card.
\DECLARESUBTYPE{ItemEnvelope}{ItemPacket}
\PRESETS{ItemEnvelope}{
  \s\MYmark{}
  }


%%%%%
%% ItemLabel macros are for small labels that would get used on
%% physreps, e.g. gun labels.  The ``mark'' is left blank, since
%% it isn't used for these.
\DECLARESUBTYPE{ItemLabel}{Item}
\PRESETS{ItemLabel}{
  \s\MYmark{}
  }


%%%%%
%% \icard[<extras>]{<name>}{<number>}{<text>}
%% \specialicard[<extras>]{<name>}{<number>}{<text>}{<mark>}
%% \itempacket[<extras>]{<name>}{<number>}{<text>}{<mark>}{<contents>}
%% \itemfold{<name>}{<number>}{<text>}{<contents>}
%% \itemenvelope{<name>}{<number>}{<text>}{<contents>}
%% \itemlabel{<name>}{<number>}{<text>}
%%
%% These are wrappers around \INSTANCE, useful for 1-shots.
%%
%% For \icard, \specialicard, and \itempacket, the optional <extras>
%% (in []'s) is for things like \unstash and \bulky{3}.  For example,
%% \icard[\prop\contain{2}]{..}{..}{..}{..} gives an item that has a
%% prop and 3-hands capacity.
%%
%% The last arg (#5) to \specialicard is for anything extra you may
%% want in the ``mark''
\newinstance{Item}{\icard[4][]}{
  \s\MYname{#2}\s\MYnumber{#3}\s\MYtext{#4}#1}
\newinstance{Item}{\specialicard[5][]}{
  \s\MYname{#2}\s\MYnumber{#3}\s\MYtext{#4}\s\MYmark{#5}#1}
\newinstance{ItemPacket}{\itempacket[6][]}{
  \s\MYname{#2}\s\MYnumber{#3}\s\MYtext{#4}\s\MYmark{#5}\s\MYcontents{#6}#1}
\newinstance{ItemFold}{\itemfold[4]}{
  \s\MYname{#1}\s\MYnumber{#2}\s\MYtext{#3}\s\MYcontents{#4}}
\newinstance{ItemEnvelope}{\itemenvelope[4]}{
  \s\MYname{#1}\s\MYnumber{#2}\s\MYtext{#3}\s\MYcontents{#4}}
\newinstance{ItemLabel}{\itemlabel[3]}{
  \s\MYname{#1}\s\MYnumber{#2}\s\MYtext{#3}}


%%%%%%%%%%%%%%%%%%%%%%%%%%%%%%%%%%%%%%%%%%%%%%%%%%%%%%%%%%%%%%%%%%

\NEW{Item}{\iTest}{
  \s\MYname	{Test Item}
  \s\MYnumber	{0000}
  \s\MYtext	{A Test Item Card}
  }

\NEW{ItemPacket}{\iTestPacket}{
  \s\MYname	{Test Item}
  \s\MYnumber	{0000}
  \s\MYtext	{A Test Item with a big red button.  Open packet if
		you press the big red button.}
  \s\MYcontents	{The item beeps at you.}
  }

\NEW{ItemFold}{\iTestFold}{
  \s\MYname	{Test Food}
  \s\MYnumber	{0000}
  \s\MYtext	{open if you eat}
  \s\MYcontents	{It tastes yummy.}
  }

\NEW{ItemEnvelope}{\iTestEnvelope}{
  \s\MYname	{Test Food}
  \s\MYnumber	{0000}
  \s\MYtext	{open if you eat}
  \s\MYcontents	{It tastes yummy.}
  }

\NEW{ItemLabel}{\iTestLabel}{
  \s\MYname	{Test Gun Label}
  \s\MYnumber	{0000}
  \s\MYtext	{Disc gun, loadable to 20 shots.}
  }

\NEW{Item}{\iWhatzit}{
  \s\MYname	{Whatzit}
  \s\MYnumber	{12345}
  \s\MYtext	{If you press it, open packet a.  If you twirl it, open
		packet b.  If you pull it, open packet c.}
  \bulky	{1}
  \s\MYsigns	{\signstrip{a}{it goes ``beep.''}
		\signstrip{b}{it goes ``whoop.''}
		\signstrip{c}{it goes ``bang.''}
		}
  \s\MYabils	{\ability{Stop Crying}{By futzing with the Whatzit, you
		can make babies stop crying.}{I make the baby stop
		crying.}
		}
  }


%%%%%%%%%%%%%%%%%%%%%%%%%%%%%%%%%%%%%%%%%%%%%%%%%%%%%%%%%%%%%%%%%%


%%COS Character Items
\NEW{ItemEnvelope}{\iBeansNB}{ %%Prototype beans. \cCurse{} starts with these
  \s\MYname	{Beans}
  \s\MYnumber	{tbd}
  \s\MYtext	{A handful of beans. Were they\ldots buzzing for a moment? \emph{OOC: Do not open unless directed to do so.}}
  \s\MYcontents { \textbf{Tell a GM immediately if you plant them somewhere!}}
  \s\MYitems {\iBeansMB}
}

\NEW{Item}{\iBeansMB}{
  \s\MYname	{Magic Beans}
  \s\MYnumber	{tbd}
  \s\MYtext	{A handful of beans. They buzz loudly and even rattle a little in your hand.}
}

\NEW{ItemEnvelope}{\iSignetRing}{ %%\cHedonist{} starts with this
  \s\MYname	{A Fancy Ring}
  \s\MYnumber	{tbd}
  \s\MYtext	{A very fancy ring. You may only open this item if specifically told to do so by a mechanic.}
  \s\MYcontents	{This is a signet ring of the royal family of \pFarm{}. It can be used to endorse documents as official. What is it doing here? Only the \cQueen{\Majesty}, or \cQueen{\their} heir should ever have this.}
}


\NEW{ItemEnvelope}{\iFolderOfNotes}{ %%\cDisney{} starts with this
  \s\MYname	{A Folder of Notes}
  \s\MYnumber	{tbd}
  \s\MYtext	{A very fancy ring. You may only open this item if specifically told to do so by a mechanic.}
  \s\MYcontents	{This is a signet ring of the royal family of \pFarm{}. It can be used to endorse documents as official. What is it doing here? Only the \cQueen{\Majesty}, or \cQueen{\their} heir should ever have this.}
}

%%FPF Character Items


\NEW{Item}{\iScholarship}{ %%Antichup starts with this
  \s\MYname	{A Scholarship Document}
  \s\MYnumber	{tbd}
  \s\MYtext	{A scholarship from the Church of \cTechGod{} for \cScholarship{}. It is currently authorized under \cAntiChup{} but could be transferred to another member of the \pTech{} clergy if both the current authorizer and the new authorizer consent. Technically \cScholarship{} can't object to such a transfer, but doing so without notifying \cScholarship{\them} would be unorthodox. \emph{OOC: Update this item as necessary if the authorizer changes.}}
}

\NEW{Item}{\iVCDBlueprint}{ %%TechStar starts with this
  \s\MYname	{Blueprint for Vid-Com Devices}
  \s\MYnumber	{tbd}
  \s\MYtext	{A detailed blue-print for how to build the new fangled long distance communication devices developed by \cTechStar{} as part of the Tech Star Competition last year.
	}
}

%%WR Character Items
\NEW{Item}{\iCursedLetter}{ %%\cInitiate starts with this
  \s\MYname	{A handwritten letter}
  \s\MYnumber	{tbd}
  \s\MYtext	{This letter is written on the highest quality paper,  by someone with exquisite calligraphy skills. It reads \emph{Dearest \cInitiate{}, It has taken us entirely too long to find you after your parents abandoned \pFarm{} for \pShip{}. We very much wish to know our grand\cIntiate{\child}, who we have come to know is greatly skilled in magic. We implore you to visit us, that we might tempt you to stay, and inherit all that is rightfully yours. Signed, Your loving grandparents.}}
}

%%The Relics
\NEW{ItemEnvelope}{\iChalice}{ %%Dual? Flow?
  \s\MYname	{The Hallowed Chalice}
  \s\MYnumber	{tbd}
  \bulky	{1}
  \s\MYtext	{open if you have the ability to check alignment, or if you are directed by a mechanic to change the alignment. Replace contents once complete.}
  \s\MYcontents	{Current Alignment: \pShip{}}
}

\NEW{ItemEnvelope}{\iNet}{  %%Dual? Ebb?
  \s\MYname	{Net of Two Phases}
  \s\MYnumber	{tbd}
  \bulky	{1}
  \s\MYtext	{open if you have the ability to check alignment, or if you are directed by a mechanic to change the alignment. Replace contents once complete.}
  \s\MYcontents	{Current Alignment: Neutral}
}

\NEW{ItemEnvelope}{\iMirror}{  
  \s\MYname	{Cassiopeia’s Mirror}
  \s\MYnumber	{tbd}
  \bulky	{1}
  \s\MYtext	{open if you have the ability to check alignment, or if you are directed by a mechanic to change the alignment. Replace contents once complete.}
  \s\MYcontents	{Current Alignment: \pTech{}}
}

\NEW{ItemEnvelope}{\iLariat}{  
  \s\MYname	{Lariat of Truth}
  \s\MYnumber	{tbd}
  \bulky	{1}
  \s\MYtext	{open if you have the ability to check alignment, or if you are directed by a mechanic to change the alignment. Replace contents once complete.}
  \s\MYcontents	{Current Alignment: \pTech{}}
}

\NEW{ItemEnvelope}{\iScythe}{  
  \s\MYname	{Scythe of Silence}
  \s\MYnumber	{tbd}
  \bulky	{1}
  \s\MYtext	{open if you have the ability to check alignment, or if you are directed by a mechanic to change the alignment. Replace contents once complete.}
  \s\MYcontents	{Current Alignment: \pFarm{}}
}

\NEW{ItemEnvelope}{\iPitcher}{  
  \s\MYname	{Golden Amphora}
  \s\MYnumber	{tbd}
  \bulky	{1}
  \s\MYtext	{open if you have the ability to check alignment, or if you are directed by a mechanic to change the alignment. Replace contents once complete.}
  \s\MYcontents	{Current Alignment: \pFarm{}}
}

\NEW{ItemEnvelope}{\iHorseshoe}{  
  \s\MYname	{Silver Horseshoe} %Starts with ChupLeader
  \s\MYnumber	{tbd}
  \bulky	{1}
  \s\MYtext	{open if you have the ability to check alignment, or if you are directed by a mechanic to change the alignment. Replace contents once complete.}
  \s\MYcontents	{Current Alignment: \pFarm{}}
}

\NEW{ItemEnvelope}{\iLocket}{  
  \s\MYname	{Shadow Locket} %Starts with Disney
  \s\MYnumber	{tbd}
  \bulky	{1}
  \s\MYtext	{open if you have the ability to check alignment, or if you are directed by a mechanic to change the alignment. Replace contents once complete.}
  \s\MYcontents	{Current Alignment: \pFarm{}}
}

%%%%%%%%%%%%% Bunker Repair Items %%%%%%%%%%%%%%
\NEW{Item}{\iCornerPiece}{
  \s\MYname	{Corner Piece}
  \s\MYnumber	{tbd}
  \s\MYtext	{A fragment of the magical protections from the bunkers. Will need to be assembled with others to make a complete picture.}
}


\NEW{Item}{\iEdgePiece}{
  \s\MYname	{Edge Piece}
  \s\MYnumber	{tbd}
  \s\MYtext	{A fragment of the magical protections from the bunkers. Will need to be assembled with others to make a complete picture.}
}

\NEW{Item}{\iCenterPiece}{
  \s\MYname	{Center Piece}
  \s\MYnumber	{tbd}
  \s\MYtext	{A fragment of the magical protections from the bunkers. Will need to be assembled with others to make a complete picture.}
}



%%%%%%%%%%%%% GM Items / GM HQ %%%%%%%%%%%%%%
\NEW{ItemEnvelope}{\iVotingStones}{
  \s\MYname	{Voting Stone}
  \s\MYnumber	{tbd}
  \s\MYtext	{(This item cannot be stolen, traded, or given away unless a mechanic enables it.) Open as directed by the ``\gVotingInstructions{}'' Greensheet. If you do not have the greensheet, someone with the sheet can explain it to you.}
  \s\MYcontents	{I cast this vote to send the storm to: \makebox[1.5in]{\hrulefill}}
}

\NEW{Item}{\iFastActingPoison}{
  \s\MYname	{Curse: Fast Acting Poison}
  \s\MYnumber	{tbd}
  \s\MYtext	{An ominous vial of fast acting poison - odorless, tasteless, and dissolves instantly in liquid. Not something you would want to be caught carrying. Unless you have a mechanic that calls for this specific item for a specific purpose (check the item number), you must notify a GM if you wish to use this curse. (It cannot be used like a normal Curse)}
}

\NEW{Item}{\iAvatarSeaSerpent}{
  \s\MYname	{A Baby Sea Serpent that radiates Ebb's Energy}
  \s\MYnumber	{51XX} %TBD
  \s\MYtext	{\textbf{See a GM for the phys rep if necessary.} This is a real live, newly hatched, sea serpent. It glows with \cEbb{}’s aura and energy, and can actually be handled and cared for in relative safety by clerics of \cEbb{}. It tries to bite everyone else. It is clearly very newly made as the avatar, and still somewhat vulnerable. If you wish to hurt or kill the baby for some reason, you must first knock it out with a normal CR attack. The serpent’s CR is listed below.
	
\vspace{2mm}
\textbf{Current CR: }
	}
}

\NEW{ItemEnvelope}{\iMindWipeCurse}{
  \s\MYname	{Curse: Mind Wipe?!?}
  \s\MYnumber	{tbd}
  \s\MYtext	{A vial of cloudy liquid. The curse stored in this object is sufficiently potent that while you cannot determine the details with a cursory investigation, anyone can tell that the effect will be akin to a significant or complete mind-wipe. This curse may be applied like any other curse with the following \textbf{addition:} After applying the curse, open the item envelope and give the research-notebook inside to whoever you activated the curse on.}
	\s\MYcontents {N/A}
	\s\MYmems	{\mReversingMindWipe{}}
}
  
%%%%%%%%%%%%% Garden Items %%%%%%%%%%%%%%
  
\NEW{Item}{\iNightshade}{ %% No mechanic yet
  \s\MYname	{Nightshade}
  \s\MYnumber	{tbd}
  \s\MYtext	{A beautiful nightshade, innocent, white flowers belying the poison contained in its glistening, midnight blue fruits.}
}

\NEW{Item}{\iMoonflower}{ %% Using Casseopia's Mirror to look into the past
  \s\MYname	{Moonflower}
  \s\MYnumber	{tbd}
  \s\MYtext	{A gentle moonflower, ghostly white as its namesake, swaying silently in the breeze. Its scent is deep and intoxicating as moonshadow.}
}


\NEW{Item}{\iMorningGlory}{ %% Using Casseopia's Mirror to look into the past
  \s\MYname	{Morning Glory}
  \s\MYnumber	{tbd}
  \s\MYtext	{A bright pink morning glory flower, glistening with dew, unopened buds promising more vibrant blooms yet to come.}
}

\NEW{Item}{\iFlameOrchid}{ %% No mechanic yet
  \s\MYname	{Flame Orchid}
  \s\MYnumber	{tbd}
  \s\MYtext	{A crimson orchid, dancing in the breeze like flame. Whether you were already aware or find out the hard way, every part of the plant except the flower is covered in irritating hairs that feel like being burned.}
}

\NEW{Item}{\iLimestone}{ 
  \s\MYname	{Piece of Limestone}
  \s\MYnumber	{tbd}
  \s\MYtext	{A small piece of limestone. Not particularly remarkable, but often symbolically associated with the \pShip{}.}
}

\NEW{Item}{\iMemoryCure}{
  \s\MYname	{Vial With Iridescent Liquid}
  \s\MYnumber	{tbd}
  \s\MYtext	{It’s a vial with some iridescent liquid inside it. Why is there dirt on the outside of the vial?}
}

%%%%%%%%%%%%% Graveyard Items %%%%%%%%%%%%%%

\NEW{Item}{\iStoneFlower}{ %% Using Casseopia's Mirror to look into the past; also 2 other locations TBD
  \s\MYname	{Stone Flower}
  \s\MYnumber	{tbd}
  \bulky	{1}
  \s\MYtext	{The impeccably preserved likeness of a flower, cast in pale, translucent crystal. It seems to catch the light and hold it, glowing ever so softly in all but total darkness. Whatever forces calcified this flower into its present form have been lost to time.}
}

\NEW{Item}{\iBlackCrocus}{
  \s\MYname	{Black Crocus}
  \s\MYnumber	{tbd}
  \s\MYtext	{A deceptively delicate flower that is common across the \pTech{} and often blooms so early that there is still snow on the ground, causing a striking contrast.}
}

%%%%%%%%%%%%% Training Field Items %%%%%%%%%%%%%%


\NEW{Item}{\iLily}{ %% Using Casseopia's Mirror to look into the past; WarlordDaughter's RN to cure the Warlord
  \s\MYname	{Lily}
  \s\MYnumber	{tbd}
  \s\MYtext	{A bold yellow lily, almost gold in color, proud and radiant as the noonday sun.}
}

\NEW{Item}{\iWoodenPlank}{
  \s\MYname	{Wooden Plank}
  \s\MYnumber	{tbd}
	\bulky{1}
  \s\MYtext	{An unfinished wooden plank. Good for quick construction projects.}
}

\NEW{Item}{\iWoodenBlock}{
  \s\MYname	{Block of Wood}
  \s\MYnumber	{tbd}
  \s\MYtext	{A small block of wood. Good for carving something like a toy, but not much else.}
}

%%%%%%%%%%%%% Temple Items %%%%%%%%%%%%%%
\NEW{Item}{\iOakStaff}{
  \s\MYname	{Carved Oak Staff}
  \s\MYnumber	{tbd}
	\bulky{1}
	\unstash{}
  \s\MYtext	{An intricately carved staff. This staff is one of the oldest known instances of these staffs. One is kept in each  temple across \pEarth{} and used in the ritual to promote an initiate to a full cleric.}
}

%%%%%%%%%%%%% Library Items %%%%%%%%%%%%%%
\NEW{Item}{\iBrassNails}{ %%NEED TO ASSIGN LOCATION
  \s\MYname	{Handful of Brass Nails}
  \s\MYnumber	{tbd}
  \s\MYtext	{A handful of large brass nails. Good for quick construction projects designed for channeling magical energy. Brass is a metal that is known for its ability to conduct and blend magic from multiple sources due to its nature as an alloy of copper and zinc.}
}


%%%%%%%%%%%%% Old Wing Items %%%%%%%%%%%%%%
\NEW{Item}{\iMagitechParts}{
  \s\MYname	{Misc. Magitech parts}
  \s\MYnumber	{tbd}
	\bulky{1}
  \s\MYtext	{A respectable pile of raw materials and common (and uncommon) magitech parts. Useful for constructing technological marvels.}
}

\NEW{Item}{\iEagleFeather}{
  \s\MYname	{An Eagle Feather}
  \s\MYnumber	{tbd}
  \s\MYtext	{A shed feather of an eagle. A common ingredient in creating curses. They lose their potency if plucked from a live bird; only feathers shed naturally can be used for making curses due to \cFarmGod{}'s edict against harming animals.}
}

\NEW{Item}{\iRabbitStatue}{
  \s\MYname	{A Small Statue of a White Rabbit}
  \s\MYnumber	{tbd}
  \s\MYtext	{\textbf{See a GM for the phys rep if necessary.} A well made ceramic statue of a white rabbit.
	}
}

%%%%%%%%%%%%% Maker Space %%%%%%%%%%%%%%
\NEW{ItemEnvelope}{\iProtypeMusicBox}{
  \s\MYname	{Prototype Music Box}
  \s\MYnumber	{41XX} %TBD
  \s\MYtext	{A simple music box. How quaint. It would make a nice gift for a child.}
  \s\MYcontents	{N/A}
	\s\MYitems	{\iMagicMusicBox{}}
}

\NEW{ItemEnvelope}{\iMagicMusicBox}{
  \s\MYname	{Magical Music Box}
  \s\MYnumber	{45XX} %TBD
  \s\MYtext	{\textbf{This item has a phys rep.} A music box vibrating softly with some sort of enchantment. \textbf{ONLY IF} this item has the word ``imbued'' written on it, you may open the envelope to read the ability inside. As long as you possess the music box, you may use the ability.}
  \s\MYcontents	{N/A}
	\s\MYabils	{\aHealingMusicBox{}}
}

\NEW{Item}{\iPedestalForRelic}{
  \s\MYname	{Pedestal for Relic}
  \s\MYnumber	{tbd} 
	\bulky{2}
	\unstash{}
  \s\MYtext	{\textbf{This item has a phys rep.} This is a pedestal specially constructed to hold one of the Relics to help channel their energy into the Ritual to Control the Storm.}
}

\NEW{Item}{\iToyBoat}{
  \s\MYname	{A Toy Boat}
  \s\MYnumber	{tbd} 
	\bulky{2}
	\unstash{}
  \s\MYtext	{This cute little toy boat is quite well made for being cobbled together from a roughly carved wooden hull, an eagle feather for a sail, and some bits of thread for rigging.}
}

\NEW{Item}{\iVidCom}{
  \s\MYname	{Vid-Com Device}
  \s\MYnumber	{45XX} %TBD 
  \s\MYtext	{A new fangled long distance communication device. A \textbf{pair} of imbued Vid-Com devices is required to communicate with someone \emph{(OOC: think of them like long range walkie talkies with video functionality - not as fancy or with as many utilities as a phone).} \textbf{Non-functional unless imbued.} (someone with the appropriate mechanic may choose to imbue the item, and will write ``imbued'' on this item card if they do so).}
}

%%%%%%%%%%%%% Student Lounge %%%%%%%%%%%%%%
\NEW{Item}{\iBabySeaSerpent}{
  \s\MYname	{A Baby Sea Serpent}
  \s\MYnumber	{51XX} %TBD
  \s\MYtext	{\textbf{See a GM for the phys rep if necessary.} This is a real live, newly hatched, sea serpent. How on \pEarth{} did it get here? It is very bitey. Watch your fingers! If you wish to hurt or kill the baby for some reason, you must first knock it out with a normal CR attack. The serpent’s CR is listed below:
	
\vspace{2mm}
\textbf{Current CR: 3}
	}
}


%%%%%%%%%%%%% Teacher Lounge %%%%%%%%%%%%%%
\NEW{Item}{\iThread}{
  \s\MYname	{Length of Thread}
  \s\MYnumber	{tbd}
  \s\MYtext	{A plain length of thread. Good for a small mending project or some such.}
}

\NEW{Item}{\iFancyCloth}{
  \s\MYname	{Fancy Cloth}
  \s\MYnumber	{tbd}
  \s\MYtext	{A fancy cloth with a beautiful, reflective sheen.}
}


%%%%%%%%%%%%% Advisor Lounge %%%%%%%%%%%%%%

%%%%%%%%%%%%%%%%%%%%%%%%%%%%%%%%%%%%%%%%%%%%%%%%%%%%%%%%%%%%%%%%%%
